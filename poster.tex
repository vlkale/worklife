\documentclass[serif, mathserif, final]{beamer}
%\mode<presentation>{\usetheme{Lankton}}
\usepackage[orientation=portrait,size=a0,scale=1.6, debug]{beamerposter}
\usepackage{type1cm}
\usepackage{amsmath,amsfonts,amssymb,pxfonts,xspace}
\usepackage{graphicx,ragged2e,pgffor}
\usepackage{hyperref}
\usepackage{ulem}
\usepackage{courier}
\usepackage{caption,verbatim,listings,float,subfig} %TODO: order these 
\setbeamertemplate{caption}[numbered] 
%\usepackage{calendar}
\usepackage{geometry}
\usepackage{multicol}

\usepackage{lipsum}
\usepackage{multicol}


%\usepackage{pagecolor=beige}

%TODO: remove unneeded items 
\newcommand{\bllt}{\item \small}
\newcommand{\doneTaskNoItem}[1]{\sout{#1}}
%TODO: consider changing the name of this macro to be what it's
%used for below

\newcommand{\projectTask}{\tiny \item \tiny}
\newcommand{\pitem}{\tiny \item \tiny}

\newcommand{\pitem}{\tiny \item \tiny}

\newcommand{\ptask}{\tiny \item \tiny}

\newcommand{\doneTaskNoItemNewLine}[1]{\sout{#1}}

\newcommand{\doneTask}[1]{\tiny \item \tiny \sout{#1}}
\newcommand{\doneTaskHyp}[1]{\tiny \item \tiny \textcolor{blue} {\sout{#1}}}
\newcommand{\optTask}[1]{\tiny \item \tiny \textcolor{green}{#1}}
\newcommand{\prioTask}[1]{\tiny \item \tiny \textcolor{red}{#1}}

\newcommand{\timeEst}[1]{\textit{TimeEst:}\textit{#1}}
\newcommand{\te}[1]{\textit{TimeEst:}\textit{#1}}
%\newcommand{\t}[1]{\textit{TimeEst:}\textit{#1}} - command t already
%defined in latex, so we can't use the preceding command. 
\newcommand{\deadline}[1]{\textit{Deadline:}\textit{#1}}
\newcommand{\dueBy}[1]{\textit{Deadline:}\textit{#1}}
%\newcommand{\d}[1]{\textit{Deadline:}\textit{#1}}
\newcommand{\dl}[1]{\textit{Deadline:}\textit{#1}}
\newcommand{\priority}[1]{\textit{Priority:}\textit{#1}}
\newcommand{\prio}[1]{\textit{Priority:}\textit{#1}}
\newcommand{\pr}[1]{\textit{Priority:}\textit{#1}}
%\newcommand{\p}[1]{\textit{Priority:}\textit{#1}}

\newcommand{\MyName}{Vivek~Kale}
\newcommand{\fixme}[1]{\textcolor{blue}{[FIXME: #1]}}
\newcommand{\revision}[1]{\textcolor{blue}{[FIXME comment : #1]}}
\newcommand{\regItem}[1]{\item \textcolor{cyan}{#1}}
\newcommand{\regRoutineItem}[1]{\item \textcolor{green}{\textit{Reg. Routine:} #1}}
\newcommand{\situationItem}[1]{\item \textcolor{magenta}{\textit{Situation:} #1}}
\newcommand{\comments}[1]{}

%TODO: think of document as adding around ...
%TODO: consider changing the below so you only use the text in the
%document 
%TODO: change Week plans 

%TODO: consider putting making each frames have its own document. 
%TODO: decide where social intelligence goes. 
%TODO: decide how to separate work from life in social intelligence
%notes.

\newcommand{\footleft}{Mgmt-WorkLife-weekPlan} % figure what this should be 
\newcommand{\footright}{October 2017}

\immediate\write18{/usr/bin/curl -f wttr.in  > currLocWeather.png}
%-- Main Document -------------------------------------------------
\title{Month-Week-Day Plan}\author{Vivek Kale$^1$}\institute{$^1$ University of Illinois at Urbana-Champaign}\date{\today}
%TODO: get rid of extra numbers below
\begin{document}


\begin{frame}[label=projectsPlan]
\begin{multicols}{2}
\markdownInput{/Users/vivekkale/Desktop/ViveksLaptop2/life/workLifeMgr/content_projects_scp.md}
\markdownInput{/Users/vivekkale/Desktop/ViveksLaptop2/life/workLifeMgr/content_projects_jbs.md}
\end{multicols}

\end{frame}

\begin{frame}[label=monthPlanDoc]
\begin{columns}
\begin{column}{1.0\columnwidth}
\ifdefined\POSTER
  \begin{columns}
    %-- Column 1 --------------------------------------------------- 
   \begin{column}{0.22\columnwidth}
      \begin{block}{Happiness Generation - must maintain every week.}
\else
 \underline{\bf Happiness Generation - must maintain every week.} 
\fi
        % explain happiness: show gratitude 
        \begin{itemize}
          \tiny \item \tiny happy that I have Atul to show me.  
        %\item \tiny Happy that I have normal life up to this point.  
        %\item \tiny Think about dad saying, I understand, and want
        %you to be happy.
          \item \tiny T 
        \end{itemize}
\ifdefined\POSTER
      \end{block}
\fi

\ifdefined\POSTER
    \begin{block}{Habits}
\else
 \underline{\bf \it Habits} 
\fi 
      \begin{itemize} 
        \tiny \item \tiny back straight + eye contact. 
      \item \tiny lips relaxed. 
      \item \tiny watch what happens when a hot girl talks to you. 
      \end{itemize}
\ifdefined\POSTER
    \end{block} 
  \end{column} %1
\fi

\ifdefined\POSTER
  \begin{column}{0.5\columnwidth}
    \begin{block}{10-year plan}   %-- Block 2-1 
\else 
 \underline{\bf 10-year plan}
\fi
      \begin{itemize}
      \item \small Make someone and dad proud.
      \end{itemize}
\ifdefined\POSTER
    \end{block}
\fi

\ifdefined\POSTER 
    \begin{block}{3-year plan}
\else
\underline{\bf 3-year plan} 
\fi
      \begin{itemize}
      \item \small find someone
      \item \small make 150K
      \item \small network
      \end{itemize}
\ifdefined\POSTER
    \end{block} 
\fi

\ifdefined\POSTER
    \begin{block}{6-month plan}
\else
\underline{\bf 6-month plan}
\fi
        \begin{itemize}
          \small \item \small results + thesis
          \item \small routine + worklife
          \item \small             
          \end{itemize}
\ifdefined\POSTER
    \end{block} 
\fi


\ifdefined\POSTER  
    \begin{block}{Month-by-month}
\else
\underline{\bf Month-by-month} 
\fi 
      %howto: This is most important, and should be re-visited every
      %month and definitely every three months(have fixed time within
      %4 times in the year to do this). 
      \begin{itemize} 
      %\item \small \textit{May:} Work: finish paper  + Work: comm:
      %  finish resume and cover letter.  
      %\item \small \textit{June:} Work: code org and plan + Work:
      %  finish code + worklife: org + worklife: finish emotional int +
      %  Comm: experiences + Comm: TM. 
      \item \small \textit{July:} Work: library, Work:comm: job
        introductions, Comm: meet people.
      \item \small \textit{August:} Work: library documentation and
      experimentation, Work: comm: resume prep + Work:comm:linkedIn +
      Work:comm: interviewing skills + Work:rel: get headstart o jobs (fb and dshaw) + Comm: stay in touch with girls. 
      \item \small \textit{September:} Work:library + Work: paper writing + Work:comm: interviewing. 
      \item \small \textit{October:} Work:paper +  Work:rel:interviewing + Comm: dating 
      \item \small \textit{November:} worklife: review , Work:paper, Comm: dating 
      \end{itemize}
\ifdefined\POSTER 
    \end{block}
\fi

\ifdefined\POSTER
\begin{block}{May goals}
\else
\underline{\bf May goals}\\
\fi 

\begin{enumerate} 
\item \small Work: finish writing goals
  \small \item \small Mgmt:Spaces: clean room 
\item \small worklife: practice implemented work to form habits 
\item \small Comm: meet people 
\end{enumerate}

\ifdefined\POSTER
\end{block}
\fi

\ifdefined\POSTER
\begin{block}{\small \bf Week of September $31^{st}$: job search}
\else
\underline{\bf Week of September $31^{st}$: job search}\\
\fi

\begin{enumerate}
\tiny \item \tiny Work: finish connecting with people regarding implementation. 
\item \tiny Mgmt:Spaces: take care of driver's license. 
\item \tiny 
\item \tiny Comm: meet people 
\end{enumerate}

\ifdefined\POSTER
\end{block}
\fi

\ifdefined\POSTER
\begin{block}{Weekend Plans - March $15^{th}$, 2017 to May $1^{st}$,2017} 
\else
{\underline{\bf Weekend Plans - March $15^{th}$, 2017 to May $1^{st}$,2017}}\\ 
\fi
\begin{itemize}
\item \tiny Weekend of March $17^{th}$: Visit New York City. Details:
\item \tiny Weekend of March $24^{th}$: Visit San Francisco. Details:
\item \tiny Weekend of March $31^{st}$: Visit San Francisco. Details:
\item \tiny Weekend of April $6^{th}$:  Dad is coming, cherry
  blossoms.
\item \tiny Weekend of April $13^{th}$: Visit India.
\item \tiny Weekend of April $20^{th}$: Go to New York.
\item \tiny Weekend of April $26^{th}$: Go to Los Angeles.
\end{itemize}

{\underline{\bf  Weekend plans for weekends between May $1^{st}$, 2017
    and June$1^{st}$, 2017}}\\
\begin{itemize}
\item \tiny Weekend of May $5^{th}$: Visit New York.
\item \tiny Weekend of May $12^{th}$:
\item \tiny Weekend of May $19^{th}$:
\item \tiny Weekend of May $27^{th}$:
\end{itemize}

\ifdefined\POSTER
\end{block}
\end{column}
\fi

%-- Column 3 --------------------------------------------- 

\ifdefined\POSTER
\begin{column}{0.28\linewidth}
\begin{block}{\small \bf Regular Routines}
\else
\underline{\bf Regular Routines}\\ 
\fi 

\underline{\bf Morning}
\begin{itemize}
\tiny \item \tiny Brush Teeth \timeEst{4 mins}.
\item \tiny Mouthwash, Clean tongue - it helps with breath\timeEst{1mins}
\item \tiny Eating small breakfast.
\item \tiny Check your schedule for meetings / appointments for that
day: Is anything urgent? \timeEst{5 mins}.
\item \tiny Excercise: 1 hour ( Stretch for 10 mins, Treadmill for 30
mins , Lift weights or Yoga 1 hour, Meditation for 10 mins).
\item \tiny Shave (at least every other day) \timeEst{5 mins}.
\item \tiny Shower : (shampoo+conditioner, acne wash,  soap on
underarms/underbody) \timeEst{10 mins}
\item \tiny dry hair, dry face, arms, legs, back, stomach \timeEst{3
mins}
\item \tiny Clothing: belt, check collar, shirt not half-tucked,
matching socks \timeEst{5 mins}
\item \tiny Comb hair, hair gel optional \timeEst{ 1 min}
\item \tiny Check Keys/Wallet/Cell/Badge (get a bucket) \timeEst{1
min}
\item \tiny Open meds cabinet, get water, take medicines \timeEst{1
min}
\item \tiny Wear lenses and rinse out case \timeEst{ 2 mins}
\item \tiny Deodorant or cologne depending on the day \timeEst{1 min}
\item \tiny Wash your face with a face wash - it helps with
acne \timeEst{3 mins}
\item \tiny Moisturize - apply lotion \timeEst{ 2 min}
\item \tiny Neti pot / clean out sinuses - if you’re feeling
congested \timeEst{5 min}
\item \tiny clean out ears, nose \timeEst{5 mins}
\item \tiny Remember to lock door \timeEst{1 mins}
\end{itemize}

\underline{\bf Evening}
\begin{itemize}
\tiny \item \tiny Call family: 30 mins
\item \tiny meet friends: 20 mins
\item \tiny Cleaning dishes / cleaning house: 10 mins
\item \tiny Lay out clothes for the next day, iron if necessary
\item \tiny start laundry if needed: 10 mins,
\item \tiny Collect anything you’ll need for the next day (e.g. dry
cleaning to drop off): 5 mins,
\item \tiny Check Facebook/linkedin/google+: 10 mins, matrimony stuff:10 mins,
\item \tiny Food for lunch/dinner the next day: 10 mins
\end{itemize}
\underline{\bf Night before bed}
%TODO: why isn't enumerate working???
\begin{itemize}
\tiny \item \tiny Finish computer stuff \timeEst{5 mins}
\item \tiny Get into sleeping clothes \timeEst{1 mins}
\item \tiny Before sleeping: brush teeth, floss \timeEst{2 min} 
\item \tiny Properly wash and put away contacts \timeEst{2 mins}
\item \tiny Review the days good things \timeEst{10 min} 
\item \tiny Light Reading \timeEst{ 10 min}
\item \tiny Prayer \timeEst{10 mins}
\end{itemize}

%\underline{\bf Morning}
\begin{itemize}
\tiny \item \tiny Brush Teeth \timeEst{4 mins}.
\item \tiny Mouthwash, Clean tongue - it helps with breath\timeEst{1mins}
\item \tiny Eating small breakfast.
\item \tiny Check your schedule for meetings / appointments for that
day: Is anything urgent? \timeEst{5 mins}.
\item \tiny Excercise: 1 hour ( Stretch for 10 mins, Treadmill for 30
mins , Lift weights or Yoga 1 hour, Meditation for 10 mins).
\item \tiny Shave (at least every other day) \timeEst{5 mins}.
\item \tiny Shower : (shampoo+conditioner, acne wash,  soap on
underarms/underbody) \timeEst{10 mins}
\item \tiny dry hair, dry face, arms, legs, back, stomach \timeEst{3
mins}
\item \tiny Clothing: belt, check collar, shirt not half-tucked,
matching socks \timeEst{5 mins}
\item \tiny Comb hair, hair gel optional \timeEst{ 1 min}
\item \tiny Check Keys/Wallet/Cell/Badge (get a bucket) \timeEst{1
min}
\item \tiny Open meds cabinet, get water, take medicines \timeEst{1
min}
\item \tiny Wear lenses and rinse out case \timeEst{ 2 mins}
\item \tiny Deodorant or cologne depending on the day \timeEst{1 min}
\item \tiny Wash your face with a face wash - it helps with
acne \timeEst{3 mins}
\item \tiny Moisturize - apply lotion \timeEst{ 2 min}
\item \tiny Neti pot / clean out sinuses - if you’re feeling
congested \timeEst{5 min}
\item \tiny clean out ears, nose \timeEst{5 mins}
\item \tiny Remember to lock door \timeEst{1 mins}
\end{itemize}

\underline{\bf Evening}
\begin{itemize}
\tiny \item \tiny Call family: 30 mins
\item \tiny meet friends: 20 mins
\item \tiny Cleaning dishes / cleaning house: 10 mins
\item \tiny Lay out clothes for the next day, iron if necessary
\item \tiny start laundry if needed: 10 mins,
\item \tiny Collect anything you’ll need for the next day (e.g. dry
cleaning to drop off): 5 mins,
\item \tiny Check Facebook/linkedin/google+: 10 mins, matrimony stuff:10 mins,
\item \tiny Food for lunch/dinner the next day: 10 mins
\end{itemize}
\underline{\bf Night before bed}
%TODO: why isn't enumerate working???
\begin{itemize}
\tiny \item \tiny Finish computer stuff \timeEst{5 mins}
\item \tiny Get into sleeping clothes \timeEst{1 mins}
\item \tiny Before sleeping: brush teeth, floss \timeEst{2 min} 
\item \tiny Properly wash and put away contacts \timeEst{2 mins}
\item \tiny Review the days good things \timeEst{10 min} 
\item \tiny Light Reading \timeEst{ 10 min}
\item \tiny Prayer \timeEst{10 mins}
\end{itemize}


\ifdefined\POSTER
\end{block}
\end{column}%3
\fi

\ifdefined\POSTER
\end{columns}
\fi

\end{column}
\end{columns}
\end{frame}

\begin{frame}[label=wkplanDoc]
\begin{columns}  %-- Page 2: Column 1 ---------------------------------------------------

\begin{column}{0.10\linewidth}

\begin{block}{Projects}
%Project superspace: mental health

% Project superspace: work 

% Project: Fundamental knowledge. 

% Project space: applications 

\underline{Computational Reqs} 
\underline{\textbf{Deadline:} January $29^{th}$, 2017} 
\begin{enumerate}
\pitem 
\pitem 
\end{enumerate} 

\underline{Tomography Code} 
\underline{\textbf{Deadline:} February $4^{th}$, 2017} 
\begin{enumerate}
\pitem 
\pitem 
\end{enumerate} 

\underline{Networking Infrastructure} 
\underline{\textbf{Deadline:} September $22^{nd}$, 2017} 
\begin{enumerate}
\pitem 
\pitem 
\end{enumerate} 

\underline{Publication} 
\underline{\textbf{Deadline:} September $22^{nd}$, 2017} 
\begin{enumerate}
\pitem 
\pitem 
\end{enumerate}

% Project space: programming and systems

% -- Project: OpenMP 
\underline{OMP proposal} 
\underline{\textbf{Deadline:} September $22^{nd}$, 2017} 
\begin{enumerate}
\pitem 
\pitem 
\end{enumerate} 

% -- Project: CkLoopHyb paper 
\underline{CkLoopHyb paper}
\underline{\textbf{Deadline:} September $22^{nd}$, 2017}
\begin{enumerate}
\pitem Add template to git repository.
\pitem Add writeup for dot product results on Cori.
\pitem Get PIC results with Charm++-everywhere for BW.
\pitem Need a new application. 

\ptask Update the paper technique to be consistent with experiments.  
\ptask Work: update paper .tex file with dot prod. \te{30 minutes}.
\ptask Add PAPI cache misses for problem statement.
\ptask Fix diagrams.
\ptask Figure out what to put at the top block of first quadrant.
\ptask Add content in extended paper.
\ptask Add code for PIC in bottom left quadrant.
\ptask Add projections results.
\ptask Add to related work.
\ptask Get PIC results with Charm++-everywhere, Charm++ + CkLoop for Cori.
\ptask Get results for Charm++ + CkLoop for missing data points.
\ptask Show results for dot product using Charm++ + OpenMP, and for
dot product with Charm++ + CkLoop.
%\doneTask{ Show code with Charm++ + OpenMP.}
\pitem Merge with git repository.
\pitem Send message to meet for results.
\pitem Work: message to Kathryn about poster.
\end{enumerate}

\underline{Schedule} \\ 
\begin{enumerate}
\pitem 
\end{enumerate} 
\end{block} 

%Project: interviewing 
\underline{Interviewing} 
\underline{\textbf{Deadline:}} 
\begin{enumerate}
\pitem 
\pitem 
\end{enumerate} 


%Project: extra-curricular

\begin{block}{Year Plan}
\begin{itemize}
\tiny \item \tiny January: Work: interview, application. , Mgmt:Spaces: get dl , Mgmt:Spaces: get cc, Comm: 
\item \tiny February: Work: proposal, Work: implementation, Mgmt:Spaces: 
\item \tiny March: Work: paper for IWOMP, Work: job talk prep.
\item \tiny April: Work: paper for IWOMP, Work: implementation  
\item \tiny May: Work: do paper, Work: Comm: 
\item \tiny June: Work: work on implementation with UDS 
\item \tiny July: Work: work on implementation with UDS Comm:  
\item \tiny August: Work: paper for IPDPS, 
\item \tiny September: Work: paper for IPDPS 
\item \tiny October: Work: paper for IPDPS  
\item \tiny November: Work: SC stuff, 
\item \tiny December: 
\end{itemize} 
\end{block}

\begin{block}{Month Plan}
\underline{Week of January $31^{st}$, 2018}
\begin{itemize}
\tiny \item \tiny Work: 
\item \tiny Mgmt:Spaces: 
\item \tiny Comm: 
\end{itemize}

\underline{Week of February $4^{th}$, 2018}
\begin{itemize}
\tiny \item \tiny Work: send in cover letter + Work: finish application.  
\item \tiny Mgmt:Spaces: 
\item \tiny Comm: 
\end{itemize}

\underline{Week of February $9^{th}$, 2018}
\begin{itemize}
\tiny \item \tiny Work: 
\item \tiny
\end{itemize}

\underline{Week of February $16^{th}$, 2018}
\begin{itemize}
\tiny \item \tiny 
\item \tiny
\end{itemize}

\underline{Week of February $16^{th}$, 2018}
\begin{itemize}
\tiny \item \tiny 
\item \tiny
\end{itemize}

\end{block}

\end{column}
  \begin{column}{0.10\linewidth}
    %{\textbf{\underline{Week Aspects(new name)}})
      %--- Page 2: Block 1-1 
    \begin{block}{Routines}
      { \tiny \underline{\bf Routines:} Morning: bfast, exercise/brush,
        floss, shower (hair, eyes, ears, underarms, feet), meds, belt,
        comb hair / dishes, clean kitchen floor. |  Night: running/ put
        away dishes, clean kitchen floor / brush teeth, charge phone,
        clothes for tomorrow, contacts off.}\\
      {\tiny \underline{\bf Sunday Routines:} Routines:
        Experiences,week plan.}\\ 
      {\tiny \underline{\bf Weekday Routines:}}\\
      %TODO: check if weekday routines overlaps into daily routines.
    \end{block} 
      %--- Page 2: Block 1-3
    \begin{block}{Local Happiness}
      % howto:Recognize emotions / recognize (precisely) what you're
      % worried about -> methods for eliminating worry. 
      % clean out old worries, put them in pastWorries doc.  
      % remember to define precisely 
      % order from top-to bottom
      \begin{itemize} 
        \tiny \item \tiny -
      \item \tiny Worrying about gal: think about talking to others.
      \end{itemize} 
    \end{block}
    %--- Page 2: Block 1-4 
    \begin{block}{AngersAndAnxieties}
      % howto:Recognize emotions / recognize (precisely) what you're worried about -> methods for eliminating worry. 
      % clean out old worries, put them in pastWorries doc.  
      % remember to define precisely 
      % order from top-to bottom
      \begin{itemize}
      \item \tiny Problem with Rathi: $\rightarrow$ do what she
        says. it's ok. 

        \item \tiny Issue with job: $\rightarrow$ we'll find
          something.  

%        \tiny \item \tiny Anger about UP $\rightarrow$ think about the positive
%        perspective, and the original
%      \item \tiny Anger about Mr. Kapoor $\rightarrow$  leave it
%      \item \tiny Anger about a b  $\rightarrow$ ...
      \end{itemize}
    \end{block}
      %-- Block 1-5
      \begin{block}{Life notes}
        \begin{itemize}
          \tiny \item \tiny Social Intelligence: 
        \end{itemize}
      \end{block}
      % howto:
      \begin{block}
        {\tiny {\bf Week lessons:}}
        \begin{enumerate}
        \item \tiny tm tips. 
        \item \tiny Coding problem:
        \item \tiny Chess tips: 
        \end{enumerate}
            {{\tiny {\tiny \bf  News:}} {\tiny  M:  | S: 
                | E: Fallon:  Kimmel:  Colbert: SNL:}}
            {{\tiny {\tiny \bf  Weather:}} {\tiny M: 1100 T: 1100 W: 41/34 .1r 
                 R: 51/46 .60r F: 62/36 .7r a Sa: 42/20 .10r Su: 31/17 }}
           % %\documentclass{article}
\immediate\write18{ansiweather -f 7 -s false -a false -l Champaign  > currWeather.tex}
%\begin{document}
%

%\begin{frame}{Weather}{Week of \today}
%\input{|''ansiweather -l Champaign -F -s false -a false''}
%\end{frame}

\begin{frame}{Weather}{Week of \today}

\input{./currWeather.tex}

%\begin{figure}
%\includegraphics[scale=0.4]{currLocWeather.png}
%\end{figure}

\end{frame}


%\end{document}
 
      \end{block}
  \end{column}
  %-- Column 2 ---------------------------------------------
  \begin{column}{0.6\linewidth}
    \begin{block}{Week Summary} 
      {\underline {\bf Week Plan:} 
        Work: send e-mail to Sam Williams, 
        Work: send updates to Harshitha, Work: message to , 
        Work: apply to other places, 
        Mgmt:Spaces: update MetroMile, Comm: e-mail to Rathi? 
      }\\ 
      {\underline{\bf Weekend Plan:} Celebrate Mother's Day. } \\
  
    \end{block}
    \begin{block}{Running ToDo} % find a better name 
      \item \tiny Work: apply to other places, 
        
      \doneTask{Mgmt:Spaces: get credit card. }
      \doneTask{Work: message about OpenMP stuff.}

      \begin{enumerate}

        
      \item \tiny \textbf{Curr} $\rightarrow$ 
        
      \item \tiny Work: coding practice. 
      \item \tiny Work: job talk.

      \item \tiny Mgmt:Spaces: get dentist appt.
 
      \item \tiny Comm: e-mail to Rathi?  := 
        
        Hi Rathi, 
        
        I'm in Illinois at home right now receiving severance payments from my
        previous employer. I'm working for my father's company at Charmworks
        this summer while in the Bay Area through the summer. I'll be
        at BNL most likely in September.
        
        Vivek

      %\item \tiny Work: do slides for UDS. 
      %\item \tiny Work:rel: send out cover letter. 
      
      \item \tiny Comm: upload photos.
      \item \tiny \textbf{Curr} $\rightarrow$ 
      \item \tiny Mgmt:Spaces: get glasses.   
      \item \tiny Comm: talk with Toni. 
      \item \tiny Work:rel: email to Renata. \te{10 minutes} 
      \item \tiny Comm: message to Sateja.
        
        Hey, I followed you on LinkedIn. I think your work is really good. 
        Let me know if you want to grab coffee sometime in the city. 
        
      \item \tiny \textbf{ <-- End Curr }
      \item \tiny Mgmt:wp: add photos from college.
      \item \tiny Mgmt:Spaces: post apartment on craigslist. \te{1 hour}.
      \item \tiny Comm: message to Rishi and Mike about meeting. 
      \item \tiny worklife: read social intelligence. \te{2 hour} 
      \item \tiny Work(psm): do document with slides. \te{10 minutes}.
      \item \tiny Mgmt:Spaces: buy floor mats by BMW. 
      \item \tiny Mgmt:Spaces: get face scrub
      \item \tiny Mgmt:Spaces: ask Mom to get clothes.
      \item \tiny Mgmt:Spaces: get clothes. 
      \item \tiny worklife: practice posing in a picture.  
      \item \tiny Comm: message to girl. \te{10 minutes} \pr{Urgent}. 
      \item \tiny Mgmt:wp: add tags of photos from this past weekend.  
      \item \tiny Mgmt:Spaces: call about Admiral's Club Card.
      \item \tiny Work: IM system for LLNL. 
      \item \tiny Work: get results for n-body. 
      \item \tiny Comm: message to Rupali Gr. 
        Hi Rupali, I hope that you're doing well. -Vivek

      \end{enumerate}
    \end{block} 


\begin{block}{DayPlans} 

  Sunday:   
  | Monday: 
  | Tuesday: 

  - Work: questions for letter. 
  - Work: prep for Todd. 

1. Talk about ISI situation.
2. Tell about other places interviewing. 
3. Talk about past work with LLNL, current work. 
4. Talk about meeting times. 
5. Talk about trip timings available to talk more if interested. 


  - Work: reply to NERSC. 

  - Work: reply to Sam Williams with information. 
  - Work: reply for Apple interview.

  - Mgmt:Spaces: book trip back. 
  - Mgmt:Spaces: 


  - Comm: message to 
  - Comm: message to Madhura.  

  | Wednesday: 
  | Thursday: 
| Friday: 
| Saturday:  
\end{block}

\begin{block}{Alg for Mgmt}
Alg for Mgmt:Alg for Comm   L0:  L1:  L2:  L3:  Maps L4: L5: -> L0
TODO:  and prioritize . Use colors for levels Howto:  
\end{block} 

\begin{block}{Meeting Information}
BlueJeans meeting ID: 8562450706.
Webex MeetingID: 805639779  
\end{block} 

\begin{block}{Situations} 
\begin{enumerate}
\item Meet Steve: 
\end{enumerate} 
\end{block}

\begin{block}{Rel}
\underline{Rel-chat-short}: Bhakti J., Rath S, Vinita G. , Aishwarya ,
Madhura, Anushya, Aditi 

\underline{Rel-chat-like-toinitiate}: Sateja P. , Anuja D., Shilpa G., Sneha from NJ. 

\underline{Rel-chat}:  cat1:  Toni,, S Van., Bhakti, Shalaka/Rucha, Sateja P, Anuja
D. ,  Divya, cat2: sonal mukhi, Mamta, sapana/india, Avani, cat 3:
Shilpa Ghate, Renuka, Renata, 
\end{block}
  
  % howto: the below is a routine based on the job I'm doing and the
  % circumstances I'm in. Note that this is different from routines
  % above, which is true for a long-term period of over 10 years. 
  \begin{block}{Weekly Meetings}
    \begin{itemize}
      \tiny \item \tiny IWOMP affinity meeting Thursday 7 AM PDT / 9 AM
      CDT call in: 805-639-779 pswd: places. \url{https://llnl.webex.com/llnl/j.php?MTID=m33950b76befb72c0cbc31ea2e3be720c}
    \item \tiny DEG group meeting.          \end{itemize}
  \end{block} 

      \begin{block}{Week Daily Schedule}
        % howto: arrange items from running todo into the below
        % sheet management document. 
        %NOTE: we don't use the table format because of fittting
        \begin{columns}
          \begin{column}{0.14\textwidth}{\small \underline{\bf Mon}}
            {\tiny \bf {\tiny weather:} } {\tiny 11/1 s} \\ 
            {\tiny \bf {\tiny todo}}\\ 
            \begin{itemize}
              \tiny \item \tiny Work: do experiments with multiple nodes. 
            \item \tiny Work: get experiments using real data. 
          \item \tiny Mgmt:Spaces: get the Driver's License.  
      \item \tiny Mgmt:Spaces: cancel order from LinkedIn premium +
        Mgmt:Spaces:cancel order of Youtube Red.  
            \end{itemize}
                {\small  \bf schedule}\\
                \begin{enumerate} 
                  \tiny \item \tiny 8-9AM: Regular Routines 
                \end{enumerate}
          \end{column}

          \begin{column}{0.14\textwidth}{\small \underline{\bf Tues}}
            {\bf {\tiny  weather:} } {\tiny 3/1 r} \\ 
            {\bf {\tiny todo}}\\ 
            \begin{itemize}
              \tiny \item \tiny Work: do write-up 
            \end{itemize} 
                {{\bf {\tiny  schedule}}}
                \begin{enumerate} 
                  \tiny \item \tiny 8-9AM: Regular Routines 
                \end{enumerate} 
          \end{column}
          \begin{column}{0.14\textwidth}{\small \underline{\bf Wed}}
            {\tiny \bf weather: } {\tiny 3/1 r} \\ 
            {\tiny {\bf todo}}\\
            \begin{itemize}
              \tiny \item \tiny Work: send email to Prof. Gropp. 
            \item \tiny 
            \item \tiny Comm: message to Anita. 
            \end{itemize}
                {\tiny \bf schedule}\\
                \begin{enumerate} 
                  \tiny \item \tiny 8-9AM: Regular Routines 
                \end{enumerate} 
          \end{column}

          \begin{column}{0.14\textwidth}{\small \underline{\bf Thurs}}
            {\tiny \bf weather: } {\tiny 1/-1 r }\\ 
            {\tiny \bf todo} \\ 
            \begin{itemize}
              \tiny \item \tiny Add in schedule 
            \item \tiny Work:admin: get slide updated.  
              \item \tiny Work: admin: send cc statement. 

            \end{itemize} 
                {\tiny {\bf schedule}} \\
                \begin{enumerate} 
                  \tiny \item \tiny 8-9AM: Regular Routines 
                \end{enumerate}
          \end{column} 
          
          \begin{column}{0.14\textwidth}{\small \underline{\bf Fri}}
            {\tiny \bf weather: } {\tiny 3/1 r} \\ 
            {\tiny \bf todo} \\ 
            \begin{itemize} 
            \item \tiny Work:rel: cover letter for LLNL.
            \item \tiny Work: e-mail to JP and Steve about acheivements.
            \item \tiny Work: send e-mail. 
            \item \tiny Mgmt:Spaces: give back Internet box. 
            \item \tiny Mgmt:Spaces: get the Driver's License.  
      \item \tiny Mgmt:Spaces: cancel order from LinkedIn premium +
        Mgmt:Spaces:cancel order of Youtube Red. 
            \item \tiny Comm: reply on Coffee Meets Bagel.
            \item \tiny Comm: reply to Aishwarya
            \item \tiny Comm:rel: talk with Vinita. 
                
            \end{itemize} 
                {\tiny \bf schedule} \\
                \begin{enumerate} 
                  \tiny \item \tiny 8-9AM: Regular Routines 
                \end{enumerate}
          \end{column}

          \begin{column}{0.15\textwidth}{\small \underline{\bf Sat}}
            {\tiny \bf weather: } {\tiny 11/1 .2r} \\ 
            { \tiny \bf todo} \\ 
            \begin{itemize}
              \tiny \item \tiny Add in schedule
            \item \tiny 
            \end{itemize} 
                {\tiny \bf schedule} \\
                \begin{enumerate} 
                  \tiny \item \tiny 8-9AM: Regular Routines 
                \end{enumerate}
          \end{column}
         
          \begin{column}{0.15\textwidth}{\small \underline{\bf Sun}}
            {\tiny {\bf weather:} } {\tiny 11/2 r} \\ 
            {\tiny {\bf todo}}\\
            \begin{itemize}
              \tiny \item \tiny Add in schedule
            \end{itemize} 
                {\tiny \bf schedule}\\
                \begin{enumerate} 
                  \tiny \item \tiny 8-9AM: Regular Routines 
                \end{enumerate}
          \end{column}
        \end{columns}
      \end{block}
\end{column}
    
    % ------ Page 2: Column 3
    \begin{column}{0.20\linewidth}
      % ------ Block 2.3.1 
      \begin{block}{Food Plan} 
        \begin{itemize}
          \tiny \item \tiny Google Express: DHDQQJQDRFTXDM95B
        \end{itemize}
      \end{block} 
      \begin{block}{Clothes plan} 
        \begin{itemize}
          \tiny \item \tiny Monday: 
        \item \tiny Tuesday
        \item \tiny Wednesday
        \item \tiny Thursday
        \item \tiny Friday
        \item \tiny Saturday
        \end{itemize} 
      \end{block} 
      
      %-- Block 3-3 
      \begin{block}{Situations}
        % Summary: the below are the situations that define one
        % another.
        % Howto: Put post notes of each situation from list of previous
        % week. Put high importance situations in situationslist. 
        % Put list of upcoming situations on Sunday morning with Pre. 
        \begin{itemize}
          \tiny \item \tiny 
        \item \tiny 
        \end{itemize}
      \end{block}
      % contingincies: 
      % Summary:
    \end{column}
\end{columns}

\end{frame}

\begin{frame}[label=socialInt]
\begin{block}{Jobs}
  \begin{itemize}
    \tiny \item \tiny cat1: D.E. Shaw | SUNY/BNL: applying . | IBM:
    applied,interviews.  | Sandia: applied and failed, Cray: applied and got offer, USC. 
  \item \tiny cat2: Intel: Applied and failed. Intel: re-application  completed, AMD: applied, Samsung, VMWare: Contacting Yavatkar Kaka. 
  \item \tiny cat3: Google. MongoDB: applied and failed, Facebook. 
  \item \tiny cat4: LLNL. 
  \end{itemize} 
\end{block} 

%TODO: check if below is needed. 
      %\item Friends: Amit, Sangita, Anchit, Kevin. 
      %Networking: 
  \begin{block}{Comm:friends} 
    \begin{itemize} 
    \item Palekar’s:  
      \begin{itemize}
      \item UP: 
      \item Rohun: 
      \end{itemize} 
    \item Bhagwat’s: 
    \item Kapoor’s: 
      \begin{itemize}
        \tiny \item \tiny Anchit: Interaction. Atul is smart. 
      \item \tiny Shuchi: 
      \end{itemize}
    \item Sudharshan’s: 
      \begin{itemize} 
        \tiny \item \tiny Amit:  
      \item \tiny Sangita:  
      \end{itemize}
    \item Kumar’s:
      \begin{itemize}
        \tiny \item \tiny Ashwin: 
      \item \tiny Shilpa: 
      \end{itemize}
    \item Saied’s: 
      \begin{itemize}
        \tiny \item \tiny Laila 
      \item \tiny Faisal 
      \item \tiny Ayesha
      \end{itemize} 
    \item Kukreti’s:
    \item Kumar’s:
    \item UIUC friends: 
      \begin{itemize} 
        \tiny \item \tiny Sadhna: 
      \item \tiny Rishi: 
      \item \tiny Amaeya: 
      \item \tiny Chris: 
      \item \tiny Brian: 
      \item \tiny Arif 
      \end{itemize}
    \item grad school friends 
      \begin{itemize} 
        \tiny \item \tiny  Karan: 
      \item \tiny Gopi: 
      \item \tiny Amanda:  
      \end{itemize}
    \end{itemize} 
  \end{block}
  
%  \begin{block}{Rel} 
%    \begin{itemize} 
%      \small \item \small Shwetha K/r
%    \item \small Shalaka 
%    \item \tiny DCK girl 
%    \item \tiny  Shikha Singhvi: initiated.  
%    \item \small Mamta
%    \item \small Avani 
%    \item \small SH05702678 Palo Alto - 
%    \item \small Sheila G -Sh93522198  
%    \item \small sonalisinha  s: 5103658167 and my email address is: \url{2sonalisinha@gmail.com}
%    \item \small Nandini  - 
%    \item \small Archana from San Jose. 
%    \item \small Im from DC  - B475435: parents initiated. 
%    \item \small Sruthi 
%    \item \small SH12825009 New York, New York - Finance Professional 
%    \item \small Navneet K  - 29 5'8" - SH88263211 
%    \item \small Girl from New York  -27 years
%    \item \small Shwetha Ravi 
%    \item \small Ruchira K ( SH34021888 )2-Way           
%    \item \small SH11680694 from San Francisco. 
%    \item \small Sandhya N. 22 Jun 2016 
%    \item \small Aarti S - 27 - 5' 9'' Charlotte.  
%    \item \small Anu M from Chicago, IL 
%    \item \small Pranati A from Cleveland, OH - 
%    \item \small Meghana K from New York - 
%    \item \small Ssd S - 28 , 5 5 , Boston, MA 
%    \item \small Anj B. - 30 yrs - SH13882555.
%    \end{itemize}
%  \end{block}

\end{frame} 

%\begin{frame}{Worries} 
\end{frame}

\begin{frame}{Week Plan}
\end{frame}

\begin{frame}{News}
\end{frame}

\begin{frame}[label=weather]{Weather}
\end{frame}

\begin{frame}[label=runningTodo]{Running Todo} 
\end{frame}


\begin{frame}{Mgmt:Spaces: send out dentist stuff} 
\end{frame}


\begin{frame} 
\frametitle{Work(psm): Network } 
\begin{itemize} 
\small \item \small Work(psm): Network account. 
\item \small Work(psm): Networking document. 
\item \small Work(psm): get account on USC's systems. 
\end{itemize} 
\end{frame}

\begin{frame}{ Work(ckh): related work}
\begin{itemize} 
\item \tiny Work(ckp): obtain related work. 
\end{itemize}
\end{frame} 

\begin{frame}{Work: send e-mail to Micheal Klemm} 
\end{frame}

\begin{frame}{Work: send re-imbursement for Sb} 
\end{frame}

\begin{frame}{Mgmt:Spaces: apply for visa} 
\end{frame}

\begin{frame}{Mgmt:Spaces: Get driver's license for VA} 
\end{frame}

\begin{frame}{Mgmt:Spaces: prescription for glasses}
\end{frame}


\end{document}
