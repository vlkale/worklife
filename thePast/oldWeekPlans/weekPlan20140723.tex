\documentclass[serif, mathserif, final]{beamer}
\mode<presentation>{\usetheme{Lankton}} 
%\mode<presentation>{\usetheme{Copenhagen}} 
%\mode<presentation>{\usetheme{CambridgeUS}} 
\usepackage[orientation=landscape,size=a0,scale=1.6, debug]{beamerposter}
%\usepackage[orientation=landscape,size=a0,scale=1.4,debug]{beamerposter}
%\usepackage[absolute,overlay]{textpos} % <-- can  use this to position text (solves problem of earlier yesterday with poster placement). 

%\documentclass[serif,mathserif,final]{beamer}
%\mode<poster>{\usetheme{RicePoster}}
\usepackage{amsmath,amsfonts,amssymb,pxfonts,xspace}

\usepackage{graphicx, ragged2e}
\usepackage{pgffor}
\usepackage{ulem}
\usepackage{caption}
\usepackage{courier}

\usepackage{verbatim}

%\usepackage[usenames,dvipsnames]{xcolor}    
\usepackage{listings}

\usepackage{float}
\usepackage{subfig}
\setbeamertemplate{caption}[numbered] 

\usepackage{calendar}
\usepackage{geometry}

%\graphicspath{{./figures/}}                                                   
%\usepackage{beamerposter}
%\graphicspath{{./figures/}} 
\newcommand{\bllt}{\item \small}
\newcommand{\doneTaskNoItem}[1]{\sout{#1}}
\newcommand{\doneTask}[1]{\tiny \item \tiny \sout{#1}}
\newcommand{\doneTaskHyp}[1]{\tiny \item \tiny \textcolor{blue} {\sout{#1}}}
\newcommand{\optTask}[1]{\tiny \item \tiny \textcolor{green}{#1}}
\newcommand{\prioTask}[1]{\tiny \item \tiny \textcolor{red}{#1}}
\newcommand{\timeEst}[1]{\textit{Time:} \textit{#1}}
\newcommand{\te}[1]{\textit{TimeEst:} \textit{#1}}
\newcommand{\priority}[1]{\textit{Priority:} \textit{#1}}
\newcommand{\pr}[1]{\textit{Priority:} \textit{#1}}
\newcommand{\prio}[1]{\textit{Priority:} \textit{#1}}
\newcommand{\dueBy}[1]{\textit{Deadline:} \textit{#1}}
\newcommand{\MyName}{Vivek~Kale}
\newcommand{\fixme}[1]{\textcolor{blue}{[FIXME: #1]}}
\newcommand{\revision}[1]{\textcolor{blue}{[FIXME comment : #1]}}
\newcommand{\regItem}[1]{\item \textcolor{cyan}{#1}}
\newcommand{\regRoutineItem}[1]{\item \textcolor{green}{\textit{Reg. Routine:} #1}}
\newcommand{\situationItem}[1]{\item \textcolor{magenta}{\textit{Situation:} #1}}
\newcommand{\deadline}[1]{#1}
\newcommand{\dl}[1]{#1}
\newcommand{\comments}[1]{}

%\graphicspath{{./figures/}}
%\usepackage[orientation=landscape,size=a0,scale=1.6,debug]{beamerposter}

%-- Header and footer information ----------------------------------
\newcommand{\footleft}{Mgmt-WorkLife-weekPlan} % figure what this should be 
\newcommand{\footright}{August 2015}

%------------------------------------------------------------------
%-- Main Document ------------------------------------------------- 
\title{Month-Week-Day Plan} 
\author{Vivek Kale$^1$}
\institute{$^1$ University of Illinois at Urbana-Champaign}  
\date{\today}

% TODO: make the below document neater 
% TODO: consider stretching center column 
% TODO: check how to fit week daily plan together. 
\begin{document}
    
% TODO: prioritize 
% \begin{column}{0.33\linewidth} {\textbf{\underline{Month Plan for
% August 2014}}}      %-- Block 1-1 

\begin{frame}
  \begin{columns}
    %-- Column 1 --------------------------------------------------- 
    \begin{column}{0.25\linewidth} %{\textbf{\underline{Week Aspects(new name)}}}
      \begin{block}{Habits}
        \begin{itemize}
          \tiny \item \tiny 
        \item \tiny 
        \item \tiny 
        \item \tiny 
        \end{itemize}
      \end{block}
      %-- Block 1-1 
      \begin{block}{Happiness} 
        \begin{enumerate}
          \tiny \item \tiny Think about happy thoughts about
          Urbana-Champaign. 
        \item \tiny 
        \end{enumerate}
      \end{block} 

      \begin{block}{Stresses}
        % howto:Recognize emotions / recognize (precisely) what you're worried about -> methods for eliminating worry. 
        % clean out old worries, put them in pastWorries doc.  
        % remember to define precisely 
        % order from top-to bottom
        \begin{itemize}
        \item \tiny Work: worrying about interviewing: just
          focus on the work. 
        \item \tiny Worrying about gal: think about talking to
          others 
        \item \tiny 
          % \item \tiny Comm: Worrying about not losing it (check this/ define this better) : work on sheet.  
          % there's much more to be done.  She may not be interested
          % because it's a brother. 
        \end{itemize}
      \end{block}

      \begin{block}{Real-time Confidence}
        \begin{itemize}
          \tiny \item \tiny If you do something sloppy, don't
          think about it and don't say sorry. 
        \item \tiny don't think someone is trying to hurt you if
          they say something.
        \item \tiny know the paths of least stress. 
        \item \tiny if mom does something, don't let that bite you. 
        \end{itemize}
      \end{block} 
      %-- Block 1-3
      \begin{block}{Life notes} 
        \begin{itemize} 
        \item \tiny TM: look at intelligence book. 
        \item \tiny Problem: implement game of mastermind. 
        \end{itemize}     
      \end{block}
    \end{column} %2 

    %-- Column 2 --------------------------------------------- 
    \begin{column}{0.50\linewidth}
      %--- Block 2-1
      \begin{block}{Week Summary}
        {\tiny \textbf{Week Plan:} Worklife: finish impl. diagram,
          Work: do startup work with GPUs, Work:comm: message to
          Gropp, Work: prep for meeting.} 
        {\tiny \textbf{Weekend Plan:} Meet for Lollapalooza.}  
      \end{block}


      \begin{block}{Running TODO} % find a better name 
        \begin{enumerate}

\comments{
        \item \small Work(emagpuL0) = Work(emagpu): check implementation
          changes to library + Work(emagpu): check correctness that
          GPU will run simultaneously with CPU cores + \doneTask{Work: l1/ l2:
            see if you need to add on iteration adjustment} 
          + TODO: Work(emagpuL0): ``spawns'' vs spawns  vs. 

        \item \small Work(emagpuL2) = Work(L2:  ): 
          \doneTaskNoItem{TODO: L2: to executed by the GPU  , to be
            executed by the CPU}  +  \doneTaskNoItem{TODO: L2: split
            loop iterations into statically -scheduled- and
            dynamically -scheduled-} v
          Work: TODO: L2: check if we need to say something about the
          other way of doing things.√  
          Work: TODO: L2: transform the original OpenMP loop vs. transform an
          OpenMP loop vs. transform a loop vs. transform a given √         
          +TODO: L2: alongside each other √ vs. parallel 
          +TODO: L2: and (as before)  vs. rest of the OpenMP threads
          (as before) vs [omit] 
          + Work: For hybrid static/dynamic scheduling of loop
          iterations on the GPU vs. Finally, vs. One note is that 
          +Work: for hybrid static/dynamic scheduling of these loop
          iterations  

          + TODO: using hybrid static/dynamic scheduling of these
          iterations vs. [omit] vs. parens

          +TODO: implementation adaptation 

          +TODO: based on this technique vs. technique
          
          Goal: i'm awesome and have work
          
          Point: Additional questions to consider for thesis chapter
          
          Outcome: good, and some points to consider
                 
          Work: TODO: L2: their fraction of loop iterations 
        \item \small Work(emagpuL2) = Work(L4: ): 
          Work(emagpuL4): into one fraction  
          + Work(emagpuL2/L4): may not be beneficial vs. may lack
          benefit √
          + Work(emagupL4): there are no dynamic scheduling clauses
          available √ vs. no dynamic scheduling clauses are avaialable
          + Work(emagpuL2/4) TODO: adapted √  vs. enhanced 

          + Work() : we could make a first assumption that the GPU static
          fraction would be 1.0  vs. 
          it may be good to make a first assumption that the static
          fraction is 1.0 √

          + Work: GPU might lack benefit √ vs may lack benefit 
          + Work: implementation adaptation 
}
% curr --> 

\doneTask{Work(msum): finish write-up for prepping for meeting =
  Work(msum): summarize meeting from Simon + Work(msum):summarize
  discussion about GPGPU with Prof. Gropp + Work(msum): make plan for
  upcoming} 

\item \small Work(gpuMem): find GPU memory used  = \doneTaskNoItem{ Work(gpumem): find
  variables allocated in NS\_Grid.fpp} + Work(gpuMem): find variables
  used in NS\_Grid.fpp = (talk to Tarun about this) + Work(gpuMem): go through the NS\_Grid.fpp and ModDataStruct.fpp
  to see what is being allocated + Work(gpuMem): come up with formulas in
  latex + Work(gpumem): check everything done with ModDeriv.fpp +
  Work(gpuMem): write for Prof. Gropp and Jon  + Work(gpuMem): check
  for other functions being added (talk to John Larson about this). 

\item \small worklife: finish emot. int. book  on Cecil =
  worklife: structure + worklife:integration into sheet and
  connection with other things + worklife:applications/examples \te{1 hour mins} 
  
\item \small worklife: finish slides for management =
  worklife: toastmasters + worklife: finish picture 
  
\item \small Work: do coding practice planning \te{10 mins}. \te{2 hours} 
  
\item \small Comm: message to Shilpa Talwalkar \dl{sat night}
  \te{1 hour} \prio{U}.
\item \small Comm: message to Anita
  
  % future (check this- and if it should be added)   -->   
        \end{enumerate}
      \end{block} 
      
      \begin{block}{Week Daily Schedule} 
        % howto: arrange items from running todo into the below
        % sheet management document. 
        \begin{columns} 
          \column{0.14\textwidth}{\textbf{\small \underline{Mon}}}
          \textbf{\small todo} \\ 
          \begin{itemize}
            \tiny \item \tiny 
          \item \tiny 
          \end{itemize} 
          \textbf{\small schedule} \\
          \begin{enumerate} 
            \tiny \item \tiny 8-9AM: Regular Routines 
          \end{enumerate} 

          \column{0.14\textwidth}{\textbf{\small \underline{Tue}}}
          \textbf{\small todo} \\
          \begin{itemize}
            \tiny \item \tiny             
          \end{itemize}  
          \textbf{\small schedule}\\ 
          \begin{enumerate} 
            \tiny \item \tiny
          \end{enumerate} 

          \column{0.14\textwidth}{\textbf{\small {\underline{Wed}}}}
          \textbf{\small todo} \\
          \begin{itemize}
            \tiny \item \tiny 
          \end{itemize}
          \textbf{\small schedule} \\
          \begin{enumerate}
            \tiny \item \tiny 8-9AM: Regular Routines 
             \item \tiny 9AM - 10AM: 
            \item \tiny 5PM - 6PM = 
            \item \tiny 6PM - 6:30PM = 
            \item \tiny 6:30PM - 7:30PM = 
          \end{enumerate} 
          \column{0.14\textwidth}{\textbf{\small{\underline{Thu}}}}
          \textbf{\small todo}\\
          \begin{itemize} 
         \tiny \item \tiny Work: finish prep for meeting  \te{30 mins}
         \prio{U} \dl{hourBeforeMeeting}
       \item \tiny Mgmt:Spaces: add license plate lower nails to
         rear license plate of BMW. \te{20 mins} 
       \item \tiny Comm: revise emotional int. chapter for
            applicability \te{1 hour} 
          \end{itemize}

          \textbf{\small schedule}\\
          \begin{enumerate} 
            \tiny \item \tiny 8-9AM = Regular Routines 
            \item \tiny 12-1PM = gather notes about job org. during
              session 
            \item \tiny 3:30PM - 4:30PM = meeting 
            \item \tiny 6PM - 9PM = revise emot. int. chapter and
              structure wl-cheat based on it 
          \end{enumerate} 
          
          \column{0.14\textwidth}{\textbf{\small{\underline{Fri}}}}
          \textbf{\small todo}\\ 
          \begin{itemize}
            \tiny \item \tiny Work: updates on GPu work \dl{}
            \te{} \prio{}
            \tiny \item \tiny Mgmt:Spaces: clean up the ViveksLaptop
            directory Work to have the work stuff integrated. 
            \tiny \item \tiny Comm: gather people from matrimony site \te{1 hour}. 
          \end{itemize}
          \textbf{\small schedule}\\
          \begin{enumerate}
            \tiny \item \tiny 8-9AM: Regular Routines 
          \end{enumerate}
          
          \column{0.14\textwidth}{\textbf{\small \underline{Sat}}}
          \textbf{\small todo}\\
          \begin{itemize} 
          \tiny \item \tiny
          \end{itemize}
          \textbf{\small schedule}\\ 
          \begin{enumerate}
            \tiny \item \tiny 7-7:30PM = food + organize.  
            \tiny \item \tiny 7:30PM -8PM = 
            \tiny \item \tiny 8PM -9PM = 
            \tiny \item \tiny 9:45PM - 10:30PM = go out 
            \tiny \item \tiny 2AM -3AM =  message Samir + add people
          \end{enumerate}
          
          \column{0.14\textwidth}{\textbf{\small \underline{Sun}}}
          \textbf{\small todo}\\
          \begin{itemize} 
          \item \tiny             
          \end{itemize}
          \textbf{\small schedule}\\
          \begin{enumerate} 
            \tiny \item \tiny 
          \end{enumerate} 
        \end{columns}
      \end{block}
    \end{column}%2
    
    %-- Column 3 --------------------------------------------------- 
    \begin{column}{0.20\linewidth}
      %-- Block 3-1 
      % On Sunday morn, put a quick synopsis of top news items                  
      % here.  Categorize as Comm and Money.      
      \begin{block}{News}
        \begin{itemize}   
          \tiny \item \tiny Healthcare plan 
        \item \tiny Wimbledon 
        \end{itemize}
      \end{block}
      \begin{block}{Weather} 
        \begin{itemize}
          \tiny \item \tiny 
        \item \tiny 
        \end{itemize}
      \end{block} 
      \begin{block}{Clothes plan} 
        \begin{itemize} 
          \tiny \item \tiny Monday
        \item \tiny Tuesday
        \item \tiny Wednesday
        \item \tiny Thursday
        \item \tiny Friday
        \end{itemize} 
      \end{block}
      %-- Block 3-2 

      %-- Block 3-3
      \begin{block}{Situations}
        % Howto: the below are the situations that define one another

        \begin{itemize}
        \tiny \item \tiny \textit{meet for dinner with Prof. Gropp:} Pre-:
          Post-: less talkative at age point 
        \item \tiny \textit{Neena's wedding:} Pre-:  Post-:  don't get angry
        \item \tiny \textit{Dinner at Rainbow Garden:} Pre-:  Post-:
          note the people talking about god - the conversation topic
          is placed.           
        \item \tiny \textit{Meet with Simon:}
        \item \tiny \textit{XPACC group meeting:} Pre-: discuss code
          updates for GPU with Prof Gropp (talk about how much data
          can fit in GPU) + discuss Simon comm and translating it to
          XPACC code + discuss comm and translating it 
        \item \tiny \textit{Meet with postdocs:}
        \item \tiny \textit{Meet with Jon Freund:} Pre- 
        \item \tiny \textit{Presentation}:             
        \item \tiny \textit{Movie}:

        \item \tiny \textit{BMM Travel planning:} Post: planning
          hotel , use experience from before to plan current. 
        \item \tiny \textit{Travel planning:}           
        \item \tiny \textit{Comm: tell Mona that you hate} 
        \item \tiny \textit{Comm: tell Leena and Mona} 
        \item \tiny \textit{Mgmt:Spaces:}
    \end{itemize} 

  \end{block}

\begin{block}{Rel}
\begin{itemize} 
\small \item \small Girl from SF 
\small \item \small Ruchi 
\small \item \small Vinita Ghate 
\small \item \small Mamta
\small \item \small Neha Jambhekar 
\small \item \small Meghan Joshi 
\end{itemize} 
\end{block} 
\end{column}%3 

\end{columns}

\end{frame}
\end{document}
