\begin{columns}  %-- Page 2: Column 1 ---------------------------------------------------

\begin{column}{0.10\linewidth}

\begin{block}{Projects}
%Project superspace: mental health

% Project superspace: work 

% Project: Fundamental knowledge. 

% Project space: applications 

\underline{Computational Reqs} 
\underline{\textbf{Deadline:} January $29^{th}$, 2017} 
\begin{enumerate}
\pitem 
\pitem 
\end{enumerate}

\underline{Tomography Code} 
\underline{\textbf{Deadline:} February $4^{th}$, 2017} 
\begin{enumerate}
\pitem 
\pitem 
\end{enumerate}

\underline{Networking Infrastructure} 
\underline{\textbf{Deadline:} September $22^{nd}$, 2017} 
\begin{enumerate}
\pitem 
\pitem 
\end{enumerate}

\underline{Publication} 
\underline{\textbf{Deadline:} September $22^{nd}$, 2017} 
\begin{enumerate}
\pitem 
\pitem 
\end{enumerate}

% Project space: programming and systems

% -- Project: OpenMP 
\underline{OMP proposal} 
\underline{\textbf{Deadline:} September $22^{nd}$, 2017} 
\begin{enumerate}
\pitem 
\pitem 
\end{enumerate} 

% -- Project: CkLoopHyb paper 
\underline{CkLoopHyb paper}
\underline{\textbf{Deadline:} September $22^{nd}$, 2017}
\begin{enumerate}
\pitem Add template to git repository.
\pitem Add writeup for dot product results on Cori.
\pitem Get PIC results with Charm++-everywhere for BW.
\pitem Need a new application.
\ptask Update the paper technique to be consistent with experiments.

\ptask Work: update paper .tex file with dot prod. \te{30
          minutes}.
\ptask Add PAPI cache misses for problem statement.
\ptask Fix diagrams.
\ptask Figure out what to put at the top block of first quadrant.
\ptask Add content in extended paper.
\ptask Add code for PIC in bottom left quadrant.
\ptask Add projections results.
\ptask Add to related work.
\ptask Get PIC results with Charm++-everywhere, Charm++ + CkLoop for Cori.
\ptask Get results for Charm++ + CkLoop for missing data points.
\ptask Show results for dot product using Charm++ + OpenMP, and for
dot product with Charm++ + CkLoop.
\doneTask{ Show code with Charm++ + OpenMP.}
\pitem Merge with git repository.
\pitem Send message to meet for results.
\pitem Work: message to Kathryn about poster.
\end{enumerate}

\underline{Schedule} \\ 
\begin{enumerate}
\pitem 
\end{enumerate} 
\end{block} 

%Project: interviewing 
\underline{Interviewing} 
\underline{\textbf{Deadline:}    }  
\begin{enumerate}
\pitem 
\pitem 
\end{enumerate} 

%Project: extra-curricular

\begin{block}{Year Plan}
\begin{itemize}
\tiny \item \tiny January: Work: interview, application. , Mgmt:Spaces: get dl , Mgmt:Spaces: get cc, Comm: 
\item \tiny February: Work: proposal, Work: implementation, Mgmt:Spaces: 
\item \tiny March: Work: paper for IWOMP, Work: job talk prep.
\item \tiny April: Work: paper for IWOMP, Work: implementation  
\item \tiny May: Work: do paper, Work: Comm: 
\item \tiny June: Work: work on implementation with UDS 
\item \tiny July: Work: work on implementation with UDS Comm:  
\item \tiny August: Work: paper for IPDPS, 
\item \tiny September: Work: paper for IPDPS 
\item \tiny October: Work: paper for IPDPS  
\item \tiny November: Work: SC stuff, 
\item \tiny December: 
\end{itemize} 
\end{block}


\begin{block}{Month Plan}
\underline{Week of January $31^{st}$, 2018}
\begin{itemize}
\tiny \item \tiny Work: 
\item \tiny Mgmt:Spaces: 
\item \tiny Comm: 
\end{itemize}

\underline{Week of February $4^{th}$, 2018}
\begin{itemize}
\tiny \item \tiny Work: send in cover letter + Work: finish application.  
\item \tiny Mgmt:Spaces: 
\item \tiny Comm: 
\end{itemize}


\underline{Week of February $9^{th}$, 2018}
\begin{itemize}
\tiny \item \tiny Work: 
\item \tiny
\end{itemize}

\underline{Week of February $16^{th}$, 2018}
\begin{itemize}
\tiny \item \tiny 
\item \tiny
\end{itemize}
\end{block}

\end{column}
  \begin{column}{0.10\linewidth}
    %{\textbf{\underline{Week Aspects(new name)}})
      %--- Page 2: Block 1-1 
    \begin{block}{Routines}
      { \tiny \underline{\bf Routines:} Morning: bfast, exercise/brush,
        floss, shower (hair, eyes, ears, underarms, feet), meds, belt,
        comb hair / dishes, clean kitchen floor. |  Night: running/ put
        away dishes, clean kitchen floor / brush teeth, charge phone,
        clothes for tomorrow, contacts off.}\\
      {\tiny \underline{\bf Sunday Routines:} Routines:
        Experiences,week plan.}\\ 
      {\tiny \underline{\bf Weekday Routines:}}\\
      %TODO: check if weekday routines overlaps into daily routines.
    \end{block} 
      %--- Page 2: Block 1-3
    \begin{block}{Local Happiness}
      % howto:Recognize emotions / recognize (precisely) what you're
      % worried about -> methods for eliminating worry. 
      % clean out old worries, put them in pastWorries doc.  
      % remember to define precisely 
      % order from top-to bottom
      \begin{itemize} 
        \tiny \item \tiny -
      \item \tiny Worrying about gal: think about talking to others.
      \end{itemize} 
    \end{block}
    %--- Page 2: Block 1-4 
    \begin{block}{AngersAndAnxieties}
      % howto:Recognize emotions / recognize (precisely) what you're worried about -> methods for eliminating worry. 
      % clean out old worries, put them in pastWorries doc.  
      % remember to define precisely 
      % order from top-to bottom
      \begin{itemize}
        \item \tiny Issue with USC : $\rightarrow$ things get better. 
        \tiny \item \tiny Anger about UP $\rightarrow$ think about the positive
        perspective, and the original
      \item \tiny Anger about Mr. Kapoor $\rightarrow$  leave it
      \item \tiny Anger about a b  $\rightarrow$ ...
      \end{itemize}
    \end{block}
      %-- Block 1-5
      \begin{block}{Life notes}
        \begin{itemize}
          \tiny \item \tiny Social Intelligence: 
        \end{itemize}
      \end{block}
      % howto:
      \begin{block}
        {\tiny {\bf Week lessons:}}
        \begin{enumerate}
        \item \tiny tm tips.
        \item \tiny Coding problem: 
        \item \tiny Chess tips: 
        \end{enumerate}
            {{\tiny {\tiny \bf  News:}} {\tiny  M:  | S: 
                | E: Fallon:  Kimmel:  Colbert: SNL:}}
            {{\tiny {\tiny \bf  Weather:}} {\tiny M: 1100 T: 1100 W: 41/34 .1r 
                 R: 51/46 .60r F: 62/36 .7r a Sa: 42/20 .10r Su: 31/17 }} 

           % %\documentclass{article}
\immediate\write18{ansiweather -f 7 -s false -a false -l Champaign  > currWeather.tex}
%\begin{document}
%

%\begin{frame}{Weather}{Week of \today}
%\input{|''ansiweather -l Champaign -F -s false -a false''}
%\end{frame}

\begin{frame}{Weather}{Week of \today}

\input{./currWeather.tex}

%\begin{figure}
%\includegraphics[scale=0.4]{currLocWeather.png}
%\end{figure}

\end{frame}


%\end{document}
 
      \end{block}
  \end{column}
  %-- Column 2 ---------------------------------------------
  \begin{column}{0.6\linewidth}
    \begin{block}{Week Summary}
      {\underline {\bf Week Plan:} 

        Work: send follow-up e-mails from OpenMP F2F + Mgmt:wp: add
        people on LinkedIn,  Work: do HLF
        application, Work: do monthly report, Work: ,  Mgmt:Spaces: get dl, Work:rel: send out the LLNL application + Work:rel:
        message to Nathan, Work:rel: send out cover letter, Comm: upload pictures,
 }\\

{\underline{\bf Weekend Plan:} Meet Dhanashree.} \\  

    \end{block}
    \begin{block}{Running ToDo} % find a better name 
      \begin{enumerate}



        OpenMP summary. \te{30 minutes} \pr{UI} \dl{Monday}

Hi Bill,

I’d like to let you know about a face-to-face meeting last week for
the OpenMP committee and a presentation of a substantiated proposal
for OpenMP and get your feedback on it. A substantiated version of the work on user-defined scheduling that
you and I co-authored at OpenMPCon was presented by a senior member of
the OpenMP committee, namely, Michael Klemm. My hope for the outcome
was that a first version would be added to the OpenMP 5.0
specification 

Here is the outcome of the presentation: 

The presentation has resulted in the proposed feature being created as
a ticket. 

The ticket unfortunately won't make it into the next release of OpenMP
5.0. This is understandably so, given that integratability with the
rest of OpenMP and its coverage of a variety of use cases needs to be
considered before it is added to OpenMP and fitting it in with the
several other tickets that need to be added by about September 2018 to
prepare OpenMP for release at SC 2018. 

A good result of the presentation at the face-to-face meeting is that
many people understood the idea, especially the definitions of the
user-defined loop schedulers that implement the loop schedules. Also, 
we've started to talk about actual ideas for reference implementation,
which will help to address problems of integration and also understand
how different designs for the interface and implementation perform on
hardware. I'm working with two people on the committee (Michael Klemm
along with Xinmin Tian, both from Intel) to develop a reference
implementation in OpenMP and the interface for it. Another point is
that we received good feedback from people on the Fortran subcommittee
after the presentation, supporting the feature for Fortran is something I had hand-waved in
thinking about while developing. The user-defined schedule relies on
passing to user-defined scheduler functions the pointers to objects
representing loop records for loop history. Passing pointers to
objects requires more thinking. 

While there, I also found several topics of interest, specifically,
interoperability with other runtimes systems, libraries. 

Being aware of the developing features should be useful for sucessful
integration of UDS in OpenMP. 

The affinity clause to be added to OpenMP 5.0 

My integration of the 

I met people from AMD and Intel. I didn't meet many people from IBM
for some reason. 

What you think about the results of the presentation? Do you think
that going to the 

 and if you have feedback
about them. 

Bronis: please let me know if I've said anything inaccurate based on
your understanding of the outcome of Michael's presentation last
Monday afternoon.

Vivek

\tiny \item \tiny 
Hi, 

Yeah, the commercials were nice, especially the Alexa one and the Tide one
in my opinion.

Austin was good. People critical to
the application that I've submitted to Lawrence Livermore
Nat'l Lab were there and available to talk with me. 
During a group dinner last Thursday, I sat next to my former boss from LLNL and sat across from
important people from other national labs. I got an idea of where I stand in the application
process through conversations that night and the following day. 
It'll be a month or so before I go for an
on-site interview as I plan to give a job talk for the interview and
LLNL has to arrange for a seminar open to people at the Lab, 
and LLNL has to coordinate with other people that are applying for the
position. 


to know more information about where I'm at in the application
process and 

I'm also sending e-mails to people that I met at the forum besides my
former boss to tell points about my work that strengthen my
application. 

Vivek


Austin was good. People critical to the application that I've
submitted to Lawrence Livermore Nat'l Lab were at the forum there and
were available to talk with me. I sat next to my former boss from LLNL
and sat across from important people from other national labs during a
group dinner last Thursday. I got an idea of where I stand in the
application process through conversations that night and the next
morning. It'll be a month or so before I can go for an on-site
interview as I plan to give a job talk for the interview and LLNL has
to arrange for a seminar open to people at the Lab, and LLNL has to
coordinate with other people that are applying for the position. I'm
also sending e-mails to people from the Lab that I saw at the forum
besides my former boss and that are involved in the application
process to get information of where I’m at in the application process.

and to highlight work I’ve done that strengthens my application.



\item \tiny Work: finish business cards. 

\item \tiny Work: re-imbursement for Omnigraffle. 

      \item \tiny Comm: message to Aishwarya . 

       
        Whoa. Did you party after the release?


        TODO:L0: 
        TODO:L3: 
        TODO:L4: after the release VS afterwards 

        TODO:L4: afterward VS afterwards 
        TODO:L5: 
        
        Whoa. Did you party after the release? 

        Whoa. Did you party afterward? 

      \item \tiny Mgmt:wp: message to Swati. 

      \item \tiny Work:rel: send e-mail to Prof. Gropp about reference letter. 
         
        Dear Professor Gropp, 

        I hope that you are doing well. 

        You may remember that I applied once before as a PhD student
        for the Heidelberg Laureate Forum (HLF) in 2015. You’d written a letter
        for an application to be invited to the program for me then. I
        attended an event for alumni of a HLF this past
        September. A person on the HLF's committee at the event
        encouraged me to apply again this year as a
        staff/postdoc for the HLF. I'd like to
        submit an application.

        I'd like to have you as one of my references for the
        application. The deadline for the letter is February 14,
        2018. I think that a reference from you will be
        helpful. My most recent resume is at \url{https://sites.google.com/site/vivek112/home/viveksfiles}. 
        Maybe I can give you text explaining the work that I've done since I
        started at my current position at University of Southern California/Information Sciences
        Institute. The website for the HLF is
        http://www.heidelberg-laureate-forum.org/. Can you write a letter of reference for me for the
        application to the program this year?

        I’ve sent a request for the letter through the portal to apply for the
        program. Let me know if you can write the letter for me when
        you can.
        
        Vivek

TODO:L2: ordering of first paragraph. 

TODO:L3: explanation of previous work. 

TODO:L2: Explaining that a reference from Prof. Gropp will make the
application strong VS [OMIT] -done  

TODO:L0: Talking about relevance of Prof. Gropp's recommendation
letter. 

TODO:L0: Explain deadline. 

TODO:L4: Univ. of Southern California/ ISI 

TODO:L4: Where to say I've sent a request for the letter through the
portal to apply. 

\item \tiny Work: 

      \item \tiny \textbf{Curr} $\rightarrow$ 
      \item \tiny worklife: make projects list.  


      \item \tiny Comm: send e-mail to Rathi about jobs in Boston. 

        \item \tiny Comm: send Prof. Cappello an e-mail. 
      \item \tiny Comm: send e-mail to Prof. Gropp about ref. letter. 

      \item \tiny Mgmt:Spaces: get ready for license test on Monday.

      \item \tiny Work: send e-mail to Bronis about the work. 

      \item \tiny Work:rel:cvl:upd: do cover letter for LLNL = Work: rel:
        explain reason for wanting to join LLNL + Work:rel:(cvl):
        explain your contributions.  \te{1 hour}  \dl{ }         
      \item \tiny Work:rel: send out LLNL application \te{1 hour}
        \dl{Wednesday}. 

      \item \tiny Work: send slides to Nathan. \te{30
        minutes}. \pr{UI} 

      \item \tiny Work: send e-mail about research. 

       
      \item \tiny Work:rel: do application for HLFF = Work:rel: get degree
        certificate + Work:rel: ask Steve Crago for a reference letter + Work:rel: add research statement + Work:rel: write
        letter of motivation. \te{2 hours} \dl{Friday} 
 
      \item \tiny Mgmt:wp: add people to LinkedIn =  Mgmt:wp: add Andy
        Schmidt on LinkedIn. +  Mgmt:wp: add Mike Kumbera to LinkedIn
        +  \doneTaskNoItem{Mgmt:wp: add Olga to LinkedIn} + \doneTaskNoItem{Mgmt:wp: add Ignacio to
          LLinkedIN.} + Mgmt:wp: add Marty. 
          

      \item \tiny Comm:rel: text to Aishwarya = Comm:rel: Would you like to meet this Saturday for brunch? 

      \item \tiny Comm: e-mail to Rathi about whether there are jobs in Boston. 

      \item \tiny Work: send information on publication to Steve and JP.  \te{20 minutes}.

      \item \tiny Work: send Prof. Cappello draft letter. 
      
      \item \tiny Work: send Prof. Gropp an e-mail. 
       
        
      \item \tiny Work: admin: finish re-imbursements for Denver = Work:admin: get ticket for flight to DC.
        
      \item \tiny Mgmt:Spaces: get glasses. 
      \item \tiny Mgmt:Spaces: get driver's license from VA. 
        
      \item \tiny Comm: upload pictures. 
      \item \tiny Comm: upload photos. 
        
      \item \tiny Work:admin: submit receipts for Bagel. 
        
        \tiny \item[] \tiny \textbf{Curr} $\rightarrow$
        
      \item \tiny Work: finish proposal slides. 
      \item \tiny Work:rel: do cover letter for BNL + Mgmt:Spaces: talk with Martin Kong.
        
      \item \tiny Comm: photo upload. 
                
        \item \tiny Work: send in application to PNNL.
          
        \item \tiny Mgmt:wp: add profile.
        \item \tiny Mgmt:wp: add about me. 
          
        \item \tiny Mgmt:wp: add profile 
        \item \tiny Work:rel: email to Renata. \te{10 minutes} 
        \item \tiny Comm: message to Sateja. 
          Hey, I followed you on LinkedIn. I think your work is really good. 
          Let me know if you want to grab coffee sometime in the city. 
        \item \tiny Comm:rel: talk to Shweta.
        \item \tiny Comm: talk with Toni. 
          
\item \tiny \textbf{ <-- End Curr }
      \item \tiny Mgmt:wp: add photos from college. 
      \item \tiny Comm: message to Rishi and Mike about meeting. 
      \item \tiny worklife: read social intelligence. \te{2 hour} 
      \item \tiny Mgmt:Spaces: get face scrub. 
      \item \tiny worklife: practice posing in a picture.  
      \item \tiny Work: IM system for LLNL.  
      \end{enumerate}
    \end{block} 

\begin{block}{DayPlans} 

  Sunday:   
  | Monday:  
  | Tuesday:  
  | Wednesday: 

  | Thursday: Work:admin: send Claire message,
  Work: send message to Steve about poster,  
  Work: send e-mail to Nathan, 
  Work: send application to LLNL, 
  Work:rel: send LinkedIn - Olga, Ignacio, Andy Schmidt, 
  Comm: send Rathi a message, 

| Friday: 
| Saturday:  
\end{block}

\begin{block}{Alg for Mgmt}
Alg for Mgmt:Alg for Comm   L0:  L1:  L2:  L3:  Maps L4: L5: -> L0
TODO:  and prioritize . Use colors for levels Howto:  
\end{block} 

\begin{block}{Meeting Information}
BlueJeans meeting ID: 8562450706.
Webex MeetingID: 805639779  
\end{block} 

\begin{block}{Situations} 
\begin{enumerate}
\item Meet Rathi: 
\item Meet Dhanashree:
\end{block}



\begin{block}{Rel}
\underline{Rel-chat-short}: Bhakti J., Rath S, Vinita G. , Aishwarya , Madhura

\underline{Rel-chat-like-toinitiate}: Sateja P. , Anuja D., Shilpa G., Sneha from NJ. 

\underline{Rel-chat}:  cat1  Toni,, S Van., Bhakti, Shalaka/Rucha, Sateja P, Anuja
D. ,  Divya, cat2: sonal mukhi, Mamta, sapana/india, Avani, cat 3:
Shilpa Ghate, Renuka, Renata, 
\end{block}

  
  % howto: the below is a routine based on the job I'm doing and the
  % circumstances I'm in. Note that this is different from routines
  % above, which is true for a long-term period of over 10 years. 
  \begin{block}{Weekly Meetings}
    \begin{itemize}
      \tiny \item \tiny IWOMP affinity meeting Thursday 7 AM PDT / 9 AM
      CDT call in: 805-639-779 pswd: places. \url{https://llnl.webex.com/llnl/j.php?MTID=m33950b76befb72c0cbc31ea2e3be720c}
    \item \tiny DEG group meeting.          \end{itemize}
  \end{block} 

      \begin{block}{Week Daily Schedule}
        % howto: arrange items from running todo into the below
        % sheet management document. 
        %NOTE: we don't use the table format because of fittting
        \begin{columns}
          \begin{column}{0.14\textwidth}{\small \underline{\bf Mon}}
            {\tiny \bf {\tiny weather:} } {\tiny 11/1 s} \\ 
            {\tiny \bf {\tiny todo}}\\ 
            \begin{itemize}
              \tiny \item \tiny Work: do experiments with multiple nodes. 
            \item \tiny Work: get experiments using real data.  
            \item \tiny 
            \end{itemize}
                {\small  \bf schedule}\\
                \begin{enumerate} 
                  \tiny \item \tiny 8-9AM: Regular Routines 
                \end{enumerate}
          \end{column}

          \begin{column}{0.14\textwidth}{\small \underline{\bf Tues}}
            {\bf {\tiny  weather:} } {\tiny 3/1 r} \\ 
            {\bf {\tiny todo}}\\ 
            \begin{itemize}
              \tiny \item \tiny Work: do write-up 
            \end{itemize} 
                {{\bf {\tiny  schedule}}}
                \begin{enumerate} 
                  \tiny \item \tiny 8-9AM: Regular Routines 
                \end{enumerate} 
          \end{column}
          \begin{column}{0.14\textwidth}{\small \underline{\bf Wed}}
            {\tiny \bf weather: } {\tiny 3/1 r} \\ 
            {\tiny {\bf todo}}\\
            \begin{itemize}
              \tiny \item \tiny Work: send email to Prof. Gropp. 
            \item \tiny 
            \item \tiny Comm: message to Anita. 
            \end{itemize}
                {\tiny \bf schedule}\\
                \begin{enumerate} 
                  \tiny \item \tiny 8-9AM: Regular Routines 
                \end{enumerate} 
          \end{column}

          \begin{column}{0.14\textwidth}{\small \underline{\bf Thurs}}
            {\tiny \bf weather: } {\tiny 1/-1 r }\\ 
            {\tiny \bf todo} \\ 
            \begin{itemize}
              \tiny \item \tiny Add in schedule 
            \item \tiny Work:admin: get slide updated.  
              \item \tiny Work: admin: send cc statement. 

            \end{itemize} 
                {\tiny {\bf schedule}} \\
                \begin{enumerate} 
                  \tiny \item \tiny 8-9AM: Regular Routines 
                \end{enumerate}
          \end{column} 
          
          \begin{column}{0.14\textwidth}{\small \underline{\bf Fri}}
            {\tiny \bf weather: } {\tiny 3/1 r} \\ 
            {\tiny \bf todo} \\ 
            \begin{itemize} 
            \item \tiny Work:rel: cover letter for LLNL.
            \item \tiny Work: e-mail to JP and Steve about acheivements.
            \item \tiny Work: send e-mail. 
            \item \tiny Mgmt:Spaces: give back Internet box. 
            \item \tiny Mgmt:Spaces: get the Driver's License.  
      \item \tiny Mgmt:Spaces: cancel order from LinkedIn premium +
        Mgmt:Spaces:cancel order of Youtube Red. 
            \item \tiny Comm: reply on Coffee Meets Bagel.
            \item \tiny Comm: reply to Aishwarya
            \item \tiny Comm:rel: talk with Vinita. 
                
            \end{itemize} 
                {\tiny \bf schedule} \\
                \begin{enumerate} 
                  \tiny \item \tiny 8-9AM: Regular Routines 
                \end{enumerate}
          \end{column}

          \begin{column}{0.15\textwidth}{\small \underline{\bf Sat}}
            {\tiny \bf weather: } {\tiny 11/1 .2r} \\ 
            { \tiny \bf todo} \\ 
            \begin{itemize}
              \tiny \item \tiny Add in schedule
            \item \tiny 
            \end{itemize} 
                {\tiny \bf schedule} \\
                \begin{enumerate} 
                  \tiny \item \tiny 8-9AM: Regular Routines 
                \end{enumerate}
          \end{column}
         
          \begin{column}{0.15\textwidth}{\small \underline{\bf Sun}}
            {\tiny {\bf weather:} } {\tiny 11/2 r} \\ 
            {\tiny {\bf todo}}\\
            \begin{itemize}
              \tiny \item \tiny Add in schedule
            \end{itemize} 
                {\tiny \bf schedule}\\
                \begin{enumerate} 
                  \tiny \item \tiny 8-9AM: Regular Routines 
                \end{enumerate}
          \end{column}
        \end{columns}
      \end{block}
\end{column}
    
    % ------ Page 2: Column 3
    \begin{column}{0.20\linewidth}
      % ------ Block 2.3.1 
      \begin{block}{Food Plan} 
        \begin{itemize}
          \tiny \item \tiny Google Express: DHDQQJQDRFTXDM95B
        \end{itemize}
      \end{block} 
      \begin{block}{Clothes plan} 
        \begin{itemize}
          \tiny \item \tiny Monday: 
        \item \tiny Tuesday
        \item \tiny Wednesday
        \item \tiny Thursday
        \item \tiny Friday
        \item \tiny Saturday
        \end{itemize} 
      \end{block} 
      
      %-- Block 3-3 
      \begin{block}{Situations}
        % Summary: the below are the situations that define one
        % another.
        % Howto: Put post notes of each situation from list of previous
        % week. Put high importance situations in situationslist. 
        % Put list of upcoming situations on Sunday morning with Pre. 
        \begin{itemize}
          \tiny \item \tiny 
        \item \tiny 
        \end{itemize}
      \end{block}
      % contingincies: 
      % Summary:
    \end{column}
\end{columns}
