\begin{frame}[Introduction]
\frametitle{Introduction}
\begin{itemize}
\item \small Who has it better in the job market? The one that knows a lot (body: arms wide) about
  a little(body: pinch), or the one that knows a little (body: pinch) about a lot (body: arms wide)?
\item \small There's a name for the type of people with the skill sets
  I just mentioned. The former is a generalist. The latter is a
  specialist. Specialists associated with PhDs, those who are focused.
MBA is a generalists often are associated with MBAs.
\item \small This debate between the merits of a generalist and specialist is
  age-old. And both sides seem to have an argument.
\item \small If you look online, there are several articles which I've come across
  arguing for the importance of the generalists in the modern day.
  Yet, there aren't as many articles that argue for the
  specialists (body: fingers pointed upward).
\item \small So, my goal is to fill the void and provide an argument of how specialists have the potential to be effective managers and leaders to take on managerial positions, even if they don't have actual management training.

%  Have in today's day and age. I personally beleive (body: hands to heart) that if you know a little about a lot, you'll be able
%  to go far in today's fast-paced, facebook-infused,
%  google-dominated world.

\item There are three different ways specialists have the potential to develop themselves as generalists (-- Note: get this point right===)
\end{itemize}
\end{frame}

\begin{frame}
\frametitle{Can manage in a rapidly changing environment}
\begin{itemize}
\item don't know broad skill sets, but specialists learn how to learn these broader skill sets
\item do lab experiments to find a new virus : discovery process requires learning new things as you go
\item come up with new algorithms: requires one to pick up technologies and skills sets in an uncertain environment
\end{itemize}
\end{frame}

\begin{frame}
\frametitle{Know the broader impact}
\begin{itemize}
\item Engineering to find a way to make airplanes travel direct flights from New York to Mumbai(developed here at LLNL): understanding the needs of society, that air travel is increasingly important in a globally connected world
\item A biologist cannot go into a lifelong career without knowing the broader impact of what they are doing, as they are investing a long time to do it.
% who is driven to find a cure for cancer knows that this will save the lives of many, and keeps at that solution.
\end{itemize}
\end{frame}


\begin{frame}
\frametitle{Knowing how to manage large teams}
\begin{enumerate}
\item Cross-lab collaborations are a big part of working together in
  research. It's often up to the scientists to seek out these collaborations.
\item The training of many specialists (notably PhDs) often involves some aspect of teaching undergraduate students. This involves leading students to collectively learn (by making sure to answer all questions), planning and overseeing group projects. These skills can be transferred to leadership skills.
%\item Coding requires reuse, and your code should often conform to others specifications
\end{enumerate}
\end{frame}

\begin{frame}
\frametitle{Conclusions}
\begin{itemize}
%\item Even though various articles dismiss specialists, I'm not going to dismiss
%\item If you look at the class of graduating MBAs and compare that to the class of graduating PhDs, many managers will have to make choices who gets hired.
\item Generalists may be in high-demand due to their abilities to have
  a broad range of skill sets applicable in many domains.
\item  Yet, the viewpoint that various articles suggest is that only they can be managers or be good at building teams, and there seems to suggest that specialists don't know how to manage a large group.
\item  If one considers the skills that a specialist must have to succeed in her job, this viewpoint may be a bit extreme, and not as grounded.
\item If one looks at the skill sets that a specialist have, a
  specialist can learn to adapt, understand broader impact, and work
  with others and synergize. These are skills that are desirable in a
  manager, and closely knit. Perhaps they won't be as refined managers, but they can learn on their own and pick up skills, if they worked hard at it and get practical, real-world experience doing it.
\item I think it's a matter of realizing how these skills a scientist or engineer already has can be transferred to a managerial positions.
\item It's also a matter of practicing and engaging oneself
and interact with others.
\end{itemize}
\end{frame}


\begin{frame}
\frametitle{appendix }

\item \small
\item \small Have you ever wondered: If I just knew a little more breadth in my skill set, could I advance
myself professionally? Or perhaps have you wondered: if I just focused
on learning some specific skill really well, could I advance myself
professionally?

\end{frame}
