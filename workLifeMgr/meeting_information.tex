\documentclass{beamer}
%\usetheme[
%          titlepagelogo=bold50-illinois,% Logo for the first page
%          pageofpages=of,% String used between the current page and the total page count
%          titleline=true,% Show a line below the frame title
%          ]{TorinoTh}

\usetheme{Luebeck}

\useoutertheme{infolines}

\usepackage{beamerthemesplit}
\usepackage{graphics}
\usepackage{graphicx}
\usepackage{hyperref}
%\usepackage{comment}
\usepackage{ulem}
\usepackage{color}
\usepackage{xspace}

\usepackage{amssymb}
\usepackage{caption}
\usepackage{courier,cite}
%\usepackage[usenames,dvipsnames]{xcolor}
%\usepackage{color,cite}
%\usepackage{scalefnt}
\usepackage{listings}





%\newcommand{\comments}[1]{}

%\setbeamertemplate{footline}{\insertframenumber/\inserttotalframenumber}

\usepackage{outlines} 
\usepackage{enumitem}
\setenumerate[1]{label=\Roman*.}
\setenumerate[2]{label=\Alph*.}
\setenumerate[3]{label=\roman*.}
\setenumerate[4]{label=\alph*.}

\usepackage{flushend}
\usepackage{float}
\usepackage{subfig}
\usepackage{tikz}

\lstset{
    language=C++,
    basicstyle=\tiny\ttfamily, % Standardschrift
    numbers=left,               % Ort der Zeilennummern
    numberstyle=\footnotesize\ttfamily,          % Stil der Zeilennummern
    %stepnumber=2,               % Abstand zwischen den Zeilennummern
    numbersep=6pt,              % Abstand der Nummern zum Text
    tabsize=2,                  % Groesse von Tabs
    extendedchars=true,         %
         breaklines=true,            % Zeilen werden Umgebrochen
        frame=single,
         keywordstyle=[1]\textbf,    % Stil der Keywords
         stringstyle=\ttfamily,
         showspaces=false,           % Leerzeichen anzeigen ?
         showtabs=false,             % Tabs anzeigen ?
         backgroundcolor=\color{white!97!black},
         showstringspaces=false      % Leerzeichen in Strings anzeigen ?
}

\newcommand{\htor}[1]{\textcolor{OliveGreen}{[htor: #1]}}
\newcommand{\tg}[1]{\textcolor{orange}{[tg: #1]}}
\newcommand{\vivek}[1]{\textcolor{cyan}{[vivek: #1]}}

\newcommand{\comments}[1]{}
\newcommand{\bllt}{\item \small}
\newcommand{\MyName}{Vivek~Kale}

\newcommand{\doneTaskNoItem}[1]{\sout{#1}}
\newcommand{\doneTask}[1]{\tiny \item \tiny \sout{#1}}
\newcommand{\doneTaskHyp}[1]{\tiny \item \tiny \textcolor{blue}{\sout{#1}}}
\newcommand{\optTask}[1]{\tiny \item \tiny \textcolor{green}{#1}}
\newcommand{\prioTask}[1]{\tiny \item \tiny \textcolor{red}{#1}}

\newcommand{\timeEst}[1]{\textit{TimeEst:}\textit{#1}}
\newcommand{\te}[1]{\textit{TimeEst:}\textit{#1}}
\newcommand{\deadline}[1]{\textit{Deadline:}\textit{#1}}
\newcommand{\dueBy}[1]{\textit{Deadline:}\textit{#1}}
\newcommand{\dl}[1]{\textit{Deadline:}\textit{#1}}
\newcommand{\priority}[1]{\textit{Priority:}\textit{#1}}
\newcommand{\prio}[1]{\textit{Priority:}\textit{#1}}
\newcommand{\pr}[1]{\textit{Priority:}\textit{#1}}

\newcommand{\fixme}[1]{\textcolor{blue}{[FIXME: #1]}}
\newcommand{\todo}[1]{\textcolor{red}{[TODO: #1]}}
\newcommand{\revision}[1]{\textcolor{blue}{[FIXME comment : #1]}}
\newcommand{\regItem}[1]{\item \textcolor{cyan}{#1}}
\newcommand{\regRoutineItem}[1]{\item
  \textcolor{green}{\textit{Reg. Routine:} #1}}
\newcommand{\situationItem}[1]{\item
  \textcolor{magenta}{\textit{Situation:} #1}}

\title{Notes}
\author{\MyName}
\date{\today}
\title{Meeting Information}
\author[\today]{
  Vivek~Kale\inst{1}
}

%\institute[University of Illinois at Urbana-Champaign]{
%  \inst{1}%
%  Department of Computer Science\\
%  University of Illinois at Urbana-Champaign
%}

\institute[UIUC]{
  \inst{1}%
 Department of Computer Science\\
 University of Southern California
}

\date{\today}
\subject{Computational Sciences}

\begin{document}
\AtBeginSection[]
{
\begin{frame}
\frametitle{Outline}
\tableofcontents[part=1, pausesections]
\end{frame}
}

\titlepage
%\section{Prelim Outline Notes}
%\input{./content_defense}
\begin{frame}[label=mtgPSMimg]
\frametitle{Meeting by Vivek}
\underline{\bf Join on computer}
\begin{itemize}
\small \item \small Dial: 199.48.152.152
\item \small Meeting ID: 5061993
\end{itemize}
\underline{\bf Join by phone}
\begin{itemize}
\small \item \small 408.740.7256 or 888.240.2560
\item \small Meeting ID: 5061993
\item \small Press \#
\end{itemize}
\end{frame}

\begin{frame}[label=omp_lang]
\frametitle{Meeting for OMP-lang}
\underline{\bf Join WebEx meeting}
\begin{itemize}
\small \item \small Meet Tuesday at 10AM CT /11AM ET
\small \item \small \url{https://llnl.webex.com/llnl/j.php?MTID=md575d91e6134ce9b0aaf14bc65c011eb}
\small \item \small \url{http://tinyurl.com/ompltc}
%\item \small Meeting number: 805 639 779
\item \small Meeting number: 808 344 165
\item \small Meeting password: openmp
\end{itemize}
\underline{\bf Join by phone}
\begin{itemize}
\small \item \small +1-415-655-0001 US TOLL
\item \small Access code: 808 344 165
\end{itemize}
\end{frame}

\begin{frame}[label=omp_affinity]
\frametitle{Meeting for omp-affinity}
\underline{\bf Join WebEx meeting}
\begin{itemize}
\small \item \small  \url{https://llnl.webex.com/llnl/j.php?MTID=md575d91e6134ce9b0aaf14bc65c011eb}
\item \small Meeting number: 846 335 992
\item \small Meeting password: places
\end{itemize}
\underline{\bf Join by phone}
\begin{itemize}
\item +1-415-655-0001 US TOLL
\item Access code: 846 335 992
\end{itemize}
\end{frame}

%TODO: add this. 

\begin{frame}[label=omp_task]
\frametitle{Meeting for omp-task}
\underline{\bf Join WebEx meeting}
\begin{itemize}
\small \item \small \url{}   
% \url{https://llnl.webex.com/llnl/j.php?MTID=md575d91e6134ce9b0aaf14bc65c011eb}
\item \small Meeting number: 846 335 992
\item \small Meeting password: places
\end{itemize}
\underline{\bf Join by phone}
\begin{itemize}
\item +1-415-655-0001 US TOLL
\item Access code: 846 335 992
\end{itemize}
\end{frame}


\begin{frame}[label=omp_device]
\frametitle{Meeting for omp-device}
\underline{\bf Join WebEx meeting}
\begin{itemize}
\small \item \small \url{}   
% \url{https://llnl.webex.com/llnl/j.php?MTID=md575d91e6134ce9b0aaf14bc65c011eb}
\item \small Meeting number: 846 335 992
\item \small Meeting password: places
\end{itemize}
\underline{\bf Join by phone}
\begin{itemize}
\item +1-415-655-0001 US TOLL
\item Access code: 846 335 992
\end{itemize}
\end{frame}

%\input{acks}
%\input{thanks}


%\bibliographystyle{abbrv}
%\bibliography{bibliography}

\end{document}
