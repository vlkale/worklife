%!TEX root = researchQuestionsAndData.tex

%TODO: fix minima explanation 
%TODO: think about a new graph for static fraction 
%TODO: split explanations of graphs in first slides, into separate graphs 
%TODO: get different noise graphs. 
%TODO: figure out explanation for histogram graphs. 
%TODO: figure out rebound code that maps to heartbeat sim. and show that here.  
%TODO: make sure that rebound code works in terms of mapping to heartbeat  (check with someone) 
%TODO: ensure naming consistent besf.


%TODO: worklife: see how you can connect different levels ..

%Ridhima : honda cr-v 
% dad: acura rlx 
%vivek: bmw 3


\begin{frame}
\frametitle{Overview} 
\begin{enumerate} 
\item Work 
\item Mgmt
\item Comm 
\end{enumerate} 
\end{frame} 

%\input{/Users/vivekkale/Desktop/ViveksLaptop/Work/pres/statdyn/thesis/content_researchQuestionsAndData}

\begin{frame}
\frametitle{Overview: Comm}
\begin{enumerate} 
\tiny \item \tiny 
\end{enumerate} 

\end{frame} 

\begin{frame}
\frametitle{Overview: Work}
\begin{enumerate}
\item Introduction and Motivation for Schedulers
\item Lightweight Scheduling with further optimizations + Scalability 
\item Scheduling for Both Transient and Persistent Load imbalances
\item Model-guided Optimization and Optimizations taking into account MPI Slack
\item Balancing Load Balance and Locality in different dimensions
\item Combining all schedulers and Application Programmer Usability
\item Conclusions and Future Work
\end{enumerate}
\end{frame}

\begin{frame}
\frametitle{Jobs}
{\small To get a prioritized list, we show from least feasible to most
  feasible and easiest to start from clean slate to hardest. It also
  prioritizes easiest to develop yourself to hardest. } 
\begin{enumerate} 
\small \item \small BNL: 
\item \small - 
\begin{itemize} 
\tiny \item \tiny D.E. Shaw 
\item \tiny Yahoo,  GS, Salesforce
\item \tiny  Google: 
\end{itemize}
 
\item \small broad national labs. 
\begin{itemize}
\tiny  \item \tiny PNNL or LBNL: 
\end{itemize} 

\item \small Academia 
\begin{itemize} 
\tiny \item \tiny Stanford 
\item \tiny MIT Media 
\end{itemize} 
\item \small Research focused jobs
\begin{itemize} 
\tiny \item \tiny  LLNL or Google
\end{itemize} 
\end{enumerate} 

\begin{itemize}
\tiny \item \tiny would be good to end up at LLNL long-term. 
\item \tiny 
\end{itemize} 
\end{frame} 

\begin{frame} 
\frametitle{Mediums to Use to Do Slides}

\begin{itemize} 
\tiny \item \tiny  
\item \tiny 
\end{itemize} 
\end{frame}

\begin{frame}
\frametitle{Cars Instances}  
{\tiny These are representative car instances that I think are
  suitable for me.} 
\begin{itemize} 
%TODO: add in 328i sport line (blk/grey color)
\small \item \small  2013 BMW 328i xDdrive. \$31,223. CPO. 24,340 mi. Colors: Blk
Exterior/Tan interior/WoodGrain. VIN (for Carfax):
WBA3B5G54DNS05235. Dealer ad: \url{http://www.cars.com/vehicledetail/detail/627013937/overview/.}
\item \small 2012 BMW 335i. \$34,998. Colors: Black/Black 32,003
  mi. CPO. M sport package. Tech package. Navigation. Heads up display. VIN(for
  Carfax):  WBA3A9C51CFX59924
  \url{http://www.cars.com/vehicledetail/detail/628928072/overview/}. Phone: 
%\item \small 2010 BMW M3.  \$38,525. CPO. 29,144 mi.  VIN (for Carfax):
%WBSWD9C55AP363244. \url{http://www.cars.com/vehicledetail/detail/623505105/overview/} 
\end{itemize}
\end{frame}

\comments{
\begin{frame} 
\frametitle{Long-term Goals: Heidelberg} 

\begin{outline}[enumerate] 
\tiny \1 {\tiny  Intro }
\tiny \2 {\tiny Opening: History:  Advisor , MPI.} 
\tiny \2 {\tiny My Work aims to solve scheduling problem with
  low-overhead.} 
%\tiny \2 {\tiny The key question I want to answer is how I can make a 
%long-term impact based on the thesis work I have tried to solve. } 

\tiny \2 {\tiny The conference can help me in 3 ways to accomplish this goal. }
\tiny \2 {\tiny Long-term Goal: Using my PhD thesis as the basis, 
provide new insight or develop an exemplar technology in the area of
computer science and mathematics.}
 % Don't say that the
% conference is the event that shapes
% your life. 

\tiny \1 {\tiny First, helps to strengthen work within my area of
  specialization.}
%\tiny \2 {\tiny Teach others what I am doing , who come from different
%areas of mathematics and computer science.} 
%\tiny {\2 This process can help to increase awareness in other 
%disciplines of my work, not just in the area of my speciality.} 

\tiny \2 {\tiny Explaining my work to experts in a variety of 
disciplines in computer science, my hope is to 
define and formulating the problem in ways done in another area of computer
science, e.g., Algorithms or Theoretical Computer Science, or 
in areas of Mathematics.} 
\tiny \2 {\tiny Particularly new mathematical formulations of the work can
  fill gaps in the work, and make the work stronger.}  

% the dissertation.}
%can help to provide additional insights of the work.} 
%\tiny \2 {\tiny Shed light on how to further strengthen the work.}
%\tiny \2 {\tiny It can also involve describing different aspects of my
%  solution. For example, software engineering aspect less
%  emphasized. Formalization through Theory of Computation could allow 
%for better runtime software, and increased error correctness of the solution, and to identify new techniques.}
%\tiny \2 {\tiny Teach others what I'm doing.} 

\tiny \1 {\tiny Impact to other areas of computer science and
  mathematics.} 
\tiny \2 {\tiny Could my work provide insight to new solutions in other areas such as
distributed systems? Could my scheduling techniques be beneficial
here? The opportunity to talk to others in distributed systems would
be beneficial.}

\tiny \1 {\tiny  Variety of backgrounds in computer science, and not
  just a specific field, it can help me understand implications of my
  thesis in areas other than computer science.} 
\tiny \2 {\tiny Architectures: Fast Hardware Synchronization. }  
\tiny \2  {\tiny Distributing Systems: Cloud computing .} 
\tiny \2 {\tiny Algorithms: Formulate as a theoretical problem. } 
\tiny \1 {\tiny It would benefit helping me develop long-term goals
  by having continued impact that my field ideas can have on society
  for years to come.} 
\tiny \2 {\tiny Because of the laureates continued involvement in the
  research community, they are uniquely aware of the current problems
  in society, and have a perspective on what can shape the next
  generation.}
\tiny \2 {\tiny How does my thesis work even when you consider new advances in physics, where quantum computing can change my thesis?} 

\tiny \1 {\tiny Forum can help teach how handle life.} 
\tiny \2 {\tiny The way you handle your life can make a difference in what changes you make.} 
\tiny \2 {\tiny Conferences are the norm, rather journals.} 
\tiny \2 {\tiny  A forum like this can help teach me how to remain competitive in this field without breaking down.} 
\tiny \1 {\tiny Conclusions:}
\tiny \2 {\tiny Getting other experts’ opinions will allow one to formulate a
particularly well-rounded opinion and vision for where I want to go.} 
\tiny \2 {\tiny This forum would help me shape my longer-term views, goals and
ideas of how to succeed in my career.} 
\tiny \2 {\tiny I believe that coming to a forum with other people 
  provides an opportunity for me to define my career goals, something
  I would like to be focused on as I work to complete my PhD.} 
\end{outline} 


%I also believe that by sharing my work with others and
%listening to others work, I can become more knowledgeable in my
%field. 

% {\tiny There is a scheduling session in every conference of the 6
% conferences. Computation is a fast-paced field. It requires management
% of time. } 
\end{frame} 
}

\begin{frame} 
\frametitle{Heidelberg Laureate Forum Essay}
{\tiny This defines the long-term goals, which is explained in the
  Heidelberg essay}.\\ 
\begin{outline}[enumerate] 
\tiny \1 {\tiny Intro: Long term goals and how this forum helps. } 

\tiny \2 {\tiny Define Long term goal : offer a novel perspective or 
new exemplar technology in the area of mathematics or computer
science.} 

\tiny \2 {\tiny By being selected for the forum (and subsequently
  attending it), I will be able to better accomplish this goal in
  three ways.}
 
\tiny \1 {\tiny I can learn from the experts about future trends in technology, which will help me navigate .} 

% Experts who know past trends, can help predict future
% trends and how areas of specialization will be affected.} 

\tiny \2 {\tiny Understanding of how future trends can impact what I am doing.} 
\tiny \2 {\tiny Experts who know past trends. How will quantum
  computing impact?} 
\tiny \1 {\tiny Attending forum will help me strengthen basis of PhD research (from which my career will follow). Mathematicians point of view. } 
\tiny \2 {\tiny New ways of formulating the problem will fill gaps
 (improve ? strengthen? collabs..), if any. %TODO: consider
                                %theoretical .. but is ok. 
} 
\tiny \1  {\tiny Broader impact to field of computer science. } 
\tiny \2 {\tiny  Implications of Scheduling to Cloud?} 
\tiny \1 {\tiny Conclusions: Attending will give well-grounded ,
  well-rounded viewpoint to help me pursue(achieve) long-term goals.} 
\end{outline}  
\end{frame}


\begin{frame} 
\frametitle{Reference Letter for Heidelberg Forum by Gropp} 
\begin{enumerate} 
\tiny \item \tiny 
\item \tiny 
\end{enumerate}

\end{frame} 


\begin{frame} 
\frametitle{Reference Letter for Heidelberg Forum by Cappello} 
\begin{outline}[enumerate] 
\tiny \1 {\tiny Intro: } 

\tiny \2 {\tiny Intent of Letter. }

\tiny \2 {\tiny Knowledge of Vivek and relationship to Vivek } 

\tiny \2 {\tiny My credentials, and who I've worked with.} 

\tiny \1 {\tiny Problem definition} 

\tiny \2 {\tiny Just doing MPI+OpenMP process/thread not good enough} 

\tiny \2 {\tiny Additionally, load imbalance problem across cores.} 

\tiny \2 {\tiny Bulk-synchrony needed for applications to be
  correct. }
\tiny \2 {\tiny Many applications are loosely synchronous, so impact
  of solving this is high.}

\tiny \1 { \tiny Vivek worked on solving the problem for MPI+pthread
  reg. mesh code and CALU.  } 

\tiny \1 {\tiny He then worked on additional techniques such as
  weighted } 
\tiny \1 {\tiny With this, he was awarded a Lawrene Fellowship, where
  he worked on slack-conscious scheduling. } 

\tiny \1 {\tiny Recently, he returned to work with Simplice to show
  how combining scheduling techniques can be beneficial. This was
  published in SC} 

\tiny \1 {\tiny Conclusion  }

\tiny \2 {\tiny Critical problem currently, and fundamental to
  computer science.  Problem will remain as long we need supercomputing
  for advancements in science and engineering, and thus solution will
  still be relevant.} 

\tiny \2 {\tiny Already established strong contacts; benefit from
  Heidelberg through interactions contribute to discussions} 

\tiny \2 { My highest recommendations} 
\end{outline} 

\end{frame}

\begin{frame} 
\frametitle{ Reference Letter}
\end{frame} 

\begin{frame} 
\frametitle{Cover Letter for SUNY sb}

\end{frame} 




\begin{frame} 
\frametitle{Cover Letter for D.E. Shaw}

\begin{itemize}
\tiny \item \tiny 
\item \tiny 

\end{itemize}
\end{frame}

\begin{frame}
\frametitle{Questions for D.E. Shaw Application}
\begin{enumerate} 
\tiny \item \tiny Please describe what you consider your most
significant accomplishment as a hardware/software architect, design
engineer, or researcher.
\end{enumerate} 
\end{frame} 


\begin{frame}  
\frametitle{ DE Shaw Question 1: Project Descriptions }  
{\tiny What were the sizes (in terms of number of lines of code, number of
gates, etc.) of the two to four largest designs you've worked on? If
the applications were group projects, what percent of each project was
your contribution? What programming languages were used? Briefly
describe the purpose and functionality of each project.} 

\begin{enumerate} 
\item \tiny DPLASMA: Helped develop libraries for Communication-avoiding libraries. 

\item \tiny OpenMP runtime. 
\begin{itemize} 
\tiny \item \tiny 30000 lines of code (and may grow further). 
\item \tiny C++ for runtime. python for plotting. 
\item \tiny It was a project led by my supervisor. 75\% of the project
  was my contribution. 

\item \tiny Purpose: Provide a scheduling library which consists of
  different scheduling strategies. This was developed as a standalone project,
  but actually sits as a component within the above runtime. 

\item \tiny Functionality: In MPI\_Init(), uses ROSE.  
Uses PMPI to adjust scheduler, Uses ROSE to automate compiler
transformation. %TODO: make more clear
\end{itemize} 

\item \tiny Slack-conscious low-overhead Scheduling runtime. 

\begin{itemize} 
\tiny \item \tiny 40000 lines of code. 
\item \tiny C++ for runtime. python for plotting. 
\item \tiny It was a group project. 75\% of the project was my
  contribution. 
\item \tiny Purpose: Provide a runtime for people to use my
  low-overhead scheduling techniques.  
% runtime for people to use my
%  low-overhead scheduling techniques.  
\item \tiny Functionality: In MPI\_Init(), uses ROSE.  
Uses PMPI to adjust scheduler, Uses ROSE to automate compiler
transformation. %TODO: make more clear 
\end{itemize} 

\item \tiny 
\end{enumerate} 




\end{frame} 

\begin{frame} 
\frametitle{DE Shaw Question 2: Areas of Interest} 
{\tiny A research or engineering position within DESRES can focus on
  any or all of a range of areas, from software design and algorithm
  analysis to computer architecture, chip design, and
  verification. What area or areas best match your expertise and
  interest?} 

\begin{itemize} 

\tiny \item \tiny Architecture. 
\item \tiny High-Performance Computing: 
\item \tiny Performance Modeling and Theoretical Analysis
\item \tiny Dense Matrix Factorizations 
\item \tiny Scientific Computing and Applications. 

\end{itemize} 

\end{frame} 
