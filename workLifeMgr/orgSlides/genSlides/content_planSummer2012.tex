\begin{frame} [Tasks] 
\frametitle{Task: \textit{MPI shared memory extensions} + Numerical linear algebra work (Estimated time: 2 weeks, 1 week complete)}
\begin{enumerate} 
\item Finish shared mem extensions paper (\textit{complete}). \\ 
\item Tests/application of shared mem extensions for simple 3D stencil app (in progress tommorrow).  \\ 
\item Integration and implementation of lightweight scheduling through allocated shared memory.     \\
\end{enumerate} 
\end{frame}  

\begin{frame} [Tasks] 
\frametitle{Task: MPI shared mem extensions + \textit{Numerical linear algebra work}, cont'd (Estimated time: 2 weeks, 1 week complete)}
\begin{enumerate} 
\item IPDPS presentation : lightweight scheduling for Communication-Avoiding 
Dense Matrix Factorizations (\textit{complete}).  \\
\item Quick review AMG performance model.  No need to integrate lightweight scheduling 
into this performance model. I just need to review it as part 
of the knowledge for my thesis. (We've already applied lightweight scheduling theoretical 
analysis to Communication Avoiding LU and Comm Avoiding QR, and this is good enough ). \\ 
\end{enumerate}
\end{frame} 

\begin{frame} [Tasks] 
\frametitle{Task: Tools setup (Estimated time: 5-6 days)} 
\begin{enumerate} 
\item Make sure lightweight sched strategy is usable in 
basic application (through linking using \texttt{ar rcs ...}). \\ 
\item  make sure it is portable through \texttt{autoconf} or 
\texttt{cmake} (preferable is \texttt{autoconf} since I've used it 
before, but am going to look into \texttt{cmake}). \\ 
\item Collaboration support: git repo setup, organize current svn for paper \\ 
\item Publicizing: simple wiki + website.  \\ 
\item Code documentation: Doxygen, or really just a simple systematic mechanism for documentation/commenting that can easily be seen. \\ 
\item Results organization: make simple \texttt{gnuplot} script to make common plots needed. \\  
\item Setup and organize figures/diagrams for lightweight scheduling  and slack-conscious lightweight sched \\ 
\end{enumerate} 
\end{frame} 

\begin{frame} [Tasks] 
\frametitle{Task: Additional Experimentation on BG/Q and BW (Est time: 3 days)}
\begin{enumerate} 
\item Do fully dynamic sched. (got this setup) \\
\item Check slack-conscious sched on BG/Q again. \\ 
\item Do experimentation with different compilers: gcc and xl , on BG/Q  \\
\item Check experimentation with different OpenMP thread configuration settings \\ 
\end{enumerate}  
\end{frame} 

\begin{frame} [Tasks] 
\frametitle{Task: Simulator setup and integration of slack-conscious scheduler (Estimated time: 3-4 days)}
\begin{enumerate} 
\item Make sure it works properly on laptop, write a readme for running. \\
\item Incorporate slack-conscious scheduling. \\
\item May want to consider noise injection as an alternative to simulation.  Main 
goal of this task is to see if this helps the argument of the thesis, or if 
we should drop it.  (Issue with simulator being that someone needs to validate and maintain it).  \\
\end{enumerate} 
\end{frame} 

%\begin{frame} [Tasks] 
%\frametitle{Task: Revise current Concepts/Theoretical Analysis (Est time: 3 days)} 
%\begin{enumerate} 
%\item Theoretical Analysis: have simple version ready which adds in slack-conscious scheduling. \\ 
%\item Strategy Explanation: Apply it to a simple example. \\ 
%\item Apply it to AMG. 
%\end{enumerate} 
%\end{frame} 

%\begin{frame} [Tasks] 
%\frametitle{Task: Risk Analysis (Est time:  4 days)} 
%\begin{enumerate} 
%\item Review statistics and game theory. 
%\item Look through noise theoretical work by Tsasfir.
%\item connect game theory and statistic to noise, so that we 
%can adjust static fraction for sched. 
%\end{enumerate} 
%\end{frame} 

\begin{frame}
\frametitle{Schedule(Summer 2012)}
\begin{itemize}

\tiny \item \tiny May 13th - 22nd:  MPI paper, IPDPS2012 presentation, website+ wiki + collab space setup \\  

\item \tiny May 22nd - June 3rd: application of shared mem extensions (use lightweight sched) 
to 3D stencil code (as a comparison to OpenMP prog model), work on follow-up tests for SC paper. 

\item \tiny June 3rd - June 13th: finish application of shared mem extensions to 3D stencil code,  
setup the library creation for slack-conscious sched using software agreed upon above(e.g.
\texttt{autotools} , \texttt{ar rcs}).  \\  

\item \tiny June 13th - June 15th: travel back to LLNL, review some basic statistics and some game theory. \\  

\item \tiny June 15th - July 15th: Risk Analysis + Theory + Formalization of Slack-conscious sched in software \\ 

\item \tiny July 15th - Aug 15th: apply to real code; look at AMG and PF3D. \\ 

\item \tiny Aug 15th - Sept 1st: gather results and do write-up for fall conference (PPoPP or maybe IPDPS target) \\ 

\end{itemize} 
\end{frame} 

%\begin{frame} 
%\frametitle{Schedule(Fall 2012)}
%\begin{itemize}
%\tiny \item \tiny Sept 1st - Oct 1st: IPDPS write-up for risk analysis.
%\item \tiny Oct 1st - Dec 1st : Journal with Prof Grigori and Ulrike + MPI shared memory programming model + prelim  
%\item \tiny Dec 1st - Jan 15th : organize code, thesis writing plan and organization
%\item \tiny Jan 15th - May 15th : thesis writing ,  Gordon Bell submission or SC submission for LBM work.   
%\item \tiny May 15th - Aug 15th : thesis writing,  defense
%\end{itemize} 
%\end{frame} 
