\begin{frame}
\frametitle{Intro}
\begin{enumerate} 
\item \tiny 
\item \tiny 
\end{enumerate} 
\end{frame}


\begin{frame}
\frametitle{ Problem} 
\begin{itemize}
\item \small Noise is a problem for current generation and next-generation hpc clusters. 
\item \small We can see this through simulation, through Hoefler et. al. where ..
\item \small We see this through theory, thorugh Tsasfir et. al where ..
\item \small We see this through experimentation, as shown in Petrini et. al.  where ...
\end{itemize} 
\end{frame}  

\begin{frame}
\frametitle{Why Other Soln's don't work(listed in order of innovation date) } 
\begin{enumerate} 
\item \small co-scheduling: hard to schedule everything together, as some nodes will have timers. Also, can't coschedule hardware variations. 
\item \small low-noise OS can't help because ...
\item \small extra core (as on BG/Q) can't help because ...
\item \small basic dynamic scheduling
\begin{itemize}
\item \tiny Cilk Work-stealing 
\item \tiny OpenMP dynamic scheduling
\end{itemize}
\end{enumerate} 
\end{frame} 

\begin{frame}
\frametitle{Defining Problem Space} 
\begin{enumerate} 
\item  \small The problem space is for applications such as LU factorization, PDE's which can have fast computations. 
\item \small Something like Quantum Monte Carlo can do with heavyweight mechanisms. 
\item \small 
\item \small 
\end{enumerate} 
\end{frame}


\begin{comment}
\begin{frame}
\frametitle{What we are trying (and not trying) to do } 
\begin{enumerate} 
\item  
\item 
\item 
\end{enumerate} 
\end{frame} 
\end{comment}

\begin{frame}
\frametitle{How do we solve this problem?} 
\begin{enumerate} 
\item \small Our basic solution: use lightweight scheduling to dampen
  occasional noise within one node. 
\end{enumerate} 
\end{frame} 

\begin{frame} 
\frametitle{Justification of our solution}
\begin{itemize} 
\item \small within node is sufficient because once you go off-node,
  the cost of coherence cache misses is very high, making it very hard
  to make a lightweight mechanism work. 
\item \small importance of lightweight (and why pure virtualization
  approaches are not always adequate): pure virtualization is not
  adequate due to cost of communication latency being high. 
\item \small pure virtualization is not adequate due to cost of
  communication latency being high. 
\item \small measurement-based load balancing not sufficient due to
  not being able to predict noise patterns a priori (as noise is a
  transiently varying quantity). 
\end{itemize}
\end{frame}


\begin{frame}
\frametitle{Perf. Model for noise}  
\begin{itemize} 
\item \small (obtain from docs)
\item 
\item 
\end{itemize}
\end{frame}


\begin{frame}
\frametitle{Perf. Model for Dynamic Scheduling}  
\begin{itemize} 
\item \small (obtain from docs)
\item 
\item 
\end{itemize}
\end{frame} 

\begin{frame}
\frametitle{Perf. Model for MPI Slack}  
\begin{itemize} 
\item \small (obtain from docs)
\item \small 
\item \small 
\end{itemize}
\end{frame} 


\begin{frame}
\frametitle {How can scheduling decisions be made? }
\begin{enumerate}
\item \small \textbf{Compile time:} 
\begin{itemize} 
\item \tiny Tune static fraction. 
\item \tiny Tune tasklet size
\end{itemize}
\item \small \textbf{Runtime:}
\begin{itemize} 

\item \tiny Implementation of new scheduler which decides how tasklets are distributed in the queue. 
\end{itemize}
\item \small \textbf{Across Runs:}
\begin{itemize} 
\item \tiny Each core/thread has preference for tasklets that run on it from before.  
\item \tiny We can use patterns in persistent noise to make decisions
  about weighting work from one core to the other. 
\item \tiny We can use knowledge of slack to further avoid scheduling,
  because we know we won't amplify when there it is known to have
  adequate slack. 
\end{itemize}
\end{enumerate}
\end{frame}

\begin{frame}
\frametitle{Algorithm for lightweight Scheduling} 
\begin{enumerate} 
\item \small Our basic solution: use lightweight scheduling to dampen
  occasional noise within one node. 
\end{enumerate} 
\end{frame} 

\begin{frame} 
\frametitle{Algorithm for Slack-conscious lightweight Scheduling} 
\begin{enumerate} 
\item \small Our basic solution: use lightweight scheduling to dampen
  occasional noise within one node. 
\end{enumerate} 
\end{frame} 

\begin{frame}
\frametitle{Implementation for lightweight Scheduling} 
\begin{enumerate} 
\item \small 
\item \small 
\item \small 
\end{enumerate} 
\end{frame} 

\begin{frame}
\frametitle{Implementation for Slack-conscious Lightweight Scheduling} 
\begin{enumerate} 
\item \small 
\item \small 
\item \small 
\end{enumerate} 
\end{frame} 

\begin{frame}
\frametitle{Problem/Issue --> Solution}
\begin{itemize}
\item \small Overhead of scheduling when task size is computationally
  small --> \textit{extremely lightweight scheduling, including compile-time}. 
\item \small  Importance of locality --> \textit{take locality into account in
  decisions, while remaining lightweight}.
\item \small Predicting performance imbalance is hard --> observe
  correlation with communication delays --> use slack. Etc.  
\end{itemize} 
\end{frame}


\begin{frame}
\frametitle{Sched} 
\begin{table}[h!]
  \begin{center}
    \small
    \begin{tabular}{ | c || c | c | c | c | c | c | c |}
      \hline
      & \underline{T1} & \underline{T2} & \underline{T3} & \underline{T4} & \underline{T5} & \underline{T6} & \underline{T7} \\ 
      \hline
      \tiny \textbf{I1(<weight>)}  &  \tiny - &\tiny -  & \tiny - & \tiny -  & \tiny -  & \tiny - & \tiny - \\ 
     \hline
      \tiny \textbf{I2(<weight>)}  &  \tiny - &\tiny -  & \tiny - & \tiny -  & \tiny -  & \tiny - & \tiny - \\ 
      \hline
      \tiny \textbf{I3(<weight>)}  &  \tiny - &\tiny -  & \tiny - & \tiny -  & \tiny -  & \tiny - & \tiny - \\ 
      \hline
    \end{tabular}
  \end{center}
  \caption{<Table Caption goes here>}
\end{table}
\end{frame} 


\begin{frame} 
\frametitle{Thesis Committee} 
\begin{enumerate}
\item \small Maria Garzaran 
\item \small Prof. Wen-mei Hwu OR Prof. David Padua
\item \small Prof. William Gropp
\item \small  Bronis de Supinski
\end{enumerate}
\end{frame} 

\begin{comment}
\begin{frame}
\frametitle{}
\begin{enumerate}
\item \small \textbf{Item A}
\begin{itemize} 
\item \tiny 
\item \tiny
\end{itemize} 
\item \small \textbf{Item B}
\begin{itemize} 
\item \tiny re-imbursements
\item \tiny 
\item \tiny 
\end{itemize} 
\item \small \textbf{Item C} 
\begin{itemize} 
\item \tiny 
\item \tiny 
\item \tiny 
\end{itemize} 
\end{enumerate}
\end{frame}   
\end{comment}
