\begin{frame}
\frametitle{Opening}
\begin{itemize}
\item Shake hands 
\item Ladies and Gentleman, Distinguished Members and Guests 
\end{itemize}
\end{frame} 

\begin{frame}
\frametitle{Introduction}
\begin{itemize}
\item \small Tennis is one of the world's major sports, with 4 grand
  slam events (the largest being Wimbledon) played each year in the
  first week of July. 
\item \small Today, I wanted to talk about one of my passions, tennis. 
\item \small Today, I am going to tell you about my experience 
with tennis, as I believe it has played an important role in my life. 
\item \small I hope you will get to know me better through this.  
\end{itemize} 
\end{frame} 

\begin{frame} 
\frametitle{My Perspective of Tennis}
\begin{itemize} 
\item \small Know how to serve, going something like this.
\item \small Need to know how to hit a forehand, going something like this. 
\item \small For a backhand, you need to twist your body, grip both
  hands, and move like this. 
\item \small And finally (most importantly), you need to make sure
  that you don't smash your racket at the end of the point. 
\item \small Then, you need to be able to put these all together, and
  play your strengths and strategy. 
\end{itemize}  
\end{frame} 

\begin{frame} 
\frametitle{ Styles of Play }
\begin{itemize} 
\item \small Roger Federer has a well-balanced game, hitting all
  shots adequately well. The thing is that he gets a bit offset in
  long matches and can't run as hard.  He keeps his calm. 

\item \small Rafael Nadal plays a defensive game from the baseline. He
  can run very hard though, which allows him to win long matches. 
\item \small John McEnroe has played a mostly offensive game, trying
  to serve and volley. He keeps his calm. 
  point about his game is that he does become angry after a certain
  point. He tells umpires ``You canNOT be serious!'' a number of
  times, if he disagrees with a call.  
\item \small Comparing myself to these players, my game is more
  defensive and I stay closer to the baseline. I would say I have more
  endurance than my opponents in longer matches.  I've started to play
  more like Roger Federer though, trying more strategy to win points. 
  I do occasionally get mad after points,  but more recently I have learned to cool it down a bit. 
\end{itemize} 
\end{frame} 

\begin{frame}
\frametitle{Matches played} 
\begin{enumerate}
\item \small One match took 7 hours,  played a pretty defensive game, playing
  more forehands and backhands than volleys. I amazingly won, as I simply
  outdid my opponent in terms of stamina. 
\item \small I played another match for 2 hours, 
  where I lost the match very easily to someone who was much better
  than me. I probably learned quite a bit from this match and improved
  my level, as I was really challenged to win the point. I could learn
  from the many mistakes I made, in order to do better in a match with
 someone at my own level. 
\item \small Another match went on for a moderate 3 hours, with each
  point going on for a short time, but I coming back after a loss in
  the first set. In this match, I at first was playing my usual
  baseline game, hitting as hard as I could. 
  I then realized the opponent's weakness, which was volleying, 
so I hit many drop shots in order to get him to come in, and then miss
the ball. 
\end{enumerate} 
\end{frame}

\begin{frame} 
\frametitle{Tennis is a lot like life} 
\begin{itemize}  
\item \small Many goals we set in our lives, such as publishing a
  paper, is like a tennis match. \\ 
\item \small Just like in a tennis match, there are several basic skills we need to know in order
  to acheive our goals.  For publishing a CS conference, we should know about
  programming. 

\item \small Just like in a tennis match, there are several styles of
  meeting our goals, and each style is different based on the
  individual.  When we write our papers, we focus more on theory if
  that is our strength, and focus more on experimentation if that is
  our strength.  
 
\item \small Just like in a tennis match, experience with reaching our
  goals is key to becoming better at attaining goals. 
  We may not publish some papers, but we learn through experience how
  to publish. 

\item \small I like to watch Wimbledon on summer days. 
Watching pros play is amazing not only because of all the decisions they are 
making, but also their determination in trying to win a tennis match. 
This energy and determination motivates me in my own life, especially
during those summer days when I generally have several goals I want to
acheive. \\ 

\item \small If you need a hitting partner, feel free to get in touch
  with me. \\
  
%\item \small Each goal we set in our lives is like a tennis match.
%  The goals we set are like opportunities to further yourself. You
%  succeed in some goals, but you fail at other goals. When you succeed, you
%  become confident and can do the same thing again, or possibly challenge
%  yourself to larger goals. When you fail, you try to learn from 
%  it and see where you can improve. \\  

% \item \small Just like the amount of time it takes to play a tennis 
%  match may vary, the amount of time to reach a goal may vary (one
%  example is getting a research paper published). Sometimes, you get
%  lucky and publish quickly. Other times, you are not so lucky and
%  don't publish as fast. \\ 

%\item \small Finally, we learn to keep ourselves motivated, and not
%  get down on ourselves as we try to meet our goals. \\ 

%\item \small The broader lesson I've learned is to apply each of the
%  skills in tennis, to motivate and persevere through goals in my
%  life. \\ 

\end{itemize} 
\end{frame}

\begin{frame}
\frametitle{Closing}
\begin{itemize}
\item Shake hands 
\item Sit back down.
\end{itemize}
\end{frame} 
