%!TEX root = researchQuestionsAndData.tex

%TODO: fix minima explanation 
%TODO: think about a new graph for static fraction 
%TODO: split explanations of graphs in first slides, into separate graphs 
%TODO: get different noise graphs. 
%TODO: figure out explanation for histogram graphs. 
%TODO: figure out rebound code that maps to heartbeat sim. and show that here.  
%TODO: make sure that rebound code works in terms of mapping to heartbeat  (check with someone) 
%TODO: ensure naming consistent besf.

%Ridhima : honda cr-v 
% dad: acura rlx 
%vivek: bmw 3


\begin{frame}
\frametitle{Overview}
\begin{enumerate} 
\item Work 
\item Mgmt
\item Comm 
\end{enumerate}
\end{frame} 






\begin{frame}
\frametitle{Overview: Work}
\begin{enumerate}
\item Introduction and Motivation for Schedulers
\item Lightweight Scheduling with further optimizations + Scalability 
\item Scheduling for Both Transient and Persistent Load imbalances
\item Model-guided Optimization and Optimizations taking into account MPI Slack
\item Balancing Load Balance and Locality in different dimensions
\item Combining all schedulers and Application Programmer Usability
\item Conclusions and Future Work
\end{enumerate}
\end{frame}


%!TEX root = researchQuestionsAndData.tex

%TODO: fix minima explanation 
%TODO: think about a new graph for static fraction 
%TODO: split explanations of graphs in first slides, into separate graphs 
%TODO: get different noise graphs. 
%TODO: figure out explanation for histogram graphs. 
%TODO: figure out rebound code that maps to heartbeat sim. and show that here.  
%TODO: make sure that rebound code works in terms of mapping to heartbeat  (check with someone) 
%TODO: ensure naming consistent besf.

%Ridhima : honda cr-v 
% dad: acura rlx 
%vivek: bmw 3


\begin{frame}
\frametitle{Overview}
\begin{enumerate} 
\item Work 
\item Mgmt
\item Comm 
\end{enumerate}
\end{frame} 






\begin{frame}
\frametitle{Overview: Work}
\begin{enumerate}
\item Introduction and Motivation for Schedulers
\item Lightweight Scheduling with further optimizations + Scalability 
\item Scheduling for Both Transient and Persistent Load imbalances
\item Model-guided Optimization and Optimizations taking into account MPI Slack
\item Balancing Load Balance and Locality in different dimensions
\item Combining all schedulers and Application Programmer Usability
\item Conclusions and Future Work
\end{enumerate}
\end{frame}


%!TEX root = researchQuestionsAndData.tex

%TODO: fix minima explanation 
%TODO: think about a new graph for static fraction 
%TODO: split explanations of graphs in first slides, into separate graphs 
%TODO: get different noise graphs. 
%TODO: figure out explanation for histogram graphs. 
%TODO: figure out rebound code that maps to heartbeat sim. and show that here.  
%TODO: make sure that rebound code works in terms of mapping to heartbeat  (check with someone) 
%TODO: ensure naming consistent besf.

%Ridhima : honda cr-v 
% dad: acura rlx 
%vivek: bmw 3


\begin{frame}
\frametitle{Overview}
\begin{enumerate} 
\item Work 
\item Mgmt
\item Comm 
\end{enumerate}
\end{frame} 






\begin{frame}
\frametitle{Overview: Work}
\begin{enumerate}
\item Introduction and Motivation for Schedulers
\item Lightweight Scheduling with further optimizations + Scalability 
\item Scheduling for Both Transient and Persistent Load imbalances
\item Model-guided Optimization and Optimizations taking into account MPI Slack
\item Balancing Load Balance and Locality in different dimensions
\item Combining all schedulers and Application Programmer Usability
\item Conclusions and Future Work
\end{enumerate}
\end{frame}


%!TEX root = researchQuestionsAndData.tex

%TODO: fix minima explanation 
%TODO: think about a new graph for static fraction 
%TODO: split explanations of graphs in first slides, into separate graphs 
%TODO: get different noise graphs. 
%TODO: figure out explanation for histogram graphs. 
%TODO: figure out rebound code that maps to heartbeat sim. and show that here.  
%TODO: make sure that rebound code works in terms of mapping to heartbeat  (check with someone) 
%TODO: ensure naming consistent besf.

%Ridhima : honda cr-v 
% dad: acura rlx 
%vivek: bmw 3


\begin{frame}
\frametitle{Overview}
\begin{enumerate} 
\item Work 
\item Mgmt
\item Comm 
\end{enumerate}
\end{frame} 






\begin{frame}
\frametitle{Overview: Work}
\begin{enumerate}
\item Introduction and Motivation for Schedulers
\item Lightweight Scheduling with further optimizations + Scalability 
\item Scheduling for Both Transient and Persistent Load imbalances
\item Model-guided Optimization and Optimizations taking into account MPI Slack
\item Balancing Load Balance and Locality in different dimensions
\item Combining all schedulers and Application Programmer Usability
\item Conclusions and Future Work
\end{enumerate}
\end{frame}


\input{content_workSlides}

\begin{frame}
\frametitle{Overview: Mgmt} 

\end{frame} 

\begin{frame}
\frametitle{Overview: Comm} 


\end{frame} 









%TODO: add in simple bulk-synchronous code here. 
%TODO: justify why these old graphs are fine. 
\begin{frame}
\frametitle{Example Synchronous MPI code}
\lstinputlisting{../listings/mpi-bulk-synch.c}
\end{frame}


\begin{frame}
\frametitle{Transient Load Imbalance and its Mitigation}
\begin{columns} 
\column{0.5\columnwidth}
{\small Performance Irregularities on individual nodes can lead to
  significant slowdown of synchronous MPI codes. }
\column{0.5\columnwidth}
{\small One promising approach, given multi-core nodes, is to only
re-distribute work within each node (this is inexpensive). } 
\end{columns} 
%{$\rightarrow$ \small Within-node performance irregularities that
%induce load imbalances are as significant a factor for applications as
%across-node load imbalances. The below gives an intuitive
%explanation of why this is so.}

%TODO: animate this to show  right one slide later. 
\begin{figure}[ht]
  \label{fig:NoiseTransImbPic}
  \label{fig:AmplificationPic1}\subfloat[\label{fig:AmplificationPic1}
    \tiny Noise occuring on different nodes in different
    iterations delays every iteration. 
  ]{\includegraphics[height=1.0in]{../pictures/NoiseAmplificationPic4}}
  \label{fig:AmplificationPic3}\hspace*{0.25in}\subfloat[\label{fig:AmplificationPic3}
    \tiny Execution times is significantly reduced, if we assume load
    can be perfectly re-distributed within each node. 
  ]{\includegraphics[height=1.0in]{../pictures/NoiseMitigatedPic5}}
  \caption{\label{fig:NoiseTransImbPic} Schematics of application
    timelines showing how impact of noise can be mitigated by
    idealized within-node work re-distribution.} 
\end{figure}
\end{frame}
%TODO: finalize that this should go here. We explain that the problem
%will become even greater when laod imbalance is.  
\begin{frame}[label=workredistributionNoise]
\frametitle{Within-node Load Imbalance May Help}
\begin{figure}[ht!]
\label{fig:stat-histo}
\begin{center}
\includegraphics[scale=0.16]{../images/IterTimesHisto-outliers-static}\\
%\includegraphics[scale=0.29]{../images/IterTimesHisto-outliers-dynamic}
%\includegraphics[scale=0.35]{../images/IterTimingsHisto-dynamic}
\end{center}
\caption{\label{fig:stat-histo} \small Iteration timing distribution across 1000 timesteps of the
statically scheduled\comments{\hyperlink{3Dmesh}{\beamerbutton{3D
      reg. mesh}}} 3D regular mesh computation on one node of Power5+. }
\end{figure} 

\comments{
\begin{figure}[h!]
\label{fig:work-redistribution-histo-ideal}
\begin{center}
\fbox{\includegraphics[scale=0.23]{../images/emptySchedImage}}\\
{\tiny todo - need to add (idealized) iteration timing histogram here with single bar at
  6.75 ms} 
\end{center}
%TODO: could re-word the below further. 
\caption{\label{fig:work-redistribution-histo-ideal} \small  }
\end{figure} 
}
\begin{itemize} 
\tiny  \item \tiny Iteration timing distribution
  for 3D regular mesh across 1000 timesteps using no work
  re-distribution\comments{across cores} during each iteration cause
  bi-modal distribution. The second mode on the right causes 
  performance degradations. 
 \item \tiny To minimize the impact of noise to application performance, 
  we would need to re-distribute work in a way that causes the above
  histogram be the one with minimal spread and mean.
\end{itemize} 

\end{frame}

\begin{frame}[label=mitigationWithinNode]
\frametitle{Within-node Persistent Load Imbalance and Mitigation of It}
\begin{figure}[ht]
\label{fig:AppPersistentImbPic}
\subfloat[\tiny A modeled application timeline having persistent load
imbalance across nodes, and cores during execution.]{\includegraphics[height=1.0in]{../pictures/PersistentCores}}
\label{fig:AmplificationPic3}\hspace*{0.25in}\subfloat[\label{fig:AmplificationPic3}\tiny
  A modeled application timeline with work re-distributed across cores,
to reduce load imbalance.]{\includegraphics[height=1.0in]{../pictures/PersistentNode}}
\caption{\label{fig:AppPersistentImbPic} Application-induced load imbalances, which are typically jagged and persistent, are also mitigated by within-node re-balancing. }
\end{figure}
\end{frame}

\begin{frame}[label=ImbWithinNode]
%\frametitle{Performance Doesn't Scale with Cores as Exp}
\frametitle{Within-node Imbalance May Help}
{\small We measure the load on each core for an n-body application code called Rebound.}\\
{\small Mitigated Imbalance: Assuming load within each node is perfectly balanced across its cores without any overhead.}
\hspace*{2.5in}\includegraphics[scale=.12]{../pictures/persistentImb}\hspace*{1in}\includegraphics[scale=.12]{../pictures/mitigatedImb}
\vspace*{-0.1in}
\begin{center}
\input{../plots/bh-withinAcrossImb-cab}
\end{center}
%TODO: show calculation for this. 
\begin{center}
{\tiny Imbalance is measured as $\frac{{max\{load_{i}\}}}{avg\{load_{i}\}}$, where $i$ is the processing element.}
%TODO word the below better 
\end{center}
\end{frame}

%TODO: see if we can change title 
\begin{frame}[label=dynLoadBal]
\frametitle{So, can dynamic load balancing fix this?}
\begin{itemize}
\item Dynamic load balancing within a node can improve performance. 
\begin{enumerate}
\item For both kinds of imbalances. 
\item Persistent imbalance can also be addressed by across-node balancing (avail. in Charm++, Zoltan), but this is complementary.
\end{enumerate}
\vspace*{.5in}
\item However, within-node dynamic load balancing faces many challenges. 
\end{itemize}
\end{frame}

\begin{frame}[label=idleTimeCalc]
\frametitle{Challenge 1: Idle Time due to Within-node Imbalance}
{\tiny How we measured idle time:}
\begin{center}
\lstinputlisting{../listings/omp-CG-idleTimeCalc.c}
\end{center}
\end{frame}

\begin{frame}[label=basictimings]
\frametitle{Challenge 1: Idle Time due to Within-node Imbalance}
{\tiny \textit{\textbf{Setup and Methodology:}} Used NASLU OpenMP
  benchmark code and Barnes-Hut pthread application code. Ran on one
  node of LLNL cab cluster, which is an Intel Xeon 16-core with CHAOS
  operating system.\comments{using gcc optimization level 'O3'}. \comments{having OpenMP3 and mvapich3}}
\visible<1->{
\begin{figure}[t]
\begin{center} 
\includegraphics[scale=0.35]{../plots/dmTime-nbody-cab-idleonly}
\includegraphics[scale=0.35]{../plots/dmTime-NASLU-cab-idleonly}
\end{center}
\caption{\small Breakdown of execution time for a load imbalanced code
  (left) and a load balanced code (right) on cab, shown for the
  3\comments{available}OpenMP scheduling strategies.}
\end{figure}}

\visible<1->{
\begin{itemize}
\tiny {\item \tiny  Dynamic or guided scheduling eliminates idle times in both codes.}
\tiny {\item \tiny However, for NASLU, the total execution time actually increases with both dynamic and guided scheduling.}
%TODO : make within-node dynamic balancing clear. 
\begin{itemize}
\tiny {\item \tiny  Recall that (within-node) dynamic load balancing is needed even for naturally balanced applications, to handle noise.}
\end{itemize}
\tiny {\item \tiny  For n-body, the execution time does not decrease as much as we expect, based on the idle time in the static scheduling.}
\end{itemize}}
\end{frame}

%TODO: word the expected overhead of idle time better. 
\begin{frame}
\frametitle{Why is the Non-idle Time Increasing?}
%\begin{itemize}
%\tiny \item \tiny Scheduling overhead?
%\end{itemize}
\begin{figure}[t]
\begin{center}
\includegraphics[scale=0.35]{../plots/dmTime-nbody-cab-idleonly}
\includegraphics[scale=0.35]{../plots/dmTime-NASLU-cab-idleonly}
\end{center}
\end{figure}
\begin{center}
{\small Breakdown of execution time for a load imbalanced code (left) and a load balanced code (right) on cab.}
\end{center}
{\small (Answer: Mostly data movement, as we will see in the next few slides.)}
\end{frame}

%TODO: check cause/challenge here and in thesis. 
\begin{frame}
\frametitle{Challenge 2: Synchronization Overhead}
{\tiny \textbf{\textit{Methodology}}: Obtain call-path profile from
  hpcToolkit which gives percent time spent in \texttt{omp\_lock()}
  OpenMP runtime function.}
{\tiny comp: computation time,  dq: locking and synchronization, 
Dm:  data movement, due to extra cache misses, etc. } 
\begin{figure}[t]
\label{fig:dmTimes-cab}
\begin{center}
\includegraphics[scale=0.35]{../plots/dmTime-nbody-cab}
\includegraphics[scale=0.35]{../plots/dmTime-NASLU-cab}
\end{center}
%\caption{\label{fig:dmTimes-cab} \tiny Breakdown of time. Synchronization overheads are shown in green.}
\end{figure}
\begin{center}
{\small Breakdown of time. Synchronization overheads are shown in green.}
\end{center}
\begin{itemize}
\tiny \item \tiny Synchronization overheads exist when using dynamic or guided scheduling.
\item \tiny While seemingly insignificant, they account for roughly 5\% percent of execution time for both of these applications. 
\end{itemize}
\end{frame}

%TODO: figure out whether we need data movement on BG/Q. 
%TODO: figure out whether we need to show data movement on bw. 
%TODO: see if you should show static time, to explain prefetching benefit. 
%TODO: separate NUMA and non-NUMA slides.
%TODO: add in details on architecture/platform settings. 
%TODO: add in job script specifics.
%TODO: add number of nodes run on.
%TODO: add in timestep length and problem size. 
%TODO: explain why for different setups 

\begin{frame}
\frametitle{Challenge 3: Data Movement}
{\tiny \textit{\textbf{Methodology}}: Measured L2 and L3 cache misses using PAPI counters, placed just before and just after OpenMP threaded computation region. \comments{(Note: will add code in next slide if I can} } 
\visible<1->{
\begin{figure}[t]
\label{fig:cacheMisses-cab} 
\begin{center}
\includegraphics[scale=0.30]{../plots/L2-cacheMisses-nbody-cab}
\includegraphics[scale=0.30]{../plots/L2-cacheMisses-NASLU-cab}\\
\includegraphics[scale=0.30]{../plots/L3-cacheMisses-nbody-cab}
\includegraphics[scale=0.30]{../plots/L3-cacheMisses-NASLU-cab}
\end{center}
\end{figure}
\begin{center}
{\small Cache misses with L2 misses on top and L3 on bottom, for different OpenMP scheduling strategies.}
\end{center}
}
\visible<1->{
\begin{itemize}
{\tiny \item \tiny For both codes, cache misses increase significantly with dynamic or 
guided scheduling, suggesting an  explanation for the performance loss.}
\end{itemize}
} 
\comments{ {\footnotesize Cache miss calcs in thesis write-up  } }
\end{frame} 

\begin{frame}
\frametitle{Calculations for Cache Misses for NASLU} 
{\small To verify that cache misses are in fact impacting performance 
(and is in fact the cause of the increase in data movement time), we show calculations below. }\\
{\tiny L2 cache miss latency is 100 - 300 cycles (with a 2.66 Ghz processor, this is 4 - 8 ns). }\\
{\tiny L3 cache miss latency is 100 ns. }\\
\begin{figure}[ht]
\label{fig:cacheMisses-cab-forCalcs}
\begin{center}
\includegraphics[scale=0.20]{../plots/L2-cacheMisses-NASLU-cab}\includegraphics[scale=0.20]{../plots/dmTime-NASLU-cab}
\end{center}
\end{figure} 

\begin{itemize}
\tiny \item \tiny We need to account for 485.52 - 83.53 - 24.65 = 
376.34 seconds of time spent in dyn sched. Note that we don't need the
execution time breakdown here, since there is low idle time for this code. 
\item \tiny Cache miss time for L2 is $\frac{200}{2.66 \times 10^9}
  \cdot (3845- 2257 \times 10^6) = 65.04$ s. 
\item \tiny 376.34 - 65.04 = 311.44 seconds to account for.
\item \tiny Cache miss time for L3 is $(100 \cdot 10^-9) \cdot (1676
  - 631) \times 10^6 =  97.82 $ s. 
\item \tiny Still 311.44 - 97.64 = 214.20 seconds to account for. 
  The cost of L3 cache miss may be much higher than 100 ns,
  because access to main memory by multiple threads requires threads to wait in memory
  queue. In the worst case, i.e., when all threads access main memory simultaneously, 
  the latency to memory is factor of $(16+(16-mem\_queue\_depth)\times
  100$ ns of the original latency, where mem\_queue\_depth=12 on this
  machine. 
  %We didn't take
  %into account the execution time for memory buffers for , which can
  %make 
\item \tiny Additionally, to verify that memory bandwidth was the
  factor, we ran STREAM on one core and STREAM running all cores, and
  recorded the bandwidth for both runs. 
\item \tiny Memory bw of STREAM triad on one core is 14837.24 MB/s, but on 16
  cores it is only 103669.71 MB/s. The effective mem. latency
  increases by factor of 2.4x, giving 238.4 s instead of 97.64 s for
  Cache miss time for L3. 
\item \tiny The remainder of time is now 311 - 238.34 = 80 s. This may
  due to L2 being higher (also, note that the L2 latency was reported at
  1 core), or by inefficiencies in parallelization. 
\end{itemize} 
\end{frame}

\comments{
\begin{frame}
\frametitle{Calculations for Cache Misses for bh}
Similarly, for n-body: 
\begin{figure}[ht]
\label{fig:cacheMisses-cab-forCalcs}
\begin{center}
\includegraphics[scale=0.30]{../plots/L2-cacheMisses-nbody-cab}\\
\includegraphics[scale=0.43]{../plots/dmTime-nbody-cab}
\end{center}
\end{figure}
\end{frame}
}
%TODO: consider new title 
% possible Question from audience: why is guided different from mixed
% static/dynamic?  Answer: because it doesn't preserve outer iteration
% data locality, and because on the first chunks, there is dqueue
% overhead. 
% Also, the sizes of the tasks can be controlled in guided, but can't
% handle staggering, i.e., spatial locality. 

\begin{frame}
\frametitle{Thesis Objective}
\underline{\textbf{Thesis Objective:}} Design a set of new scheduling strategies that handles all three causes of the problem, i.e., thread idle
time, data movement, and synchronization overhead simultaneously for
many applications and platforms, in the context of bulk-synchronous and loosely synchronous
MPI applications. 
% all at once, which  OpenMP static scheduling and OpenMP dynamic scheduling. 
\end{frame}
%TODO: change the below to the decomp 

\begin{frame}
\frametitle{Key Idea of Solution}
\visible<1->{\small Can we combine a static and dynamic scheduling scheme, which
  simultaneously reduces scheduler overhead and load imbalance
  (respectively) in an intelligent manner?}\\ 
\visible<1->{
\begin{columns}
\column{0.9\textwidth}
In spite of noise, most of the time, all cores are available to
the application. 
\begin{enumerate}
  \small \item \small Allocate a fixed fraction of
  loop iterations statically. (Then do remaining loop
  iterations dynamically). 
\item \small Ratio of static loop iterations to all loop iterations is the \textit{static fraction}.
\item \small \textit{How do we select the static fraction?}  
\end{enumerate}
\end{columns}
}
\end{frame}

%Note:  add in the cost model .
%TODO: check that we brought in ULT load imbalance.

\begin{frame}
\frametitle{Overview of Scheduling Strategies}
\begin{columns}
\column{0.95\textwidth}
       {\tiny \underline{\textbf{Available OpenMP Schedulers}}} \\
       \begin{itemize}
         \tiny \item \tiny \textcolor{black}{OpenMP static scheduling (\textit{static} or \textit{OpenMP-static})}\\ 
       \item \tiny \textcolor{black}{OpenMP dynamic scheduling (\textit{dynamic} or \textit{OpenMP-dynamic})}\\
       \item \tiny  \textcolor{black} {OpenMP guided scheduling(\textit{guided} or \textit{OpenMP-guided})}\\
       \end{itemize}
           {\tiny  \underline{\textbf{Scheduling Strategy Optimizations}}}\\
           \begin{itemize}
             \tiny \item \tiny \textcolor{black}{mixed static/dynamic scheduling(\textit{half})}
           \item \tiny \textcolor{black}{hybrid static/dynamic scheduling(\textit{besf})}
           \item \tiny \textcolor{black} {lightweight scheduling \textit{ligh}, which includes locality-aware scheduling(\textit{loca}) and skewed workloads(\textit{skew})}
           \item \tiny \textcolor{black} {weighted locality-sensitive scheduling (\textit{weig}, \textit{uWsched} or
             \textit{uWldB}), which consists of hybrid static/dynamic
             scheduling and weighted allocation (\textit{wSched} or \textit{wldB})}.
           \item \tiny \textcolor{black} {model-guided Hybrid
             Static/Dynamic Scheduling (\textit{Static-Hybrid}), and
             slack-conscious hybrid scheduling
             (\textit{callsite\_fd})}. Slack-conscious
             hybrid static/dynamic scheduling is referred to as Adaptive Scheduling in the paper.  
           \item \tiny \textcolor{black} {staggered hybrid static/dynamic scheduling (\textit{sds} or \textit{vSched})}
           \item \tiny \textcolor{black}{{Example combination of schedulers (\textit{allStrat} or \textit{full}) }}
           \end{itemize}
               {\tiny \underline{\textbf {Methodology for Tuning Scheduler Parameters}}}
        \begin{itemize}
          \tiny \item \tiny \textcolor{black} {experimental tuning}
        \item \tiny \textcolor{black} {model-guided optimization} %TODO: change to pruning search space 
        \item \tiny \textcolor{black} {runtime adjustment} %TODO: add within-node runtime adjustment and across node adjustment 
        \end{itemize}
\end{columns}
\end{frame}

\begin{frame}
\frametitle{Hybrid Static/Dynamic Scheduling and Lightweight Scheduling} 
\comments{
\column{0.5\columnwidth} 
\begin{enumerate}
\item \textbf{Introduction:}
\begin{itemize}
\tiny \item \tiny Show basic scheduling scheme and provide explanation of why it is beneficial. 
\item \tiny Explain inefficiency of load balanced code due to noise, with just one node. 
\item \tiny Show additional ways to provide low-overhead dynamic scheduling. 
\item \tiny Intuitive explanation of amplification problem, and benefits of our scheduling strategy to scale. 
\end{itemize} 
\item \textbf{Technique Overview}:
  \begin{itemize}
  \tiny \item \tiny \textbf{Scheduling Strategies}:Mixed static/dynamic, Hybrid Static/Dynamic Scheduling,Lightweight scheduling (which includes locality-aware scheduling and skewed workloads)
  \small \item \small \textbf{Tuning Methodology}: Experimental Tuning  
\end{itemize}
\item \textbf{Architectures:}
\begin{itemize} 
\tiny \item \tiny \textbf{cab}: Intel Xeon Cluster at LLNL having 16 cores per node, CHAOS operating system, mpich2, openMP 3.4. 
\item \tiny \textbf{bp}: IBM Power5+ Cluster at Univ. of Illinois having 16 cores per node, IBM AIX Operating System, ibmmpi, lomp. 
\end{itemize}
\item \textbf{Applications:}
\begin{itemize}
\tiny \item \tiny NASLU: 
\item \tiny nbody: 
\item \tiny 3D stencil: 
\end{itemize}
\end{enumerate}
}

\begin{columns}[T]
  \column{0.5\textwidth}
  \textbf{Introduction:}\\
    \begin{itemize} 
      \tiny \item \tiny Show basic scheduling scheme and provide explanation of why it is beneficial. 
    \item \tiny Explain inefficiency of load balanced code due to noise, with just one node. 
    \item \tiny Show additional ways to provide low-overhead dynamic scheduling. 
    \item \tiny Intuitive explanation of amplification problem, and benefits of our scheduling strategy to scale. 
    \end{itemize}
    \vrule{10in}{1pt}
    %\vrule width 10in height 1pt
    \column{0.5\textwidth}  
           {\tiny  \underline{\textbf{Scheduling Strategy Optimizations:}}}\\
           \begin{itemize}
             \tiny \item \tiny mixed static/dynamic $\rightarrow$ hybrid static/dynamic  (base). 
           \item \tiny Lightweight scheduling (which includes locality-aware scheduling and skewed workloads). 
           \item \tiny \textcolor{white} {weighted locality-sensitive scheduling}
           \item \tiny \textcolor{white} {slack-conscious scheduling}
           \item \tiny \textcolor{white} {staggered hybrid static/dynamic scheduling}
           \end{itemize}
               {\tiny \underline{\textbf { Methodology for Tuning Scheduler Parameters:}}}\\
               \begin{itemize}
                \tiny \item \tiny experimental tuning
               \item \tiny \textcolor{white} {model-guided optimization} %TODO: change to pruning search space 
               \item \tiny \textcolor{white} {runtime adjustment}  %TODO: add within-node runtime adjustment and across node adjustment 
               \end{itemize}
\end{columns}  
\end{frame} 

\begin{frame}
\frametitle{Static and Dynamic Scheduling}
\includegraphics[scale=0.30]{../images/legend-dynamic} \\
\begin{columns}
  \column{0.5\columnwidth}
  \lstinputlisting{../listings/threadedCompRegion-static.c}
  \column{0.5\columnwidth}
  \begin{center}
    \includegraphics[scale=0.35]{../images/threadedCompRegion-static} \\
  \end{center} 
\end{columns}
\begin{columns}
\column{0.5\columnwidth}
\lstinputlisting{../listings/threadedCompRegion-dynamic.c}
\column{0.5\columnwidth}
  \begin{center}
    \includegraphics[scale=0.35]{../images/threadedCompRegion-dynamic}\\
  \end{center}
\end{columns}
%\end{block}

\begin{columns}
\column{0.5\columnwidth} 
- 
\column{0.5\columnwidth}
  \begin{center}
    \includegraphics[scale=0.35]{../images/emptySchedImage}
  \end{center}
\end{columns}
\end{frame}

\begin{frame}[label=mixedstatdyn]
\frametitle{Mixed Static/Dynamic Scheduling}
%\begin{block}
\begin{columns}
  \column{0.5\columnwidth}
  \lstinputlisting{../listings/threadedCompRegion-static.c}
  \column{0.5\columnwidth}
  \includegraphics[scale=0.15]{../images/legend-dynamic}\\
  \begin{center}
    \includegraphics[scale=0.35]{../images/threadedCompRegion-static} \\
  \end{center}
\end{columns}
%\end{block} 
%\begin{block}
\begin{columns}
\column{0.5\columnwidth}
\lstinputlisting{../listings/threadedCompRegion-dynamic.c}
\column{0.5\columnwidth}
  \begin{center}
    \includegraphics[scale=0.35]{../images/threadedCompRegion-dynamic}\\
  \end{center}
\end{columns}
%\end{block}
\begin{columns}
\column{0.5\columnwidth}
\lstinputlisting{../listings/threadedCompRegion-mixed.c}
\column{0.5\columnwidth}
\begin{center}
\includegraphics[scale=0.35]{../images/threadedCompRegion-mixed}
\end{center}
\end{columns}
\end{frame}

\begin{frame}[label=halftimings]
\frametitle{Naive Mixed Static/Dynamic Scheduling}
{\tiny Setup and Methodology is the same as that used in
  \hyperlink{basictimings}{\beamerbutton{here}}.} 
%Used NASLU OpenMP benchmark code and Barnes-Hut pthread application
%code. Ran on one node of LLNL cab cluster, which is an Intel Xeon
%16-core with CHAOS operating system.\comments{using gcc optimization
%level 'O3'}. \comments{having OpenMP3 and mvapich3}} 
\begin{figure}[t]
\begin{center}
\includegraphics[scale=0.34]{../plots/dmTime-nbody-cab-withsd}
\includegraphics[scale=0.34]{../plots/dmTime-NASLU-cab-withsd}\\
\end{center}
\label{fig:dmTimes-NASLU-cab}
\end{figure}
\begin{center}
{\small Set static fraction to 0.5: In \textit{half}, threads do the
  first half of their iterations statically. See rightmost bar. }
\end{center}
\begin{itemize}
\tiny \item {\tiny We expect \textit{half} to cut the dynamic overhead in half.}
\item {\tiny The \textit{half} scheduling provides 10.1\% gain over
  OpenMP guided for the nbody code, due to preservation of locality
  across invocations of threaded computation regions and thus less
  data movement across invocations of the threaded computation
  region.}
\item {\tiny For the NASLU code, the \textit{half} scheduling still has high overhead.}
\end{itemize}
\end{frame}

\begin{frame}[label=dmTimes]
\frametitle{Data Movement with Naive Mixed Static/Dynamic Scheduling}
\visible<1->{
\begin{figure}[t]
\label{fig:cacheMisses-cab-withsd}
\begin{center}
\includegraphics[scale=0.30]{../plots/L2-cacheMisses-nbody-cab-withsd}
\includegraphics[scale=0.30]{../plots/L2-cacheMisses-NASLU-cab-withsd}\\
\includegraphics[scale=0.30]{../plots/L3-cacheMisses-nbody-cab-withsd}
\includegraphics[scale=0.30]{../plots/L3-cacheMisses-NASLU-cab-withsd}
\end{center}
%\caption{\label{fig:cacheMisses-cab-withsd} L2 cache misses (top) and L3 cache misses (bottom).}
\end{figure}
\begin{center}
{\small L2 cache misses shown in the top graphs; L3 in the
bottom. \comments{The \textit{half} is added on the rightmost bar.}
% The data shown is based on 100 trials run on a single node of the
% respective machines. Data movement time is obtained by subtracting
% the sum of idle time, dequeue time and computation time, from total
% execution time.
}
\end{center}
}
\visible<1->{\begin{itemize} 
{\tiny \item \tiny With \textit{half}, cache misses decrease with respect to OpenMP guided, for both nbody and NASLU codes.}
\end{itemize}}
\end{frame}

\begin{frame}[label=noPreset]
\frametitle{Problem: A Preset Static Fraction Doesn't Work Effectively for all Application-architecture Pairs}
\begin{itemize}
\item The above shows that the basic static/dynamic scheduling 
improves performance only for nbody on cab. So, we make the strategy more flexible. 
\end{itemize}
\begin{itemize}
\item We expose the static fraction to the application programmer.  
\item What is right blend of static and dynamic scheduling to use? 
\item The application programmer tunes the blend of static and dynamic scheduling.
\item We refer to this as \textit{Hybrid Static/Dynamic Scheduling}. 
\end{itemize}
\end{frame}

\begin{frame}[label=hybridstatdyn]
\frametitle{Hybrid Static/Dynamic Scheduling}
\begin{columns}
  \column{0.5\columnwidth}
  \lstinputlisting{../listings/threadedCompRegion-static.c}
  \column{0.5\columnwidth}
  \includegraphics[scale=0.15]{../images/legend-dynamic}\\
  \begin{center}
    \includegraphics[scale=0.35]{../images/threadedCompRegion-static} \\
  \end{center} 
\end{columns}
%\end{block} 
%\begin{block}
\begin{columns}
\column{0.5\columnwidth}
\lstinputlisting{../listings/threadedCompRegion-dynamic.c}
\column{0.5\columnwidth}
  \begin{center}
    \includegraphics[scale=0.35]{../images/threadedCompRegion-dynamic}\\
  \end{center}
\end{columns}
%\end{block} 

\begin{columns}
\column{0.5\columnwidth}
%TODO: fix code  to be hybrid sched
\lstinputlisting{../listings/threadedCompRegion-hybrid.c}
\column{0.5\columnwidth}
\begin{center}
\includegraphics[scale=0.35]{../images/threadedCompRegion-hybrid}
\end{center}
\end{columns}
\end{frame}

%TODO: think of a better name for title. 
\begin{frame}[label=expTuningSF]
\frametitle{Experimentally Tuning the Static Fraction}
%TODO: explain why increments of 0.01 
%TODO: explain number of trials run 
{\tiny We vary the static fraction environment variable in increments of 0.01. \comments{(We do 0.01 to get enough data points, but more refinement might give less improvement. )} } 
\begin{figure}[ht]
\begin{center} 
\includegraphics[scale=0.19]{../plots/expTuningSF-nbody-cab-WallTime}
\includegraphics[scale=0.19]{../plots/expTuningSF-NASLU-cab-WallTime}\\
\end{center} 
\begin{center}
{\small Performance for different static fractions when using hybrid static/dynamic scheduling strategy for nbody and NASLU. }
\end{center}
\end{figure}

\begin{itemize}
\tiny \item \tiny For n-body, fully dynamic is better than fully static, but static fraction of 49\% gives even better performance. 
\item \tiny For NASLU, the fully static is much better than fully
  dynamic. However, static fraction of 96\% is the best (although
  compared with the 100\% dynamic, the difference between ``best'' and
  fully static may appear small, it is still a significant
  improvement).  
\item \tiny Note that the best static fraction varies across
  application. 
%Comment: make note about half
\end{itemize}
\end{frame} 

\begin{frame}[label=halftimings]
\frametitle{Hybrid Static/Dynamic Scheduling}
\begin{figure}[t]
\begin{center}
\includegraphics[scale=0.34]{../plots/dmTime-nbody-cab-withsdandbesf}
\includegraphics[scale=0.34]{../plots/dmTime-NASLU-cab-withsdandbesf}\\
\end{center}
\label{fig:dmTimes-NASLU-cab}
\end{figure}
\begin{center}
{\small In \textit{besf}, we use the best static fraction. See rightmost bar.}
\end{center}
%TODO: fix this so it doesn't duplicate

\visible<1->{
\begin{itemize}
\tiny \item \tiny When using \textit{besf} scheduling, cache misses decrease w.r.t. \textit{half}, for both nbody and NASLU codes. This is
because we tune to balance the tradeoff between load balance and locality, to minimize the costs in each of the three challenges, simultaneously.
\item \tiny Paper [statdyn1] studies this idea in more depth.
\end{itemize}}
\end{frame}

\begin{frame}
\frametitle{Data Movement with Hybrid Static/Dynamic Scheduling}
\visible<1->{
\begin{figure}[t]
\label{fig:cacheMisses-cab}
\begin{center}
\includegraphics[scale=0.30]{../plots/L2-cacheMisses-nbody-cab-withbesf}
\includegraphics[scale=0.30]{../plots/L2-cacheMisses-NASLU-cab-withbesf}\\
\includegraphics[scale=0.30]{../plots/L3-cacheMisses-nbody-cab-withbesf}
\includegraphics[scale=0.30]{../plots/L3-cacheMisses-NASLU-cab-withbesf}
\end{center}
%\caption{\label{fig:cacheMisses-nbody-cab} L2 cache misses (top) and L3 cache misses (bottom).}
\caption{\small Cache misses for \textit{besf}, added on the rightmost bar. }
\end{figure}
}
%\begin{center}
%The data shown is based on a 100 trials run on a single node of the respective machines. Data movement time is obtained by subtracting the sum of idle time, dequeue time and computation time, from total execution time.
%} 
%\end{center} 
\visible<1->{
\begin{itemize} 
\tiny \item \tiny When using \textit{besf} scheduling, cache misses 
decrease w.r.t. \textit{half}, for both nbody and NASLU codes. This is
because we tune to balance the tradeoff between load balance and
locality, to minimize the costs in each of the three challenges,
simultaneously. 
\item \tiny Paper [statdyn1] \comments{\footcite{dynwork}} studies
  this idea in more depth. 
\end{itemize}}
\end{frame}

\begin{frame}[label=3Dmesh]
\frametitle{3D Regular Mesh process+thread Domain Decomposition}
\begin{columns}
  \column{0.35\textwidth}
  \begin{center}
    \includegraphics[scale=0.35]{../images/appTimestep-noslack}
 \end{center}
\column{0.65\textwidth}

\begin{figure}[ht]
  \label{fig:3DStencil-domainDecomp-stat}
  \begin{center}
    \includegraphics[scale=0.18]{../images/3DStencil-domainDecomp-stat}
  \end{center}
  \caption{\label{fig:3DStencil-domainDecomp-stat} {\small 3D Regular Mesh domain
  decomposition across processes, along with thread partitioning
  of work within each process.}}
\end{figure}
\end{columns}
\end{frame}


\begin{frame}[label=perfVar]
\frametitle{Performance Variation and Absolute Performance for Static Scheduling}
{\tiny \underline{\textbf{Setup and Methodology:}} Run on IBM Power5+ cluster at UIUC which has 16 cores per node\comments{32MB L1 cache and 128MB L2 cache} and AIX operating system.}
\begin{figure}[h!]
\label{fig:stat-histo}
\begin{center}
\includegraphics[scale=0.29]{../images/IterTimesHisto-outliers-static}\\
%\includegraphics[scale=0.35]{../images/IterTimingsHisto-dynamic}
\end{center}
\caption{\label{fig:stat-histo} \small Iteration timing distribution across 1000 timesteps of the statically scheduled\comments{\hyperlink{3Dmesh}{\beamerbutton{3D reg. mesh}}} 3D regular mesh computation.}
\end{figure}
\begin{center}
{\small $\rightarrow$ \textcolor{gray}{Bi-modal distribution for static scheduling.} \comments{Uni-modal distribution for dynamic scheduling, but increased spread and higher mean for dynamic scheduling.}}
\end{center}
\end{frame}

\begin{frame}[label=lightweightDynDefs]
\frametitle{Lightweight Dynamic Scheduling}
\textbf{Tasklet}: A tasklet specifies a set of iterations and records the core that it was executed on in the preceding iteration. 
\begin{enumerate}
\item \textbf{Skewed Workload}: The distribution of tasklets in the shared queue, from beginning to end, is 16 tasklets of size 16, 32 tasklet of size 8, and 64 tasklets of size 4. 
\item \textbf{Locality-aware Scheduling}: On each iteration, a thread tries to search for the tasklet that it executed during the preceding iteration. 
%\item \textbf{Work-stealing}: On each iteration, a thread tries to search for the tasklet that it executed during the preceding iteration. 
\end{enumerate}
\end{frame}

\begin{frame}[label=HybStatDynAppliedto3DMesh]
\frametitle{Hybrid Static/Dynamic Scheduling Applied to 3D Regular Mesh}
\begin{columns}
  \column{0.10\textwidth}

  \column{0.90\textwidth}
  \begin{figure}
    \label{fig:3DStencil-domainDecomp-sd}
    \begin{center}
      \includegraphics[scale=0.18]{../images/3DStencil-domainDecomp-sd}
    \end{center}
    \caption{\label{fig:3DStencil-domainDecomp-hybrid} \small 3D Regular Mesh domain decomposition with Hybrid Static/Dynamic Scheduling Strategy Applied.}
  \end{figure}
\end{columns}
\end{frame}

\begin{frame}
\frametitle{Hybrid Static/Dynamic Scheduling with Lightweight Dynamic Scheduling in Dynamic Section}
{\tiny Setup and methodology the same as \hyperlink{perfVar}{\textcolor{blue}{here}}.} 
\begin{figure}[h!]
\label{fig:skewed2}
\begin{center}
\includegraphics[scale=0.28]{../images/IterTimesHisto-skewedWorkload2}
\end{center}
\caption{\label{fig:skewed2} Iteration timing histograms for 100\% dynamic scheduling, 100\% dynamic scheduling with locality, and 50\% dynamic scheduling.}
\end{figure}
\begin{itemize}
\tiny \item \tiny 50\% dynamic provides better absolute performance,
as well as less performance variation than 100\% dynamic scheduling
without locality and 100\% dynamic scheduling with locality. 
% \tiny \item \tiny Each percent dynamic has a minima for the number of
%  tasklets in the queue, and skewed does well for larger percent
%  dynamic. 
%\item \tiny The best performance is with 4 planes per planes per
%tasklet, so we use this for the subsequent experimentation. 
%As we vary the number of tasklets in the queue, ther
\end{itemize}
\end{frame}

\begin{frame}[label=noiseAmpExample]
\frametitle{Amplification Problem in the Context of Noise}
\begin{itemize}
\item \small On one node running 1000 iterations of length 10 ms. 
\begin{itemize}
\tiny \item \tiny Assume noise (PI) is of length 5 ms, and it occurs approximately every second.
\item \tiny Execution time without PI 10 ms * 1000 iterations = 10 seconds. 
\item \tiny With PI: 10 out of 1000 iterations will be affected (that is delayed by 5 ms) .  So, the total delay is 50 ms.
\item \tiny Execution time is 10.05 seconds, i.e., 0.5\% delay. 
\end{itemize}
\item \small On two nodes (weak scaling):
\begin{itemize}
\item \tiny Probability that an iteration is delayed on one of the nodes is 1\%, same as above.
\item \tiny Probability that a given iteration is delayed on any processor is : $1.0 - (1 - 0.01)^2$ = $1.99$\% .
\item \tiny The delay is then $1000 \times 0.0199 \times 5$  = $99.5$ ms for a total execution time of $10.0995$ s. 
\end{itemize}
\item \small On p nodes :
\begin{itemize}
\item \tiny The delay is $1000 \times (1.0 - 0.99^p) \times 5$ ms
\item \tiny For 10 nodes: $1000 \times 0.0956 \times 5$ ms =  478 ms
\item \tiny For 100 nodes: $1000 \times 0.634 \times 5$ ms = 3170 ms
\item \tiny For 1000 nodes: $1000 \times 0.9999 \times 5$ = 4999.5 ms
\end{itemize}
\item \small That's a 50\% performance loss. 
\item \small As $p$ increases, the probability that one node experiences a delay on a given iteration starts approaching 100\%. 
\small \item \small Caveat for loosely synchronous: amplification will come at a larger number of nodes than for bulk-synchronous, though magnitude of performance degradation may be the same as bulk-synchronous code. 
\end{itemize}
\end{frame}

\begin{frame}[label=noiseampformula]
\frametitle{Amplification Problem in Context of Noise} 
\begin{itemize}
\small \item \small Let $T_p$ be iteration time on one node with $p$ cores, with no noise. We assume weak scaling. 
\item \small  Let $\delta$ be the noise duration.
\item \small Let $\phi$ be the expected number of noise events per second, on one node. \item \small We assume global synchronization with zero overhead, e.g., allreduce.
\end{itemize}

\textbf{----------------------------------------------------------}
\begin{itemize}
\small \item \small Probability of noise happening in one iteration on one node = $\phi \times T_p$. 
\item \small Probability that there is no delay on any nodes while running on k nodes is $(1 - \phi \times T_p)^k$. 
\item \small Probability that \textit{some} node experiences a delay in a given iteration is $1 - (1- \phi \times T_p)^k$. 
\item \small So, expected execution time of an iteration on k nodes is: $E[T_k]  >=  T_p + \delta \times  [1 - (1- \phi \times T_p)^k]$. 
\end{itemize}
%( Caveat of slack) 
\end{frame}

\begin{frame}
\frametitle{Improvement of Scalability for 3D Stencil}
\begin{figure}
\includegraphics[scale=0.34]{../images/stencil-scaling}
\caption{Scaling 3D Regular Mesh for different schedulers.}
\begin{itemize}
\tiny {\item \tiny Mixed static/dynamic scheduling results in improved 
scalability over both static scheduling and dynamic scheduling.}
\end{itemize}
\end{figure}
\end{frame}

%TODO: show graph for fully dynamic scheduling.
%\vivek{fix layout}
\begin{frame}
\frametitle{Impact on Scaling}
\begin{figure}[ht!]
\label{fig:scaling-histograms}
\begin{center}
\includegraphics[scale=0.20]{../images/scaling-histograms}
\end{center}
%TODO: word the below better
\caption{\label{fig:scaling-histograms} Iteration timing histograms 
for static scheduling~(left) for mixed static/dynamic scheduling~(right) as we increase the 
number of nodes.}
\end{figure}

\begin{enumerate}
\tiny \item \tiny Distribution of two equal peaks goes to one small
peak and one large peak. 
% Performance variation increases with increasing numbers of
% nodes
\item \tiny The mean of the distribution for mixed 
static/dynamic scheduling stays consistent 
as we increase the number of nodes. 
\item \tiny The mean of the distribution at 64 nodes is better 
  with static/dynamic scheduling than that of static
  scheduling. %TODO: think of a better point here. 
\end{enumerate}
\end{frame}

\begin{frame}
\frametitle{Improvement Over an Already Optimized Dense Matrix Factorization using Hybrid Static/Dynamic Scheduling}
%\frametitle{For Problem 1: Does our scheduling approach perform competitively or better than other industry wide schedulers?}
\begin{columns}
\column{0.5\textwidth}
\begin{figure}
\includegraphics[scale=0.22]{../plots/calu-fastNUMA4-perf}
\caption{\tiny Comparison of hybrid static/dynamic scheduled CALU
  to statically scheduled CALU (first version) as well as 
  other industry standards such as PLASMA and MKL.\comments{showing significant
  performance improvements over the baseline (statically scheduled)
  CALU, particularly with large matrix sizes.}}
\visible<1->{\tiny $\rightarrow$ We get significant
  performance improvements over the baseline (statically scheduled)
  CALU, particularly with large matrix sizes. Our scheduler is 30\% 
  better than PLASMA library's implementation of the same 
  numerical algorithm, and 38\% better than Intel MKL library's 
  impementation of the same numerical algorithm.}
%\end{itemize}
%(Problem: PLASMA has inefficient work-stealing algorithm.) 
\end{figure}
\column{0.5\textwidth}
\begin{figure}
\includegraphics[scale=0.10]{../plots/calu-fastNUMA3-histogram}
\caption{\tiny Performance Variation for static CALU, dynamic CALU, and hybrid static/dynamic CALU.}
\end{figure}
{\tiny $\rightarrow$ Hybrid static/dynamic scheduled CALU reduces
  performance variability while maintaining absolute performance
  better than that of statically scheduled CALU and dynamically
  scheduled CALU, and this is consistent with the performance
  variability results for 3D stencil.} 
\end{columns}
\end{frame}

%TODO: update this
\begin{frame}
\frametitle{Problems with the Basic Lightweight Scheduling Approach}
%\textbf{Two problems with the above basic approach:} 
\begin{enumerate}
\small \item \small Persistent load imbalances not handled. How do we handle both
persistent and transient imbalances together? 
\item \small Experimental tuning: can we reduce the search by using a
model?
\item \small Across-node load imbalance not taken into account in 
scheduling strategy. Can we use runtime-guided adjustments (per-node static
fraction based on slack) to improve performance?
\item \small Our scheduler does not consider spatial locality. How do
  we optimize the mixed static/dynamic scheduling scheme so that we
  improve spatial locality and temporal locality?
\end{enumerate}
\end{frame}

\begin{frame}
\frametitle{Weighted Locality-sensitive Scheduling}
\begin{columns}[T]
% \column{0.50\textwidth}
  \begin{column}{0.5\textwidth}
    {\small \underline{\textbf{Introduction}}} \\
    \begin{itemize}
      \tiny \item \tiny Noise not completely transient; 
      can be a mixture of persistent core noise and transient noise, as observed
      in Jaguar (see next slide). 
    \item \tiny We modify our lightweight scheduling scheme to handle the
      mixture of persistent core noise and transient node-level OS noise. 
    \item \tiny For architectures with 
      the mentioned mixture of noise events, we get noted performance
      gains of 10.4\% over our basic lightweight scheduling scheme. 
    \item \tiny The approach described can be generalized to work for any
      type of mixture of persistent and transient load imbalance, not just
      mixtures of persistent and transient noise. 
    \end{itemize}
  \end{column}
  % \column{0.50\textwidth}
  \begin{column}{0.5\textwidth}
    {\tiny  \underline{\textbf{Scheduling Strategy Optimizations}}}\\
    \begin{itemize}
      \tiny \item \tiny \textcolor{gray}{ mixed static/dynamic (\textit{half}) $\rightarrow$ hybrid static/dynamic \textit{besf} (base)}   
    \item \tiny \textcolor{gray}{ lightweight scheduling (\textit{ligh}) (which includes locality-aware scheduling (\textit{loca}) and skewed workloads \textit{skew})}
    \item \tiny \textcolor{black} {weighted locality-sensitive scheduling (\textit{weig})}
    \item \tiny \textcolor{white} {slack-conscious scheduling: \textit{slac} :  model-guided\textit{Static-Hybrid},   slack-conscious(\textit{callsite\_fd})}
    \item \tiny \textcolor{white} {staggered hybrid static/dynamic scheduling: \textit{ssds}}
    \end{itemize}
        {\tiny \underline{\textbf {Methodology for Tuning Scheduler Parameters}}}
        \begin{itemize}
          \tiny \item \tiny \textcolor{black} {experimental tuning}
        \item \tiny \textcolor{white} {model-guided optimization} %TODO: change to pruning search space 
        \item \tiny \textcolor{black} {runtime adjustment} %TODO: add within-node runtime adjustment and across node adjustment 
        \end{itemize}
  \end{column}
\end{columns}
\end{frame} 

%TODO: figure out whether we need to use coupling. 
%TODO: word the below better 
\begin{frame}
  \frametitle{Persistent Imbalances (Noise-induced) Not Handled} 
  \begin{figure}
    %\includegraphics[scale=0.44]{../images/noise_pes_jaguar}
    \subfloat[\tiny Jaguar quasi-persistence in noise.]{\includegraphics[scale=0.41]{../images/noise_pes_jaguar}}
    \subfloat[\tiny Ranger quasi-persistence in noise.]{ \includegraphics[scale=0.41]{../images/noise_pes_ranger}}
    \caption{\small Time spent by each core in a 1200 core run, for a load balanced code.}
  \end{figure}
  \begin{itemize}
    \tiny \item \tiny The two top bands of indicate slow cores, while
    bottom bands show fast cores (particularly for Jaguar).
  \item \tiny Across all nodes, the subset of cores being fast and
    subset slow is almost the same. 
  \item \tiny We refer to the above phenomenon as persistent core noise,
    or  noise-induced persistent imbalances. 
  %(and note that this could be generalized as a fine-grained application-induced persistent load imbalance across cores).%TODO: explain more on why  
  \end{itemize}
\end{frame}

%TODO: check the question presented in the title of the slide.
\begin{frame}
\frametitle{Blending Weighted Factoring and Mixed Static/Dynamic Scheduling to Handle Persistent Core Noise}
%TODO: add regular diagram for weighted decomp. 
{\small Weighted Factoring: assign different number of iterations to different cores based on history.}
{\small We combine weighted with hybrid static/dynamic scheduling, and
  then tune the static fraction.}\\
{\small Earlier, the number of iterations allocated to thread i was
 $f_s \cdot \frac{N}{p}$}
{\small With weighted, it is: $w_i \cdot \frac{f_s \cdot N}{p}$,
where $w_i$ is calculated by measuring the time spent in the static
portion by different threads, $s_i$, as: $w_i = \frac{avg\{s_i\}}{s_i}$}.
% begining iteration for thread i, b_i = b_{i - 1} +
% w_i*\frac{f_s*N}{p} , with b_0 = 0 
\begin{columns}
  \column{0.5\textwidth}
         {\small Weighted Locality-sensitive Scheduling Strategy Applied to 3D Reg. Mesh.\\}
         \column{0.5\textwidth}
         \begin{center}
           \fbox{\includegraphics[scale=0.2]{../images/weighted_decomp}}\\
         \end{center}
\end{columns}
\end{frame}

\begin{frame}
\frametitle{Weighted Locality-Sensitive Scheduling Results}
\begin{itemize}
\tiny \item \tiny Ranger and Jaguar both experience noise; Ranger is
somewhat noisier.
\item \tiny Only Jaguar has core differentation; some cores experience 
more high-frequency noise than the others.
%\item \tiny 
\end{itemize}

\begin{figure}
\label{fig:uWSched-stencil}
\begin{center}
\includegraphics[scale=0.39]{../plots/timings_jaguar}
\includegraphics[scale=0.39]{../plots/timings_ranger}\\
\end{center}
\caption{\label{fig:uWSched-stencil} 3D Stencil scaling with weighted
  locality-sensitive scheduling for Ranger~(left) and Jaguar~(right).}
\end{figure}

\begin{enumerate}
\tiny \item \tiny Weighted micro-scheduling is the best on Ranger,
although micro-scheduling is almost as good (dynamic load balancing
most important). 
\item \tiny Jaguar: Weighted is better on fewer than 64 nodes and
  micro scheduling is better beyond that. 
\item \tiny Jaguar: Weighted micro-scheduling is better than both of these. 
\end{enumerate}
{\tiny $\rightarrow$ Combining schedulers has no disadvantage.}
%TODO: figure out the avg/max for load imbalance to calculate static fraction. 
%TODO: max - avg . avg / max ; 
\end{frame}

\begin{frame}
\frametitle{Slack-conscious Hybrid Static/Dynamic Scheduling}
\begin{columns}[T]
  \column{0.5\textwidth}
         {\small  \underline{\textbf{Introduction:}}} \\
         \begin{itemize}
           \tiny \item \tiny We first discuss limitations of
           experimentally tuning the lightweight scheduling scheme,
           including the problem of slack. 
         \item \tiny With this, we explain the two techniques of this
           chapter: model-guided lightweight scheduling and
           slack-conscious scheduling technique. 
         \item \tiny We evaluate our slack prediction mechanisms,
           showing scaling results for our model-guided lightweight
           scheduling. 
         \item \tiny We show why our slack-conscious runtime
           adjustment is essential and beneficial at large-scale, by
           showing speedup of slack-conscious scheduling over best
           static fraction (\textit{besf}) scheduling strategy. 
         \end{itemize}
         %\vrule[0.5em]{1pt}{3.5in]
\vrule{} 
         \column{0.5\textwidth}
                {\tiny  \underline{\textbf{Scheduling Strategy Optimizations}}} \\
                \begin{itemize}
                  \tiny \item \tiny \textcolor{gray} {mixed static/dynamic $\rightarrow$ hybrid static/dynamic  (base)}
                \item \tiny  \textcolor{gray} {lightweight scheduling (which includes locality-aware scheduling and skewed workloads)}
                \item \tiny \textcolor{gray} {weighted locality-sensitive scheduling}
                \item \tiny \textcolor{black} {slack-conscious scheduling}
                \item \tiny \textcolor{white} {staggered hybrid static/dynamic scheduling}
                \end{itemize}
                    {\tiny \underline{\textbf { Methodology for Tuning Scheduler Parameters}}}
                    \begin{itemize}
                      \tiny \item \tiny \textcolor{white} {experimental tuning}
                    \item \tiny \textcolor{black} {model-guided optimization} %TODO: change to pruning search space 
                    \item \tiny \textcolor{black} {runtime adjustment}  %TODO: add within-node runtime adjustment and across node adjustment 
                \end{itemize}
\end{columns}
\end{frame}

%TODO: get another graph - this may not show the results as well. 
%TODO: show a caption

\begin{frame}
\frametitle{Large and Complex Search Space}
%TODO: get new picture 
%Also, Could maybe show results for enumeration of search space, and that it's large here. 
%TODO: fix the below 
%Put in modeled graph with two endpoints, showing non-continuous search space. 
\begin{enumerate} 
\item Difficult to find exact static fraction for single node. 
\item Best static fraction changes per node because MPI collective operations involve ``slack''(explained more in depth on next slide).
\end{enumerate}
%\begin{figure}
%\includegraphics[scale=0.20]{../plots/vary-sf-for-diff-threads}
%\caption{\small Graph showing the performance of different static fractions
%  used. Here, we have tried different static fractions, 
%  in increments of 0.1. The minima here is difficult to attain.}
%\includegraphics[scale=0.28]{../plots/best-fs-vs-nodes}
%\end{figure}
\end{frame}

%TODO: add in other data sets.  
%TODO: add in max/avg. 

\comments{
\begin{frame}
\frametitle{Can We Use Theoretical Analysis to Prune the Large Search Space?}
% Model-guided optimization, but with experimental tuning after
% scheduler parameter from model is determined. 
\begin{center}
\includegraphics[scale=0.26]{../plots/app-scaling-strat-SNAP-fastNUMA2}
\includegraphics[scale=0.26]{../plots/app-scaling-strat-nbody-fastNUMA2}\\
\end{center}
\begin{center}
{\small Using a theoretical analysis to acheive the minima helps to provide 13-16\% performance gains for a particle simulation (nbody) and SNAP (reg. mesh), each tested on a 16-core non-NUMA architecture and 16-core NUMA architecture.}
\end{center}
\end{frame}
}

\begin{frame}
\frametitle{Different MPI processes Have Different Cost to Application Critical Path}
{\tiny \underline{\textbf{MPI Slack}} is the time by which a process
  can delay arrival at a synchronizing point without impacting application execution time.}
%
\begin{table}[tr]
\label{fig:collectiveSlack-useq}
  \begin{center}
    \begin{tabular}{ | c || c | c || c |}
      \hline
      \textbf{Collective} &      \textbf{max} & \textbf{avg ($\sigma$)} & \textbf{$\sigma_i$} \\ 
\hline
      \texttt{Allreduce} &             307    &    199 (.792) &            .102 \\ 
\hline
      \texttt{Alltoall}  &             280    &    149 (v.821) &            .176 \\ 
\hline
      \texttt{Barrier}  &              250    &    139 (.178)  &           .061 \\ 
\hline
      \texttt{Allgather} &             268    &    189 (.527) &            .115 \\
\hline
      \texttt{Reduce\_scatter} &       459    &    296  (.649)  &          .129 \\ 
\hline
    \end{tabular}
  \end{center}
  \caption{\label{fig:collectiveSlack-useq} {\small Different slack statistics (in $\mu$s) shown for experiments with different collectives at the end of an iteration. Additional evidence and impact of slack esp. at large scale is discussed in [slackPerf] and [adagio].}}
\end{table}

\begin{itemize}
\tiny \item \tiny Slack distribution is different for different collectives.
\item \tiny Slack varies across processes.
\item \tiny The distribution of slack on different processes evolves slowly for different applications. 
\end{itemize}
\end{frame}

\begin{frame}
\frametitle{Terminology and Performance Model}
\begin{table}[tr]\centering
\begin{tabular} {|c|p{.75\columnwidth}|} \hline
\multicolumn{2}{|c|}{\textit{Model outputs}} \\\hline
$f_d$       & {\tiny Fraction of work scheduled dynamically.} \\\hline
$f_d'$      & {\tiny Fraction of work scheduled dynamically,\newline
                      reduced for slack.} \\\hline
%
\hline
\multicolumn{2}{|c|}{\textit{Model inputs}} \\\hline
$q$                     & {\tiny Overhead of dequeueing single iteration.} \\\hline
$S_i$                   & {\tiny Slack duration on process $i$.} \\\hline
$\delta$        & {\tiny Expected noise duration.} \\\hline
$T_p$       & {\tiny Execution time on $p$ cores with fully static scheduling and no noise.} \\\hline
$N$         & {\tiny Total number of iterations across all threads.}\\\hline
% 
$t_1$       & {\tiny Duration of single iteration on a single thread, on one core.} \\\hline
$d$                     & {\tiny Duration of single dynamic iteration on a single thread.} \\\hline
%$\eta$          & Compute delay due to noise and queue overhead \\\hline
%$\alpha_i$      &  Delay added to the critical path by node $i$\\\hline
\end{tabular}
\caption{\small Overview of all model parameters. Note that we use
  the dynamic fraction $f_d$ instead of static fraction $f_s$ to simplify our
  theoretical analysis.\label{tab:parameters}}
\end{table}
\end{frame}

\begin{frame}[label=mghybridstatdyn]
\frametitle{Model-guided Hybrid Static/Dynamic Scheduling}
{\small Instead of tuning the dynamic fraction, we establish a model to
determine it based on noise duration $\delta$, and other model parameters. \\}
\begin{center}
  \fbox{\includegraphics[scale=0.46]{../images/hybSched-nonoisenoslack}} \\
\end{center}

{\small Considering the above model, along with our goal of avoiding load balancing until it is absolutely needed, 
the dynamic fraction applied to each MPI process is: \\}
\begin{center}
  \fbox{{$f_d = \frac{p\delta}{N\left(t_1 + q\right)}$}}\\
\end{center}
    {\small We refer to the strategy described above as Model-guided Hybrid
      Static/Dynamic Scheduling. This strategy is referred to as 
      \textit{Static-Hybrid} in the graphs.} 
\end{frame}

\begin{frame}
\frametitle{Considering Slack Factor for Hybrid Scheduling}
{\small With this, let us now consider the slack factor for each MPI process.} 

\begin{figure}
  \begin{center}
    \includegraphics[scale=0.46]{../images/slackConsciousSched} \\
  \end{center}
\end{figure}

\begin{itemize}
\tiny \item \tiny Slack allows us to reduce the dynamic fraction
further, because a small amount of delay due to noise will lead to no
delay in the completion of an MPI operation. For example, if you are
in a Reduction and waiting for data from your spanning tree
children. 
\item \tiny Note that the slack is different on different nodes,
  depending on the implementation of the collective operation. 
\item \tiny Therefore, the dynamic fraction will be different on
  different processors. Nodes with large slack can get away with 
small dynamic fraction without impacting application performance.
\end{itemize}
\end{frame}

\begin{frame}[label=slackfs]
\frametitle{Slack-conscious Hybrid Static/Dynamic Scheduling
  a.k.a. Adaptive Scheduling} 
\begin{figure}
  \begin{center}
    \includegraphics[scale=0.41]{../images/slackConsciousSched} \\
  \end{center}
\end{figure}
\begin{center}
\fbox{{$f_d = \frac{p\delta}{N\left(t_1 + q\right)}$}}
\fbox{ {$f_d' = f_d - \frac{S}{(p-1)\frac{Nt_1}{p} - Nq}$}}
\end{center}
\begin{center}
{\tiny Note that $f_d'$ and $S$ are local variables within each MPI process. }
\end{center}

%TODO: 
{\small The dynamic fraction $f_d$ of a particular node
 is reduced by a value proportionate to the slack on that node, 
to get the new $f_d'$.} \\

{\small To estimate the slack on each processor, we use ideas from
 Adagio (see next slide for impl). \\ We compared multiple methods,
 and chose the \textit{callsite\_fd} method.} \\

{\small Reducing dynamic fraction reduces overall overhead.}

\end{frame}

\comments{
\begin{frame} 
\frametitle{Slack Prediction Mechanism and Usage}  
\begin{enumerate}
\tiny \item \tiny On each process, retrieve that process's invocation of the
  last MPI collective, where the invocation of the last MPI collective is
retrieved through the callsite slack-prediction method (see paper for
details on implementation details of slack prediction).
 \item \tiny Given the identifier of the last the MPI collective call
invoked, estimate that collective call's slack value from the
history of slack values stored by the slack-conscious runtime.
The slack estimate is based on the slack value recorded
in the previous MPI collective invocation, as is done in the
cited work of Adagio.
 \item \tiny On each process, adjust its dynamic fraction based on the
   slack value. This adjustment is done using the formula 5 in section
   3 of Chapter 4  and implementation of section 4 of
   Chapter 4. The static fraction used in the
   loop bound is 1- $f_d$.
\end{enumerate}
\end{frame} 
}

\begin{frame}
\frametitle{Software Architecture}
\begin{figure}
\begin{center}
\label{code:architecture}
\includegraphics[scale=0.56]{../images/architecture}
\end{center}
\end{figure} 

\begin{itemize}
\tiny \item \tiny The white rectangles taken by themselves show the
interaction between the MPI+OpenMP application code and the
MPI \& OpenMP runtimes.  
%how a basic MPI+OpenMP
%application is run.  
\item \tiny Our contributions in this work are shown in gray
  rectangles. %Involves source-to-source transformation and adaptiv
\item \tiny Predict static fraction call (arrow going from compilation
  to Adaptive runtime library) is how MPI runtime
  communicates with OpenMP runtime to tune OpenMP scheduler parameters
  (and other parameters). 
\end{itemize}
\end{frame}

\begin{frame}
\frametitle{Code Transformation}
\vspace*{-0.4in}
\begin{columns}
\column{0.5\textwidth}
\begin{figure}
\label{code:origOMPcode}
%\subfloat[\label{code:origOMPcode}{\tiny Loop transformed for slack-conscious scheduling.}]
\begin{center} 
\lstinputlisting{../listings/omp-static.c}
\end{center} 
\end{figure}
\column{0.5\textwidth}
\begin{figure}
\label{code:refactoredOMPcode}
\begin{center} 
\lstinputlisting{../listings/omp-hybrid.c}
\end{center} 
%\caption{\label{code:refactoredOMPcode}\tiny Loop transformed for
%  slack-conscious scheduling.}
\end{figure}
\end{columns}
\begin{enumerate}
\tiny \item \tiny The \texttt{decide\_static\_fraction()} and 
\texttt{endLoop()} functions are calls to the slack-conscious runtime. 
\item \tiny The runtime maintains a \texttt{LoopTimeRecord} struct
  \texttt{lr} across invocations for a threaded computation region. 
  The \texttt{LoopTimeRecord} struct includes storage for
  parameters of loop time, slack, est. sched overhead time, and number of
  iterations, along with various timers obtained at different points during
  the execution of the OpenMP threaded computation region. 
\item \tiny Using wrappers around the MPI collective, the 
  slack-conscious runtime predicts slack, and calculates optimal
  static fraction using the model. 
\item \tiny For the first invocation of the threaded computation
  region, \texttt{lr} is empty, so the runtime system uses static
  fraction of 100\%. 
  % the simpler model without slack to determine the
  % static fraction. 
  % calls around the MPI comm. 
\end{enumerate}
\end{frame}

\begin{frame}[label=runtimeparameterestimation]
\frametitle{Scheduler Parameter Estimation}
%{\small To calculate the value of $\delta$, $\frac{N \cdot t_1}{p}$ 
%  and $q$ for computing the slack-conscious static fraction
%  based on \hyperlink{slackfs}{\beamerbutton{Slide 46}}}:}
\begin{itemize}
\small \item \small The value of the noise parameter $\delta$ is 
estimated in the \texttt{MPI\_Init()} function: we subtract the
maximum and minimum times for the iterations of a basic OpenMP loop
performing a square root operation. 
\comments{and then do an
MPI\_Allreduce() with the max operator to collect the maximum across
all iteration times.}
\item \small We estimate the scheduler overhead $q$ by subtracting 
performance of a statically scheduled OpenMP loop doing a dot product
from a dynamically scheduled OpenMP loop doing a dot product.  
\item \small The time $\frac{N \cdot t_1}{p}$ is estimated by taking 
the time taken for statically scheduled section. 
 \comments{of the last 5 invocations of the threaded computation 
region and taking the minimum execution time of these times.} We 
obtain $t_1$ by dividing this time by the number of iterations
executed statically, and use it to calculate $\frac{N\cdot t_1}{p}$.  
% during the execution of the threaded computation
% region, 

%Note that this gives an estimate of 
%scheduler overhead, in order to give a tighter bound on the static 
%fraction. 
\end{itemize}
%\comments{ 
\begin{enumerate}
\tiny \item \tiny We perform our estimates within the MPI\_Init() 
function (intercepted by PMPI), using a wrapper script. Our calculations needed are done
after the real MPI\_Init() has finished. 
%\tiny \item \tiny The value for noise estimated is 5.4\% greater than that of netgauge
%noise measurements.(possible future direction for netgauge is to make
%it output noise event length delta, rather than percent noise). 
%\tiny \item \tiny Note that values noise and scheduler overhead are estimates, rather
%than the exact values; as we will see in the results section, 
%these estimates are adequate to give us significant performance
%gains. 
\end{enumerate}
%}
\end{frame}

\begin{frame}
\frametitle{Process-local Slack Measurement}
%{\small To calculate the slack S, for computing the slack-conscious static fraction based on 
%\hyperlink{slackfs}{\beamerbutton{Slide 46}}:} 
\begin{itemize} 
\small \item \small \textbf{Callpath.} Associate slack trace values
with only the type of collective operation that had the slack. 
\item \small \textbf{Collective.} Storing the full callpath requires
  overhead of unwinding the stack. Associate slack trace values with
  only the type of the collective operation that had the slack.
\item \small \textbf{Naive.} Forget slack tracing. Predict
  slack based on a precomputed experiment. Specifically, 
  run 1,000 invocations of a simple MPI\_Allreduce() with a
  message size of 4 bytes on all MPI processes before the
  application begins, i.e., right after MPI\_Init(). 
\end{itemize}

\begin{enumerate}
\tiny \item \tiny Tradeoff between runtime overhead for the slack measurement, and the 
advantage due to Adaptive Hybrid Static/Dynamic Scheduling.
\item \tiny Note that this implementation is based on the Adagio
runtime, and that we did not use the Adagio runtime directly. 
\end{enumerate}
\end{frame}

\begin{frame}
\frametitle{Slack Prediction Mechanism Overheads and Errors}
\begin{itemize}
\tiny \item \tiny \textit{Slack prediction error} for a particular MPI
region is the percentage difference 
between the predicted slack before the MPI region begins 
and the actual slack recorded at the end of the MPI region.
\item \tiny \textit{Slack prediction overhead} for a particular MPI
  region is the time spent in our slack prediction runtime.
\end{itemize}
\begin{figure}
\includegraphics[scale=0.30]{../plots/slackErr-fastNUMA2}
\includegraphics[scale=0.30]{../plots/slackOvhd-fastNUMA2} \\
\includegraphics[scale=0.30]{../plots/slackErr-rzuseq}
\includegraphics[scale=0.30]{../plots/slackOvhd-rzuseq}
%TODO: see if schemes or runtimes is better
\caption{\tiny Overheads and errors of different available slack
  prediction schemes in our runtime system. Each process sums the
  total error and overhead across all MPI regions, and we report below
  average overhead and error across MPI processes.}
\end{figure}
\end{frame}

\begin{frame}
\frametitle{Improving Performance through Model-guided Optimization And Slack-Conscious Scheduling}
%\frametitle{Can We Use This Slack Information to Adjust the Scheduler Parameters So As to Reduce this Cost?}
\begin{figure}
\includegraphics[scale=0.34]{../plots/app-scaling-fastNUMA2}
\caption{\small Speedup of slack-conscious hybrid static/dynamic scheduling over OpenMP static scheduling for cab with varying number of nodes.}
\end{figure}
\begin{itemize}
\tiny \item \tiny Using a theoretical analysis to acheive the minima
with Static-Hybrid helps to provide 13-16\% performance gains for a
particle simulation (nbody) and SNAP (reg. mesh), each tested on a
16-core non-NUMA architecture and 16-core NUMA architecture.
\item \tiny Slack-conscious scheduling  provides additional gains
  of 23.5\% for NASLU-MZ and 25.5\% for SNAP.
\end{itemize}
\end{frame}

\begin{frame}
\frametitle{Improving Performance through Model-guided Optimization And Slack-Conscious Scheduling}
%\frametitle{Can We Use This Slack Information to Adjust the Scheduler Parameters So As to Reduce this Cost?}
\begin{figure}
\includegraphics[scale=0.34]{../plots/app-scaling-rzuseq}
\caption{\small Speedup of slack-conscious static fraction over OpenMP
  static for rzuseqwith varying number of nodes.}
\end{figure}
\begin{itemize}
\tiny \item \tiny Using our scheduling strategy does not incur
significant performance degradation (less than 1\% on all node counts
tested) for a non-noisy machine. 
\item \tiny We are able obtain some performance gain (5.5\% at 1024 
  nodes) for our amg experiment which has a small amount of application load imbalance in the code.
\end{itemize}
\end{frame}

\begin{frame}[label=bestfsvsadpt]
%\frametitle{Why Should We Care About Model-guided and Runtime-guided Optimizations to Our Lightweight Scheduling Strategies?}
\frametitle{Benefits of Using Runtime Adjustment over Experimental Tuning}
\begin{figure}
\includegraphics[scale=0.50]{../plots/best-fs-vs-adpt}
\caption{\small Shows how runtime-adjusted static fraction for each application run on cab
works better than the tuning approach (which is best static fraction
(\textit{besf}) strategy). Speedup of slack-conscious static fraction
over best static fraction is shown.}
%This shows the importance of model-guided optimization and
%runtime-based adjustment (esp. with increasing nodes). }
\end{figure}
\end{frame}

\begin{frame}[label=locOptHybrid]
\frametitle{Locality-Optimized Mixed Static/Dynamic Scheduling}
\begin{columns}[T]
\column{0.5\textwidth}
{\small \underline{\textbf{Introduction:}}} \\
\begin{itemize}
  \comments{\tiny \item \tiny We need to avoid disturbing spatial locality while also making our scheduler as lightweight as possible.}
  \tiny \item \tiny We first describe our locality-optimizations,
  explaining why the original disturbs spatial locality and how we can
  improve it while maintaining our original lightweight scheduling
  scheme's benefits. 
\item \tiny We describe our implementation of a library that is
  invoked within an OpenMP threaded computation to schedule
  iterations, explaining that it is low overhead compared to OpenMP
  static.  
%\item \tiny We show results for a Barnes-hut pthread code and regular mesh OpenMP code using our scheduling library. 
\item \tiny We show results for a Barnes-hut and regular mesh code 
  using our scheduling library.
\item \tiny We show results that indicate that our locality-optimized 
strategy controls the performance variation (discussed
\hyperlink{perfVar}{\textcolor{blue}{here}}) just as well or better than
does the basic hybrid static/dynamic scheduling. 
%\comments{(which is beneficial for scalability)}. 
\end{itemize}
\vrule{}
\column{0.5\textwidth}
       {\tiny  \underline{\textbf{Scheduling Strategy Optimizations}}}\\
       \begin{itemize}
         \tiny \item \tiny \textcolor{gray}{ mixed static/dynamic $\rightarrow$ hybrid static/dynamic  (base)}
       \item \tiny  \textcolor{gray}{lightweight scheduling (which includes locality-aware scheduling and skewed workloads)}
       \item \tiny \textcolor{gray} {weighted locality-sensitive scheduling}
       \item \tiny \textcolor{gray} {slack-conscious scheduling}
       \item \tiny \textcolor{black} {staggered hybrid static/dynamic scheduling}
       \end{itemize}
    {\tiny \underline{\textbf { Methodology for Tuning Scheduler Parameters}}}
    \begin{itemize}
      \tiny \item \tiny \textcolor{black} {experimental tuning}
    \item \tiny \textcolor{white} {model-guided optimization} %TODO: change to pruning search space 
    \item \tiny \textcolor{white} {runtime adjustment} %TODO: add within-node runtime adjustment and across node adjustment
    \end{itemize}
\end{columns}
\end{frame}

\comments{
low-probability term  for terms  -     precise  approximation .    

-    correctness of the equality . 
- intuition 

-  storyboard   

simple techniques can make your code faster.   
}

\begin{frame}
\frametitle{Improving Spatial Locality in Dynamically Scheduled Iterations}
{\tiny Ignoring slack for the moment, consider the diagram shown in slide 45
  showing hybrid static/dynamic scheduling. \\}
{\tiny  From this diagram, consider the sequence of loop iterations that
  all 4 threads work on. (Note that the x-axis below is increasing
  iteration numbers, and not time ). \\} 
\begin{center}
\includegraphics[scale=0.35]{../images/sdSched} \\
\end{center}
\begin{itemize}
\small \item \small The consecutive iterations typically access consecutive memory locations. 
\item \small Spatial locality benefits static iterations, but is lost in dynamic iterations. 
\item \small Given this, we could get additional performance benefit if we bias distribution of dynamic iterations, so that each thread is likely to execute dynamic iterations 
that are contiguous with its static iterations.  
\end{itemize}
\end{frame}

\begin{frame}[label=sdsSched]
%\frametitle{Question: Can we Re-distribute the Iterations So that Spatial Locality Isn't Lost?}
\frametitle{Spatial Locality in the Hybrid Static/dynamic Scheduling Approach}
\begin{center}
\begin{figure}
\includegraphics[scale=0.35]{../images/sdsSched}
%TODO: update the below, to word better for intent. 
\caption{\small We stagger the dynamic iterations in order to improve spatial locality of the basic static/dynamic scheduling.}
\end{figure}
\end{center}

\begin{itemize}
\small \item \small Strategy: Each thread finishes its static iterations, then does dynamic iterations marked for it if available, and only then looks for other dynamic work. 
\item \small Spatial locality will not be lost in common cases: 
\begin{enumerate}
\small \item \small When there is no imbalance. 
\item \small Even when there is imbalance espeicially for noise, most threads are able to complete their own dynamic iterations.
\end{enumerate}
\end{itemize}
\end{frame}

\begin{frame}[label=efficientImpl]
%\frametitle{Question: Can we develop an efficient scheduler library that handles different tradeoffs between load balance and locality?}
\frametitle{Note on Efficient Implementation of Scheduler Library}
\begin{center}
\input{../plots/runtime-costs}
\end{center}
\begin{itemize}
\small \item \small Overhead of our library is relatively small. 
\end{itemize}
\end{frame}

\begin{frame}[label=claresults]
%TODO: add in varying number of cores to show improvements. 
\frametitle{Performance Improvements with Varying Problem Sizes}
\begin{figure}
\includegraphics[scale=0.40]{../plots/bh-16cores}
\includegraphics[scale=0.35]{../plots/st-16cores}
\caption{\small Shows performance improvements when using our vSched
  library (having staggered static/dynamic scheduling strategy), for a
  Barnes-Hut pthread particle simulation code (right) from the
  Lonestar Benchmark Suite and the SNAP 2D regular mesh code (right).}
\end{figure}
\begin{itemize}
\tiny \item \tiny Using our staggered hybrid static/dynamic
scheduling (\textit{stag}) for the 2D regular mesh code
with 65536 mesh points, we get reasonable performance gain of 6.45\% 
over OpenMP static scheduling. 
\item \tiny Using our staggered static/dynamic scheduling approach for
 the Barnes-Hut code at 100,000 particles, we get significant gains of
 18.53\%.
\item \tiny Our strategy provides increased performance gains as we
  increase the problem size. 
\end{itemize}
\end{frame}

\begin{frame}[label=controlPerfVar]
\frametitle{Controlling Performance Variability}
%\frametitle{Do the optimizations maintain performance consistency characteristics of static/dynamic scheduli}
\begin{center}
\input{../plots/runtime-stddev}
\end{center}
\begin{itemize}
\tiny \item \tiny With dynamic scheduling, performance can vary 
from run to run (even for static/dynamic scheduling).
\item \tiny Runtime standard deviations are controlled with our
  scheduler optimizations done in vSched, probably because of more
  persistent and predictable behavior(because each timestep each
  thread is likely executing the same dynamic iterations). 
\end{itemize}
\end{frame}

\begin{frame}[label=combinedintro]
\frametitle{Scheduler Composition}
\begin{columns}
\column{0.5\textwidth}
{\small \textbf{\underline{Introduction}:}}\\
\begin{itemize}
\tiny \item \tiny Some schedulers may benefit in some cases, others elsewhere.
\item \tiny Problem: Machine and application has all above mentioned factors involved. 
\item \tiny We show how to combine all schedulers. We also apply it to 3 different application codes. 
\item \tiny We discuss application programmer usability. 
\item \tiny Finally, we re-visit the amplification problem, and discuss how our schedulers work at large-scale.
\end{itemize}

\vrule{}
 
\column{0.5\textwidth}
{\tiny  \underline{\textbf{Scheduling Strategy Optimizations:}}}\\
\begin{itemize}
  \tiny \item \tiny mixed static/dynamic (base)
\item \tiny lightweight scheduling 
\item \tiny weighted locality-sensitive
\item \tiny slack-conscious scheduling
\item \tiny constrained staggering
\item \tiny $\rightarrow$ Each opens up new opportunities to balance the tradeoff between load balance and locality. % The main \point is that there are many dimensions of scheduler 
\end{itemize}

{\tiny \underline{\textbf { Methodology for Tuning Scheduler Parameters}}}\\
\begin{itemize}
  \tiny \item \tiny experimental auto-tuning
\item \tiny model-guided optimization
\item \tiny runtime adjustment
\item \tiny $\rightarrow$ Each are important mechanisms to find the best tradeoff between load balance and locality.
\end{itemize}
\end{columns}
\end{frame}

\begin{frame}[label=combinedSched]
\frametitle{Components of Example Composed Scheduler}
\begin{itemize}
\small \item \small \textbf{uSched}: Use model-guided optimization
  to determine the static fraction (we have separate cases for load
  imbalanced due to noise and load imbalance due to the application), 
  and then experimentally tune the static fraction $0.05$ below static
  fraction and $0.05$ above static fraction ($0.05$ is a configurable
  parameter). 
\item \small \textbf{vSched}: Almost the same as the \textit{ssds}
  scheduling strategy. The difference here is that we use the 
  static fraction obtained from \textbf{uSch} above
  rather than experimentally tuning the static fraction. 
\item \small \textbf{callsite\_fd}: Almost the same as the Slack-conscious Scheduling strategy. 
  The difference here is that initial static fraction to use is obtained from \textbf{uSched} above. 
\item \small \textbf{comboSched}: We start with the staggered static/dynamic
  scheduling scheme defined in \textbf{vSched} above, and then do the 
  slack-conscious adjustment described in slack-conscious scheduling section.
\end{itemize}
\end{frame} 

\comments{
\begin{frame}[label=combined] 
\frametitle{Scheduler Composition of Example Composed Scheduler} 
\begin{figure}[ht]
\label{fig:sched-composition}
\begin{center}
\includegraphics[scale=0.35]{../pictures/sched-composition-diagram}
\end{center}
\caption{\label{fig:sched-composition} Composition of 3 schedulers
which make ComboSched.}
\end{figure} 
\end{frame} 
}

\begin{frame}[label=combinednbody]
\frametitle{Combined Scheduling Results for Rebound}
{\tiny Rebound(*) is an MPI+OpenMP particle simulation code involving large load
imbalances as well as irregular data access patterns.}
\begin{figure}
\includegraphics[scale=0.43]{../plots/app-scaling-strat-nbody-fastNUMA2}
\caption{\tiny Percent speedup over OpenMP static scheduling for each
  scheduling strategy for the particle simulation code. Run on
  \texttt{cab} using 16 threads per node (and 1 MPI process per
  node).}

\end{figure}
\begin{itemize}
\tiny \item \tiny Combining different techniques seems to add on
benefits, i.e., they don't cancel benefit out. 
\item \tiny Because they are based on complementary factors. Spatial
  locality in the static/dynamic scheduling vs. reducing dynamic
  scheduling. 
%\item \tiny better combination could lead to even more improvement 
\end{itemize}

{ \tiny *Rebound is actually the name a suite of particle simulation application
  codes, and we considered the shearing sheet simulation. Here, we use the  
  name Rebound to identify the shearing sheet application run in our experiments.  }
\end{frame}

\begin{frame}[label=combinedfe]
\frametitle{Combined Scheduling Results for MiniFE}
{\tiny  CORAL miniFE MPI+OpenMP code runs a finite element method. 
  This involves irregular computation, and relatively low load imbalance.} 
\begin{figure}
%\includegraphics[scale=0.38]{../plots/app-scaling-strat-SNAP-craybw}\\
\includegraphics[scale=0.43]{../plots/app-scaling-strat-SNAP-fastNUMA2}
\caption{\tiny Shows how the application of full scheduling to miniFE 
works better than the application of other approaches, by showing speedup over OpenMP
static scheduling for each scheduling strategy for different numbers
of nodes. Run on \texttt{cab} using 16 threads per node
(and 1 MPI process per node).}
\end{figure}

\begin{itemize}
\tiny \item \tiny Dynamic scheduling causes significant degradation 
even though there is imbalance. 
\item \tiny Our scheduling strategies provides notable performance 
improvement over static scheduling, though gains less than performance gains for n-body.
\item \tiny One reason is that code\comments{'s computation} has 
relatively significant amount of across-node load imblance relative
to within-node load imbalance, while n-body has little to no load
imbalance. 
\end{itemize}

\end{frame}

\begin{frame}[label=combinedregmesh]
\frametitle{Combined Scheduling Results for Regular Mesh}
{\tiny CORAL SNAP is an MPI+OpenMP code that does a regular mesh
  computation used in heat diffusion. This computation is similar to
  that in chapter 2.}
\begin{figure}
\includegraphics[scale=0.43]{../plots/app-scaling-strat-fe-fastNUMA2}\\
\caption{\tiny Shows how the full scheduling for SNAP 
works better than other approaches, by showing speedup over OpenMP
static scheduling for each scheduling strategy for different numbers
of nodes. The above is run on \texttt{cab} using 16 threads per node, 
and 1 MPI process per node.}
\end{figure}

\begin{itemize}
\tiny \item \tiny Even when dynamic strategies worsen performance, the
combined strategies show a small benefit. 
%\item \tiny uSched doesn't do as well 
\end{itemize}
\end{frame}

\comments{
\begin{frame} 
\frametitle{Code with Scheduling Composition} 
\begin{figure}[h!t]
              \label{code:dvSchedMacro}
              \begin{center}
                \lstinputlisting{../listings/dvSchedMacro-CG.c}
             \end{center}
              \caption{\label{code:dvSchedMacro}{\small NAS CG code
                  modification using our scheduling lib\
rary.}}
            \end{figure}
\end{frame} 
}

\begin{frame}[label=programmerEffort]
\frametitle{Is Our Method Easy to Use for the Application Programmer?}
{\tiny Rebound: (-add total loc-) SNAP: (-add total loc-)  miniFE: (-add total loc-) \\}
\input{../plots/linesOfCodeChanged}
%Experiment: Show lines of code modified, as shown in fourth quadrant
%of SC poster. 

\begin{itemize}
\tiny \item \tiny Lines of code changed is minimal for all three codes. 
\item \tiny Code transformation via ROSE \comments{(and changes to
  GOMP)} can reduce or eliminate need for code change.
\end{itemize}
%TODO: explain
\end{frame}

\comments{
\begin{frame} 
\frametitle{Load Imbalance Metrics With Increasing Numbers of Nodes}
\begin{figure}
\label{fig:bh-withinAcrossImb-cab}
\begin{center}
  \subfloat[Load Imbalance Across Nodes] {\includegraphics[width=0.5\textwidth] {../plots/imb-maxavg-nbody-cab-inodes}}
  \subfloat[Load Imbalance Across Cores] {\includegraphics[width=0.5\textwidth] {../plots/imb-maxavg-nbody-cab-icores}}
\end{center}
\caption{\label{fig:bh-withinAcrossImb-cab}{Standard load imbalance metrics taken for a galaxy 
    simulation running on 1024 nodes of \textit{cab}, which is an Intel Westmere
    machine with 16 cores per node. The code is run with one MPI process per core.}}
% Overheads of our scheduling runtime.\comments{shown through comparison between our static scheduling implementation and OpenMP static scheduling.}
\end{figure} 
\end{frame}
}
\comments{
\begin{frame}[label=ampProbRevisit]
\frametitle{Amplification Problem Revisited}
{\tiny (Ignore for now.)} 
\begin{figure}
\label{fig:modeledScurve}
\begin{center}
\includegraphics[scale=0.38]{../plots/amplification-model-Scurve1}
\end{center}
\caption{\label{fig:modeledScurve}\small Modeled scaling curve based
  on formula for expected length of the $k^{th}$ iteration,
  described~\hyperlink{noiseampformula}{\beamerbutton{here}}. In the formula, we use I=1000, $\delta$ = 5 ms, $\phi$ = 1.}
\end{figure}
\end{frame}
}

\begin{frame}[label=ampProbRevisit]
\frametitle{Impact to Extreme-Scale}
\begin{itemize} 
 \small \item \small As we go to future machines including exascale
\begin{enumerate} 
 \tiny  \item \tiny  number of nodes increases 
\item \tiny cores per node increases 
\end{enumerate} 
\item most importantly, static and dynamic variabilty increases. 
\begin{enumerate}
\tiny \item \tiny Operating voltage might decrease, which increases
  variability and decreases reliability. 
\item \tiny  Fault rates increase; these will be contained
through multiple mechanisms locally, but this will lead to transient
load imbalances. 
 \end{enumerate} 
\item \small Applications will need strong scaling, as exascale reports say,
  because memory per core will be lower, and scientists would want to
  solve the same problem faster rather than a larger problem at the
  same speed. 
\item \small All of the above reasons will increase the impact of
  amplification. 
\item \small Thereby increasing the need for the kind of schedulers we have developed in
  this dissertation 
\begin{itemize} 
 \item \small Number of cores per node will be substantially higher, which our
techniques can leverage.
\end{itemize} 

\item {\small the multiple techniques that we developed are in
  extensible software library that can be extended for exascale
  machines with newer features and future sophisticated applications. }
% can be
% selectively combined in different 
\end{itemize} 

\comments{
adaptive componets of hardware and software, low power  node 

The number of errors will increase, but it is expected that errors
will be contained localily 

impact of transient events is larger impact of applications .. 
This will lead to   
}
\end{frame} 

\begin{frame}[label=relatedWork]
\frametitle{Related Work}
\begin{itemize}
\item \small DPLASMA: dynamic scheduling for specific Numerical Linear
  Algebra application, not tunable static fraction. 
\item \small Work-stealing: Ignores locality, and can't tune the
  dynamic scheduler. 
\item \small OpenMP loop scheduling: Sarkar, Chapman, deSupinski.
\item \small TBB, Habanero.
%Some work by Sriram et al tries to fix outer
%iteration locality 
\item \small Co-scheduling: Schedule all events together to make only 
  some timesteps be impacted by noise, but can't co-schedule all
  events. 
\item \small Hyperthreading with SMT at LLNL: shows benefits with SMT 
  and can be lightweight, but is a hardware solution (not tunable to
  avoid overheads). 
\item \small Runtime for Irregular applications (Galois): Transient
  load imbalances, but dealing with more frequent irregularity on each
  timestep, and load imbalances are coarse-grained. 
\item \small UPC task scheduler: Can move work across threads with
  some affinity, but this can't be tuned (all is done at runtime).
\end{itemize}
%*Some work/discussions have been done to integrate our scheduler into this runtime
\end{frame}

\begin{frame}[label=conclusions]
  \frametitle{Conclusions}
  \begin{itemize}
  \item \small Performance irregularities create scaling challenges
    and are difficult to handle through conventional techniques. 
  \item \small Locality-aware hybrid static/dynamic scheduling
    provides a way to cost efficiently handle within-node 
    load imbalance induced by performance irregularity. 
  \item \small Formalization and Numerical Linear Algebra Application
    Case Studies. 
  \item \small Additional scheduling techniques and optimizations:
    model-guided determination of $f_d$, adaptive per-thread static fraction, slack-conscious static
    fraction (staggered scheduling for spatial locality in dynamic
    section). 
  \item \small Our runtime does not introduce significant overheads. 
\begin{itemize} 
  \item \small So, even if runtime slowdowns due to performance
    irregularity are small, it can be used.
\end{itemize}  
  \item \small Feasible to integrate into applications through
    compiler transformations and adaptive runtime library. 
  \item \small Demonstrated a novel set of scheduling strategies that
    can be used to overcome challenges posed by dynamic
    irregularities, while minimizing data movement and synchronization
    costs. 
\item \small Directions for future work: (a) can use within MPI-shared memory extensions and
  MPI+X models; and (b) can combine with other across-node load balancers.
%  \item \small Can further refine accuracy of slack prediction through
%   risk model.
%  \item \small Can use mixed non-blocking + blocking collectives,
%  inspired by mixed static/dynamic scheduling.  
    % \item \small Plan to provide more loosely coupled interface between OpenMP and MPI, or between any two programming models.
  \end{itemize}
\end{frame}

%Acknowlegdments of people
%\input{ack}
%NOTE: this marks the end of the presentation. 
%TODO: figure out if we should add ``questions'' slide.

%\input{thanks}
%People involved in the project 
%\input{people} 


\begin{frame}
\frametitle{Overview: Mgmt} 

\end{frame} 

\begin{frame}
\frametitle{Overview: Comm} 


\end{frame} 









%TODO: add in simple bulk-synchronous code here. 
%TODO: justify why these old graphs are fine. 
\begin{frame}
\frametitle{Example Synchronous MPI code}
\lstinputlisting{../listings/mpi-bulk-synch.c}
\end{frame}


\begin{frame}
\frametitle{Transient Load Imbalance and its Mitigation}
\begin{columns} 
\column{0.5\columnwidth}
{\small Performance Irregularities on individual nodes can lead to
  significant slowdown of synchronous MPI codes. }
\column{0.5\columnwidth}
{\small One promising approach, given multi-core nodes, is to only
re-distribute work within each node (this is inexpensive). } 
\end{columns} 
%{$\rightarrow$ \small Within-node performance irregularities that
%induce load imbalances are as significant a factor for applications as
%across-node load imbalances. The below gives an intuitive
%explanation of why this is so.}

%TODO: animate this to show  right one slide later. 
\begin{figure}[ht]
  \label{fig:NoiseTransImbPic}
  \label{fig:AmplificationPic1}\subfloat[\label{fig:AmplificationPic1}
    \tiny Noise occuring on different nodes in different
    iterations delays every iteration. 
  ]{\includegraphics[height=1.0in]{../pictures/NoiseAmplificationPic4}}
  \label{fig:AmplificationPic3}\hspace*{0.25in}\subfloat[\label{fig:AmplificationPic3}
    \tiny Execution times is significantly reduced, if we assume load
    can be perfectly re-distributed within each node. 
  ]{\includegraphics[height=1.0in]{../pictures/NoiseMitigatedPic5}}
  \caption{\label{fig:NoiseTransImbPic} Schematics of application
    timelines showing how impact of noise can be mitigated by
    idealized within-node work re-distribution.} 
\end{figure}
\end{frame}
%TODO: finalize that this should go here. We explain that the problem
%will become even greater when laod imbalance is.  
\begin{frame}[label=workredistributionNoise]
\frametitle{Within-node Load Imbalance May Help}
\begin{figure}[ht!]
\label{fig:stat-histo}
\begin{center}
\includegraphics[scale=0.16]{../images/IterTimesHisto-outliers-static}\\
%\includegraphics[scale=0.29]{../images/IterTimesHisto-outliers-dynamic}
%\includegraphics[scale=0.35]{../images/IterTimingsHisto-dynamic}
\end{center}
\caption{\label{fig:stat-histo} \small Iteration timing distribution across 1000 timesteps of the
statically scheduled\comments{\hyperlink{3Dmesh}{\beamerbutton{3D
      reg. mesh}}} 3D regular mesh computation on one node of Power5+. }
\end{figure} 

\comments{
\begin{figure}[h!]
\label{fig:work-redistribution-histo-ideal}
\begin{center}
\fbox{\includegraphics[scale=0.23]{../images/emptySchedImage}}\\
{\tiny todo - need to add (idealized) iteration timing histogram here with single bar at
  6.75 ms} 
\end{center}
%TODO: could re-word the below further. 
\caption{\label{fig:work-redistribution-histo-ideal} \small  }
\end{figure} 
}
\begin{itemize} 
\tiny  \item \tiny Iteration timing distribution
  for 3D regular mesh across 1000 timesteps using no work
  re-distribution\comments{across cores} during each iteration cause
  bi-modal distribution. The second mode on the right causes 
  performance degradations. 
 \item \tiny To minimize the impact of noise to application performance, 
  we would need to re-distribute work in a way that causes the above
  histogram be the one with minimal spread and mean.
\end{itemize} 

\end{frame}

\begin{frame}[label=mitigationWithinNode]
\frametitle{Within-node Persistent Load Imbalance and Mitigation of It}
\begin{figure}[ht]
\label{fig:AppPersistentImbPic}
\subfloat[\tiny A modeled application timeline having persistent load
imbalance across nodes, and cores during execution.]{\includegraphics[height=1.0in]{../pictures/PersistentCores}}
\label{fig:AmplificationPic3}\hspace*{0.25in}\subfloat[\label{fig:AmplificationPic3}\tiny
  A modeled application timeline with work re-distributed across cores,
to reduce load imbalance.]{\includegraphics[height=1.0in]{../pictures/PersistentNode}}
\caption{\label{fig:AppPersistentImbPic} Application-induced load imbalances, which are typically jagged and persistent, are also mitigated by within-node re-balancing. }
\end{figure}
\end{frame}

\begin{frame}[label=ImbWithinNode]
%\frametitle{Performance Doesn't Scale with Cores as Exp}
\frametitle{Within-node Imbalance May Help}
{\small We measure the load on each core for an n-body application code called Rebound.}\\
{\small Mitigated Imbalance: Assuming load within each node is perfectly balanced across its cores without any overhead.}
\hspace*{2.5in}\includegraphics[scale=.12]{../pictures/persistentImb}\hspace*{1in}\includegraphics[scale=.12]{../pictures/mitigatedImb}
\vspace*{-0.1in}
\begin{center}
\input{../plots/bh-withinAcrossImb-cab}
\end{center}
%TODO: show calculation for this. 
\begin{center}
{\tiny Imbalance is measured as $\frac{{max\{load_{i}\}}}{avg\{load_{i}\}}$, where $i$ is the processing element.}
%TODO word the below better 
\end{center}
\end{frame}

%TODO: see if we can change title 
\begin{frame}[label=dynLoadBal]
\frametitle{So, can dynamic load balancing fix this?}
\begin{itemize}
\item Dynamic load balancing within a node can improve performance. 
\begin{enumerate}
\item For both kinds of imbalances. 
\item Persistent imbalance can also be addressed by across-node balancing (avail. in Charm++, Zoltan), but this is complementary.
\end{enumerate}
\vspace*{.5in}
\item However, within-node dynamic load balancing faces many challenges. 
\end{itemize}
\end{frame}

\begin{frame}[label=idleTimeCalc]
\frametitle{Challenge 1: Idle Time due to Within-node Imbalance}
{\tiny How we measured idle time:}
\begin{center}
\lstinputlisting{../listings/omp-CG-idleTimeCalc.c}
\end{center}
\end{frame}

\begin{frame}[label=basictimings]
\frametitle{Challenge 1: Idle Time due to Within-node Imbalance}
{\tiny \textit{\textbf{Setup and Methodology:}} Used NASLU OpenMP
  benchmark code and Barnes-Hut pthread application code. Ran on one
  node of LLNL cab cluster, which is an Intel Xeon 16-core with CHAOS
  operating system.\comments{using gcc optimization level 'O3'}. \comments{having OpenMP3 and mvapich3}}
\visible<1->{
\begin{figure}[t]
\begin{center} 
\includegraphics[scale=0.35]{../plots/dmTime-nbody-cab-idleonly}
\includegraphics[scale=0.35]{../plots/dmTime-NASLU-cab-idleonly}
\end{center}
\caption{\small Breakdown of execution time for a load imbalanced code
  (left) and a load balanced code (right) on cab, shown for the
  3\comments{available}OpenMP scheduling strategies.}
\end{figure}}

\visible<1->{
\begin{itemize}
\tiny {\item \tiny  Dynamic or guided scheduling eliminates idle times in both codes.}
\tiny {\item \tiny However, for NASLU, the total execution time actually increases with both dynamic and guided scheduling.}
%TODO : make within-node dynamic balancing clear. 
\begin{itemize}
\tiny {\item \tiny  Recall that (within-node) dynamic load balancing is needed even for naturally balanced applications, to handle noise.}
\end{itemize}
\tiny {\item \tiny  For n-body, the execution time does not decrease as much as we expect, based on the idle time in the static scheduling.}
\end{itemize}}
\end{frame}

%TODO: word the expected overhead of idle time better. 
\begin{frame}
\frametitle{Why is the Non-idle Time Increasing?}
%\begin{itemize}
%\tiny \item \tiny Scheduling overhead?
%\end{itemize}
\begin{figure}[t]
\begin{center}
\includegraphics[scale=0.35]{../plots/dmTime-nbody-cab-idleonly}
\includegraphics[scale=0.35]{../plots/dmTime-NASLU-cab-idleonly}
\end{center}
\end{figure}
\begin{center}
{\small Breakdown of execution time for a load imbalanced code (left) and a load balanced code (right) on cab.}
\end{center}
{\small (Answer: Mostly data movement, as we will see in the next few slides.)}
\end{frame}

%TODO: check cause/challenge here and in thesis. 
\begin{frame}
\frametitle{Challenge 2: Synchronization Overhead}
{\tiny \textbf{\textit{Methodology}}: Obtain call-path profile from
  hpcToolkit which gives percent time spent in \texttt{omp\_lock()}
  OpenMP runtime function.}
{\tiny comp: computation time,  dq: locking and synchronization, 
Dm:  data movement, due to extra cache misses, etc. } 
\begin{figure}[t]
\label{fig:dmTimes-cab}
\begin{center}
\includegraphics[scale=0.35]{../plots/dmTime-nbody-cab}
\includegraphics[scale=0.35]{../plots/dmTime-NASLU-cab}
\end{center}
%\caption{\label{fig:dmTimes-cab} \tiny Breakdown of time. Synchronization overheads are shown in green.}
\end{figure}
\begin{center}
{\small Breakdown of time. Synchronization overheads are shown in green.}
\end{center}
\begin{itemize}
\tiny \item \tiny Synchronization overheads exist when using dynamic or guided scheduling.
\item \tiny While seemingly insignificant, they account for roughly 5\% percent of execution time for both of these applications. 
\end{itemize}
\end{frame}

%TODO: figure out whether we need data movement on BG/Q. 
%TODO: figure out whether we need to show data movement on bw. 
%TODO: see if you should show static time, to explain prefetching benefit. 
%TODO: separate NUMA and non-NUMA slides.
%TODO: add in details on architecture/platform settings. 
%TODO: add in job script specifics.
%TODO: add number of nodes run on.
%TODO: add in timestep length and problem size. 
%TODO: explain why for different setups 

\begin{frame}
\frametitle{Challenge 3: Data Movement}
{\tiny \textit{\textbf{Methodology}}: Measured L2 and L3 cache misses using PAPI counters, placed just before and just after OpenMP threaded computation region. \comments{(Note: will add code in next slide if I can} } 
\visible<1->{
\begin{figure}[t]
\label{fig:cacheMisses-cab} 
\begin{center}
\includegraphics[scale=0.30]{../plots/L2-cacheMisses-nbody-cab}
\includegraphics[scale=0.30]{../plots/L2-cacheMisses-NASLU-cab}\\
\includegraphics[scale=0.30]{../plots/L3-cacheMisses-nbody-cab}
\includegraphics[scale=0.30]{../plots/L3-cacheMisses-NASLU-cab}
\end{center}
\end{figure}
\begin{center}
{\small Cache misses with L2 misses on top and L3 on bottom, for different OpenMP scheduling strategies.}
\end{center}
}
\visible<1->{
\begin{itemize}
{\tiny \item \tiny For both codes, cache misses increase significantly with dynamic or 
guided scheduling, suggesting an  explanation for the performance loss.}
\end{itemize}
} 
\comments{ {\footnotesize Cache miss calcs in thesis write-up  } }
\end{frame} 

\begin{frame}
\frametitle{Calculations for Cache Misses for NASLU} 
{\small To verify that cache misses are in fact impacting performance 
(and is in fact the cause of the increase in data movement time), we show calculations below. }\\
{\tiny L2 cache miss latency is 100 - 300 cycles (with a 2.66 Ghz processor, this is 4 - 8 ns). }\\
{\tiny L3 cache miss latency is 100 ns. }\\
\begin{figure}[ht]
\label{fig:cacheMisses-cab-forCalcs}
\begin{center}
\includegraphics[scale=0.20]{../plots/L2-cacheMisses-NASLU-cab}\includegraphics[scale=0.20]{../plots/dmTime-NASLU-cab}
\end{center}
\end{figure} 

\begin{itemize}
\tiny \item \tiny We need to account for 485.52 - 83.53 - 24.65 = 
376.34 seconds of time spent in dyn sched. Note that we don't need the
execution time breakdown here, since there is low idle time for this code. 
\item \tiny Cache miss time for L2 is $\frac{200}{2.66 \times 10^9}
  \cdot (3845- 2257 \times 10^6) = 65.04$ s. 
\item \tiny 376.34 - 65.04 = 311.44 seconds to account for.
\item \tiny Cache miss time for L3 is $(100 \cdot 10^-9) \cdot (1676
  - 631) \times 10^6 =  97.82 $ s. 
\item \tiny Still 311.44 - 97.64 = 214.20 seconds to account for. 
  The cost of L3 cache miss may be much higher than 100 ns,
  because access to main memory by multiple threads requires threads to wait in memory
  queue. In the worst case, i.e., when all threads access main memory simultaneously, 
  the latency to memory is factor of $(16+(16-mem\_queue\_depth)\times
  100$ ns of the original latency, where mem\_queue\_depth=12 on this
  machine. 
  %We didn't take
  %into account the execution time for memory buffers for , which can
  %make 
\item \tiny Additionally, to verify that memory bandwidth was the
  factor, we ran STREAM on one core and STREAM running all cores, and
  recorded the bandwidth for both runs. 
\item \tiny Memory bw of STREAM triad on one core is 14837.24 MB/s, but on 16
  cores it is only 103669.71 MB/s. The effective mem. latency
  increases by factor of 2.4x, giving 238.4 s instead of 97.64 s for
  Cache miss time for L3. 
\item \tiny The remainder of time is now 311 - 238.34 = 80 s. This may
  due to L2 being higher (also, note that the L2 latency was reported at
  1 core), or by inefficiencies in parallelization. 
\end{itemize} 
\end{frame}

\comments{
\begin{frame}
\frametitle{Calculations for Cache Misses for bh}
Similarly, for n-body: 
\begin{figure}[ht]
\label{fig:cacheMisses-cab-forCalcs}
\begin{center}
\includegraphics[scale=0.30]{../plots/L2-cacheMisses-nbody-cab}\\
\includegraphics[scale=0.43]{../plots/dmTime-nbody-cab}
\end{center}
\end{figure}
\end{frame}
}
%TODO: consider new title 
% possible Question from audience: why is guided different from mixed
% static/dynamic?  Answer: because it doesn't preserve outer iteration
% data locality, and because on the first chunks, there is dqueue
% overhead. 
% Also, the sizes of the tasks can be controlled in guided, but can't
% handle staggering, i.e., spatial locality. 

\begin{frame}
\frametitle{Thesis Objective}
\underline{\textbf{Thesis Objective:}} Design a set of new scheduling strategies that handles all three causes of the problem, i.e., thread idle
time, data movement, and synchronization overhead simultaneously for
many applications and platforms, in the context of bulk-synchronous and loosely synchronous
MPI applications. 
% all at once, which  OpenMP static scheduling and OpenMP dynamic scheduling. 
\end{frame}
%TODO: change the below to the decomp 

\begin{frame}
\frametitle{Key Idea of Solution}
\visible<1->{\small Can we combine a static and dynamic scheduling scheme, which
  simultaneously reduces scheduler overhead and load imbalance
  (respectively) in an intelligent manner?}\\ 
\visible<1->{
\begin{columns}
\column{0.9\textwidth}
In spite of noise, most of the time, all cores are available to
the application. 
\begin{enumerate}
  \small \item \small Allocate a fixed fraction of
  loop iterations statically. (Then do remaining loop
  iterations dynamically). 
\item \small Ratio of static loop iterations to all loop iterations is the \textit{static fraction}.
\item \small \textit{How do we select the static fraction?}  
\end{enumerate}
\end{columns}
}
\end{frame}

%Note:  add in the cost model .
%TODO: check that we brought in ULT load imbalance.

\begin{frame}
\frametitle{Overview of Scheduling Strategies}
\begin{columns}
\column{0.95\textwidth}
       {\tiny \underline{\textbf{Available OpenMP Schedulers}}} \\
       \begin{itemize}
         \tiny \item \tiny \textcolor{black}{OpenMP static scheduling (\textit{static} or \textit{OpenMP-static})}\\ 
       \item \tiny \textcolor{black}{OpenMP dynamic scheduling (\textit{dynamic} or \textit{OpenMP-dynamic})}\\
       \item \tiny  \textcolor{black} {OpenMP guided scheduling(\textit{guided} or \textit{OpenMP-guided})}\\
       \end{itemize}
           {\tiny  \underline{\textbf{Scheduling Strategy Optimizations}}}\\
           \begin{itemize}
             \tiny \item \tiny \textcolor{black}{mixed static/dynamic scheduling(\textit{half})}
           \item \tiny \textcolor{black}{hybrid static/dynamic scheduling(\textit{besf})}
           \item \tiny \textcolor{black} {lightweight scheduling \textit{ligh}, which includes locality-aware scheduling(\textit{loca}) and skewed workloads(\textit{skew})}
           \item \tiny \textcolor{black} {weighted locality-sensitive scheduling (\textit{weig}, \textit{uWsched} or
             \textit{uWldB}), which consists of hybrid static/dynamic
             scheduling and weighted allocation (\textit{wSched} or \textit{wldB})}.
           \item \tiny \textcolor{black} {model-guided Hybrid
             Static/Dynamic Scheduling (\textit{Static-Hybrid}), and
             slack-conscious hybrid scheduling
             (\textit{callsite\_fd})}. Slack-conscious
             hybrid static/dynamic scheduling is referred to as Adaptive Scheduling in the paper.  
           \item \tiny \textcolor{black} {staggered hybrid static/dynamic scheduling (\textit{sds} or \textit{vSched})}
           \item \tiny \textcolor{black}{{Example combination of schedulers (\textit{allStrat} or \textit{full}) }}
           \end{itemize}
               {\tiny \underline{\textbf {Methodology for Tuning Scheduler Parameters}}}
        \begin{itemize}
          \tiny \item \tiny \textcolor{black} {experimental tuning}
        \item \tiny \textcolor{black} {model-guided optimization} %TODO: change to pruning search space 
        \item \tiny \textcolor{black} {runtime adjustment} %TODO: add within-node runtime adjustment and across node adjustment 
        \end{itemize}
\end{columns}
\end{frame}

\begin{frame}
\frametitle{Hybrid Static/Dynamic Scheduling and Lightweight Scheduling} 
\comments{
\column{0.5\columnwidth} 
\begin{enumerate}
\item \textbf{Introduction:}
\begin{itemize}
\tiny \item \tiny Show basic scheduling scheme and provide explanation of why it is beneficial. 
\item \tiny Explain inefficiency of load balanced code due to noise, with just one node. 
\item \tiny Show additional ways to provide low-overhead dynamic scheduling. 
\item \tiny Intuitive explanation of amplification problem, and benefits of our scheduling strategy to scale. 
\end{itemize} 
\item \textbf{Technique Overview}:
  \begin{itemize}
  \tiny \item \tiny \textbf{Scheduling Strategies}:Mixed static/dynamic, Hybrid Static/Dynamic Scheduling,Lightweight scheduling (which includes locality-aware scheduling and skewed workloads)
  \small \item \small \textbf{Tuning Methodology}: Experimental Tuning  
\end{itemize}
\item \textbf{Architectures:}
\begin{itemize} 
\tiny \item \tiny \textbf{cab}: Intel Xeon Cluster at LLNL having 16 cores per node, CHAOS operating system, mpich2, openMP 3.4. 
\item \tiny \textbf{bp}: IBM Power5+ Cluster at Univ. of Illinois having 16 cores per node, IBM AIX Operating System, ibmmpi, lomp. 
\end{itemize}
\item \textbf{Applications:}
\begin{itemize}
\tiny \item \tiny NASLU: 
\item \tiny nbody: 
\item \tiny 3D stencil: 
\end{itemize}
\end{enumerate}
}

\begin{columns}[T]
  \column{0.5\textwidth}
  \textbf{Introduction:}\\
    \begin{itemize} 
      \tiny \item \tiny Show basic scheduling scheme and provide explanation of why it is beneficial. 
    \item \tiny Explain inefficiency of load balanced code due to noise, with just one node. 
    \item \tiny Show additional ways to provide low-overhead dynamic scheduling. 
    \item \tiny Intuitive explanation of amplification problem, and benefits of our scheduling strategy to scale. 
    \end{itemize}
    \vrule{10in}{1pt}
    %\vrule width 10in height 1pt
    \column{0.5\textwidth}  
           {\tiny  \underline{\textbf{Scheduling Strategy Optimizations:}}}\\
           \begin{itemize}
             \tiny \item \tiny mixed static/dynamic $\rightarrow$ hybrid static/dynamic  (base). 
           \item \tiny Lightweight scheduling (which includes locality-aware scheduling and skewed workloads). 
           \item \tiny \textcolor{white} {weighted locality-sensitive scheduling}
           \item \tiny \textcolor{white} {slack-conscious scheduling}
           \item \tiny \textcolor{white} {staggered hybrid static/dynamic scheduling}
           \end{itemize}
               {\tiny \underline{\textbf { Methodology for Tuning Scheduler Parameters:}}}\\
               \begin{itemize}
                \tiny \item \tiny experimental tuning
               \item \tiny \textcolor{white} {model-guided optimization} %TODO: change to pruning search space 
               \item \tiny \textcolor{white} {runtime adjustment}  %TODO: add within-node runtime adjustment and across node adjustment 
               \end{itemize}
\end{columns}  
\end{frame} 

\begin{frame}
\frametitle{Static and Dynamic Scheduling}
\includegraphics[scale=0.30]{../images/legend-dynamic} \\
\begin{columns}
  \column{0.5\columnwidth}
  \lstinputlisting{../listings/threadedCompRegion-static.c}
  \column{0.5\columnwidth}
  \begin{center}
    \includegraphics[scale=0.35]{../images/threadedCompRegion-static} \\
  \end{center} 
\end{columns}
\begin{columns}
\column{0.5\columnwidth}
\lstinputlisting{../listings/threadedCompRegion-dynamic.c}
\column{0.5\columnwidth}
  \begin{center}
    \includegraphics[scale=0.35]{../images/threadedCompRegion-dynamic}\\
  \end{center}
\end{columns}
%\end{block}

\begin{columns}
\column{0.5\columnwidth} 
- 
\column{0.5\columnwidth}
  \begin{center}
    \includegraphics[scale=0.35]{../images/emptySchedImage}
  \end{center}
\end{columns}
\end{frame}

\begin{frame}[label=mixedstatdyn]
\frametitle{Mixed Static/Dynamic Scheduling}
%\begin{block}
\begin{columns}
  \column{0.5\columnwidth}
  \lstinputlisting{../listings/threadedCompRegion-static.c}
  \column{0.5\columnwidth}
  \includegraphics[scale=0.15]{../images/legend-dynamic}\\
  \begin{center}
    \includegraphics[scale=0.35]{../images/threadedCompRegion-static} \\
  \end{center}
\end{columns}
%\end{block} 
%\begin{block}
\begin{columns}
\column{0.5\columnwidth}
\lstinputlisting{../listings/threadedCompRegion-dynamic.c}
\column{0.5\columnwidth}
  \begin{center}
    \includegraphics[scale=0.35]{../images/threadedCompRegion-dynamic}\\
  \end{center}
\end{columns}
%\end{block}
\begin{columns}
\column{0.5\columnwidth}
\lstinputlisting{../listings/threadedCompRegion-mixed.c}
\column{0.5\columnwidth}
\begin{center}
\includegraphics[scale=0.35]{../images/threadedCompRegion-mixed}
\end{center}
\end{columns}
\end{frame}

\begin{frame}[label=halftimings]
\frametitle{Naive Mixed Static/Dynamic Scheduling}
{\tiny Setup and Methodology is the same as that used in
  \hyperlink{basictimings}{\beamerbutton{here}}.} 
%Used NASLU OpenMP benchmark code and Barnes-Hut pthread application
%code. Ran on one node of LLNL cab cluster, which is an Intel Xeon
%16-core with CHAOS operating system.\comments{using gcc optimization
%level 'O3'}. \comments{having OpenMP3 and mvapich3}} 
\begin{figure}[t]
\begin{center}
\includegraphics[scale=0.34]{../plots/dmTime-nbody-cab-withsd}
\includegraphics[scale=0.34]{../plots/dmTime-NASLU-cab-withsd}\\
\end{center}
\label{fig:dmTimes-NASLU-cab}
\end{figure}
\begin{center}
{\small Set static fraction to 0.5: In \textit{half}, threads do the
  first half of their iterations statically. See rightmost bar. }
\end{center}
\begin{itemize}
\tiny \item {\tiny We expect \textit{half} to cut the dynamic overhead in half.}
\item {\tiny The \textit{half} scheduling provides 10.1\% gain over
  OpenMP guided for the nbody code, due to preservation of locality
  across invocations of threaded computation regions and thus less
  data movement across invocations of the threaded computation
  region.}
\item {\tiny For the NASLU code, the \textit{half} scheduling still has high overhead.}
\end{itemize}
\end{frame}

\begin{frame}[label=dmTimes]
\frametitle{Data Movement with Naive Mixed Static/Dynamic Scheduling}
\visible<1->{
\begin{figure}[t]
\label{fig:cacheMisses-cab-withsd}
\begin{center}
\includegraphics[scale=0.30]{../plots/L2-cacheMisses-nbody-cab-withsd}
\includegraphics[scale=0.30]{../plots/L2-cacheMisses-NASLU-cab-withsd}\\
\includegraphics[scale=0.30]{../plots/L3-cacheMisses-nbody-cab-withsd}
\includegraphics[scale=0.30]{../plots/L3-cacheMisses-NASLU-cab-withsd}
\end{center}
%\caption{\label{fig:cacheMisses-cab-withsd} L2 cache misses (top) and L3 cache misses (bottom).}
\end{figure}
\begin{center}
{\small L2 cache misses shown in the top graphs; L3 in the
bottom. \comments{The \textit{half} is added on the rightmost bar.}
% The data shown is based on 100 trials run on a single node of the
% respective machines. Data movement time is obtained by subtracting
% the sum of idle time, dequeue time and computation time, from total
% execution time.
}
\end{center}
}
\visible<1->{\begin{itemize} 
{\tiny \item \tiny With \textit{half}, cache misses decrease with respect to OpenMP guided, for both nbody and NASLU codes.}
\end{itemize}}
\end{frame}

\begin{frame}[label=noPreset]
\frametitle{Problem: A Preset Static Fraction Doesn't Work Effectively for all Application-architecture Pairs}
\begin{itemize}
\item The above shows that the basic static/dynamic scheduling 
improves performance only for nbody on cab. So, we make the strategy more flexible. 
\end{itemize}
\begin{itemize}
\item We expose the static fraction to the application programmer.  
\item What is right blend of static and dynamic scheduling to use? 
\item The application programmer tunes the blend of static and dynamic scheduling.
\item We refer to this as \textit{Hybrid Static/Dynamic Scheduling}. 
\end{itemize}
\end{frame}

\begin{frame}[label=hybridstatdyn]
\frametitle{Hybrid Static/Dynamic Scheduling}
\begin{columns}
  \column{0.5\columnwidth}
  \lstinputlisting{../listings/threadedCompRegion-static.c}
  \column{0.5\columnwidth}
  \includegraphics[scale=0.15]{../images/legend-dynamic}\\
  \begin{center}
    \includegraphics[scale=0.35]{../images/threadedCompRegion-static} \\
  \end{center} 
\end{columns}
%\end{block} 
%\begin{block}
\begin{columns}
\column{0.5\columnwidth}
\lstinputlisting{../listings/threadedCompRegion-dynamic.c}
\column{0.5\columnwidth}
  \begin{center}
    \includegraphics[scale=0.35]{../images/threadedCompRegion-dynamic}\\
  \end{center}
\end{columns}
%\end{block} 

\begin{columns}
\column{0.5\columnwidth}
%TODO: fix code  to be hybrid sched
\lstinputlisting{../listings/threadedCompRegion-hybrid.c}
\column{0.5\columnwidth}
\begin{center}
\includegraphics[scale=0.35]{../images/threadedCompRegion-hybrid}
\end{center}
\end{columns}
\end{frame}

%TODO: think of a better name for title. 
\begin{frame}[label=expTuningSF]
\frametitle{Experimentally Tuning the Static Fraction}
%TODO: explain why increments of 0.01 
%TODO: explain number of trials run 
{\tiny We vary the static fraction environment variable in increments of 0.01. \comments{(We do 0.01 to get enough data points, but more refinement might give less improvement. )} } 
\begin{figure}[ht]
\begin{center} 
\includegraphics[scale=0.19]{../plots/expTuningSF-nbody-cab-WallTime}
\includegraphics[scale=0.19]{../plots/expTuningSF-NASLU-cab-WallTime}\\
\end{center} 
\begin{center}
{\small Performance for different static fractions when using hybrid static/dynamic scheduling strategy for nbody and NASLU. }
\end{center}
\end{figure}

\begin{itemize}
\tiny \item \tiny For n-body, fully dynamic is better than fully static, but static fraction of 49\% gives even better performance. 
\item \tiny For NASLU, the fully static is much better than fully
  dynamic. However, static fraction of 96\% is the best (although
  compared with the 100\% dynamic, the difference between ``best'' and
  fully static may appear small, it is still a significant
  improvement).  
\item \tiny Note that the best static fraction varies across
  application. 
%Comment: make note about half
\end{itemize}
\end{frame} 

\begin{frame}[label=halftimings]
\frametitle{Hybrid Static/Dynamic Scheduling}
\begin{figure}[t]
\begin{center}
\includegraphics[scale=0.34]{../plots/dmTime-nbody-cab-withsdandbesf}
\includegraphics[scale=0.34]{../plots/dmTime-NASLU-cab-withsdandbesf}\\
\end{center}
\label{fig:dmTimes-NASLU-cab}
\end{figure}
\begin{center}
{\small In \textit{besf}, we use the best static fraction. See rightmost bar.}
\end{center}
%TODO: fix this so it doesn't duplicate

\visible<1->{
\begin{itemize}
\tiny \item \tiny When using \textit{besf} scheduling, cache misses decrease w.r.t. \textit{half}, for both nbody and NASLU codes. This is
because we tune to balance the tradeoff between load balance and locality, to minimize the costs in each of the three challenges, simultaneously.
\item \tiny Paper [statdyn1] studies this idea in more depth.
\end{itemize}}
\end{frame}

\begin{frame}
\frametitle{Data Movement with Hybrid Static/Dynamic Scheduling}
\visible<1->{
\begin{figure}[t]
\label{fig:cacheMisses-cab}
\begin{center}
\includegraphics[scale=0.30]{../plots/L2-cacheMisses-nbody-cab-withbesf}
\includegraphics[scale=0.30]{../plots/L2-cacheMisses-NASLU-cab-withbesf}\\
\includegraphics[scale=0.30]{../plots/L3-cacheMisses-nbody-cab-withbesf}
\includegraphics[scale=0.30]{../plots/L3-cacheMisses-NASLU-cab-withbesf}
\end{center}
%\caption{\label{fig:cacheMisses-nbody-cab} L2 cache misses (top) and L3 cache misses (bottom).}
\caption{\small Cache misses for \textit{besf}, added on the rightmost bar. }
\end{figure}
}
%\begin{center}
%The data shown is based on a 100 trials run on a single node of the respective machines. Data movement time is obtained by subtracting the sum of idle time, dequeue time and computation time, from total execution time.
%} 
%\end{center} 
\visible<1->{
\begin{itemize} 
\tiny \item \tiny When using \textit{besf} scheduling, cache misses 
decrease w.r.t. \textit{half}, for both nbody and NASLU codes. This is
because we tune to balance the tradeoff between load balance and
locality, to minimize the costs in each of the three challenges,
simultaneously. 
\item \tiny Paper [statdyn1] \comments{\footcite{dynwork}} studies
  this idea in more depth. 
\end{itemize}}
\end{frame}

\begin{frame}[label=3Dmesh]
\frametitle{3D Regular Mesh process+thread Domain Decomposition}
\begin{columns}
  \column{0.35\textwidth}
  \begin{center}
    \includegraphics[scale=0.35]{../images/appTimestep-noslack}
 \end{center}
\column{0.65\textwidth}

\begin{figure}[ht]
  \label{fig:3DStencil-domainDecomp-stat}
  \begin{center}
    \includegraphics[scale=0.18]{../images/3DStencil-domainDecomp-stat}
  \end{center}
  \caption{\label{fig:3DStencil-domainDecomp-stat} {\small 3D Regular Mesh domain
  decomposition across processes, along with thread partitioning
  of work within each process.}}
\end{figure}
\end{columns}
\end{frame}


\begin{frame}[label=perfVar]
\frametitle{Performance Variation and Absolute Performance for Static Scheduling}
{\tiny \underline{\textbf{Setup and Methodology:}} Run on IBM Power5+ cluster at UIUC which has 16 cores per node\comments{32MB L1 cache and 128MB L2 cache} and AIX operating system.}
\begin{figure}[h!]
\label{fig:stat-histo}
\begin{center}
\includegraphics[scale=0.29]{../images/IterTimesHisto-outliers-static}\\
%\includegraphics[scale=0.35]{../images/IterTimingsHisto-dynamic}
\end{center}
\caption{\label{fig:stat-histo} \small Iteration timing distribution across 1000 timesteps of the statically scheduled\comments{\hyperlink{3Dmesh}{\beamerbutton{3D reg. mesh}}} 3D regular mesh computation.}
\end{figure}
\begin{center}
{\small $\rightarrow$ \textcolor{gray}{Bi-modal distribution for static scheduling.} \comments{Uni-modal distribution for dynamic scheduling, but increased spread and higher mean for dynamic scheduling.}}
\end{center}
\end{frame}

\begin{frame}[label=lightweightDynDefs]
\frametitle{Lightweight Dynamic Scheduling}
\textbf{Tasklet}: A tasklet specifies a set of iterations and records the core that it was executed on in the preceding iteration. 
\begin{enumerate}
\item \textbf{Skewed Workload}: The distribution of tasklets in the shared queue, from beginning to end, is 16 tasklets of size 16, 32 tasklet of size 8, and 64 tasklets of size 4. 
\item \textbf{Locality-aware Scheduling}: On each iteration, a thread tries to search for the tasklet that it executed during the preceding iteration. 
%\item \textbf{Work-stealing}: On each iteration, a thread tries to search for the tasklet that it executed during the preceding iteration. 
\end{enumerate}
\end{frame}

\begin{frame}[label=HybStatDynAppliedto3DMesh]
\frametitle{Hybrid Static/Dynamic Scheduling Applied to 3D Regular Mesh}
\begin{columns}
  \column{0.10\textwidth}

  \column{0.90\textwidth}
  \begin{figure}
    \label{fig:3DStencil-domainDecomp-sd}
    \begin{center}
      \includegraphics[scale=0.18]{../images/3DStencil-domainDecomp-sd}
    \end{center}
    \caption{\label{fig:3DStencil-domainDecomp-hybrid} \small 3D Regular Mesh domain decomposition with Hybrid Static/Dynamic Scheduling Strategy Applied.}
  \end{figure}
\end{columns}
\end{frame}

\begin{frame}
\frametitle{Hybrid Static/Dynamic Scheduling with Lightweight Dynamic Scheduling in Dynamic Section}
{\tiny Setup and methodology the same as \hyperlink{perfVar}{\textcolor{blue}{here}}.} 
\begin{figure}[h!]
\label{fig:skewed2}
\begin{center}
\includegraphics[scale=0.28]{../images/IterTimesHisto-skewedWorkload2}
\end{center}
\caption{\label{fig:skewed2} Iteration timing histograms for 100\% dynamic scheduling, 100\% dynamic scheduling with locality, and 50\% dynamic scheduling.}
\end{figure}
\begin{itemize}
\tiny \item \tiny 50\% dynamic provides better absolute performance,
as well as less performance variation than 100\% dynamic scheduling
without locality and 100\% dynamic scheduling with locality. 
% \tiny \item \tiny Each percent dynamic has a minima for the number of
%  tasklets in the queue, and skewed does well for larger percent
%  dynamic. 
%\item \tiny The best performance is with 4 planes per planes per
%tasklet, so we use this for the subsequent experimentation. 
%As we vary the number of tasklets in the queue, ther
\end{itemize}
\end{frame}

\begin{frame}[label=noiseAmpExample]
\frametitle{Amplification Problem in the Context of Noise}
\begin{itemize}
\item \small On one node running 1000 iterations of length 10 ms. 
\begin{itemize}
\tiny \item \tiny Assume noise (PI) is of length 5 ms, and it occurs approximately every second.
\item \tiny Execution time without PI 10 ms * 1000 iterations = 10 seconds. 
\item \tiny With PI: 10 out of 1000 iterations will be affected (that is delayed by 5 ms) .  So, the total delay is 50 ms.
\item \tiny Execution time is 10.05 seconds, i.e., 0.5\% delay. 
\end{itemize}
\item \small On two nodes (weak scaling):
\begin{itemize}
\item \tiny Probability that an iteration is delayed on one of the nodes is 1\%, same as above.
\item \tiny Probability that a given iteration is delayed on any processor is : $1.0 - (1 - 0.01)^2$ = $1.99$\% .
\item \tiny The delay is then $1000 \times 0.0199 \times 5$  = $99.5$ ms for a total execution time of $10.0995$ s. 
\end{itemize}
\item \small On p nodes :
\begin{itemize}
\item \tiny The delay is $1000 \times (1.0 - 0.99^p) \times 5$ ms
\item \tiny For 10 nodes: $1000 \times 0.0956 \times 5$ ms =  478 ms
\item \tiny For 100 nodes: $1000 \times 0.634 \times 5$ ms = 3170 ms
\item \tiny For 1000 nodes: $1000 \times 0.9999 \times 5$ = 4999.5 ms
\end{itemize}
\item \small That's a 50\% performance loss. 
\item \small As $p$ increases, the probability that one node experiences a delay on a given iteration starts approaching 100\%. 
\small \item \small Caveat for loosely synchronous: amplification will come at a larger number of nodes than for bulk-synchronous, though magnitude of performance degradation may be the same as bulk-synchronous code. 
\end{itemize}
\end{frame}

\begin{frame}[label=noiseampformula]
\frametitle{Amplification Problem in Context of Noise} 
\begin{itemize}
\small \item \small Let $T_p$ be iteration time on one node with $p$ cores, with no noise. We assume weak scaling. 
\item \small  Let $\delta$ be the noise duration.
\item \small Let $\phi$ be the expected number of noise events per second, on one node. \item \small We assume global synchronization with zero overhead, e.g., allreduce.
\end{itemize}

\textbf{----------------------------------------------------------}
\begin{itemize}
\small \item \small Probability of noise happening in one iteration on one node = $\phi \times T_p$. 
\item \small Probability that there is no delay on any nodes while running on k nodes is $(1 - \phi \times T_p)^k$. 
\item \small Probability that \textit{some} node experiences a delay in a given iteration is $1 - (1- \phi \times T_p)^k$. 
\item \small So, expected execution time of an iteration on k nodes is: $E[T_k]  >=  T_p + \delta \times  [1 - (1- \phi \times T_p)^k]$. 
\end{itemize}
%( Caveat of slack) 
\end{frame}

\begin{frame}
\frametitle{Improvement of Scalability for 3D Stencil}
\begin{figure}
\includegraphics[scale=0.34]{../images/stencil-scaling}
\caption{Scaling 3D Regular Mesh for different schedulers.}
\begin{itemize}
\tiny {\item \tiny Mixed static/dynamic scheduling results in improved 
scalability over both static scheduling and dynamic scheduling.}
\end{itemize}
\end{figure}
\end{frame}

%TODO: show graph for fully dynamic scheduling.
%\vivek{fix layout}
\begin{frame}
\frametitle{Impact on Scaling}
\begin{figure}[ht!]
\label{fig:scaling-histograms}
\begin{center}
\includegraphics[scale=0.20]{../images/scaling-histograms}
\end{center}
%TODO: word the below better
\caption{\label{fig:scaling-histograms} Iteration timing histograms 
for static scheduling~(left) for mixed static/dynamic scheduling~(right) as we increase the 
number of nodes.}
\end{figure}

\begin{enumerate}
\tiny \item \tiny Distribution of two equal peaks goes to one small
peak and one large peak. 
% Performance variation increases with increasing numbers of
% nodes
\item \tiny The mean of the distribution for mixed 
static/dynamic scheduling stays consistent 
as we increase the number of nodes. 
\item \tiny The mean of the distribution at 64 nodes is better 
  with static/dynamic scheduling than that of static
  scheduling. %TODO: think of a better point here. 
\end{enumerate}
\end{frame}

\begin{frame}
\frametitle{Improvement Over an Already Optimized Dense Matrix Factorization using Hybrid Static/Dynamic Scheduling}
%\frametitle{For Problem 1: Does our scheduling approach perform competitively or better than other industry wide schedulers?}
\begin{columns}
\column{0.5\textwidth}
\begin{figure}
\includegraphics[scale=0.22]{../plots/calu-fastNUMA4-perf}
\caption{\tiny Comparison of hybrid static/dynamic scheduled CALU
  to statically scheduled CALU (first version) as well as 
  other industry standards such as PLASMA and MKL.\comments{showing significant
  performance improvements over the baseline (statically scheduled)
  CALU, particularly with large matrix sizes.}}
\visible<1->{\tiny $\rightarrow$ We get significant
  performance improvements over the baseline (statically scheduled)
  CALU, particularly with large matrix sizes. Our scheduler is 30\% 
  better than PLASMA library's implementation of the same 
  numerical algorithm, and 38\% better than Intel MKL library's 
  impementation of the same numerical algorithm.}
%\end{itemize}
%(Problem: PLASMA has inefficient work-stealing algorithm.) 
\end{figure}
\column{0.5\textwidth}
\begin{figure}
\includegraphics[scale=0.10]{../plots/calu-fastNUMA3-histogram}
\caption{\tiny Performance Variation for static CALU, dynamic CALU, and hybrid static/dynamic CALU.}
\end{figure}
{\tiny $\rightarrow$ Hybrid static/dynamic scheduled CALU reduces
  performance variability while maintaining absolute performance
  better than that of statically scheduled CALU and dynamically
  scheduled CALU, and this is consistent with the performance
  variability results for 3D stencil.} 
\end{columns}
\end{frame}

%TODO: update this
\begin{frame}
\frametitle{Problems with the Basic Lightweight Scheduling Approach}
%\textbf{Two problems with the above basic approach:} 
\begin{enumerate}
\small \item \small Persistent load imbalances not handled. How do we handle both
persistent and transient imbalances together? 
\item \small Experimental tuning: can we reduce the search by using a
model?
\item \small Across-node load imbalance not taken into account in 
scheduling strategy. Can we use runtime-guided adjustments (per-node static
fraction based on slack) to improve performance?
\item \small Our scheduler does not consider spatial locality. How do
  we optimize the mixed static/dynamic scheduling scheme so that we
  improve spatial locality and temporal locality?
\end{enumerate}
\end{frame}

\begin{frame}
\frametitle{Weighted Locality-sensitive Scheduling}
\begin{columns}[T]
% \column{0.50\textwidth}
  \begin{column}{0.5\textwidth}
    {\small \underline{\textbf{Introduction}}} \\
    \begin{itemize}
      \tiny \item \tiny Noise not completely transient; 
      can be a mixture of persistent core noise and transient noise, as observed
      in Jaguar (see next slide). 
    \item \tiny We modify our lightweight scheduling scheme to handle the
      mixture of persistent core noise and transient node-level OS noise. 
    \item \tiny For architectures with 
      the mentioned mixture of noise events, we get noted performance
      gains of 10.4\% over our basic lightweight scheduling scheme. 
    \item \tiny The approach described can be generalized to work for any
      type of mixture of persistent and transient load imbalance, not just
      mixtures of persistent and transient noise. 
    \end{itemize}
  \end{column}
  % \column{0.50\textwidth}
  \begin{column}{0.5\textwidth}
    {\tiny  \underline{\textbf{Scheduling Strategy Optimizations}}}\\
    \begin{itemize}
      \tiny \item \tiny \textcolor{gray}{ mixed static/dynamic (\textit{half}) $\rightarrow$ hybrid static/dynamic \textit{besf} (base)}   
    \item \tiny \textcolor{gray}{ lightweight scheduling (\textit{ligh}) (which includes locality-aware scheduling (\textit{loca}) and skewed workloads \textit{skew})}
    \item \tiny \textcolor{black} {weighted locality-sensitive scheduling (\textit{weig})}
    \item \tiny \textcolor{white} {slack-conscious scheduling: \textit{slac} :  model-guided\textit{Static-Hybrid},   slack-conscious(\textit{callsite\_fd})}
    \item \tiny \textcolor{white} {staggered hybrid static/dynamic scheduling: \textit{ssds}}
    \end{itemize}
        {\tiny \underline{\textbf {Methodology for Tuning Scheduler Parameters}}}
        \begin{itemize}
          \tiny \item \tiny \textcolor{black} {experimental tuning}
        \item \tiny \textcolor{white} {model-guided optimization} %TODO: change to pruning search space 
        \item \tiny \textcolor{black} {runtime adjustment} %TODO: add within-node runtime adjustment and across node adjustment 
        \end{itemize}
  \end{column}
\end{columns}
\end{frame} 

%TODO: figure out whether we need to use coupling. 
%TODO: word the below better 
\begin{frame}
  \frametitle{Persistent Imbalances (Noise-induced) Not Handled} 
  \begin{figure}
    %\includegraphics[scale=0.44]{../images/noise_pes_jaguar}
    \subfloat[\tiny Jaguar quasi-persistence in noise.]{\includegraphics[scale=0.41]{../images/noise_pes_jaguar}}
    \subfloat[\tiny Ranger quasi-persistence in noise.]{ \includegraphics[scale=0.41]{../images/noise_pes_ranger}}
    \caption{\small Time spent by each core in a 1200 core run, for a load balanced code.}
  \end{figure}
  \begin{itemize}
    \tiny \item \tiny The two top bands of indicate slow cores, while
    bottom bands show fast cores (particularly for Jaguar).
  \item \tiny Across all nodes, the subset of cores being fast and
    subset slow is almost the same. 
  \item \tiny We refer to the above phenomenon as persistent core noise,
    or  noise-induced persistent imbalances. 
  %(and note that this could be generalized as a fine-grained application-induced persistent load imbalance across cores).%TODO: explain more on why  
  \end{itemize}
\end{frame}

%TODO: check the question presented in the title of the slide.
\begin{frame}
\frametitle{Blending Weighted Factoring and Mixed Static/Dynamic Scheduling to Handle Persistent Core Noise}
%TODO: add regular diagram for weighted decomp. 
{\small Weighted Factoring: assign different number of iterations to different cores based on history.}
{\small We combine weighted with hybrid static/dynamic scheduling, and
  then tune the static fraction.}\\
{\small Earlier, the number of iterations allocated to thread i was
 $f_s \cdot \frac{N}{p}$}
{\small With weighted, it is: $w_i \cdot \frac{f_s \cdot N}{p}$,
where $w_i$ is calculated by measuring the time spent in the static
portion by different threads, $s_i$, as: $w_i = \frac{avg\{s_i\}}{s_i}$}.
% begining iteration for thread i, b_i = b_{i - 1} +
% w_i*\frac{f_s*N}{p} , with b_0 = 0 
\begin{columns}
  \column{0.5\textwidth}
         {\small Weighted Locality-sensitive Scheduling Strategy Applied to 3D Reg. Mesh.\\}
         \column{0.5\textwidth}
         \begin{center}
           \fbox{\includegraphics[scale=0.2]{../images/weighted_decomp}}\\
         \end{center}
\end{columns}
\end{frame}

\begin{frame}
\frametitle{Weighted Locality-Sensitive Scheduling Results}
\begin{itemize}
\tiny \item \tiny Ranger and Jaguar both experience noise; Ranger is
somewhat noisier.
\item \tiny Only Jaguar has core differentation; some cores experience 
more high-frequency noise than the others.
%\item \tiny 
\end{itemize}

\begin{figure}
\label{fig:uWSched-stencil}
\begin{center}
\includegraphics[scale=0.39]{../plots/timings_jaguar}
\includegraphics[scale=0.39]{../plots/timings_ranger}\\
\end{center}
\caption{\label{fig:uWSched-stencil} 3D Stencil scaling with weighted
  locality-sensitive scheduling for Ranger~(left) and Jaguar~(right).}
\end{figure}

\begin{enumerate}
\tiny \item \tiny Weighted micro-scheduling is the best on Ranger,
although micro-scheduling is almost as good (dynamic load balancing
most important). 
\item \tiny Jaguar: Weighted is better on fewer than 64 nodes and
  micro scheduling is better beyond that. 
\item \tiny Jaguar: Weighted micro-scheduling is better than both of these. 
\end{enumerate}
{\tiny $\rightarrow$ Combining schedulers has no disadvantage.}
%TODO: figure out the avg/max for load imbalance to calculate static fraction. 
%TODO: max - avg . avg / max ; 
\end{frame}

\begin{frame}
\frametitle{Slack-conscious Hybrid Static/Dynamic Scheduling}
\begin{columns}[T]
  \column{0.5\textwidth}
         {\small  \underline{\textbf{Introduction:}}} \\
         \begin{itemize}
           \tiny \item \tiny We first discuss limitations of
           experimentally tuning the lightweight scheduling scheme,
           including the problem of slack. 
         \item \tiny With this, we explain the two techniques of this
           chapter: model-guided lightweight scheduling and
           slack-conscious scheduling technique. 
         \item \tiny We evaluate our slack prediction mechanisms,
           showing scaling results for our model-guided lightweight
           scheduling. 
         \item \tiny We show why our slack-conscious runtime
           adjustment is essential and beneficial at large-scale, by
           showing speedup of slack-conscious scheduling over best
           static fraction (\textit{besf}) scheduling strategy. 
         \end{itemize}
         %\vrule[0.5em]{1pt}{3.5in]
\vrule{} 
         \column{0.5\textwidth}
                {\tiny  \underline{\textbf{Scheduling Strategy Optimizations}}} \\
                \begin{itemize}
                  \tiny \item \tiny \textcolor{gray} {mixed static/dynamic $\rightarrow$ hybrid static/dynamic  (base)}
                \item \tiny  \textcolor{gray} {lightweight scheduling (which includes locality-aware scheduling and skewed workloads)}
                \item \tiny \textcolor{gray} {weighted locality-sensitive scheduling}
                \item \tiny \textcolor{black} {slack-conscious scheduling}
                \item \tiny \textcolor{white} {staggered hybrid static/dynamic scheduling}
                \end{itemize}
                    {\tiny \underline{\textbf { Methodology for Tuning Scheduler Parameters}}}
                    \begin{itemize}
                      \tiny \item \tiny \textcolor{white} {experimental tuning}
                    \item \tiny \textcolor{black} {model-guided optimization} %TODO: change to pruning search space 
                    \item \tiny \textcolor{black} {runtime adjustment}  %TODO: add within-node runtime adjustment and across node adjustment 
                \end{itemize}
\end{columns}
\end{frame}

%TODO: get another graph - this may not show the results as well. 
%TODO: show a caption

\begin{frame}
\frametitle{Large and Complex Search Space}
%TODO: get new picture 
%Also, Could maybe show results for enumeration of search space, and that it's large here. 
%TODO: fix the below 
%Put in modeled graph with two endpoints, showing non-continuous search space. 
\begin{enumerate} 
\item Difficult to find exact static fraction for single node. 
\item Best static fraction changes per node because MPI collective operations involve ``slack''(explained more in depth on next slide).
\end{enumerate}
%\begin{figure}
%\includegraphics[scale=0.20]{../plots/vary-sf-for-diff-threads}
%\caption{\small Graph showing the performance of different static fractions
%  used. Here, we have tried different static fractions, 
%  in increments of 0.1. The minima here is difficult to attain.}
%\includegraphics[scale=0.28]{../plots/best-fs-vs-nodes}
%\end{figure}
\end{frame}

%TODO: add in other data sets.  
%TODO: add in max/avg. 

\comments{
\begin{frame}
\frametitle{Can We Use Theoretical Analysis to Prune the Large Search Space?}
% Model-guided optimization, but with experimental tuning after
% scheduler parameter from model is determined. 
\begin{center}
\includegraphics[scale=0.26]{../plots/app-scaling-strat-SNAP-fastNUMA2}
\includegraphics[scale=0.26]{../plots/app-scaling-strat-nbody-fastNUMA2}\\
\end{center}
\begin{center}
{\small Using a theoretical analysis to acheive the minima helps to provide 13-16\% performance gains for a particle simulation (nbody) and SNAP (reg. mesh), each tested on a 16-core non-NUMA architecture and 16-core NUMA architecture.}
\end{center}
\end{frame}
}

\begin{frame}
\frametitle{Different MPI processes Have Different Cost to Application Critical Path}
{\tiny \underline{\textbf{MPI Slack}} is the time by which a process
  can delay arrival at a synchronizing point without impacting application execution time.}
%
\begin{table}[tr]
\label{fig:collectiveSlack-useq}
  \begin{center}
    \begin{tabular}{ | c || c | c || c |}
      \hline
      \textbf{Collective} &      \textbf{max} & \textbf{avg ($\sigma$)} & \textbf{$\sigma_i$} \\ 
\hline
      \texttt{Allreduce} &             307    &    199 (.792) &            .102 \\ 
\hline
      \texttt{Alltoall}  &             280    &    149 (v.821) &            .176 \\ 
\hline
      \texttt{Barrier}  &              250    &    139 (.178)  &           .061 \\ 
\hline
      \texttt{Allgather} &             268    &    189 (.527) &            .115 \\
\hline
      \texttt{Reduce\_scatter} &       459    &    296  (.649)  &          .129 \\ 
\hline
    \end{tabular}
  \end{center}
  \caption{\label{fig:collectiveSlack-useq} {\small Different slack statistics (in $\mu$s) shown for experiments with different collectives at the end of an iteration. Additional evidence and impact of slack esp. at large scale is discussed in [slackPerf] and [adagio].}}
\end{table}

\begin{itemize}
\tiny \item \tiny Slack distribution is different for different collectives.
\item \tiny Slack varies across processes.
\item \tiny The distribution of slack on different processes evolves slowly for different applications. 
\end{itemize}
\end{frame}

\begin{frame}
\frametitle{Terminology and Performance Model}
\begin{table}[tr]\centering
\begin{tabular} {|c|p{.75\columnwidth}|} \hline
\multicolumn{2}{|c|}{\textit{Model outputs}} \\\hline
$f_d$       & {\tiny Fraction of work scheduled dynamically.} \\\hline
$f_d'$      & {\tiny Fraction of work scheduled dynamically,\newline
                      reduced for slack.} \\\hline
%
\hline
\multicolumn{2}{|c|}{\textit{Model inputs}} \\\hline
$q$                     & {\tiny Overhead of dequeueing single iteration.} \\\hline
$S_i$                   & {\tiny Slack duration on process $i$.} \\\hline
$\delta$        & {\tiny Expected noise duration.} \\\hline
$T_p$       & {\tiny Execution time on $p$ cores with fully static scheduling and no noise.} \\\hline
$N$         & {\tiny Total number of iterations across all threads.}\\\hline
% 
$t_1$       & {\tiny Duration of single iteration on a single thread, on one core.} \\\hline
$d$                     & {\tiny Duration of single dynamic iteration on a single thread.} \\\hline
%$\eta$          & Compute delay due to noise and queue overhead \\\hline
%$\alpha_i$      &  Delay added to the critical path by node $i$\\\hline
\end{tabular}
\caption{\small Overview of all model parameters. Note that we use
  the dynamic fraction $f_d$ instead of static fraction $f_s$ to simplify our
  theoretical analysis.\label{tab:parameters}}
\end{table}
\end{frame}

\begin{frame}[label=mghybridstatdyn]
\frametitle{Model-guided Hybrid Static/Dynamic Scheduling}
{\small Instead of tuning the dynamic fraction, we establish a model to
determine it based on noise duration $\delta$, and other model parameters. \\}
\begin{center}
  \fbox{\includegraphics[scale=0.46]{../images/hybSched-nonoisenoslack}} \\
\end{center}

{\small Considering the above model, along with our goal of avoiding load balancing until it is absolutely needed, 
the dynamic fraction applied to each MPI process is: \\}
\begin{center}
  \fbox{{$f_d = \frac{p\delta}{N\left(t_1 + q\right)}$}}\\
\end{center}
    {\small We refer to the strategy described above as Model-guided Hybrid
      Static/Dynamic Scheduling. This strategy is referred to as 
      \textit{Static-Hybrid} in the graphs.} 
\end{frame}

\begin{frame}
\frametitle{Considering Slack Factor for Hybrid Scheduling}
{\small With this, let us now consider the slack factor for each MPI process.} 

\begin{figure}
  \begin{center}
    \includegraphics[scale=0.46]{../images/slackConsciousSched} \\
  \end{center}
\end{figure}

\begin{itemize}
\tiny \item \tiny Slack allows us to reduce the dynamic fraction
further, because a small amount of delay due to noise will lead to no
delay in the completion of an MPI operation. For example, if you are
in a Reduction and waiting for data from your spanning tree
children. 
\item \tiny Note that the slack is different on different nodes,
  depending on the implementation of the collective operation. 
\item \tiny Therefore, the dynamic fraction will be different on
  different processors. Nodes with large slack can get away with 
small dynamic fraction without impacting application performance.
\end{itemize}
\end{frame}

\begin{frame}[label=slackfs]
\frametitle{Slack-conscious Hybrid Static/Dynamic Scheduling
  a.k.a. Adaptive Scheduling} 
\begin{figure}
  \begin{center}
    \includegraphics[scale=0.41]{../images/slackConsciousSched} \\
  \end{center}
\end{figure}
\begin{center}
\fbox{{$f_d = \frac{p\delta}{N\left(t_1 + q\right)}$}}
\fbox{ {$f_d' = f_d - \frac{S}{(p-1)\frac{Nt_1}{p} - Nq}$}}
\end{center}
\begin{center}
{\tiny Note that $f_d'$ and $S$ are local variables within each MPI process. }
\end{center}

%TODO: 
{\small The dynamic fraction $f_d$ of a particular node
 is reduced by a value proportionate to the slack on that node, 
to get the new $f_d'$.} \\

{\small To estimate the slack on each processor, we use ideas from
 Adagio (see next slide for impl). \\ We compared multiple methods,
 and chose the \textit{callsite\_fd} method.} \\

{\small Reducing dynamic fraction reduces overall overhead.}

\end{frame}

\comments{
\begin{frame} 
\frametitle{Slack Prediction Mechanism and Usage}  
\begin{enumerate}
\tiny \item \tiny On each process, retrieve that process's invocation of the
  last MPI collective, where the invocation of the last MPI collective is
retrieved through the callsite slack-prediction method (see paper for
details on implementation details of slack prediction).
 \item \tiny Given the identifier of the last the MPI collective call
invoked, estimate that collective call's slack value from the
history of slack values stored by the slack-conscious runtime.
The slack estimate is based on the slack value recorded
in the previous MPI collective invocation, as is done in the
cited work of Adagio.
 \item \tiny On each process, adjust its dynamic fraction based on the
   slack value. This adjustment is done using the formula 5 in section
   3 of Chapter 4  and implementation of section 4 of
   Chapter 4. The static fraction used in the
   loop bound is 1- $f_d$.
\end{enumerate}
\end{frame} 
}

\begin{frame}
\frametitle{Software Architecture}
\begin{figure}
\begin{center}
\label{code:architecture}
\includegraphics[scale=0.56]{../images/architecture}
\end{center}
\end{figure} 

\begin{itemize}
\tiny \item \tiny The white rectangles taken by themselves show the
interaction between the MPI+OpenMP application code and the
MPI \& OpenMP runtimes.  
%how a basic MPI+OpenMP
%application is run.  
\item \tiny Our contributions in this work are shown in gray
  rectangles. %Involves source-to-source transformation and adaptiv
\item \tiny Predict static fraction call (arrow going from compilation
  to Adaptive runtime library) is how MPI runtime
  communicates with OpenMP runtime to tune OpenMP scheduler parameters
  (and other parameters). 
\end{itemize}
\end{frame}

\begin{frame}
\frametitle{Code Transformation}
\vspace*{-0.4in}
\begin{columns}
\column{0.5\textwidth}
\begin{figure}
\label{code:origOMPcode}
%\subfloat[\label{code:origOMPcode}{\tiny Loop transformed for slack-conscious scheduling.}]
\begin{center} 
\lstinputlisting{../listings/omp-static.c}
\end{center} 
\end{figure}
\column{0.5\textwidth}
\begin{figure}
\label{code:refactoredOMPcode}
\begin{center} 
\lstinputlisting{../listings/omp-hybrid.c}
\end{center} 
%\caption{\label{code:refactoredOMPcode}\tiny Loop transformed for
%  slack-conscious scheduling.}
\end{figure}
\end{columns}
\begin{enumerate}
\tiny \item \tiny The \texttt{decide\_static\_fraction()} and 
\texttt{endLoop()} functions are calls to the slack-conscious runtime. 
\item \tiny The runtime maintains a \texttt{LoopTimeRecord} struct
  \texttt{lr} across invocations for a threaded computation region. 
  The \texttt{LoopTimeRecord} struct includes storage for
  parameters of loop time, slack, est. sched overhead time, and number of
  iterations, along with various timers obtained at different points during
  the execution of the OpenMP threaded computation region. 
\item \tiny Using wrappers around the MPI collective, the 
  slack-conscious runtime predicts slack, and calculates optimal
  static fraction using the model. 
\item \tiny For the first invocation of the threaded computation
  region, \texttt{lr} is empty, so the runtime system uses static
  fraction of 100\%. 
  % the simpler model without slack to determine the
  % static fraction. 
  % calls around the MPI comm. 
\end{enumerate}
\end{frame}

\begin{frame}[label=runtimeparameterestimation]
\frametitle{Scheduler Parameter Estimation}
%{\small To calculate the value of $\delta$, $\frac{N \cdot t_1}{p}$ 
%  and $q$ for computing the slack-conscious static fraction
%  based on \hyperlink{slackfs}{\beamerbutton{Slide 46}}}:}
\begin{itemize}
\small \item \small The value of the noise parameter $\delta$ is 
estimated in the \texttt{MPI\_Init()} function: we subtract the
maximum and minimum times for the iterations of a basic OpenMP loop
performing a square root operation. 
\comments{and then do an
MPI\_Allreduce() with the max operator to collect the maximum across
all iteration times.}
\item \small We estimate the scheduler overhead $q$ by subtracting 
performance of a statically scheduled OpenMP loop doing a dot product
from a dynamically scheduled OpenMP loop doing a dot product.  
\item \small The time $\frac{N \cdot t_1}{p}$ is estimated by taking 
the time taken for statically scheduled section. 
 \comments{of the last 5 invocations of the threaded computation 
region and taking the minimum execution time of these times.} We 
obtain $t_1$ by dividing this time by the number of iterations
executed statically, and use it to calculate $\frac{N\cdot t_1}{p}$.  
% during the execution of the threaded computation
% region, 

%Note that this gives an estimate of 
%scheduler overhead, in order to give a tighter bound on the static 
%fraction. 
\end{itemize}
%\comments{ 
\begin{enumerate}
\tiny \item \tiny We perform our estimates within the MPI\_Init() 
function (intercepted by PMPI), using a wrapper script. Our calculations needed are done
after the real MPI\_Init() has finished. 
%\tiny \item \tiny The value for noise estimated is 5.4\% greater than that of netgauge
%noise measurements.(possible future direction for netgauge is to make
%it output noise event length delta, rather than percent noise). 
%\tiny \item \tiny Note that values noise and scheduler overhead are estimates, rather
%than the exact values; as we will see in the results section, 
%these estimates are adequate to give us significant performance
%gains. 
\end{enumerate}
%}
\end{frame}

\begin{frame}
\frametitle{Process-local Slack Measurement}
%{\small To calculate the slack S, for computing the slack-conscious static fraction based on 
%\hyperlink{slackfs}{\beamerbutton{Slide 46}}:} 
\begin{itemize} 
\small \item \small \textbf{Callpath.} Associate slack trace values
with only the type of collective operation that had the slack. 
\item \small \textbf{Collective.} Storing the full callpath requires
  overhead of unwinding the stack. Associate slack trace values with
  only the type of the collective operation that had the slack.
\item \small \textbf{Naive.} Forget slack tracing. Predict
  slack based on a precomputed experiment. Specifically, 
  run 1,000 invocations of a simple MPI\_Allreduce() with a
  message size of 4 bytes on all MPI processes before the
  application begins, i.e., right after MPI\_Init(). 
\end{itemize}

\begin{enumerate}
\tiny \item \tiny Tradeoff between runtime overhead for the slack measurement, and the 
advantage due to Adaptive Hybrid Static/Dynamic Scheduling.
\item \tiny Note that this implementation is based on the Adagio
runtime, and that we did not use the Adagio runtime directly. 
\end{enumerate}
\end{frame}

\begin{frame}
\frametitle{Slack Prediction Mechanism Overheads and Errors}
\begin{itemize}
\tiny \item \tiny \textit{Slack prediction error} for a particular MPI
region is the percentage difference 
between the predicted slack before the MPI region begins 
and the actual slack recorded at the end of the MPI region.
\item \tiny \textit{Slack prediction overhead} for a particular MPI
  region is the time spent in our slack prediction runtime.
\end{itemize}
\begin{figure}
\includegraphics[scale=0.30]{../plots/slackErr-fastNUMA2}
\includegraphics[scale=0.30]{../plots/slackOvhd-fastNUMA2} \\
\includegraphics[scale=0.30]{../plots/slackErr-rzuseq}
\includegraphics[scale=0.30]{../plots/slackOvhd-rzuseq}
%TODO: see if schemes or runtimes is better
\caption{\tiny Overheads and errors of different available slack
  prediction schemes in our runtime system. Each process sums the
  total error and overhead across all MPI regions, and we report below
  average overhead and error across MPI processes.}
\end{figure}
\end{frame}

\begin{frame}
\frametitle{Improving Performance through Model-guided Optimization And Slack-Conscious Scheduling}
%\frametitle{Can We Use This Slack Information to Adjust the Scheduler Parameters So As to Reduce this Cost?}
\begin{figure}
\includegraphics[scale=0.34]{../plots/app-scaling-fastNUMA2}
\caption{\small Speedup of slack-conscious hybrid static/dynamic scheduling over OpenMP static scheduling for cab with varying number of nodes.}
\end{figure}
\begin{itemize}
\tiny \item \tiny Using a theoretical analysis to acheive the minima
with Static-Hybrid helps to provide 13-16\% performance gains for a
particle simulation (nbody) and SNAP (reg. mesh), each tested on a
16-core non-NUMA architecture and 16-core NUMA architecture.
\item \tiny Slack-conscious scheduling  provides additional gains
  of 23.5\% for NASLU-MZ and 25.5\% for SNAP.
\end{itemize}
\end{frame}

\begin{frame}
\frametitle{Improving Performance through Model-guided Optimization And Slack-Conscious Scheduling}
%\frametitle{Can We Use This Slack Information to Adjust the Scheduler Parameters So As to Reduce this Cost?}
\begin{figure}
\includegraphics[scale=0.34]{../plots/app-scaling-rzuseq}
\caption{\small Speedup of slack-conscious static fraction over OpenMP
  static for rzuseqwith varying number of nodes.}
\end{figure}
\begin{itemize}
\tiny \item \tiny Using our scheduling strategy does not incur
significant performance degradation (less than 1\% on all node counts
tested) for a non-noisy machine. 
\item \tiny We are able obtain some performance gain (5.5\% at 1024 
  nodes) for our amg experiment which has a small amount of application load imbalance in the code.
\end{itemize}
\end{frame}

\begin{frame}[label=bestfsvsadpt]
%\frametitle{Why Should We Care About Model-guided and Runtime-guided Optimizations to Our Lightweight Scheduling Strategies?}
\frametitle{Benefits of Using Runtime Adjustment over Experimental Tuning}
\begin{figure}
\includegraphics[scale=0.50]{../plots/best-fs-vs-adpt}
\caption{\small Shows how runtime-adjusted static fraction for each application run on cab
works better than the tuning approach (which is best static fraction
(\textit{besf}) strategy). Speedup of slack-conscious static fraction
over best static fraction is shown.}
%This shows the importance of model-guided optimization and
%runtime-based adjustment (esp. with increasing nodes). }
\end{figure}
\end{frame}

\begin{frame}[label=locOptHybrid]
\frametitle{Locality-Optimized Mixed Static/Dynamic Scheduling}
\begin{columns}[T]
\column{0.5\textwidth}
{\small \underline{\textbf{Introduction:}}} \\
\begin{itemize}
  \comments{\tiny \item \tiny We need to avoid disturbing spatial locality while also making our scheduler as lightweight as possible.}
  \tiny \item \tiny We first describe our locality-optimizations,
  explaining why the original disturbs spatial locality and how we can
  improve it while maintaining our original lightweight scheduling
  scheme's benefits. 
\item \tiny We describe our implementation of a library that is
  invoked within an OpenMP threaded computation to schedule
  iterations, explaining that it is low overhead compared to OpenMP
  static.  
%\item \tiny We show results for a Barnes-hut pthread code and regular mesh OpenMP code using our scheduling library. 
\item \tiny We show results for a Barnes-hut and regular mesh code 
  using our scheduling library.
\item \tiny We show results that indicate that our locality-optimized 
strategy controls the performance variation (discussed
\hyperlink{perfVar}{\textcolor{blue}{here}}) just as well or better than
does the basic hybrid static/dynamic scheduling. 
%\comments{(which is beneficial for scalability)}. 
\end{itemize}
\vrule{}
\column{0.5\textwidth}
       {\tiny  \underline{\textbf{Scheduling Strategy Optimizations}}}\\
       \begin{itemize}
         \tiny \item \tiny \textcolor{gray}{ mixed static/dynamic $\rightarrow$ hybrid static/dynamic  (base)}
       \item \tiny  \textcolor{gray}{lightweight scheduling (which includes locality-aware scheduling and skewed workloads)}
       \item \tiny \textcolor{gray} {weighted locality-sensitive scheduling}
       \item \tiny \textcolor{gray} {slack-conscious scheduling}
       \item \tiny \textcolor{black} {staggered hybrid static/dynamic scheduling}
       \end{itemize}
    {\tiny \underline{\textbf { Methodology for Tuning Scheduler Parameters}}}
    \begin{itemize}
      \tiny \item \tiny \textcolor{black} {experimental tuning}
    \item \tiny \textcolor{white} {model-guided optimization} %TODO: change to pruning search space 
    \item \tiny \textcolor{white} {runtime adjustment} %TODO: add within-node runtime adjustment and across node adjustment
    \end{itemize}
\end{columns}
\end{frame}

\comments{
low-probability term  for terms  -     precise  approximation .    

-    correctness of the equality . 
- intuition 

-  storyboard   

simple techniques can make your code faster.   
}

\begin{frame}
\frametitle{Improving Spatial Locality in Dynamically Scheduled Iterations}
{\tiny Ignoring slack for the moment, consider the diagram shown in slide 45
  showing hybrid static/dynamic scheduling. \\}
{\tiny  From this diagram, consider the sequence of loop iterations that
  all 4 threads work on. (Note that the x-axis below is increasing
  iteration numbers, and not time ). \\} 
\begin{center}
\includegraphics[scale=0.35]{../images/sdSched} \\
\end{center}
\begin{itemize}
\small \item \small The consecutive iterations typically access consecutive memory locations. 
\item \small Spatial locality benefits static iterations, but is lost in dynamic iterations. 
\item \small Given this, we could get additional performance benefit if we bias distribution of dynamic iterations, so that each thread is likely to execute dynamic iterations 
that are contiguous with its static iterations.  
\end{itemize}
\end{frame}

\begin{frame}[label=sdsSched]
%\frametitle{Question: Can we Re-distribute the Iterations So that Spatial Locality Isn't Lost?}
\frametitle{Spatial Locality in the Hybrid Static/dynamic Scheduling Approach}
\begin{center}
\begin{figure}
\includegraphics[scale=0.35]{../images/sdsSched}
%TODO: update the below, to word better for intent. 
\caption{\small We stagger the dynamic iterations in order to improve spatial locality of the basic static/dynamic scheduling.}
\end{figure}
\end{center}

\begin{itemize}
\small \item \small Strategy: Each thread finishes its static iterations, then does dynamic iterations marked for it if available, and only then looks for other dynamic work. 
\item \small Spatial locality will not be lost in common cases: 
\begin{enumerate}
\small \item \small When there is no imbalance. 
\item \small Even when there is imbalance espeicially for noise, most threads are able to complete their own dynamic iterations.
\end{enumerate}
\end{itemize}
\end{frame}

\begin{frame}[label=efficientImpl]
%\frametitle{Question: Can we develop an efficient scheduler library that handles different tradeoffs between load balance and locality?}
\frametitle{Note on Efficient Implementation of Scheduler Library}
\begin{center}
\input{../plots/runtime-costs}
\end{center}
\begin{itemize}
\small \item \small Overhead of our library is relatively small. 
\end{itemize}
\end{frame}

\begin{frame}[label=claresults]
%TODO: add in varying number of cores to show improvements. 
\frametitle{Performance Improvements with Varying Problem Sizes}
\begin{figure}
\includegraphics[scale=0.40]{../plots/bh-16cores}
\includegraphics[scale=0.35]{../plots/st-16cores}
\caption{\small Shows performance improvements when using our vSched
  library (having staggered static/dynamic scheduling strategy), for a
  Barnes-Hut pthread particle simulation code (right) from the
  Lonestar Benchmark Suite and the SNAP 2D regular mesh code (right).}
\end{figure}
\begin{itemize}
\tiny \item \tiny Using our staggered hybrid static/dynamic
scheduling (\textit{stag}) for the 2D regular mesh code
with 65536 mesh points, we get reasonable performance gain of 6.45\% 
over OpenMP static scheduling. 
\item \tiny Using our staggered static/dynamic scheduling approach for
 the Barnes-Hut code at 100,000 particles, we get significant gains of
 18.53\%.
\item \tiny Our strategy provides increased performance gains as we
  increase the problem size. 
\end{itemize}
\end{frame}

\begin{frame}[label=controlPerfVar]
\frametitle{Controlling Performance Variability}
%\frametitle{Do the optimizations maintain performance consistency characteristics of static/dynamic scheduli}
\begin{center}
\input{../plots/runtime-stddev}
\end{center}
\begin{itemize}
\tiny \item \tiny With dynamic scheduling, performance can vary 
from run to run (even for static/dynamic scheduling).
\item \tiny Runtime standard deviations are controlled with our
  scheduler optimizations done in vSched, probably because of more
  persistent and predictable behavior(because each timestep each
  thread is likely executing the same dynamic iterations). 
\end{itemize}
\end{frame}

\begin{frame}[label=combinedintro]
\frametitle{Scheduler Composition}
\begin{columns}
\column{0.5\textwidth}
{\small \textbf{\underline{Introduction}:}}\\
\begin{itemize}
\tiny \item \tiny Some schedulers may benefit in some cases, others elsewhere.
\item \tiny Problem: Machine and application has all above mentioned factors involved. 
\item \tiny We show how to combine all schedulers. We also apply it to 3 different application codes. 
\item \tiny We discuss application programmer usability. 
\item \tiny Finally, we re-visit the amplification problem, and discuss how our schedulers work at large-scale.
\end{itemize}

\vrule{}
 
\column{0.5\textwidth}
{\tiny  \underline{\textbf{Scheduling Strategy Optimizations:}}}\\
\begin{itemize}
  \tiny \item \tiny mixed static/dynamic (base)
\item \tiny lightweight scheduling 
\item \tiny weighted locality-sensitive
\item \tiny slack-conscious scheduling
\item \tiny constrained staggering
\item \tiny $\rightarrow$ Each opens up new opportunities to balance the tradeoff between load balance and locality. % The main \point is that there are many dimensions of scheduler 
\end{itemize}

{\tiny \underline{\textbf { Methodology for Tuning Scheduler Parameters}}}\\
\begin{itemize}
  \tiny \item \tiny experimental auto-tuning
\item \tiny model-guided optimization
\item \tiny runtime adjustment
\item \tiny $\rightarrow$ Each are important mechanisms to find the best tradeoff between load balance and locality.
\end{itemize}
\end{columns}
\end{frame}

\begin{frame}[label=combinedSched]
\frametitle{Components of Example Composed Scheduler}
\begin{itemize}
\small \item \small \textbf{uSched}: Use model-guided optimization
  to determine the static fraction (we have separate cases for load
  imbalanced due to noise and load imbalance due to the application), 
  and then experimentally tune the static fraction $0.05$ below static
  fraction and $0.05$ above static fraction ($0.05$ is a configurable
  parameter). 
\item \small \textbf{vSched}: Almost the same as the \textit{ssds}
  scheduling strategy. The difference here is that we use the 
  static fraction obtained from \textbf{uSch} above
  rather than experimentally tuning the static fraction. 
\item \small \textbf{callsite\_fd}: Almost the same as the Slack-conscious Scheduling strategy. 
  The difference here is that initial static fraction to use is obtained from \textbf{uSched} above. 
\item \small \textbf{comboSched}: We start with the staggered static/dynamic
  scheduling scheme defined in \textbf{vSched} above, and then do the 
  slack-conscious adjustment described in slack-conscious scheduling section.
\end{itemize}
\end{frame} 

\comments{
\begin{frame}[label=combined] 
\frametitle{Scheduler Composition of Example Composed Scheduler} 
\begin{figure}[ht]
\label{fig:sched-composition}
\begin{center}
\includegraphics[scale=0.35]{../pictures/sched-composition-diagram}
\end{center}
\caption{\label{fig:sched-composition} Composition of 3 schedulers
which make ComboSched.}
\end{figure} 
\end{frame} 
}

\begin{frame}[label=combinednbody]
\frametitle{Combined Scheduling Results for Rebound}
{\tiny Rebound(*) is an MPI+OpenMP particle simulation code involving large load
imbalances as well as irregular data access patterns.}
\begin{figure}
\includegraphics[scale=0.43]{../plots/app-scaling-strat-nbody-fastNUMA2}
\caption{\tiny Percent speedup over OpenMP static scheduling for each
  scheduling strategy for the particle simulation code. Run on
  \texttt{cab} using 16 threads per node (and 1 MPI process per
  node).}

\end{figure}
\begin{itemize}
\tiny \item \tiny Combining different techniques seems to add on
benefits, i.e., they don't cancel benefit out. 
\item \tiny Because they are based on complementary factors. Spatial
  locality in the static/dynamic scheduling vs. reducing dynamic
  scheduling. 
%\item \tiny better combination could lead to even more improvement 
\end{itemize}

{ \tiny *Rebound is actually the name a suite of particle simulation application
  codes, and we considered the shearing sheet simulation. Here, we use the  
  name Rebound to identify the shearing sheet application run in our experiments.  }
\end{frame}

\begin{frame}[label=combinedfe]
\frametitle{Combined Scheduling Results for MiniFE}
{\tiny  CORAL miniFE MPI+OpenMP code runs a finite element method. 
  This involves irregular computation, and relatively low load imbalance.} 
\begin{figure}
%\includegraphics[scale=0.38]{../plots/app-scaling-strat-SNAP-craybw}\\
\includegraphics[scale=0.43]{../plots/app-scaling-strat-SNAP-fastNUMA2}
\caption{\tiny Shows how the application of full scheduling to miniFE 
works better than the application of other approaches, by showing speedup over OpenMP
static scheduling for each scheduling strategy for different numbers
of nodes. Run on \texttt{cab} using 16 threads per node
(and 1 MPI process per node).}
\end{figure}

\begin{itemize}
\tiny \item \tiny Dynamic scheduling causes significant degradation 
even though there is imbalance. 
\item \tiny Our scheduling strategies provides notable performance 
improvement over static scheduling, though gains less than performance gains for n-body.
\item \tiny One reason is that code\comments{'s computation} has 
relatively significant amount of across-node load imblance relative
to within-node load imbalance, while n-body has little to no load
imbalance. 
\end{itemize}

\end{frame}

\begin{frame}[label=combinedregmesh]
\frametitle{Combined Scheduling Results for Regular Mesh}
{\tiny CORAL SNAP is an MPI+OpenMP code that does a regular mesh
  computation used in heat diffusion. This computation is similar to
  that in chapter 2.}
\begin{figure}
\includegraphics[scale=0.43]{../plots/app-scaling-strat-fe-fastNUMA2}\\
\caption{\tiny Shows how the full scheduling for SNAP 
works better than other approaches, by showing speedup over OpenMP
static scheduling for each scheduling strategy for different numbers
of nodes. The above is run on \texttt{cab} using 16 threads per node, 
and 1 MPI process per node.}
\end{figure}

\begin{itemize}
\tiny \item \tiny Even when dynamic strategies worsen performance, the
combined strategies show a small benefit. 
%\item \tiny uSched doesn't do as well 
\end{itemize}
\end{frame}

\comments{
\begin{frame} 
\frametitle{Code with Scheduling Composition} 
\begin{figure}[h!t]
              \label{code:dvSchedMacro}
              \begin{center}
                \lstinputlisting{../listings/dvSchedMacro-CG.c}
             \end{center}
              \caption{\label{code:dvSchedMacro}{\small NAS CG code
                  modification using our scheduling lib\
rary.}}
            \end{figure}
\end{frame} 
}

\begin{frame}[label=programmerEffort]
\frametitle{Is Our Method Easy to Use for the Application Programmer?}
{\tiny Rebound: (-add total loc-) SNAP: (-add total loc-)  miniFE: (-add total loc-) \\}
\input{../plots/linesOfCodeChanged}
%Experiment: Show lines of code modified, as shown in fourth quadrant
%of SC poster. 

\begin{itemize}
\tiny \item \tiny Lines of code changed is minimal for all three codes. 
\item \tiny Code transformation via ROSE \comments{(and changes to
  GOMP)} can reduce or eliminate need for code change.
\end{itemize}
%TODO: explain
\end{frame}

\comments{
\begin{frame} 
\frametitle{Load Imbalance Metrics With Increasing Numbers of Nodes}
\begin{figure}
\label{fig:bh-withinAcrossImb-cab}
\begin{center}
  \subfloat[Load Imbalance Across Nodes] {\includegraphics[width=0.5\textwidth] {../plots/imb-maxavg-nbody-cab-inodes}}
  \subfloat[Load Imbalance Across Cores] {\includegraphics[width=0.5\textwidth] {../plots/imb-maxavg-nbody-cab-icores}}
\end{center}
\caption{\label{fig:bh-withinAcrossImb-cab}{Standard load imbalance metrics taken for a galaxy 
    simulation running on 1024 nodes of \textit{cab}, which is an Intel Westmere
    machine with 16 cores per node. The code is run with one MPI process per core.}}
% Overheads of our scheduling runtime.\comments{shown through comparison between our static scheduling implementation and OpenMP static scheduling.}
\end{figure} 
\end{frame}
}
\comments{
\begin{frame}[label=ampProbRevisit]
\frametitle{Amplification Problem Revisited}
{\tiny (Ignore for now.)} 
\begin{figure}
\label{fig:modeledScurve}
\begin{center}
\includegraphics[scale=0.38]{../plots/amplification-model-Scurve1}
\end{center}
\caption{\label{fig:modeledScurve}\small Modeled scaling curve based
  on formula for expected length of the $k^{th}$ iteration,
  described~\hyperlink{noiseampformula}{\beamerbutton{here}}. In the formula, we use I=1000, $\delta$ = 5 ms, $\phi$ = 1.}
\end{figure}
\end{frame}
}

\begin{frame}[label=ampProbRevisit]
\frametitle{Impact to Extreme-Scale}
\begin{itemize} 
 \small \item \small As we go to future machines including exascale
\begin{enumerate} 
 \tiny  \item \tiny  number of nodes increases 
\item \tiny cores per node increases 
\end{enumerate} 
\item most importantly, static and dynamic variabilty increases. 
\begin{enumerate}
\tiny \item \tiny Operating voltage might decrease, which increases
  variability and decreases reliability. 
\item \tiny  Fault rates increase; these will be contained
through multiple mechanisms locally, but this will lead to transient
load imbalances. 
 \end{enumerate} 
\item \small Applications will need strong scaling, as exascale reports say,
  because memory per core will be lower, and scientists would want to
  solve the same problem faster rather than a larger problem at the
  same speed. 
\item \small All of the above reasons will increase the impact of
  amplification. 
\item \small Thereby increasing the need for the kind of schedulers we have developed in
  this dissertation 
\begin{itemize} 
 \item \small Number of cores per node will be substantially higher, which our
techniques can leverage.
\end{itemize} 

\item {\small the multiple techniques that we developed are in
  extensible software library that can be extended for exascale
  machines with newer features and future sophisticated applications. }
% can be
% selectively combined in different 
\end{itemize} 

\comments{
adaptive componets of hardware and software, low power  node 

The number of errors will increase, but it is expected that errors
will be contained localily 

impact of transient events is larger impact of applications .. 
This will lead to   
}
\end{frame} 

\begin{frame}[label=relatedWork]
\frametitle{Related Work}
\begin{itemize}
\item \small DPLASMA: dynamic scheduling for specific Numerical Linear
  Algebra application, not tunable static fraction. 
\item \small Work-stealing: Ignores locality, and can't tune the
  dynamic scheduler. 
\item \small OpenMP loop scheduling: Sarkar, Chapman, deSupinski.
\item \small TBB, Habanero.
%Some work by Sriram et al tries to fix outer
%iteration locality 
\item \small Co-scheduling: Schedule all events together to make only 
  some timesteps be impacted by noise, but can't co-schedule all
  events. 
\item \small Hyperthreading with SMT at LLNL: shows benefits with SMT 
  and can be lightweight, but is a hardware solution (not tunable to
  avoid overheads). 
\item \small Runtime for Irregular applications (Galois): Transient
  load imbalances, but dealing with more frequent irregularity on each
  timestep, and load imbalances are coarse-grained. 
\item \small UPC task scheduler: Can move work across threads with
  some affinity, but this can't be tuned (all is done at runtime).
\end{itemize}
%*Some work/discussions have been done to integrate our scheduler into this runtime
\end{frame}

\begin{frame}[label=conclusions]
  \frametitle{Conclusions}
  \begin{itemize}
  \item \small Performance irregularities create scaling challenges
    and are difficult to handle through conventional techniques. 
  \item \small Locality-aware hybrid static/dynamic scheduling
    provides a way to cost efficiently handle within-node 
    load imbalance induced by performance irregularity. 
  \item \small Formalization and Numerical Linear Algebra Application
    Case Studies. 
  \item \small Additional scheduling techniques and optimizations:
    model-guided determination of $f_d$, adaptive per-thread static fraction, slack-conscious static
    fraction (staggered scheduling for spatial locality in dynamic
    section). 
  \item \small Our runtime does not introduce significant overheads. 
\begin{itemize} 
  \item \small So, even if runtime slowdowns due to performance
    irregularity are small, it can be used.
\end{itemize}  
  \item \small Feasible to integrate into applications through
    compiler transformations and adaptive runtime library. 
  \item \small Demonstrated a novel set of scheduling strategies that
    can be used to overcome challenges posed by dynamic
    irregularities, while minimizing data movement and synchronization
    costs. 
\item \small Directions for future work: (a) can use within MPI-shared memory extensions and
  MPI+X models; and (b) can combine with other across-node load balancers.
%  \item \small Can further refine accuracy of slack prediction through
%   risk model.
%  \item \small Can use mixed non-blocking + blocking collectives,
%  inspired by mixed static/dynamic scheduling.  
    % \item \small Plan to provide more loosely coupled interface between OpenMP and MPI, or between any two programming models.
  \end{itemize}
\end{frame}

%Acknowlegdments of people
%\begin{frame}[Acknowledgements]
\frametitle{Acknowledgements}
%\begin{itemize}
%\item -
%\item -
%\item -
%\end{itemize}
\end{frame}

\begin{frame}[People] 
\frametitle{People} 
\begin{itemize} 
\item William Gropp (UIUC)
\item Torsten Hoefler (UIUC)
\item Bronis de Supinski (LLNL)
\item Todd Gamblin (LLNL)
\item Amanda Peters (collaborators for Human Heartbeat simulation at Harvard University) 
\item Simplice Donfack, Laura Grigori, Ulrike Meier Yang (collaborators for numerical linear algebra at INRIA and LLNL) 
\item Jim Dinan and Darius Buntinas 
and Mike Heroux (collaborators Programming Models) 
\end{itemize}
\end{frame}


%NOTE: this marks the end of the presentation. 
%TODO: figure out if we should add ``questions'' slide.

%\input{thanks}
%People involved in the project 
%\input{people} 


\begin{frame}
\frametitle{Overview: Mgmt} 

\end{frame} 

\begin{frame}
\frametitle{Overview: Comm} 


\end{frame} 









%TODO: add in simple bulk-synchronous code here. 
%TODO: justify why these old graphs are fine. 
\begin{frame}
\frametitle{Example Synchronous MPI code}
\lstinputlisting{../listings/mpi-bulk-synch.c}
\end{frame}


\begin{frame}
\frametitle{Transient Load Imbalance and its Mitigation}
\begin{columns} 
\column{0.5\columnwidth}
{\small Performance Irregularities on individual nodes can lead to
  significant slowdown of synchronous MPI codes. }
\column{0.5\columnwidth}
{\small One promising approach, given multi-core nodes, is to only
re-distribute work within each node (this is inexpensive). } 
\end{columns} 
%{$\rightarrow$ \small Within-node performance irregularities that
%induce load imbalances are as significant a factor for applications as
%across-node load imbalances. The below gives an intuitive
%explanation of why this is so.}

%TODO: animate this to show  right one slide later. 
\begin{figure}[ht]
  \label{fig:NoiseTransImbPic}
  \label{fig:AmplificationPic1}\subfloat[\label{fig:AmplificationPic1}
    \tiny Noise occuring on different nodes in different
    iterations delays every iteration. 
  ]{\includegraphics[height=1.0in]{../pictures/NoiseAmplificationPic4}}
  \label{fig:AmplificationPic3}\hspace*{0.25in}\subfloat[\label{fig:AmplificationPic3}
    \tiny Execution times is significantly reduced, if we assume load
    can be perfectly re-distributed within each node. 
  ]{\includegraphics[height=1.0in]{../pictures/NoiseMitigatedPic5}}
  \caption{\label{fig:NoiseTransImbPic} Schematics of application
    timelines showing how impact of noise can be mitigated by
    idealized within-node work re-distribution.} 
\end{figure}
\end{frame}
%TODO: finalize that this should go here. We explain that the problem
%will become even greater when laod imbalance is.  
\begin{frame}[label=workredistributionNoise]
\frametitle{Within-node Load Imbalance May Help}
\begin{figure}[ht!]
\label{fig:stat-histo}
\begin{center}
\includegraphics[scale=0.16]{../images/IterTimesHisto-outliers-static}\\
%\includegraphics[scale=0.29]{../images/IterTimesHisto-outliers-dynamic}
%\includegraphics[scale=0.35]{../images/IterTimingsHisto-dynamic}
\end{center}
\caption{\label{fig:stat-histo} \small Iteration timing distribution across 1000 timesteps of the
statically scheduled\comments{\hyperlink{3Dmesh}{\beamerbutton{3D
      reg. mesh}}} 3D regular mesh computation on one node of Power5+. }
\end{figure} 

\comments{
\begin{figure}[h!]
\label{fig:work-redistribution-histo-ideal}
\begin{center}
\fbox{\includegraphics[scale=0.23]{../images/emptySchedImage}}\\
{\tiny todo - need to add (idealized) iteration timing histogram here with single bar at
  6.75 ms} 
\end{center}
%TODO: could re-word the below further. 
\caption{\label{fig:work-redistribution-histo-ideal} \small  }
\end{figure} 
}
\begin{itemize} 
\tiny  \item \tiny Iteration timing distribution
  for 3D regular mesh across 1000 timesteps using no work
  re-distribution\comments{across cores} during each iteration cause
  bi-modal distribution. The second mode on the right causes 
  performance degradations. 
 \item \tiny To minimize the impact of noise to application performance, 
  we would need to re-distribute work in a way that causes the above
  histogram be the one with minimal spread and mean.
\end{itemize} 

\end{frame}

\begin{frame}[label=mitigationWithinNode]
\frametitle{Within-node Persistent Load Imbalance and Mitigation of It}
\begin{figure}[ht]
\label{fig:AppPersistentImbPic}
\subfloat[\tiny A modeled application timeline having persistent load
imbalance across nodes, and cores during execution.]{\includegraphics[height=1.0in]{../pictures/PersistentCores}}
\label{fig:AmplificationPic3}\hspace*{0.25in}\subfloat[\label{fig:AmplificationPic3}\tiny
  A modeled application timeline with work re-distributed across cores,
to reduce load imbalance.]{\includegraphics[height=1.0in]{../pictures/PersistentNode}}
\caption{\label{fig:AppPersistentImbPic} Application-induced load imbalances, which are typically jagged and persistent, are also mitigated by within-node re-balancing. }
\end{figure}
\end{frame}

\begin{frame}[label=ImbWithinNode]
%\frametitle{Performance Doesn't Scale with Cores as Exp}
\frametitle{Within-node Imbalance May Help}
{\small We measure the load on each core for an n-body application code called Rebound.}\\
{\small Mitigated Imbalance: Assuming load within each node is perfectly balanced across its cores without any overhead.}
\hspace*{2.5in}\includegraphics[scale=.12]{../pictures/persistentImb}\hspace*{1in}\includegraphics[scale=.12]{../pictures/mitigatedImb}
\vspace*{-0.1in}
\begin{center}
\input{../plots/bh-withinAcrossImb-cab}
\end{center}
%TODO: show calculation for this. 
\begin{center}
{\tiny Imbalance is measured as $\frac{{max\{load_{i}\}}}{avg\{load_{i}\}}$, where $i$ is the processing element.}
%TODO word the below better 
\end{center}
\end{frame}

%TODO: see if we can change title 
\begin{frame}[label=dynLoadBal]
\frametitle{So, can dynamic load balancing fix this?}
\begin{itemize}
\item Dynamic load balancing within a node can improve performance. 
\begin{enumerate}
\item For both kinds of imbalances. 
\item Persistent imbalance can also be addressed by across-node balancing (avail. in Charm++, Zoltan), but this is complementary.
\end{enumerate}
\vspace*{.5in}
\item However, within-node dynamic load balancing faces many challenges. 
\end{itemize}
\end{frame}

\begin{frame}[label=idleTimeCalc]
\frametitle{Challenge 1: Idle Time due to Within-node Imbalance}
{\tiny How we measured idle time:}
\begin{center}
\lstinputlisting{../listings/omp-CG-idleTimeCalc.c}
\end{center}
\end{frame}

\begin{frame}[label=basictimings]
\frametitle{Challenge 1: Idle Time due to Within-node Imbalance}
{\tiny \textit{\textbf{Setup and Methodology:}} Used NASLU OpenMP
  benchmark code and Barnes-Hut pthread application code. Ran on one
  node of LLNL cab cluster, which is an Intel Xeon 16-core with CHAOS
  operating system.\comments{using gcc optimization level 'O3'}. \comments{having OpenMP3 and mvapich3}}
\visible<1->{
\begin{figure}[t]
\begin{center} 
\includegraphics[scale=0.35]{../plots/dmTime-nbody-cab-idleonly}
\includegraphics[scale=0.35]{../plots/dmTime-NASLU-cab-idleonly}
\end{center}
\caption{\small Breakdown of execution time for a load imbalanced code
  (left) and a load balanced code (right) on cab, shown for the
  3\comments{available}OpenMP scheduling strategies.}
\end{figure}}

\visible<1->{
\begin{itemize}
\tiny {\item \tiny  Dynamic or guided scheduling eliminates idle times in both codes.}
\tiny {\item \tiny However, for NASLU, the total execution time actually increases with both dynamic and guided scheduling.}
%TODO : make within-node dynamic balancing clear. 
\begin{itemize}
\tiny {\item \tiny  Recall that (within-node) dynamic load balancing is needed even for naturally balanced applications, to handle noise.}
\end{itemize}
\tiny {\item \tiny  For n-body, the execution time does not decrease as much as we expect, based on the idle time in the static scheduling.}
\end{itemize}}
\end{frame}

%TODO: word the expected overhead of idle time better. 
\begin{frame}
\frametitle{Why is the Non-idle Time Increasing?}
%\begin{itemize}
%\tiny \item \tiny Scheduling overhead?
%\end{itemize}
\begin{figure}[t]
\begin{center}
\includegraphics[scale=0.35]{../plots/dmTime-nbody-cab-idleonly}
\includegraphics[scale=0.35]{../plots/dmTime-NASLU-cab-idleonly}
\end{center}
\end{figure}
\begin{center}
{\small Breakdown of execution time for a load imbalanced code (left) and a load balanced code (right) on cab.}
\end{center}
{\small (Answer: Mostly data movement, as we will see in the next few slides.)}
\end{frame}

%TODO: check cause/challenge here and in thesis. 
\begin{frame}
\frametitle{Challenge 2: Synchronization Overhead}
{\tiny \textbf{\textit{Methodology}}: Obtain call-path profile from
  hpcToolkit which gives percent time spent in \texttt{omp\_lock()}
  OpenMP runtime function.}
{\tiny comp: computation time,  dq: locking and synchronization, 
Dm:  data movement, due to extra cache misses, etc. } 
\begin{figure}[t]
\label{fig:dmTimes-cab}
\begin{center}
\includegraphics[scale=0.35]{../plots/dmTime-nbody-cab}
\includegraphics[scale=0.35]{../plots/dmTime-NASLU-cab}
\end{center}
%\caption{\label{fig:dmTimes-cab} \tiny Breakdown of time. Synchronization overheads are shown in green.}
\end{figure}
\begin{center}
{\small Breakdown of time. Synchronization overheads are shown in green.}
\end{center}
\begin{itemize}
\tiny \item \tiny Synchronization overheads exist when using dynamic or guided scheduling.
\item \tiny While seemingly insignificant, they account for roughly 5\% percent of execution time for both of these applications. 
\end{itemize}
\end{frame}

%TODO: figure out whether we need data movement on BG/Q. 
%TODO: figure out whether we need to show data movement on bw. 
%TODO: see if you should show static time, to explain prefetching benefit. 
%TODO: separate NUMA and non-NUMA slides.
%TODO: add in details on architecture/platform settings. 
%TODO: add in job script specifics.
%TODO: add number of nodes run on.
%TODO: add in timestep length and problem size. 
%TODO: explain why for different setups 

\begin{frame}
\frametitle{Challenge 3: Data Movement}
{\tiny \textit{\textbf{Methodology}}: Measured L2 and L3 cache misses using PAPI counters, placed just before and just after OpenMP threaded computation region. \comments{(Note: will add code in next slide if I can} } 
\visible<1->{
\begin{figure}[t]
\label{fig:cacheMisses-cab} 
\begin{center}
\includegraphics[scale=0.30]{../plots/L2-cacheMisses-nbody-cab}
\includegraphics[scale=0.30]{../plots/L2-cacheMisses-NASLU-cab}\\
\includegraphics[scale=0.30]{../plots/L3-cacheMisses-nbody-cab}
\includegraphics[scale=0.30]{../plots/L3-cacheMisses-NASLU-cab}
\end{center}
\end{figure}
\begin{center}
{\small Cache misses with L2 misses on top and L3 on bottom, for different OpenMP scheduling strategies.}
\end{center}
}
\visible<1->{
\begin{itemize}
{\tiny \item \tiny For both codes, cache misses increase significantly with dynamic or 
guided scheduling, suggesting an  explanation for the performance loss.}
\end{itemize}
} 
\comments{ {\footnotesize Cache miss calcs in thesis write-up  } }
\end{frame} 

\begin{frame}
\frametitle{Calculations for Cache Misses for NASLU} 
{\small To verify that cache misses are in fact impacting performance 
(and is in fact the cause of the increase in data movement time), we show calculations below. }\\
{\tiny L2 cache miss latency is 100 - 300 cycles (with a 2.66 Ghz processor, this is 4 - 8 ns). }\\
{\tiny L3 cache miss latency is 100 ns. }\\
\begin{figure}[ht]
\label{fig:cacheMisses-cab-forCalcs}
\begin{center}
\includegraphics[scale=0.20]{../plots/L2-cacheMisses-NASLU-cab}\includegraphics[scale=0.20]{../plots/dmTime-NASLU-cab}
\end{center}
\end{figure} 

\begin{itemize}
\tiny \item \tiny We need to account for 485.52 - 83.53 - 24.65 = 
376.34 seconds of time spent in dyn sched. Note that we don't need the
execution time breakdown here, since there is low idle time for this code. 
\item \tiny Cache miss time for L2 is $\frac{200}{2.66 \times 10^9}
  \cdot (3845- 2257 \times 10^6) = 65.04$ s. 
\item \tiny 376.34 - 65.04 = 311.44 seconds to account for.
\item \tiny Cache miss time for L3 is $(100 \cdot 10^-9) \cdot (1676
  - 631) \times 10^6 =  97.82 $ s. 
\item \tiny Still 311.44 - 97.64 = 214.20 seconds to account for. 
  The cost of L3 cache miss may be much higher than 100 ns,
  because access to main memory by multiple threads requires threads to wait in memory
  queue. In the worst case, i.e., when all threads access main memory simultaneously, 
  the latency to memory is factor of $(16+(16-mem\_queue\_depth)\times
  100$ ns of the original latency, where mem\_queue\_depth=12 on this
  machine. 
  %We didn't take
  %into account the execution time for memory buffers for , which can
  %make 
\item \tiny Additionally, to verify that memory bandwidth was the
  factor, we ran STREAM on one core and STREAM running all cores, and
  recorded the bandwidth for both runs. 
\item \tiny Memory bw of STREAM triad on one core is 14837.24 MB/s, but on 16
  cores it is only 103669.71 MB/s. The effective mem. latency
  increases by factor of 2.4x, giving 238.4 s instead of 97.64 s for
  Cache miss time for L3. 
\item \tiny The remainder of time is now 311 - 238.34 = 80 s. This may
  due to L2 being higher (also, note that the L2 latency was reported at
  1 core), or by inefficiencies in parallelization. 
\end{itemize} 
\end{frame}

\comments{
\begin{frame}
\frametitle{Calculations for Cache Misses for bh}
Similarly, for n-body: 
\begin{figure}[ht]
\label{fig:cacheMisses-cab-forCalcs}
\begin{center}
\includegraphics[scale=0.30]{../plots/L2-cacheMisses-nbody-cab}\\
\includegraphics[scale=0.43]{../plots/dmTime-nbody-cab}
\end{center}
\end{figure}
\end{frame}
}
%TODO: consider new title 
% possible Question from audience: why is guided different from mixed
% static/dynamic?  Answer: because it doesn't preserve outer iteration
% data locality, and because on the first chunks, there is dqueue
% overhead. 
% Also, the sizes of the tasks can be controlled in guided, but can't
% handle staggering, i.e., spatial locality. 

\begin{frame}
\frametitle{Thesis Objective}
\underline{\textbf{Thesis Objective:}} Design a set of new scheduling strategies that handles all three causes of the problem, i.e., thread idle
time, data movement, and synchronization overhead simultaneously for
many applications and platforms, in the context of bulk-synchronous and loosely synchronous
MPI applications. 
% all at once, which  OpenMP static scheduling and OpenMP dynamic scheduling. 
\end{frame}
%TODO: change the below to the decomp 

\begin{frame}
\frametitle{Key Idea of Solution}
\visible<1->{\small Can we combine a static and dynamic scheduling scheme, which
  simultaneously reduces scheduler overhead and load imbalance
  (respectively) in an intelligent manner?}\\ 
\visible<1->{
\begin{columns}
\column{0.9\textwidth}
In spite of noise, most of the time, all cores are available to
the application. 
\begin{enumerate}
  \small \item \small Allocate a fixed fraction of
  loop iterations statically. (Then do remaining loop
  iterations dynamically). 
\item \small Ratio of static loop iterations to all loop iterations is the \textit{static fraction}.
\item \small \textit{How do we select the static fraction?}  
\end{enumerate}
\end{columns}
}
\end{frame}

%Note:  add in the cost model .
%TODO: check that we brought in ULT load imbalance.

\begin{frame}
\frametitle{Overview of Scheduling Strategies}
\begin{columns}
\column{0.95\textwidth}
       {\tiny \underline{\textbf{Available OpenMP Schedulers}}} \\
       \begin{itemize}
         \tiny \item \tiny \textcolor{black}{OpenMP static scheduling (\textit{static} or \textit{OpenMP-static})}\\ 
       \item \tiny \textcolor{black}{OpenMP dynamic scheduling (\textit{dynamic} or \textit{OpenMP-dynamic})}\\
       \item \tiny  \textcolor{black} {OpenMP guided scheduling(\textit{guided} or \textit{OpenMP-guided})}\\
       \end{itemize}
           {\tiny  \underline{\textbf{Scheduling Strategy Optimizations}}}\\
           \begin{itemize}
             \tiny \item \tiny \textcolor{black}{mixed static/dynamic scheduling(\textit{half})}
           \item \tiny \textcolor{black}{hybrid static/dynamic scheduling(\textit{besf})}
           \item \tiny \textcolor{black} {lightweight scheduling \textit{ligh}, which includes locality-aware scheduling(\textit{loca}) and skewed workloads(\textit{skew})}
           \item \tiny \textcolor{black} {weighted locality-sensitive scheduling (\textit{weig}, \textit{uWsched} or
             \textit{uWldB}), which consists of hybrid static/dynamic
             scheduling and weighted allocation (\textit{wSched} or \textit{wldB})}.
           \item \tiny \textcolor{black} {model-guided Hybrid
             Static/Dynamic Scheduling (\textit{Static-Hybrid}), and
             slack-conscious hybrid scheduling
             (\textit{callsite\_fd})}. Slack-conscious
             hybrid static/dynamic scheduling is referred to as Adaptive Scheduling in the paper.  
           \item \tiny \textcolor{black} {staggered hybrid static/dynamic scheduling (\textit{sds} or \textit{vSched})}
           \item \tiny \textcolor{black}{{Example combination of schedulers (\textit{allStrat} or \textit{full}) }}
           \end{itemize}
               {\tiny \underline{\textbf {Methodology for Tuning Scheduler Parameters}}}
        \begin{itemize}
          \tiny \item \tiny \textcolor{black} {experimental tuning}
        \item \tiny \textcolor{black} {model-guided optimization} %TODO: change to pruning search space 
        \item \tiny \textcolor{black} {runtime adjustment} %TODO: add within-node runtime adjustment and across node adjustment 
        \end{itemize}
\end{columns}
\end{frame}

\begin{frame}
\frametitle{Hybrid Static/Dynamic Scheduling and Lightweight Scheduling} 
\comments{
\column{0.5\columnwidth} 
\begin{enumerate}
\item \textbf{Introduction:}
\begin{itemize}
\tiny \item \tiny Show basic scheduling scheme and provide explanation of why it is beneficial. 
\item \tiny Explain inefficiency of load balanced code due to noise, with just one node. 
\item \tiny Show additional ways to provide low-overhead dynamic scheduling. 
\item \tiny Intuitive explanation of amplification problem, and benefits of our scheduling strategy to scale. 
\end{itemize} 
\item \textbf{Technique Overview}:
  \begin{itemize}
  \tiny \item \tiny \textbf{Scheduling Strategies}:Mixed static/dynamic, Hybrid Static/Dynamic Scheduling,Lightweight scheduling (which includes locality-aware scheduling and skewed workloads)
  \small \item \small \textbf{Tuning Methodology}: Experimental Tuning  
\end{itemize}
\item \textbf{Architectures:}
\begin{itemize} 
\tiny \item \tiny \textbf{cab}: Intel Xeon Cluster at LLNL having 16 cores per node, CHAOS operating system, mpich2, openMP 3.4. 
\item \tiny \textbf{bp}: IBM Power5+ Cluster at Univ. of Illinois having 16 cores per node, IBM AIX Operating System, ibmmpi, lomp. 
\end{itemize}
\item \textbf{Applications:}
\begin{itemize}
\tiny \item \tiny NASLU: 
\item \tiny nbody: 
\item \tiny 3D stencil: 
\end{itemize}
\end{enumerate}
}

\begin{columns}[T]
  \column{0.5\textwidth}
  \textbf{Introduction:}\\
    \begin{itemize} 
      \tiny \item \tiny Show basic scheduling scheme and provide explanation of why it is beneficial. 
    \item \tiny Explain inefficiency of load balanced code due to noise, with just one node. 
    \item \tiny Show additional ways to provide low-overhead dynamic scheduling. 
    \item \tiny Intuitive explanation of amplification problem, and benefits of our scheduling strategy to scale. 
    \end{itemize}
    \vrule{10in}{1pt}
    %\vrule width 10in height 1pt
    \column{0.5\textwidth}  
           {\tiny  \underline{\textbf{Scheduling Strategy Optimizations:}}}\\
           \begin{itemize}
             \tiny \item \tiny mixed static/dynamic $\rightarrow$ hybrid static/dynamic  (base). 
           \item \tiny Lightweight scheduling (which includes locality-aware scheduling and skewed workloads). 
           \item \tiny \textcolor{white} {weighted locality-sensitive scheduling}
           \item \tiny \textcolor{white} {slack-conscious scheduling}
           \item \tiny \textcolor{white} {staggered hybrid static/dynamic scheduling}
           \end{itemize}
               {\tiny \underline{\textbf { Methodology for Tuning Scheduler Parameters:}}}\\
               \begin{itemize}
                \tiny \item \tiny experimental tuning
               \item \tiny \textcolor{white} {model-guided optimization} %TODO: change to pruning search space 
               \item \tiny \textcolor{white} {runtime adjustment}  %TODO: add within-node runtime adjustment and across node adjustment 
               \end{itemize}
\end{columns}  
\end{frame} 

\begin{frame}
\frametitle{Static and Dynamic Scheduling}
\includegraphics[scale=0.30]{../images/legend-dynamic} \\
\begin{columns}
  \column{0.5\columnwidth}
  \lstinputlisting{../listings/threadedCompRegion-static.c}
  \column{0.5\columnwidth}
  \begin{center}
    \includegraphics[scale=0.35]{../images/threadedCompRegion-static} \\
  \end{center} 
\end{columns}
\begin{columns}
\column{0.5\columnwidth}
\lstinputlisting{../listings/threadedCompRegion-dynamic.c}
\column{0.5\columnwidth}
  \begin{center}
    \includegraphics[scale=0.35]{../images/threadedCompRegion-dynamic}\\
  \end{center}
\end{columns}
%\end{block}

\begin{columns}
\column{0.5\columnwidth} 
- 
\column{0.5\columnwidth}
  \begin{center}
    \includegraphics[scale=0.35]{../images/emptySchedImage}
  \end{center}
\end{columns}
\end{frame}

\begin{frame}[label=mixedstatdyn]
\frametitle{Mixed Static/Dynamic Scheduling}
%\begin{block}
\begin{columns}
  \column{0.5\columnwidth}
  \lstinputlisting{../listings/threadedCompRegion-static.c}
  \column{0.5\columnwidth}
  \includegraphics[scale=0.15]{../images/legend-dynamic}\\
  \begin{center}
    \includegraphics[scale=0.35]{../images/threadedCompRegion-static} \\
  \end{center}
\end{columns}
%\end{block} 
%\begin{block}
\begin{columns}
\column{0.5\columnwidth}
\lstinputlisting{../listings/threadedCompRegion-dynamic.c}
\column{0.5\columnwidth}
  \begin{center}
    \includegraphics[scale=0.35]{../images/threadedCompRegion-dynamic}\\
  \end{center}
\end{columns}
%\end{block}
\begin{columns}
\column{0.5\columnwidth}
\lstinputlisting{../listings/threadedCompRegion-mixed.c}
\column{0.5\columnwidth}
\begin{center}
\includegraphics[scale=0.35]{../images/threadedCompRegion-mixed}
\end{center}
\end{columns}
\end{frame}

\begin{frame}[label=halftimings]
\frametitle{Naive Mixed Static/Dynamic Scheduling}
{\tiny Setup and Methodology is the same as that used in
  \hyperlink{basictimings}{\beamerbutton{here}}.} 
%Used NASLU OpenMP benchmark code and Barnes-Hut pthread application
%code. Ran on one node of LLNL cab cluster, which is an Intel Xeon
%16-core with CHAOS operating system.\comments{using gcc optimization
%level 'O3'}. \comments{having OpenMP3 and mvapich3}} 
\begin{figure}[t]
\begin{center}
\includegraphics[scale=0.34]{../plots/dmTime-nbody-cab-withsd}
\includegraphics[scale=0.34]{../plots/dmTime-NASLU-cab-withsd}\\
\end{center}
\label{fig:dmTimes-NASLU-cab}
\end{figure}
\begin{center}
{\small Set static fraction to 0.5: In \textit{half}, threads do the
  first half of their iterations statically. See rightmost bar. }
\end{center}
\begin{itemize}
\tiny \item {\tiny We expect \textit{half} to cut the dynamic overhead in half.}
\item {\tiny The \textit{half} scheduling provides 10.1\% gain over
  OpenMP guided for the nbody code, due to preservation of locality
  across invocations of threaded computation regions and thus less
  data movement across invocations of the threaded computation
  region.}
\item {\tiny For the NASLU code, the \textit{half} scheduling still has high overhead.}
\end{itemize}
\end{frame}

\begin{frame}[label=dmTimes]
\frametitle{Data Movement with Naive Mixed Static/Dynamic Scheduling}
\visible<1->{
\begin{figure}[t]
\label{fig:cacheMisses-cab-withsd}
\begin{center}
\includegraphics[scale=0.30]{../plots/L2-cacheMisses-nbody-cab-withsd}
\includegraphics[scale=0.30]{../plots/L2-cacheMisses-NASLU-cab-withsd}\\
\includegraphics[scale=0.30]{../plots/L3-cacheMisses-nbody-cab-withsd}
\includegraphics[scale=0.30]{../plots/L3-cacheMisses-NASLU-cab-withsd}
\end{center}
%\caption{\label{fig:cacheMisses-cab-withsd} L2 cache misses (top) and L3 cache misses (bottom).}
\end{figure}
\begin{center}
{\small L2 cache misses shown in the top graphs; L3 in the
bottom. \comments{The \textit{half} is added on the rightmost bar.}
% The data shown is based on 100 trials run on a single node of the
% respective machines. Data movement time is obtained by subtracting
% the sum of idle time, dequeue time and computation time, from total
% execution time.
}
\end{center}
}
\visible<1->{\begin{itemize} 
{\tiny \item \tiny With \textit{half}, cache misses decrease with respect to OpenMP guided, for both nbody and NASLU codes.}
\end{itemize}}
\end{frame}

\begin{frame}[label=noPreset]
\frametitle{Problem: A Preset Static Fraction Doesn't Work Effectively for all Application-architecture Pairs}
\begin{itemize}
\item The above shows that the basic static/dynamic scheduling 
improves performance only for nbody on cab. So, we make the strategy more flexible. 
\end{itemize}
\begin{itemize}
\item We expose the static fraction to the application programmer.  
\item What is right blend of static and dynamic scheduling to use? 
\item The application programmer tunes the blend of static and dynamic scheduling.
\item We refer to this as \textit{Hybrid Static/Dynamic Scheduling}. 
\end{itemize}
\end{frame}

\begin{frame}[label=hybridstatdyn]
\frametitle{Hybrid Static/Dynamic Scheduling}
\begin{columns}
  \column{0.5\columnwidth}
  \lstinputlisting{../listings/threadedCompRegion-static.c}
  \column{0.5\columnwidth}
  \includegraphics[scale=0.15]{../images/legend-dynamic}\\
  \begin{center}
    \includegraphics[scale=0.35]{../images/threadedCompRegion-static} \\
  \end{center} 
\end{columns}
%\end{block} 
%\begin{block}
\begin{columns}
\column{0.5\columnwidth}
\lstinputlisting{../listings/threadedCompRegion-dynamic.c}
\column{0.5\columnwidth}
  \begin{center}
    \includegraphics[scale=0.35]{../images/threadedCompRegion-dynamic}\\
  \end{center}
\end{columns}
%\end{block} 

\begin{columns}
\column{0.5\columnwidth}
%TODO: fix code  to be hybrid sched
\lstinputlisting{../listings/threadedCompRegion-hybrid.c}
\column{0.5\columnwidth}
\begin{center}
\includegraphics[scale=0.35]{../images/threadedCompRegion-hybrid}
\end{center}
\end{columns}
\end{frame}

%TODO: think of a better name for title. 
\begin{frame}[label=expTuningSF]
\frametitle{Experimentally Tuning the Static Fraction}
%TODO: explain why increments of 0.01 
%TODO: explain number of trials run 
{\tiny We vary the static fraction environment variable in increments of 0.01. \comments{(We do 0.01 to get enough data points, but more refinement might give less improvement. )} } 
\begin{figure}[ht]
\begin{center} 
\includegraphics[scale=0.19]{../plots/expTuningSF-nbody-cab-WallTime}
\includegraphics[scale=0.19]{../plots/expTuningSF-NASLU-cab-WallTime}\\
\end{center} 
\begin{center}
{\small Performance for different static fractions when using hybrid static/dynamic scheduling strategy for nbody and NASLU. }
\end{center}
\end{figure}

\begin{itemize}
\tiny \item \tiny For n-body, fully dynamic is better than fully static, but static fraction of 49\% gives even better performance. 
\item \tiny For NASLU, the fully static is much better than fully
  dynamic. However, static fraction of 96\% is the best (although
  compared with the 100\% dynamic, the difference between ``best'' and
  fully static may appear small, it is still a significant
  improvement).  
\item \tiny Note that the best static fraction varies across
  application. 
%Comment: make note about half
\end{itemize}
\end{frame} 

\begin{frame}[label=halftimings]
\frametitle{Hybrid Static/Dynamic Scheduling}
\begin{figure}[t]
\begin{center}
\includegraphics[scale=0.34]{../plots/dmTime-nbody-cab-withsdandbesf}
\includegraphics[scale=0.34]{../plots/dmTime-NASLU-cab-withsdandbesf}\\
\end{center}
\label{fig:dmTimes-NASLU-cab}
\end{figure}
\begin{center}
{\small In \textit{besf}, we use the best static fraction. See rightmost bar.}
\end{center}
%TODO: fix this so it doesn't duplicate

\visible<1->{
\begin{itemize}
\tiny \item \tiny When using \textit{besf} scheduling, cache misses decrease w.r.t. \textit{half}, for both nbody and NASLU codes. This is
because we tune to balance the tradeoff between load balance and locality, to minimize the costs in each of the three challenges, simultaneously.
\item \tiny Paper [statdyn1] studies this idea in more depth.
\end{itemize}}
\end{frame}

\begin{frame}
\frametitle{Data Movement with Hybrid Static/Dynamic Scheduling}
\visible<1->{
\begin{figure}[t]
\label{fig:cacheMisses-cab}
\begin{center}
\includegraphics[scale=0.30]{../plots/L2-cacheMisses-nbody-cab-withbesf}
\includegraphics[scale=0.30]{../plots/L2-cacheMisses-NASLU-cab-withbesf}\\
\includegraphics[scale=0.30]{../plots/L3-cacheMisses-nbody-cab-withbesf}
\includegraphics[scale=0.30]{../plots/L3-cacheMisses-NASLU-cab-withbesf}
\end{center}
%\caption{\label{fig:cacheMisses-nbody-cab} L2 cache misses (top) and L3 cache misses (bottom).}
\caption{\small Cache misses for \textit{besf}, added on the rightmost bar. }
\end{figure}
}
%\begin{center}
%The data shown is based on a 100 trials run on a single node of the respective machines. Data movement time is obtained by subtracting the sum of idle time, dequeue time and computation time, from total execution time.
%} 
%\end{center} 
\visible<1->{
\begin{itemize} 
\tiny \item \tiny When using \textit{besf} scheduling, cache misses 
decrease w.r.t. \textit{half}, for both nbody and NASLU codes. This is
because we tune to balance the tradeoff between load balance and
locality, to minimize the costs in each of the three challenges,
simultaneously. 
\item \tiny Paper [statdyn1] \comments{\footcite{dynwork}} studies
  this idea in more depth. 
\end{itemize}}
\end{frame}

\begin{frame}[label=3Dmesh]
\frametitle{3D Regular Mesh process+thread Domain Decomposition}
\begin{columns}
  \column{0.35\textwidth}
  \begin{center}
    \includegraphics[scale=0.35]{../images/appTimestep-noslack}
 \end{center}
\column{0.65\textwidth}

\begin{figure}[ht]
  \label{fig:3DStencil-domainDecomp-stat}
  \begin{center}
    \includegraphics[scale=0.18]{../images/3DStencil-domainDecomp-stat}
  \end{center}
  \caption{\label{fig:3DStencil-domainDecomp-stat} {\small 3D Regular Mesh domain
  decomposition across processes, along with thread partitioning
  of work within each process.}}
\end{figure}
\end{columns}
\end{frame}


\begin{frame}[label=perfVar]
\frametitle{Performance Variation and Absolute Performance for Static Scheduling}
{\tiny \underline{\textbf{Setup and Methodology:}} Run on IBM Power5+ cluster at UIUC which has 16 cores per node\comments{32MB L1 cache and 128MB L2 cache} and AIX operating system.}
\begin{figure}[h!]
\label{fig:stat-histo}
\begin{center}
\includegraphics[scale=0.29]{../images/IterTimesHisto-outliers-static}\\
%\includegraphics[scale=0.35]{../images/IterTimingsHisto-dynamic}
\end{center}
\caption{\label{fig:stat-histo} \small Iteration timing distribution across 1000 timesteps of the statically scheduled\comments{\hyperlink{3Dmesh}{\beamerbutton{3D reg. mesh}}} 3D regular mesh computation.}
\end{figure}
\begin{center}
{\small $\rightarrow$ \textcolor{gray}{Bi-modal distribution for static scheduling.} \comments{Uni-modal distribution for dynamic scheduling, but increased spread and higher mean for dynamic scheduling.}}
\end{center}
\end{frame}

\begin{frame}[label=lightweightDynDefs]
\frametitle{Lightweight Dynamic Scheduling}
\textbf{Tasklet}: A tasklet specifies a set of iterations and records the core that it was executed on in the preceding iteration. 
\begin{enumerate}
\item \textbf{Skewed Workload}: The distribution of tasklets in the shared queue, from beginning to end, is 16 tasklets of size 16, 32 tasklet of size 8, and 64 tasklets of size 4. 
\item \textbf{Locality-aware Scheduling}: On each iteration, a thread tries to search for the tasklet that it executed during the preceding iteration. 
%\item \textbf{Work-stealing}: On each iteration, a thread tries to search for the tasklet that it executed during the preceding iteration. 
\end{enumerate}
\end{frame}

\begin{frame}[label=HybStatDynAppliedto3DMesh]
\frametitle{Hybrid Static/Dynamic Scheduling Applied to 3D Regular Mesh}
\begin{columns}
  \column{0.10\textwidth}

  \column{0.90\textwidth}
  \begin{figure}
    \label{fig:3DStencil-domainDecomp-sd}
    \begin{center}
      \includegraphics[scale=0.18]{../images/3DStencil-domainDecomp-sd}
    \end{center}
    \caption{\label{fig:3DStencil-domainDecomp-hybrid} \small 3D Regular Mesh domain decomposition with Hybrid Static/Dynamic Scheduling Strategy Applied.}
  \end{figure}
\end{columns}
\end{frame}

\begin{frame}
\frametitle{Hybrid Static/Dynamic Scheduling with Lightweight Dynamic Scheduling in Dynamic Section}
{\tiny Setup and methodology the same as \hyperlink{perfVar}{\textcolor{blue}{here}}.} 
\begin{figure}[h!]
\label{fig:skewed2}
\begin{center}
\includegraphics[scale=0.28]{../images/IterTimesHisto-skewedWorkload2}
\end{center}
\caption{\label{fig:skewed2} Iteration timing histograms for 100\% dynamic scheduling, 100\% dynamic scheduling with locality, and 50\% dynamic scheduling.}
\end{figure}
\begin{itemize}
\tiny \item \tiny 50\% dynamic provides better absolute performance,
as well as less performance variation than 100\% dynamic scheduling
without locality and 100\% dynamic scheduling with locality. 
% \tiny \item \tiny Each percent dynamic has a minima for the number of
%  tasklets in the queue, and skewed does well for larger percent
%  dynamic. 
%\item \tiny The best performance is with 4 planes per planes per
%tasklet, so we use this for the subsequent experimentation. 
%As we vary the number of tasklets in the queue, ther
\end{itemize}
\end{frame}

\begin{frame}[label=noiseAmpExample]
\frametitle{Amplification Problem in the Context of Noise}
\begin{itemize}
\item \small On one node running 1000 iterations of length 10 ms. 
\begin{itemize}
\tiny \item \tiny Assume noise (PI) is of length 5 ms, and it occurs approximately every second.
\item \tiny Execution time without PI 10 ms * 1000 iterations = 10 seconds. 
\item \tiny With PI: 10 out of 1000 iterations will be affected (that is delayed by 5 ms) .  So, the total delay is 50 ms.
\item \tiny Execution time is 10.05 seconds, i.e., 0.5\% delay. 
\end{itemize}
\item \small On two nodes (weak scaling):
\begin{itemize}
\item \tiny Probability that an iteration is delayed on one of the nodes is 1\%, same as above.
\item \tiny Probability that a given iteration is delayed on any processor is : $1.0 - (1 - 0.01)^2$ = $1.99$\% .
\item \tiny The delay is then $1000 \times 0.0199 \times 5$  = $99.5$ ms for a total execution time of $10.0995$ s. 
\end{itemize}
\item \small On p nodes :
\begin{itemize}
\item \tiny The delay is $1000 \times (1.0 - 0.99^p) \times 5$ ms
\item \tiny For 10 nodes: $1000 \times 0.0956 \times 5$ ms =  478 ms
\item \tiny For 100 nodes: $1000 \times 0.634 \times 5$ ms = 3170 ms
\item \tiny For 1000 nodes: $1000 \times 0.9999 \times 5$ = 4999.5 ms
\end{itemize}
\item \small That's a 50\% performance loss. 
\item \small As $p$ increases, the probability that one node experiences a delay on a given iteration starts approaching 100\%. 
\small \item \small Caveat for loosely synchronous: amplification will come at a larger number of nodes than for bulk-synchronous, though magnitude of performance degradation may be the same as bulk-synchronous code. 
\end{itemize}
\end{frame}

\begin{frame}[label=noiseampformula]
\frametitle{Amplification Problem in Context of Noise} 
\begin{itemize}
\small \item \small Let $T_p$ be iteration time on one node with $p$ cores, with no noise. We assume weak scaling. 
\item \small  Let $\delta$ be the noise duration.
\item \small Let $\phi$ be the expected number of noise events per second, on one node. \item \small We assume global synchronization with zero overhead, e.g., allreduce.
\end{itemize}

\textbf{----------------------------------------------------------}
\begin{itemize}
\small \item \small Probability of noise happening in one iteration on one node = $\phi \times T_p$. 
\item \small Probability that there is no delay on any nodes while running on k nodes is $(1 - \phi \times T_p)^k$. 
\item \small Probability that \textit{some} node experiences a delay in a given iteration is $1 - (1- \phi \times T_p)^k$. 
\item \small So, expected execution time of an iteration on k nodes is: $E[T_k]  >=  T_p + \delta \times  [1 - (1- \phi \times T_p)^k]$. 
\end{itemize}
%( Caveat of slack) 
\end{frame}

\begin{frame}
\frametitle{Improvement of Scalability for 3D Stencil}
\begin{figure}
\includegraphics[scale=0.34]{../images/stencil-scaling}
\caption{Scaling 3D Regular Mesh for different schedulers.}
\begin{itemize}
\tiny {\item \tiny Mixed static/dynamic scheduling results in improved 
scalability over both static scheduling and dynamic scheduling.}
\end{itemize}
\end{figure}
\end{frame}

%TODO: show graph for fully dynamic scheduling.
%\vivek{fix layout}
\begin{frame}
\frametitle{Impact on Scaling}
\begin{figure}[ht!]
\label{fig:scaling-histograms}
\begin{center}
\includegraphics[scale=0.20]{../images/scaling-histograms}
\end{center}
%TODO: word the below better
\caption{\label{fig:scaling-histograms} Iteration timing histograms 
for static scheduling~(left) for mixed static/dynamic scheduling~(right) as we increase the 
number of nodes.}
\end{figure}

\begin{enumerate}
\tiny \item \tiny Distribution of two equal peaks goes to one small
peak and one large peak. 
% Performance variation increases with increasing numbers of
% nodes
\item \tiny The mean of the distribution for mixed 
static/dynamic scheduling stays consistent 
as we increase the number of nodes. 
\item \tiny The mean of the distribution at 64 nodes is better 
  with static/dynamic scheduling than that of static
  scheduling. %TODO: think of a better point here. 
\end{enumerate}
\end{frame}

\begin{frame}
\frametitle{Improvement Over an Already Optimized Dense Matrix Factorization using Hybrid Static/Dynamic Scheduling}
%\frametitle{For Problem 1: Does our scheduling approach perform competitively or better than other industry wide schedulers?}
\begin{columns}
\column{0.5\textwidth}
\begin{figure}
\includegraphics[scale=0.22]{../plots/calu-fastNUMA4-perf}
\caption{\tiny Comparison of hybrid static/dynamic scheduled CALU
  to statically scheduled CALU (first version) as well as 
  other industry standards such as PLASMA and MKL.\comments{showing significant
  performance improvements over the baseline (statically scheduled)
  CALU, particularly with large matrix sizes.}}
\visible<1->{\tiny $\rightarrow$ We get significant
  performance improvements over the baseline (statically scheduled)
  CALU, particularly with large matrix sizes. Our scheduler is 30\% 
  better than PLASMA library's implementation of the same 
  numerical algorithm, and 38\% better than Intel MKL library's 
  impementation of the same numerical algorithm.}
%\end{itemize}
%(Problem: PLASMA has inefficient work-stealing algorithm.) 
\end{figure}
\column{0.5\textwidth}
\begin{figure}
\includegraphics[scale=0.10]{../plots/calu-fastNUMA3-histogram}
\caption{\tiny Performance Variation for static CALU, dynamic CALU, and hybrid static/dynamic CALU.}
\end{figure}
{\tiny $\rightarrow$ Hybrid static/dynamic scheduled CALU reduces
  performance variability while maintaining absolute performance
  better than that of statically scheduled CALU and dynamically
  scheduled CALU, and this is consistent with the performance
  variability results for 3D stencil.} 
\end{columns}
\end{frame}

%TODO: update this
\begin{frame}
\frametitle{Problems with the Basic Lightweight Scheduling Approach}
%\textbf{Two problems with the above basic approach:} 
\begin{enumerate}
\small \item \small Persistent load imbalances not handled. How do we handle both
persistent and transient imbalances together? 
\item \small Experimental tuning: can we reduce the search by using a
model?
\item \small Across-node load imbalance not taken into account in 
scheduling strategy. Can we use runtime-guided adjustments (per-node static
fraction based on slack) to improve performance?
\item \small Our scheduler does not consider spatial locality. How do
  we optimize the mixed static/dynamic scheduling scheme so that we
  improve spatial locality and temporal locality?
\end{enumerate}
\end{frame}

\begin{frame}
\frametitle{Weighted Locality-sensitive Scheduling}
\begin{columns}[T]
% \column{0.50\textwidth}
  \begin{column}{0.5\textwidth}
    {\small \underline{\textbf{Introduction}}} \\
    \begin{itemize}
      \tiny \item \tiny Noise not completely transient; 
      can be a mixture of persistent core noise and transient noise, as observed
      in Jaguar (see next slide). 
    \item \tiny We modify our lightweight scheduling scheme to handle the
      mixture of persistent core noise and transient node-level OS noise. 
    \item \tiny For architectures with 
      the mentioned mixture of noise events, we get noted performance
      gains of 10.4\% over our basic lightweight scheduling scheme. 
    \item \tiny The approach described can be generalized to work for any
      type of mixture of persistent and transient load imbalance, not just
      mixtures of persistent and transient noise. 
    \end{itemize}
  \end{column}
  % \column{0.50\textwidth}
  \begin{column}{0.5\textwidth}
    {\tiny  \underline{\textbf{Scheduling Strategy Optimizations}}}\\
    \begin{itemize}
      \tiny \item \tiny \textcolor{gray}{ mixed static/dynamic (\textit{half}) $\rightarrow$ hybrid static/dynamic \textit{besf} (base)}   
    \item \tiny \textcolor{gray}{ lightweight scheduling (\textit{ligh}) (which includes locality-aware scheduling (\textit{loca}) and skewed workloads \textit{skew})}
    \item \tiny \textcolor{black} {weighted locality-sensitive scheduling (\textit{weig})}
    \item \tiny \textcolor{white} {slack-conscious scheduling: \textit{slac} :  model-guided\textit{Static-Hybrid},   slack-conscious(\textit{callsite\_fd})}
    \item \tiny \textcolor{white} {staggered hybrid static/dynamic scheduling: \textit{ssds}}
    \end{itemize}
        {\tiny \underline{\textbf {Methodology for Tuning Scheduler Parameters}}}
        \begin{itemize}
          \tiny \item \tiny \textcolor{black} {experimental tuning}
        \item \tiny \textcolor{white} {model-guided optimization} %TODO: change to pruning search space 
        \item \tiny \textcolor{black} {runtime adjustment} %TODO: add within-node runtime adjustment and across node adjustment 
        \end{itemize}
  \end{column}
\end{columns}
\end{frame} 

%TODO: figure out whether we need to use coupling. 
%TODO: word the below better 
\begin{frame}
  \frametitle{Persistent Imbalances (Noise-induced) Not Handled} 
  \begin{figure}
    %\includegraphics[scale=0.44]{../images/noise_pes_jaguar}
    \subfloat[\tiny Jaguar quasi-persistence in noise.]{\includegraphics[scale=0.41]{../images/noise_pes_jaguar}}
    \subfloat[\tiny Ranger quasi-persistence in noise.]{ \includegraphics[scale=0.41]{../images/noise_pes_ranger}}
    \caption{\small Time spent by each core in a 1200 core run, for a load balanced code.}
  \end{figure}
  \begin{itemize}
    \tiny \item \tiny The two top bands of indicate slow cores, while
    bottom bands show fast cores (particularly for Jaguar).
  \item \tiny Across all nodes, the subset of cores being fast and
    subset slow is almost the same. 
  \item \tiny We refer to the above phenomenon as persistent core noise,
    or  noise-induced persistent imbalances. 
  %(and note that this could be generalized as a fine-grained application-induced persistent load imbalance across cores).%TODO: explain more on why  
  \end{itemize}
\end{frame}

%TODO: check the question presented in the title of the slide.
\begin{frame}
\frametitle{Blending Weighted Factoring and Mixed Static/Dynamic Scheduling to Handle Persistent Core Noise}
%TODO: add regular diagram for weighted decomp. 
{\small Weighted Factoring: assign different number of iterations to different cores based on history.}
{\small We combine weighted with hybrid static/dynamic scheduling, and
  then tune the static fraction.}\\
{\small Earlier, the number of iterations allocated to thread i was
 $f_s \cdot \frac{N}{p}$}
{\small With weighted, it is: $w_i \cdot \frac{f_s \cdot N}{p}$,
where $w_i$ is calculated by measuring the time spent in the static
portion by different threads, $s_i$, as: $w_i = \frac{avg\{s_i\}}{s_i}$}.
% begining iteration for thread i, b_i = b_{i - 1} +
% w_i*\frac{f_s*N}{p} , with b_0 = 0 
\begin{columns}
  \column{0.5\textwidth}
         {\small Weighted Locality-sensitive Scheduling Strategy Applied to 3D Reg. Mesh.\\}
         \column{0.5\textwidth}
         \begin{center}
           \fbox{\includegraphics[scale=0.2]{../images/weighted_decomp}}\\
         \end{center}
\end{columns}
\end{frame}

\begin{frame}
\frametitle{Weighted Locality-Sensitive Scheduling Results}
\begin{itemize}
\tiny \item \tiny Ranger and Jaguar both experience noise; Ranger is
somewhat noisier.
\item \tiny Only Jaguar has core differentation; some cores experience 
more high-frequency noise than the others.
%\item \tiny 
\end{itemize}

\begin{figure}
\label{fig:uWSched-stencil}
\begin{center}
\includegraphics[scale=0.39]{../plots/timings_jaguar}
\includegraphics[scale=0.39]{../plots/timings_ranger}\\
\end{center}
\caption{\label{fig:uWSched-stencil} 3D Stencil scaling with weighted
  locality-sensitive scheduling for Ranger~(left) and Jaguar~(right).}
\end{figure}

\begin{enumerate}
\tiny \item \tiny Weighted micro-scheduling is the best on Ranger,
although micro-scheduling is almost as good (dynamic load balancing
most important). 
\item \tiny Jaguar: Weighted is better on fewer than 64 nodes and
  micro scheduling is better beyond that. 
\item \tiny Jaguar: Weighted micro-scheduling is better than both of these. 
\end{enumerate}
{\tiny $\rightarrow$ Combining schedulers has no disadvantage.}
%TODO: figure out the avg/max for load imbalance to calculate static fraction. 
%TODO: max - avg . avg / max ; 
\end{frame}

\begin{frame}
\frametitle{Slack-conscious Hybrid Static/Dynamic Scheduling}
\begin{columns}[T]
  \column{0.5\textwidth}
         {\small  \underline{\textbf{Introduction:}}} \\
         \begin{itemize}
           \tiny \item \tiny We first discuss limitations of
           experimentally tuning the lightweight scheduling scheme,
           including the problem of slack. 
         \item \tiny With this, we explain the two techniques of this
           chapter: model-guided lightweight scheduling and
           slack-conscious scheduling technique. 
         \item \tiny We evaluate our slack prediction mechanisms,
           showing scaling results for our model-guided lightweight
           scheduling. 
         \item \tiny We show why our slack-conscious runtime
           adjustment is essential and beneficial at large-scale, by
           showing speedup of slack-conscious scheduling over best
           static fraction (\textit{besf}) scheduling strategy. 
         \end{itemize}
         %\vrule[0.5em]{1pt}{3.5in]
\vrule{} 
         \column{0.5\textwidth}
                {\tiny  \underline{\textbf{Scheduling Strategy Optimizations}}} \\
                \begin{itemize}
                  \tiny \item \tiny \textcolor{gray} {mixed static/dynamic $\rightarrow$ hybrid static/dynamic  (base)}
                \item \tiny  \textcolor{gray} {lightweight scheduling (which includes locality-aware scheduling and skewed workloads)}
                \item \tiny \textcolor{gray} {weighted locality-sensitive scheduling}
                \item \tiny \textcolor{black} {slack-conscious scheduling}
                \item \tiny \textcolor{white} {staggered hybrid static/dynamic scheduling}
                \end{itemize}
                    {\tiny \underline{\textbf { Methodology for Tuning Scheduler Parameters}}}
                    \begin{itemize}
                      \tiny \item \tiny \textcolor{white} {experimental tuning}
                    \item \tiny \textcolor{black} {model-guided optimization} %TODO: change to pruning search space 
                    \item \tiny \textcolor{black} {runtime adjustment}  %TODO: add within-node runtime adjustment and across node adjustment 
                \end{itemize}
\end{columns}
\end{frame}

%TODO: get another graph - this may not show the results as well. 
%TODO: show a caption

\begin{frame}
\frametitle{Large and Complex Search Space}
%TODO: get new picture 
%Also, Could maybe show results for enumeration of search space, and that it's large here. 
%TODO: fix the below 
%Put in modeled graph with two endpoints, showing non-continuous search space. 
\begin{enumerate} 
\item Difficult to find exact static fraction for single node. 
\item Best static fraction changes per node because MPI collective operations involve ``slack''(explained more in depth on next slide).
\end{enumerate}
%\begin{figure}
%\includegraphics[scale=0.20]{../plots/vary-sf-for-diff-threads}
%\caption{\small Graph showing the performance of different static fractions
%  used. Here, we have tried different static fractions, 
%  in increments of 0.1. The minima here is difficult to attain.}
%\includegraphics[scale=0.28]{../plots/best-fs-vs-nodes}
%\end{figure}
\end{frame}

%TODO: add in other data sets.  
%TODO: add in max/avg. 

\comments{
\begin{frame}
\frametitle{Can We Use Theoretical Analysis to Prune the Large Search Space?}
% Model-guided optimization, but with experimental tuning after
% scheduler parameter from model is determined. 
\begin{center}
\includegraphics[scale=0.26]{../plots/app-scaling-strat-SNAP-fastNUMA2}
\includegraphics[scale=0.26]{../plots/app-scaling-strat-nbody-fastNUMA2}\\
\end{center}
\begin{center}
{\small Using a theoretical analysis to acheive the minima helps to provide 13-16\% performance gains for a particle simulation (nbody) and SNAP (reg. mesh), each tested on a 16-core non-NUMA architecture and 16-core NUMA architecture.}
\end{center}
\end{frame}
}

\begin{frame}
\frametitle{Different MPI processes Have Different Cost to Application Critical Path}
{\tiny \underline{\textbf{MPI Slack}} is the time by which a process
  can delay arrival at a synchronizing point without impacting application execution time.}
%
\begin{table}[tr]
\label{fig:collectiveSlack-useq}
  \begin{center}
    \begin{tabular}{ | c || c | c || c |}
      \hline
      \textbf{Collective} &      \textbf{max} & \textbf{avg ($\sigma$)} & \textbf{$\sigma_i$} \\ 
\hline
      \texttt{Allreduce} &             307    &    199 (.792) &            .102 \\ 
\hline
      \texttt{Alltoall}  &             280    &    149 (v.821) &            .176 \\ 
\hline
      \texttt{Barrier}  &              250    &    139 (.178)  &           .061 \\ 
\hline
      \texttt{Allgather} &             268    &    189 (.527) &            .115 \\
\hline
      \texttt{Reduce\_scatter} &       459    &    296  (.649)  &          .129 \\ 
\hline
    \end{tabular}
  \end{center}
  \caption{\label{fig:collectiveSlack-useq} {\small Different slack statistics (in $\mu$s) shown for experiments with different collectives at the end of an iteration. Additional evidence and impact of slack esp. at large scale is discussed in [slackPerf] and [adagio].}}
\end{table}

\begin{itemize}
\tiny \item \tiny Slack distribution is different for different collectives.
\item \tiny Slack varies across processes.
\item \tiny The distribution of slack on different processes evolves slowly for different applications. 
\end{itemize}
\end{frame}

\begin{frame}
\frametitle{Terminology and Performance Model}
\begin{table}[tr]\centering
\begin{tabular} {|c|p{.75\columnwidth}|} \hline
\multicolumn{2}{|c|}{\textit{Model outputs}} \\\hline
$f_d$       & {\tiny Fraction of work scheduled dynamically.} \\\hline
$f_d'$      & {\tiny Fraction of work scheduled dynamically,\newline
                      reduced for slack.} \\\hline
%
\hline
\multicolumn{2}{|c|}{\textit{Model inputs}} \\\hline
$q$                     & {\tiny Overhead of dequeueing single iteration.} \\\hline
$S_i$                   & {\tiny Slack duration on process $i$.} \\\hline
$\delta$        & {\tiny Expected noise duration.} \\\hline
$T_p$       & {\tiny Execution time on $p$ cores with fully static scheduling and no noise.} \\\hline
$N$         & {\tiny Total number of iterations across all threads.}\\\hline
% 
$t_1$       & {\tiny Duration of single iteration on a single thread, on one core.} \\\hline
$d$                     & {\tiny Duration of single dynamic iteration on a single thread.} \\\hline
%$\eta$          & Compute delay due to noise and queue overhead \\\hline
%$\alpha_i$      &  Delay added to the critical path by node $i$\\\hline
\end{tabular}
\caption{\small Overview of all model parameters. Note that we use
  the dynamic fraction $f_d$ instead of static fraction $f_s$ to simplify our
  theoretical analysis.\label{tab:parameters}}
\end{table}
\end{frame}

\begin{frame}[label=mghybridstatdyn]
\frametitle{Model-guided Hybrid Static/Dynamic Scheduling}
{\small Instead of tuning the dynamic fraction, we establish a model to
determine it based on noise duration $\delta$, and other model parameters. \\}
\begin{center}
  \fbox{\includegraphics[scale=0.46]{../images/hybSched-nonoisenoslack}} \\
\end{center}

{\small Considering the above model, along with our goal of avoiding load balancing until it is absolutely needed, 
the dynamic fraction applied to each MPI process is: \\}
\begin{center}
  \fbox{{$f_d = \frac{p\delta}{N\left(t_1 + q\right)}$}}\\
\end{center}
    {\small We refer to the strategy described above as Model-guided Hybrid
      Static/Dynamic Scheduling. This strategy is referred to as 
      \textit{Static-Hybrid} in the graphs.} 
\end{frame}

\begin{frame}
\frametitle{Considering Slack Factor for Hybrid Scheduling}
{\small With this, let us now consider the slack factor for each MPI process.} 

\begin{figure}
  \begin{center}
    \includegraphics[scale=0.46]{../images/slackConsciousSched} \\
  \end{center}
\end{figure}

\begin{itemize}
\tiny \item \tiny Slack allows us to reduce the dynamic fraction
further, because a small amount of delay due to noise will lead to no
delay in the completion of an MPI operation. For example, if you are
in a Reduction and waiting for data from your spanning tree
children. 
\item \tiny Note that the slack is different on different nodes,
  depending on the implementation of the collective operation. 
\item \tiny Therefore, the dynamic fraction will be different on
  different processors. Nodes with large slack can get away with 
small dynamic fraction without impacting application performance.
\end{itemize}
\end{frame}

\begin{frame}[label=slackfs]
\frametitle{Slack-conscious Hybrid Static/Dynamic Scheduling
  a.k.a. Adaptive Scheduling} 
\begin{figure}
  \begin{center}
    \includegraphics[scale=0.41]{../images/slackConsciousSched} \\
  \end{center}
\end{figure}
\begin{center}
\fbox{{$f_d = \frac{p\delta}{N\left(t_1 + q\right)}$}}
\fbox{ {$f_d' = f_d - \frac{S}{(p-1)\frac{Nt_1}{p} - Nq}$}}
\end{center}
\begin{center}
{\tiny Note that $f_d'$ and $S$ are local variables within each MPI process. }
\end{center}

%TODO: 
{\small The dynamic fraction $f_d$ of a particular node
 is reduced by a value proportionate to the slack on that node, 
to get the new $f_d'$.} \\

{\small To estimate the slack on each processor, we use ideas from
 Adagio (see next slide for impl). \\ We compared multiple methods,
 and chose the \textit{callsite\_fd} method.} \\

{\small Reducing dynamic fraction reduces overall overhead.}

\end{frame}

\comments{
\begin{frame} 
\frametitle{Slack Prediction Mechanism and Usage}  
\begin{enumerate}
\tiny \item \tiny On each process, retrieve that process's invocation of the
  last MPI collective, where the invocation of the last MPI collective is
retrieved through the callsite slack-prediction method (see paper for
details on implementation details of slack prediction).
 \item \tiny Given the identifier of the last the MPI collective call
invoked, estimate that collective call's slack value from the
history of slack values stored by the slack-conscious runtime.
The slack estimate is based on the slack value recorded
in the previous MPI collective invocation, as is done in the
cited work of Adagio.
 \item \tiny On each process, adjust its dynamic fraction based on the
   slack value. This adjustment is done using the formula 5 in section
   3 of Chapter 4  and implementation of section 4 of
   Chapter 4. The static fraction used in the
   loop bound is 1- $f_d$.
\end{enumerate}
\end{frame} 
}

\begin{frame}
\frametitle{Software Architecture}
\begin{figure}
\begin{center}
\label{code:architecture}
\includegraphics[scale=0.56]{../images/architecture}
\end{center}
\end{figure} 

\begin{itemize}
\tiny \item \tiny The white rectangles taken by themselves show the
interaction between the MPI+OpenMP application code and the
MPI \& OpenMP runtimes.  
%how a basic MPI+OpenMP
%application is run.  
\item \tiny Our contributions in this work are shown in gray
  rectangles. %Involves source-to-source transformation and adaptiv
\item \tiny Predict static fraction call (arrow going from compilation
  to Adaptive runtime library) is how MPI runtime
  communicates with OpenMP runtime to tune OpenMP scheduler parameters
  (and other parameters). 
\end{itemize}
\end{frame}

\begin{frame}
\frametitle{Code Transformation}
\vspace*{-0.4in}
\begin{columns}
\column{0.5\textwidth}
\begin{figure}
\label{code:origOMPcode}
%\subfloat[\label{code:origOMPcode}{\tiny Loop transformed for slack-conscious scheduling.}]
\begin{center} 
\lstinputlisting{../listings/omp-static.c}
\end{center} 
\end{figure}
\column{0.5\textwidth}
\begin{figure}
\label{code:refactoredOMPcode}
\begin{center} 
\lstinputlisting{../listings/omp-hybrid.c}
\end{center} 
%\caption{\label{code:refactoredOMPcode}\tiny Loop transformed for
%  slack-conscious scheduling.}
\end{figure}
\end{columns}
\begin{enumerate}
\tiny \item \tiny The \texttt{decide\_static\_fraction()} and 
\texttt{endLoop()} functions are calls to the slack-conscious runtime. 
\item \tiny The runtime maintains a \texttt{LoopTimeRecord} struct
  \texttt{lr} across invocations for a threaded computation region. 
  The \texttt{LoopTimeRecord} struct includes storage for
  parameters of loop time, slack, est. sched overhead time, and number of
  iterations, along with various timers obtained at different points during
  the execution of the OpenMP threaded computation region. 
\item \tiny Using wrappers around the MPI collective, the 
  slack-conscious runtime predicts slack, and calculates optimal
  static fraction using the model. 
\item \tiny For the first invocation of the threaded computation
  region, \texttt{lr} is empty, so the runtime system uses static
  fraction of 100\%. 
  % the simpler model without slack to determine the
  % static fraction. 
  % calls around the MPI comm. 
\end{enumerate}
\end{frame}

\begin{frame}[label=runtimeparameterestimation]
\frametitle{Scheduler Parameter Estimation}
%{\small To calculate the value of $\delta$, $\frac{N \cdot t_1}{p}$ 
%  and $q$ for computing the slack-conscious static fraction
%  based on \hyperlink{slackfs}{\beamerbutton{Slide 46}}}:}
\begin{itemize}
\small \item \small The value of the noise parameter $\delta$ is 
estimated in the \texttt{MPI\_Init()} function: we subtract the
maximum and minimum times for the iterations of a basic OpenMP loop
performing a square root operation. 
\comments{and then do an
MPI\_Allreduce() with the max operator to collect the maximum across
all iteration times.}
\item \small We estimate the scheduler overhead $q$ by subtracting 
performance of a statically scheduled OpenMP loop doing a dot product
from a dynamically scheduled OpenMP loop doing a dot product.  
\item \small The time $\frac{N \cdot t_1}{p}$ is estimated by taking 
the time taken for statically scheduled section. 
 \comments{of the last 5 invocations of the threaded computation 
region and taking the minimum execution time of these times.} We 
obtain $t_1$ by dividing this time by the number of iterations
executed statically, and use it to calculate $\frac{N\cdot t_1}{p}$.  
% during the execution of the threaded computation
% region, 

%Note that this gives an estimate of 
%scheduler overhead, in order to give a tighter bound on the static 
%fraction. 
\end{itemize}
%\comments{ 
\begin{enumerate}
\tiny \item \tiny We perform our estimates within the MPI\_Init() 
function (intercepted by PMPI), using a wrapper script. Our calculations needed are done
after the real MPI\_Init() has finished. 
%\tiny \item \tiny The value for noise estimated is 5.4\% greater than that of netgauge
%noise measurements.(possible future direction for netgauge is to make
%it output noise event length delta, rather than percent noise). 
%\tiny \item \tiny Note that values noise and scheduler overhead are estimates, rather
%than the exact values; as we will see in the results section, 
%these estimates are adequate to give us significant performance
%gains. 
\end{enumerate}
%}
\end{frame}

\begin{frame}
\frametitle{Process-local Slack Measurement}
%{\small To calculate the slack S, for computing the slack-conscious static fraction based on 
%\hyperlink{slackfs}{\beamerbutton{Slide 46}}:} 
\begin{itemize} 
\small \item \small \textbf{Callpath.} Associate slack trace values
with only the type of collective operation that had the slack. 
\item \small \textbf{Collective.} Storing the full callpath requires
  overhead of unwinding the stack. Associate slack trace values with
  only the type of the collective operation that had the slack.
\item \small \textbf{Naive.} Forget slack tracing. Predict
  slack based on a precomputed experiment. Specifically, 
  run 1,000 invocations of a simple MPI\_Allreduce() with a
  message size of 4 bytes on all MPI processes before the
  application begins, i.e., right after MPI\_Init(). 
\end{itemize}

\begin{enumerate}
\tiny \item \tiny Tradeoff between runtime overhead for the slack measurement, and the 
advantage due to Adaptive Hybrid Static/Dynamic Scheduling.
\item \tiny Note that this implementation is based on the Adagio
runtime, and that we did not use the Adagio runtime directly. 
\end{enumerate}
\end{frame}

\begin{frame}
\frametitle{Slack Prediction Mechanism Overheads and Errors}
\begin{itemize}
\tiny \item \tiny \textit{Slack prediction error} for a particular MPI
region is the percentage difference 
between the predicted slack before the MPI region begins 
and the actual slack recorded at the end of the MPI region.
\item \tiny \textit{Slack prediction overhead} for a particular MPI
  region is the time spent in our slack prediction runtime.
\end{itemize}
\begin{figure}
\includegraphics[scale=0.30]{../plots/slackErr-fastNUMA2}
\includegraphics[scale=0.30]{../plots/slackOvhd-fastNUMA2} \\
\includegraphics[scale=0.30]{../plots/slackErr-rzuseq}
\includegraphics[scale=0.30]{../plots/slackOvhd-rzuseq}
%TODO: see if schemes or runtimes is better
\caption{\tiny Overheads and errors of different available slack
  prediction schemes in our runtime system. Each process sums the
  total error and overhead across all MPI regions, and we report below
  average overhead and error across MPI processes.}
\end{figure}
\end{frame}

\begin{frame}
\frametitle{Improving Performance through Model-guided Optimization And Slack-Conscious Scheduling}
%\frametitle{Can We Use This Slack Information to Adjust the Scheduler Parameters So As to Reduce this Cost?}
\begin{figure}
\includegraphics[scale=0.34]{../plots/app-scaling-fastNUMA2}
\caption{\small Speedup of slack-conscious hybrid static/dynamic scheduling over OpenMP static scheduling for cab with varying number of nodes.}
\end{figure}
\begin{itemize}
\tiny \item \tiny Using a theoretical analysis to acheive the minima
with Static-Hybrid helps to provide 13-16\% performance gains for a
particle simulation (nbody) and SNAP (reg. mesh), each tested on a
16-core non-NUMA architecture and 16-core NUMA architecture.
\item \tiny Slack-conscious scheduling  provides additional gains
  of 23.5\% for NASLU-MZ and 25.5\% for SNAP.
\end{itemize}
\end{frame}

\begin{frame}
\frametitle{Improving Performance through Model-guided Optimization And Slack-Conscious Scheduling}
%\frametitle{Can We Use This Slack Information to Adjust the Scheduler Parameters So As to Reduce this Cost?}
\begin{figure}
\includegraphics[scale=0.34]{../plots/app-scaling-rzuseq}
\caption{\small Speedup of slack-conscious static fraction over OpenMP
  static for rzuseqwith varying number of nodes.}
\end{figure}
\begin{itemize}
\tiny \item \tiny Using our scheduling strategy does not incur
significant performance degradation (less than 1\% on all node counts
tested) for a non-noisy machine. 
\item \tiny We are able obtain some performance gain (5.5\% at 1024 
  nodes) for our amg experiment which has a small amount of application load imbalance in the code.
\end{itemize}
\end{frame}

\begin{frame}[label=bestfsvsadpt]
%\frametitle{Why Should We Care About Model-guided and Runtime-guided Optimizations to Our Lightweight Scheduling Strategies?}
\frametitle{Benefits of Using Runtime Adjustment over Experimental Tuning}
\begin{figure}
\includegraphics[scale=0.50]{../plots/best-fs-vs-adpt}
\caption{\small Shows how runtime-adjusted static fraction for each application run on cab
works better than the tuning approach (which is best static fraction
(\textit{besf}) strategy). Speedup of slack-conscious static fraction
over best static fraction is shown.}
%This shows the importance of model-guided optimization and
%runtime-based adjustment (esp. with increasing nodes). }
\end{figure}
\end{frame}

\begin{frame}[label=locOptHybrid]
\frametitle{Locality-Optimized Mixed Static/Dynamic Scheduling}
\begin{columns}[T]
\column{0.5\textwidth}
{\small \underline{\textbf{Introduction:}}} \\
\begin{itemize}
  \comments{\tiny \item \tiny We need to avoid disturbing spatial locality while also making our scheduler as lightweight as possible.}
  \tiny \item \tiny We first describe our locality-optimizations,
  explaining why the original disturbs spatial locality and how we can
  improve it while maintaining our original lightweight scheduling
  scheme's benefits. 
\item \tiny We describe our implementation of a library that is
  invoked within an OpenMP threaded computation to schedule
  iterations, explaining that it is low overhead compared to OpenMP
  static.  
%\item \tiny We show results for a Barnes-hut pthread code and regular mesh OpenMP code using our scheduling library. 
\item \tiny We show results for a Barnes-hut and regular mesh code 
  using our scheduling library.
\item \tiny We show results that indicate that our locality-optimized 
strategy controls the performance variation (discussed
\hyperlink{perfVar}{\textcolor{blue}{here}}) just as well or better than
does the basic hybrid static/dynamic scheduling. 
%\comments{(which is beneficial for scalability)}. 
\end{itemize}
\vrule{}
\column{0.5\textwidth}
       {\tiny  \underline{\textbf{Scheduling Strategy Optimizations}}}\\
       \begin{itemize}
         \tiny \item \tiny \textcolor{gray}{ mixed static/dynamic $\rightarrow$ hybrid static/dynamic  (base)}
       \item \tiny  \textcolor{gray}{lightweight scheduling (which includes locality-aware scheduling and skewed workloads)}
       \item \tiny \textcolor{gray} {weighted locality-sensitive scheduling}
       \item \tiny \textcolor{gray} {slack-conscious scheduling}
       \item \tiny \textcolor{black} {staggered hybrid static/dynamic scheduling}
       \end{itemize}
    {\tiny \underline{\textbf { Methodology for Tuning Scheduler Parameters}}}
    \begin{itemize}
      \tiny \item \tiny \textcolor{black} {experimental tuning}
    \item \tiny \textcolor{white} {model-guided optimization} %TODO: change to pruning search space 
    \item \tiny \textcolor{white} {runtime adjustment} %TODO: add within-node runtime adjustment and across node adjustment
    \end{itemize}
\end{columns}
\end{frame}

\comments{
low-probability term  for terms  -     precise  approximation .    

-    correctness of the equality . 
- intuition 

-  storyboard   

simple techniques can make your code faster.   
}

\begin{frame}
\frametitle{Improving Spatial Locality in Dynamically Scheduled Iterations}
{\tiny Ignoring slack for the moment, consider the diagram shown in slide 45
  showing hybrid static/dynamic scheduling. \\}
{\tiny  From this diagram, consider the sequence of loop iterations that
  all 4 threads work on. (Note that the x-axis below is increasing
  iteration numbers, and not time ). \\} 
\begin{center}
\includegraphics[scale=0.35]{../images/sdSched} \\
\end{center}
\begin{itemize}
\small \item \small The consecutive iterations typically access consecutive memory locations. 
\item \small Spatial locality benefits static iterations, but is lost in dynamic iterations. 
\item \small Given this, we could get additional performance benefit if we bias distribution of dynamic iterations, so that each thread is likely to execute dynamic iterations 
that are contiguous with its static iterations.  
\end{itemize}
\end{frame}

\begin{frame}[label=sdsSched]
%\frametitle{Question: Can we Re-distribute the Iterations So that Spatial Locality Isn't Lost?}
\frametitle{Spatial Locality in the Hybrid Static/dynamic Scheduling Approach}
\begin{center}
\begin{figure}
\includegraphics[scale=0.35]{../images/sdsSched}
%TODO: update the below, to word better for intent. 
\caption{\small We stagger the dynamic iterations in order to improve spatial locality of the basic static/dynamic scheduling.}
\end{figure}
\end{center}

\begin{itemize}
\small \item \small Strategy: Each thread finishes its static iterations, then does dynamic iterations marked for it if available, and only then looks for other dynamic work. 
\item \small Spatial locality will not be lost in common cases: 
\begin{enumerate}
\small \item \small When there is no imbalance. 
\item \small Even when there is imbalance espeicially for noise, most threads are able to complete their own dynamic iterations.
\end{enumerate}
\end{itemize}
\end{frame}

\begin{frame}[label=efficientImpl]
%\frametitle{Question: Can we develop an efficient scheduler library that handles different tradeoffs between load balance and locality?}
\frametitle{Note on Efficient Implementation of Scheduler Library}
\begin{center}
\input{../plots/runtime-costs}
\end{center}
\begin{itemize}
\small \item \small Overhead of our library is relatively small. 
\end{itemize}
\end{frame}

\begin{frame}[label=claresults]
%TODO: add in varying number of cores to show improvements. 
\frametitle{Performance Improvements with Varying Problem Sizes}
\begin{figure}
\includegraphics[scale=0.40]{../plots/bh-16cores}
\includegraphics[scale=0.35]{../plots/st-16cores}
\caption{\small Shows performance improvements when using our vSched
  library (having staggered static/dynamic scheduling strategy), for a
  Barnes-Hut pthread particle simulation code (right) from the
  Lonestar Benchmark Suite and the SNAP 2D regular mesh code (right).}
\end{figure}
\begin{itemize}
\tiny \item \tiny Using our staggered hybrid static/dynamic
scheduling (\textit{stag}) for the 2D regular mesh code
with 65536 mesh points, we get reasonable performance gain of 6.45\% 
over OpenMP static scheduling. 
\item \tiny Using our staggered static/dynamic scheduling approach for
 the Barnes-Hut code at 100,000 particles, we get significant gains of
 18.53\%.
\item \tiny Our strategy provides increased performance gains as we
  increase the problem size. 
\end{itemize}
\end{frame}

\begin{frame}[label=controlPerfVar]
\frametitle{Controlling Performance Variability}
%\frametitle{Do the optimizations maintain performance consistency characteristics of static/dynamic scheduli}
\begin{center}
\input{../plots/runtime-stddev}
\end{center}
\begin{itemize}
\tiny \item \tiny With dynamic scheduling, performance can vary 
from run to run (even for static/dynamic scheduling).
\item \tiny Runtime standard deviations are controlled with our
  scheduler optimizations done in vSched, probably because of more
  persistent and predictable behavior(because each timestep each
  thread is likely executing the same dynamic iterations). 
\end{itemize}
\end{frame}

\begin{frame}[label=combinedintro]
\frametitle{Scheduler Composition}
\begin{columns}
\column{0.5\textwidth}
{\small \textbf{\underline{Introduction}:}}\\
\begin{itemize}
\tiny \item \tiny Some schedulers may benefit in some cases, others elsewhere.
\item \tiny Problem: Machine and application has all above mentioned factors involved. 
\item \tiny We show how to combine all schedulers. We also apply it to 3 different application codes. 
\item \tiny We discuss application programmer usability. 
\item \tiny Finally, we re-visit the amplification problem, and discuss how our schedulers work at large-scale.
\end{itemize}

\vrule{}
 
\column{0.5\textwidth}
{\tiny  \underline{\textbf{Scheduling Strategy Optimizations:}}}\\
\begin{itemize}
  \tiny \item \tiny mixed static/dynamic (base)
\item \tiny lightweight scheduling 
\item \tiny weighted locality-sensitive
\item \tiny slack-conscious scheduling
\item \tiny constrained staggering
\item \tiny $\rightarrow$ Each opens up new opportunities to balance the tradeoff between load balance and locality. % The main \point is that there are many dimensions of scheduler 
\end{itemize}

{\tiny \underline{\textbf { Methodology for Tuning Scheduler Parameters}}}\\
\begin{itemize}
  \tiny \item \tiny experimental auto-tuning
\item \tiny model-guided optimization
\item \tiny runtime adjustment
\item \tiny $\rightarrow$ Each are important mechanisms to find the best tradeoff between load balance and locality.
\end{itemize}
\end{columns}
\end{frame}

\begin{frame}[label=combinedSched]
\frametitle{Components of Example Composed Scheduler}
\begin{itemize}
\small \item \small \textbf{uSched}: Use model-guided optimization
  to determine the static fraction (we have separate cases for load
  imbalanced due to noise and load imbalance due to the application), 
  and then experimentally tune the static fraction $0.05$ below static
  fraction and $0.05$ above static fraction ($0.05$ is a configurable
  parameter). 
\item \small \textbf{vSched}: Almost the same as the \textit{ssds}
  scheduling strategy. The difference here is that we use the 
  static fraction obtained from \textbf{uSch} above
  rather than experimentally tuning the static fraction. 
\item \small \textbf{callsite\_fd}: Almost the same as the Slack-conscious Scheduling strategy. 
  The difference here is that initial static fraction to use is obtained from \textbf{uSched} above. 
\item \small \textbf{comboSched}: We start with the staggered static/dynamic
  scheduling scheme defined in \textbf{vSched} above, and then do the 
  slack-conscious adjustment described in slack-conscious scheduling section.
\end{itemize}
\end{frame} 

\comments{
\begin{frame}[label=combined] 
\frametitle{Scheduler Composition of Example Composed Scheduler} 
\begin{figure}[ht]
\label{fig:sched-composition}
\begin{center}
\includegraphics[scale=0.35]{../pictures/sched-composition-diagram}
\end{center}
\caption{\label{fig:sched-composition} Composition of 3 schedulers
which make ComboSched.}
\end{figure} 
\end{frame} 
}

\begin{frame}[label=combinednbody]
\frametitle{Combined Scheduling Results for Rebound}
{\tiny Rebound(*) is an MPI+OpenMP particle simulation code involving large load
imbalances as well as irregular data access patterns.}
\begin{figure}
\includegraphics[scale=0.43]{../plots/app-scaling-strat-nbody-fastNUMA2}
\caption{\tiny Percent speedup over OpenMP static scheduling for each
  scheduling strategy for the particle simulation code. Run on
  \texttt{cab} using 16 threads per node (and 1 MPI process per
  node).}

\end{figure}
\begin{itemize}
\tiny \item \tiny Combining different techniques seems to add on
benefits, i.e., they don't cancel benefit out. 
\item \tiny Because they are based on complementary factors. Spatial
  locality in the static/dynamic scheduling vs. reducing dynamic
  scheduling. 
%\item \tiny better combination could lead to even more improvement 
\end{itemize}

{ \tiny *Rebound is actually the name a suite of particle simulation application
  codes, and we considered the shearing sheet simulation. Here, we use the  
  name Rebound to identify the shearing sheet application run in our experiments.  }
\end{frame}

\begin{frame}[label=combinedfe]
\frametitle{Combined Scheduling Results for MiniFE}
{\tiny  CORAL miniFE MPI+OpenMP code runs a finite element method. 
  This involves irregular computation, and relatively low load imbalance.} 
\begin{figure}
%\includegraphics[scale=0.38]{../plots/app-scaling-strat-SNAP-craybw}\\
\includegraphics[scale=0.43]{../plots/app-scaling-strat-SNAP-fastNUMA2}
\caption{\tiny Shows how the application of full scheduling to miniFE 
works better than the application of other approaches, by showing speedup over OpenMP
static scheduling for each scheduling strategy for different numbers
of nodes. Run on \texttt{cab} using 16 threads per node
(and 1 MPI process per node).}
\end{figure}

\begin{itemize}
\tiny \item \tiny Dynamic scheduling causes significant degradation 
even though there is imbalance. 
\item \tiny Our scheduling strategies provides notable performance 
improvement over static scheduling, though gains less than performance gains for n-body.
\item \tiny One reason is that code\comments{'s computation} has 
relatively significant amount of across-node load imblance relative
to within-node load imbalance, while n-body has little to no load
imbalance. 
\end{itemize}

\end{frame}

\begin{frame}[label=combinedregmesh]
\frametitle{Combined Scheduling Results for Regular Mesh}
{\tiny CORAL SNAP is an MPI+OpenMP code that does a regular mesh
  computation used in heat diffusion. This computation is similar to
  that in chapter 2.}
\begin{figure}
\includegraphics[scale=0.43]{../plots/app-scaling-strat-fe-fastNUMA2}\\
\caption{\tiny Shows how the full scheduling for SNAP 
works better than other approaches, by showing speedup over OpenMP
static scheduling for each scheduling strategy for different numbers
of nodes. The above is run on \texttt{cab} using 16 threads per node, 
and 1 MPI process per node.}
\end{figure}

\begin{itemize}
\tiny \item \tiny Even when dynamic strategies worsen performance, the
combined strategies show a small benefit. 
%\item \tiny uSched doesn't do as well 
\end{itemize}
\end{frame}

\comments{
\begin{frame} 
\frametitle{Code with Scheduling Composition} 
\begin{figure}[h!t]
              \label{code:dvSchedMacro}
              \begin{center}
                \lstinputlisting{../listings/dvSchedMacro-CG.c}
             \end{center}
              \caption{\label{code:dvSchedMacro}{\small NAS CG code
                  modification using our scheduling lib\
rary.}}
            \end{figure}
\end{frame} 
}

\begin{frame}[label=programmerEffort]
\frametitle{Is Our Method Easy to Use for the Application Programmer?}
{\tiny Rebound: (-add total loc-) SNAP: (-add total loc-)  miniFE: (-add total loc-) \\}
\input{../plots/linesOfCodeChanged}
%Experiment: Show lines of code modified, as shown in fourth quadrant
%of SC poster. 

\begin{itemize}
\tiny \item \tiny Lines of code changed is minimal for all three codes. 
\item \tiny Code transformation via ROSE \comments{(and changes to
  GOMP)} can reduce or eliminate need for code change.
\end{itemize}
%TODO: explain
\end{frame}

\comments{
\begin{frame} 
\frametitle{Load Imbalance Metrics With Increasing Numbers of Nodes}
\begin{figure}
\label{fig:bh-withinAcrossImb-cab}
\begin{center}
  \subfloat[Load Imbalance Across Nodes] {\includegraphics[width=0.5\textwidth] {../plots/imb-maxavg-nbody-cab-inodes}}
  \subfloat[Load Imbalance Across Cores] {\includegraphics[width=0.5\textwidth] {../plots/imb-maxavg-nbody-cab-icores}}
\end{center}
\caption{\label{fig:bh-withinAcrossImb-cab}{Standard load imbalance metrics taken for a galaxy 
    simulation running on 1024 nodes of \textit{cab}, which is an Intel Westmere
    machine with 16 cores per node. The code is run with one MPI process per core.}}
% Overheads of our scheduling runtime.\comments{shown through comparison between our static scheduling implementation and OpenMP static scheduling.}
\end{figure} 
\end{frame}
}
\comments{
\begin{frame}[label=ampProbRevisit]
\frametitle{Amplification Problem Revisited}
{\tiny (Ignore for now.)} 
\begin{figure}
\label{fig:modeledScurve}
\begin{center}
\includegraphics[scale=0.38]{../plots/amplification-model-Scurve1}
\end{center}
\caption{\label{fig:modeledScurve}\small Modeled scaling curve based
  on formula for expected length of the $k^{th}$ iteration,
  described~\hyperlink{noiseampformula}{\beamerbutton{here}}. In the formula, we use I=1000, $\delta$ = 5 ms, $\phi$ = 1.}
\end{figure}
\end{frame}
}

\begin{frame}[label=ampProbRevisit]
\frametitle{Impact to Extreme-Scale}
\begin{itemize} 
 \small \item \small As we go to future machines including exascale
\begin{enumerate} 
 \tiny  \item \tiny  number of nodes increases 
\item \tiny cores per node increases 
\end{enumerate} 
\item most importantly, static and dynamic variabilty increases. 
\begin{enumerate}
\tiny \item \tiny Operating voltage might decrease, which increases
  variability and decreases reliability. 
\item \tiny  Fault rates increase; these will be contained
through multiple mechanisms locally, but this will lead to transient
load imbalances. 
 \end{enumerate} 
\item \small Applications will need strong scaling, as exascale reports say,
  because memory per core will be lower, and scientists would want to
  solve the same problem faster rather than a larger problem at the
  same speed. 
\item \small All of the above reasons will increase the impact of
  amplification. 
\item \small Thereby increasing the need for the kind of schedulers we have developed in
  this dissertation 
\begin{itemize} 
 \item \small Number of cores per node will be substantially higher, which our
techniques can leverage.
\end{itemize} 

\item {\small the multiple techniques that we developed are in
  extensible software library that can be extended for exascale
  machines with newer features and future sophisticated applications. }
% can be
% selectively combined in different 
\end{itemize} 

\comments{
adaptive componets of hardware and software, low power  node 

The number of errors will increase, but it is expected that errors
will be contained localily 

impact of transient events is larger impact of applications .. 
This will lead to   
}
\end{frame} 

\begin{frame}[label=relatedWork]
\frametitle{Related Work}
\begin{itemize}
\item \small DPLASMA: dynamic scheduling for specific Numerical Linear
  Algebra application, not tunable static fraction. 
\item \small Work-stealing: Ignores locality, and can't tune the
  dynamic scheduler. 
\item \small OpenMP loop scheduling: Sarkar, Chapman, deSupinski.
\item \small TBB, Habanero.
%Some work by Sriram et al tries to fix outer
%iteration locality 
\item \small Co-scheduling: Schedule all events together to make only 
  some timesteps be impacted by noise, but can't co-schedule all
  events. 
\item \small Hyperthreading with SMT at LLNL: shows benefits with SMT 
  and can be lightweight, but is a hardware solution (not tunable to
  avoid overheads). 
\item \small Runtime for Irregular applications (Galois): Transient
  load imbalances, but dealing with more frequent irregularity on each
  timestep, and load imbalances are coarse-grained. 
\item \small UPC task scheduler: Can move work across threads with
  some affinity, but this can't be tuned (all is done at runtime).
\end{itemize}
%*Some work/discussions have been done to integrate our scheduler into this runtime
\end{frame}

\begin{frame}[label=conclusions]
  \frametitle{Conclusions}
  \begin{itemize}
  \item \small Performance irregularities create scaling challenges
    and are difficult to handle through conventional techniques. 
  \item \small Locality-aware hybrid static/dynamic scheduling
    provides a way to cost efficiently handle within-node 
    load imbalance induced by performance irregularity. 
  \item \small Formalization and Numerical Linear Algebra Application
    Case Studies. 
  \item \small Additional scheduling techniques and optimizations:
    model-guided determination of $f_d$, adaptive per-thread static fraction, slack-conscious static
    fraction (staggered scheduling for spatial locality in dynamic
    section). 
  \item \small Our runtime does not introduce significant overheads. 
\begin{itemize} 
  \item \small So, even if runtime slowdowns due to performance
    irregularity are small, it can be used.
\end{itemize}  
  \item \small Feasible to integrate into applications through
    compiler transformations and adaptive runtime library. 
  \item \small Demonstrated a novel set of scheduling strategies that
    can be used to overcome challenges posed by dynamic
    irregularities, while minimizing data movement and synchronization
    costs. 
\item \small Directions for future work: (a) can use within MPI-shared memory extensions and
  MPI+X models; and (b) can combine with other across-node load balancers.
%  \item \small Can further refine accuracy of slack prediction through
%   risk model.
%  \item \small Can use mixed non-blocking + blocking collectives,
%  inspired by mixed static/dynamic scheduling.  
    % \item \small Plan to provide more loosely coupled interface between OpenMP and MPI, or between any two programming models.
  \end{itemize}
\end{frame}

%Acknowlegdments of people
%\begin{frame}[Acknowledgements]
\frametitle{Acknowledgements}
%\begin{itemize}
%\item -
%\item -
%\item -
%\end{itemize}
\end{frame}

\begin{frame}[People] 
\frametitle{People} 
\begin{itemize} 
\item William Gropp (UIUC)
\item Torsten Hoefler (UIUC)
\item Bronis de Supinski (LLNL)
\item Todd Gamblin (LLNL)
\item Amanda Peters (collaborators for Human Heartbeat simulation at Harvard University) 
\item Simplice Donfack, Laura Grigori, Ulrike Meier Yang (collaborators for numerical linear algebra at INRIA and LLNL) 
\item Jim Dinan and Darius Buntinas 
and Mike Heroux (collaborators Programming Models) 
\end{itemize}
\end{frame}


%NOTE: this marks the end of the presentation. 
%TODO: figure out if we should add ``questions'' slide.

%\input{thanks}
%People involved in the project 
%\input{people} 


\begin{frame}
\frametitle{Overview: Mgmt} 

\end{frame} 

\begin{frame}
\frametitle{Overview: Comm} 


\end{frame} 









%TODO: add in simple bulk-synchronous code here. 
%TODO: justify why these old graphs are fine. 
\begin{frame}
\frametitle{Example Synchronous MPI code}
\lstinputlisting{../listings/mpi-bulk-synch.c}
\end{frame}


\begin{frame}
\frametitle{Transient Load Imbalance and its Mitigation}
\begin{columns} 
\column{0.5\columnwidth}
{\small Performance Irregularities on individual nodes can lead to
  significant slowdown of synchronous MPI codes. }
\column{0.5\columnwidth}
{\small One promising approach, given multi-core nodes, is to only
re-distribute work within each node (this is inexpensive). } 
\end{columns} 
%{$\rightarrow$ \small Within-node performance irregularities that
%induce load imbalances are as significant a factor for applications as
%across-node load imbalances. The below gives an intuitive
%explanation of why this is so.}

%TODO: animate this to show  right one slide later. 
\begin{figure}[ht]
  \label{fig:NoiseTransImbPic}
  \label{fig:AmplificationPic1}\subfloat[\label{fig:AmplificationPic1}
    \tiny Noise occuring on different nodes in different
    iterations delays every iteration. 
  ]{\includegraphics[height=1.0in]{../pictures/NoiseAmplificationPic4}}
  \label{fig:AmplificationPic3}\hspace*{0.25in}\subfloat[\label{fig:AmplificationPic3}
    \tiny Execution times is significantly reduced, if we assume load
    can be perfectly re-distributed within each node. 
  ]{\includegraphics[height=1.0in]{../pictures/NoiseMitigatedPic5}}
  \caption{\label{fig:NoiseTransImbPic} Schematics of application
    timelines showing how impact of noise can be mitigated by
    idealized within-node work re-distribution.} 
\end{figure}
\end{frame}
%TODO: finalize that this should go here. We explain that the problem
%will become even greater when laod imbalance is.  
\begin{frame}[label=workredistributionNoise]
\frametitle{Within-node Load Imbalance May Help}
\begin{figure}[ht!]
\label{fig:stat-histo}
\begin{center}
\includegraphics[scale=0.16]{../images/IterTimesHisto-outliers-static}\\
%\includegraphics[scale=0.29]{../images/IterTimesHisto-outliers-dynamic}
%\includegraphics[scale=0.35]{../images/IterTimingsHisto-dynamic}
\end{center}
\caption{\label{fig:stat-histo} \small Iteration timing distribution across 1000 timesteps of the
statically scheduled\comments{\hyperlink{3Dmesh}{\beamerbutton{3D
      reg. mesh}}} 3D regular mesh computation on one node of Power5+. }
\end{figure} 

\comments{
\begin{figure}[h!]
\label{fig:work-redistribution-histo-ideal}
\begin{center}
\fbox{\includegraphics[scale=0.23]{../images/emptySchedImage}}\\
{\tiny todo - need to add (idealized) iteration timing histogram here with single bar at
  6.75 ms} 
\end{center}
%TODO: could re-word the below further. 
\caption{\label{fig:work-redistribution-histo-ideal} \small  }
\end{figure} 
}
\begin{itemize} 
\tiny  \item \tiny Iteration timing distribution
  for 3D regular mesh across 1000 timesteps using no work
  re-distribution\comments{across cores} during each iteration cause
  bi-modal distribution. The second mode on the right causes 
  performance degradations. 
 \item \tiny To minimize the impact of noise to application performance, 
  we would need to re-distribute work in a way that causes the above
  histogram be the one with minimal spread and mean.
\end{itemize} 

\end{frame}

\begin{frame}[label=mitigationWithinNode]
\frametitle{Within-node Persistent Load Imbalance and Mitigation of It}
\begin{figure}[ht]
\label{fig:AppPersistentImbPic}
\subfloat[\tiny A modeled application timeline having persistent load
imbalance across nodes, and cores during execution.]{\includegraphics[height=1.0in]{../pictures/PersistentCores}}
\label{fig:AmplificationPic3}\hspace*{0.25in}\subfloat[\label{fig:AmplificationPic3}\tiny
  A modeled application timeline with work re-distributed across cores,
to reduce load imbalance.]{\includegraphics[height=1.0in]{../pictures/PersistentNode}}
\caption{\label{fig:AppPersistentImbPic} Application-induced load imbalances, which are typically jagged and persistent, are also mitigated by within-node re-balancing. }
\end{figure}
\end{frame}

\begin{frame}[label=ImbWithinNode]
%\frametitle{Performance Doesn't Scale with Cores as Exp}
\frametitle{Within-node Imbalance May Help}
{\small We measure the load on each core for an n-body application code called Rebound.}\\
{\small Mitigated Imbalance: Assuming load within each node is perfectly balanced across its cores without any overhead.}
\hspace*{2.5in}\includegraphics[scale=.12]{../pictures/persistentImb}\hspace*{1in}\includegraphics[scale=.12]{../pictures/mitigatedImb}
\vspace*{-0.1in}
\begin{center}
\input{../plots/bh-withinAcrossImb-cab}
\end{center}
%TODO: show calculation for this. 
\begin{center}
{\tiny Imbalance is measured as $\frac{{max\{load_{i}\}}}{avg\{load_{i}\}}$, where $i$ is the processing element.}
%TODO word the below better 
\end{center}
\end{frame}

%TODO: see if we can change title 
\begin{frame}[label=dynLoadBal]
\frametitle{So, can dynamic load balancing fix this?}
\begin{itemize}
\item Dynamic load balancing within a node can improve performance. 
\begin{enumerate}
\item For both kinds of imbalances. 
\item Persistent imbalance can also be addressed by across-node balancing (avail. in Charm++, Zoltan), but this is complementary.
\end{enumerate}
\vspace*{.5in}
\item However, within-node dynamic load balancing faces many challenges. 
\end{itemize}
\end{frame}

\begin{frame}[label=idleTimeCalc]
\frametitle{Challenge 1: Idle Time due to Within-node Imbalance}
{\tiny How we measured idle time:}
\begin{center}
\lstinputlisting{../listings/omp-CG-idleTimeCalc.c}
\end{center}
\end{frame}

\begin{frame}[label=basictimings]
\frametitle{Challenge 1: Idle Time due to Within-node Imbalance}
{\tiny \textit{\textbf{Setup and Methodology:}} Used NASLU OpenMP
  benchmark code and Barnes-Hut pthread application code. Ran on one
  node of LLNL cab cluster, which is an Intel Xeon 16-core with CHAOS
  operating system.\comments{using gcc optimization level 'O3'}. \comments{having OpenMP3 and mvapich3}}
\visible<1->{
\begin{figure}[t]
\begin{center} 
\includegraphics[scale=0.35]{../plots/dmTime-nbody-cab-idleonly}
\includegraphics[scale=0.35]{../plots/dmTime-NASLU-cab-idleonly}
\end{center}
\caption{\small Breakdown of execution time for a load imbalanced code
  (left) and a load balanced code (right) on cab, shown for the
  3\comments{available}OpenMP scheduling strategies.}
\end{figure}}

\visible<1->{
\begin{itemize}
\tiny {\item \tiny  Dynamic or guided scheduling eliminates idle times in both codes.}
\tiny {\item \tiny However, for NASLU, the total execution time actually increases with both dynamic and guided scheduling.}
%TODO : make within-node dynamic balancing clear. 
\begin{itemize}
\tiny {\item \tiny  Recall that (within-node) dynamic load balancing is needed even for naturally balanced applications, to handle noise.}
\end{itemize}
\tiny {\item \tiny  For n-body, the execution time does not decrease as much as we expect, based on the idle time in the static scheduling.}
\end{itemize}}
\end{frame}

%TODO: word the expected overhead of idle time better. 
\begin{frame}
\frametitle{Why is the Non-idle Time Increasing?}
%\begin{itemize}
%\tiny \item \tiny Scheduling overhead?
%\end{itemize}
\begin{figure}[t]
\begin{center}
\includegraphics[scale=0.35]{../plots/dmTime-nbody-cab-idleonly}
\includegraphics[scale=0.35]{../plots/dmTime-NASLU-cab-idleonly}
\end{center}
\end{figure}
\begin{center}
{\small Breakdown of execution time for a load imbalanced code (left) and a load balanced code (right) on cab.}
\end{center}
{\small (Answer: Mostly data movement, as we will see in the next few slides.)}
\end{frame}

%TODO: check cause/challenge here and in thesis. 
\begin{frame}
\frametitle{Challenge 2: Synchronization Overhead}
{\tiny \textbf{\textit{Methodology}}: Obtain call-path profile from
  hpcToolkit which gives percent time spent in \texttt{omp\_lock()}
  OpenMP runtime function.}
{\tiny comp: computation time,  dq: locking and synchronization, 
Dm:  data movement, due to extra cache misses, etc. } 
\begin{figure}[t]
\label{fig:dmTimes-cab}
\begin{center}
\includegraphics[scale=0.35]{../plots/dmTime-nbody-cab}
\includegraphics[scale=0.35]{../plots/dmTime-NASLU-cab}
\end{center}
%\caption{\label{fig:dmTimes-cab} \tiny Breakdown of time. Synchronization overheads are shown in green.}
\end{figure}
\begin{center}
{\small Breakdown of time. Synchronization overheads are shown in green.}
\end{center}
\begin{itemize}
\tiny \item \tiny Synchronization overheads exist when using dynamic or guided scheduling.
\item \tiny While seemingly insignificant, they account for roughly 5\% percent of execution time for both of these applications. 
\end{itemize}
\end{frame}

%TODO: figure out whether we need data movement on BG/Q. 
%TODO: figure out whether we need to show data movement on bw. 
%TODO: see if you should show static time, to explain prefetching benefit. 
%TODO: separate NUMA and non-NUMA slides.
%TODO: add in details on architecture/platform settings. 
%TODO: add in job script specifics.
%TODO: add number of nodes run on.
%TODO: add in timestep length and problem size. 
%TODO: explain why for different setups 

\begin{frame}
\frametitle{Challenge 3: Data Movement}
{\tiny \textit{\textbf{Methodology}}: Measured L2 and L3 cache misses using PAPI counters, placed just before and just after OpenMP threaded computation region. \comments{(Note: will add code in next slide if I can} } 
\visible<1->{
\begin{figure}[t]
\label{fig:cacheMisses-cab} 
\begin{center}
\includegraphics[scale=0.30]{../plots/L2-cacheMisses-nbody-cab}
\includegraphics[scale=0.30]{../plots/L2-cacheMisses-NASLU-cab}\\
\includegraphics[scale=0.30]{../plots/L3-cacheMisses-nbody-cab}
\includegraphics[scale=0.30]{../plots/L3-cacheMisses-NASLU-cab}
\end{center}
\end{figure}
\begin{center}
{\small Cache misses with L2 misses on top and L3 on bottom, for different OpenMP scheduling strategies.}
\end{center}
}
\visible<1->{
\begin{itemize}
{\tiny \item \tiny For both codes, cache misses increase significantly with dynamic or 
guided scheduling, suggesting an  explanation for the performance loss.}
\end{itemize}
} 
\comments{ {\footnotesize Cache miss calcs in thesis write-up  } }
\end{frame} 

\begin{frame}
\frametitle{Calculations for Cache Misses for NASLU} 
{\small To verify that cache misses are in fact impacting performance 
(and is in fact the cause of the increase in data movement time), we show calculations below. }\\
{\tiny L2 cache miss latency is 100 - 300 cycles (with a 2.66 Ghz processor, this is 4 - 8 ns). }\\
{\tiny L3 cache miss latency is 100 ns. }\\
\begin{figure}[ht]
\label{fig:cacheMisses-cab-forCalcs}
\begin{center}
\includegraphics[scale=0.20]{../plots/L2-cacheMisses-NASLU-cab}\includegraphics[scale=0.20]{../plots/dmTime-NASLU-cab}
\end{center}
\end{figure} 

\begin{itemize}
\tiny \item \tiny We need to account for 485.52 - 83.53 - 24.65 = 
376.34 seconds of time spent in dyn sched. Note that we don't need the
execution time breakdown here, since there is low idle time for this code. 
\item \tiny Cache miss time for L2 is $\frac{200}{2.66 \times 10^9}
  \cdot (3845- 2257 \times 10^6) = 65.04$ s. 
\item \tiny 376.34 - 65.04 = 311.44 seconds to account for.
\item \tiny Cache miss time for L3 is $(100 \cdot 10^-9) \cdot (1676
  - 631) \times 10^6 =  97.82 $ s. 
\item \tiny Still 311.44 - 97.64 = 214.20 seconds to account for. 
  The cost of L3 cache miss may be much higher than 100 ns,
  because access to main memory by multiple threads requires threads to wait in memory
  queue. In the worst case, i.e., when all threads access main memory simultaneously, 
  the latency to memory is factor of $(16+(16-mem\_queue\_depth)\times
  100$ ns of the original latency, where mem\_queue\_depth=12 on this
  machine. 
  %We didn't take
  %into account the execution time for memory buffers for , which can
  %make 
\item \tiny Additionally, to verify that memory bandwidth was the
  factor, we ran STREAM on one core and STREAM running all cores, and
  recorded the bandwidth for both runs. 
\item \tiny Memory bw of STREAM triad on one core is 14837.24 MB/s, but on 16
  cores it is only 103669.71 MB/s. The effective mem. latency
  increases by factor of 2.4x, giving 238.4 s instead of 97.64 s for
  Cache miss time for L3. 
\item \tiny The remainder of time is now 311 - 238.34 = 80 s. This may
  due to L2 being higher (also, note that the L2 latency was reported at
  1 core), or by inefficiencies in parallelization. 
\end{itemize} 
\end{frame}

\comments{
\begin{frame}
\frametitle{Calculations for Cache Misses for bh}
Similarly, for n-body: 
\begin{figure}[ht]
\label{fig:cacheMisses-cab-forCalcs}
\begin{center}
\includegraphics[scale=0.30]{../plots/L2-cacheMisses-nbody-cab}\\
\includegraphics[scale=0.43]{../plots/dmTime-nbody-cab}
\end{center}
\end{figure}
\end{frame}
}
%TODO: consider new title 
% possible Question from audience: why is guided different from mixed
% static/dynamic?  Answer: because it doesn't preserve outer iteration
% data locality, and because on the first chunks, there is dqueue
% overhead. 
% Also, the sizes of the tasks can be controlled in guided, but can't
% handle staggering, i.e., spatial locality. 

\begin{frame}
\frametitle{Thesis Objective}
\underline{\textbf{Thesis Objective:}} Design a set of new scheduling strategies that handles all three causes of the problem, i.e., thread idle
time, data movement, and synchronization overhead simultaneously for
many applications and platforms, in the context of bulk-synchronous and loosely synchronous
MPI applications. 
% all at once, which  OpenMP static scheduling and OpenMP dynamic scheduling. 
\end{frame}
%TODO: change the below to the decomp 

\begin{frame}
\frametitle{Key Idea of Solution}
\visible<1->{\small Can we combine a static and dynamic scheduling scheme, which
  simultaneously reduces scheduler overhead and load imbalance
  (respectively) in an intelligent manner?}\\ 
\visible<1->{
\begin{columns}
\column{0.9\textwidth}
In spite of noise, most of the time, all cores are available to
the application. 
\begin{enumerate}
  \small \item \small Allocate a fixed fraction of
  loop iterations statically. (Then do remaining loop
  iterations dynamically). 
\item \small Ratio of static loop iterations to all loop iterations is the \textit{static fraction}.
\item \small \textit{How do we select the static fraction?}  
\end{enumerate}
\end{columns}
}
\end{frame}

%Note:  add in the cost model .
%TODO: check that we brought in ULT load imbalance.

\begin{frame}
\frametitle{Overview of Scheduling Strategies}
\begin{columns}
\column{0.95\textwidth}
       {\tiny \underline{\textbf{Available OpenMP Schedulers}}} \\
       \begin{itemize}
         \tiny \item \tiny \textcolor{black}{OpenMP static scheduling (\textit{static} or \textit{OpenMP-static})}\\ 
       \item \tiny \textcolor{black}{OpenMP dynamic scheduling (\textit{dynamic} or \textit{OpenMP-dynamic})}\\
       \item \tiny  \textcolor{black} {OpenMP guided scheduling(\textit{guided} or \textit{OpenMP-guided})}\\
       \end{itemize}
           {\tiny  \underline{\textbf{Scheduling Strategy Optimizations}}}\\
           \begin{itemize}
             \tiny \item \tiny \textcolor{black}{mixed static/dynamic scheduling(\textit{half})}
           \item \tiny \textcolor{black}{hybrid static/dynamic scheduling(\textit{besf})}
           \item \tiny \textcolor{black} {lightweight scheduling \textit{ligh}, which includes locality-aware scheduling(\textit{loca}) and skewed workloads(\textit{skew})}
           \item \tiny \textcolor{black} {weighted locality-sensitive scheduling (\textit{weig}, \textit{uWsched} or
             \textit{uWldB}), which consists of hybrid static/dynamic
             scheduling and weighted allocation (\textit{wSched} or \textit{wldB})}.
           \item \tiny \textcolor{black} {model-guided Hybrid
             Static/Dynamic Scheduling (\textit{Static-Hybrid}), and
             slack-conscious hybrid scheduling
             (\textit{callsite\_fd})}. Slack-conscious
             hybrid static/dynamic scheduling is referred to as Adaptive Scheduling in the paper.  
           \item \tiny \textcolor{black} {staggered hybrid static/dynamic scheduling (\textit{sds} or \textit{vSched})}
           \item \tiny \textcolor{black}{{Example combination of schedulers (\textit{allStrat} or \textit{full}) }}
           \end{itemize}
               {\tiny \underline{\textbf {Methodology for Tuning Scheduler Parameters}}}
        \begin{itemize}
          \tiny \item \tiny \textcolor{black} {experimental tuning}
        \item \tiny \textcolor{black} {model-guided optimization} %TODO: change to pruning search space 
        \item \tiny \textcolor{black} {runtime adjustment} %TODO: add within-node runtime adjustment and across node adjustment 
        \end{itemize}
\end{columns}
\end{frame}

\begin{frame}
\frametitle{Hybrid Static/Dynamic Scheduling and Lightweight Scheduling} 
\comments{
\column{0.5\columnwidth} 
\begin{enumerate}
\item \textbf{Introduction:}
\begin{itemize}
\tiny \item \tiny Show basic scheduling scheme and provide explanation of why it is beneficial. 
\item \tiny Explain inefficiency of load balanced code due to noise, with just one node. 
\item \tiny Show additional ways to provide low-overhead dynamic scheduling. 
\item \tiny Intuitive explanation of amplification problem, and benefits of our scheduling strategy to scale. 
\end{itemize} 
\item \textbf{Technique Overview}:
  \begin{itemize}
  \tiny \item \tiny \textbf{Scheduling Strategies}:Mixed static/dynamic, Hybrid Static/Dynamic Scheduling,Lightweight scheduling (which includes locality-aware scheduling and skewed workloads)
  \small \item \small \textbf{Tuning Methodology}: Experimental Tuning  
\end{itemize}
\item \textbf{Architectures:}
\begin{itemize} 
\tiny \item \tiny \textbf{cab}: Intel Xeon Cluster at LLNL having 16 cores per node, CHAOS operating system, mpich2, openMP 3.4. 
\item \tiny \textbf{bp}: IBM Power5+ Cluster at Univ. of Illinois having 16 cores per node, IBM AIX Operating System, ibmmpi, lomp. 
\end{itemize}
\item \textbf{Applications:}
\begin{itemize}
\tiny \item \tiny NASLU: 
\item \tiny nbody: 
\item \tiny 3D stencil: 
\end{itemize}
\end{enumerate}
}

\begin{columns}[T]
  \column{0.5\textwidth}
  \textbf{Introduction:}\\
    \begin{itemize} 
      \tiny \item \tiny Show basic scheduling scheme and provide explanation of why it is beneficial. 
    \item \tiny Explain inefficiency of load balanced code due to noise, with just one node. 
    \item \tiny Show additional ways to provide low-overhead dynamic scheduling. 
    \item \tiny Intuitive explanation of amplification problem, and benefits of our scheduling strategy to scale. 
    \end{itemize}
    \vrule{10in}{1pt}
    %\vrule width 10in height 1pt
    \column{0.5\textwidth}  
           {\tiny  \underline{\textbf{Scheduling Strategy Optimizations:}}}\\
           \begin{itemize}
             \tiny \item \tiny mixed static/dynamic $\rightarrow$ hybrid static/dynamic  (base). 
           \item \tiny Lightweight scheduling (which includes locality-aware scheduling and skewed workloads). 
           \item \tiny \textcolor{white} {weighted locality-sensitive scheduling}
           \item \tiny \textcolor{white} {slack-conscious scheduling}
           \item \tiny \textcolor{white} {staggered hybrid static/dynamic scheduling}
           \end{itemize}
               {\tiny \underline{\textbf { Methodology for Tuning Scheduler Parameters:}}}\\
               \begin{itemize}
                \tiny \item \tiny experimental tuning
               \item \tiny \textcolor{white} {model-guided optimization} %TODO: change to pruning search space 
               \item \tiny \textcolor{white} {runtime adjustment}  %TODO: add within-node runtime adjustment and across node adjustment 
               \end{itemize}
\end{columns}  
\end{frame} 

\begin{frame}
\frametitle{Static and Dynamic Scheduling}
\includegraphics[scale=0.30]{../images/legend-dynamic} \\
\begin{columns}
  \column{0.5\columnwidth}
  \lstinputlisting{../listings/threadedCompRegion-static.c}
  \column{0.5\columnwidth}
  \begin{center}
    \includegraphics[scale=0.35]{../images/threadedCompRegion-static} \\
  \end{center} 
\end{columns}
\begin{columns}
\column{0.5\columnwidth}
\lstinputlisting{../listings/threadedCompRegion-dynamic.c}
\column{0.5\columnwidth}
  \begin{center}
    \includegraphics[scale=0.35]{../images/threadedCompRegion-dynamic}\\
  \end{center}
\end{columns}
%\end{block}

\begin{columns}
\column{0.5\columnwidth} 
- 
\column{0.5\columnwidth}
  \begin{center}
    \includegraphics[scale=0.35]{../images/emptySchedImage}
  \end{center}
\end{columns}
\end{frame}

\begin{frame}[label=mixedstatdyn]
\frametitle{Mixed Static/Dynamic Scheduling}
%\begin{block}
\begin{columns}
  \column{0.5\columnwidth}
  \lstinputlisting{../listings/threadedCompRegion-static.c}
  \column{0.5\columnwidth}
  \includegraphics[scale=0.15]{../images/legend-dynamic}\\
  \begin{center}
    \includegraphics[scale=0.35]{../images/threadedCompRegion-static} \\
  \end{center}
\end{columns}
%\end{block} 
%\begin{block}
\begin{columns}
\column{0.5\columnwidth}
\lstinputlisting{../listings/threadedCompRegion-dynamic.c}
\column{0.5\columnwidth}
  \begin{center}
    \includegraphics[scale=0.35]{../images/threadedCompRegion-dynamic}\\
  \end{center}
\end{columns}
%\end{block}
\begin{columns}
\column{0.5\columnwidth}
\lstinputlisting{../listings/threadedCompRegion-mixed.c}
\column{0.5\columnwidth}
\begin{center}
\includegraphics[scale=0.35]{../images/threadedCompRegion-mixed}
\end{center}
\end{columns}
\end{frame}

\begin{frame}[label=halftimings]
\frametitle{Naive Mixed Static/Dynamic Scheduling}
{\tiny Setup and Methodology is the same as that used in
  \hyperlink{basictimings}{\beamerbutton{here}}.} 
%Used NASLU OpenMP benchmark code and Barnes-Hut pthread application
%code. Ran on one node of LLNL cab cluster, which is an Intel Xeon
%16-core with CHAOS operating system.\comments{using gcc optimization
%level 'O3'}. \comments{having OpenMP3 and mvapich3}} 
\begin{figure}[t]
\begin{center}
\includegraphics[scale=0.34]{../plots/dmTime-nbody-cab-withsd}
\includegraphics[scale=0.34]{../plots/dmTime-NASLU-cab-withsd}\\
\end{center}
\label{fig:dmTimes-NASLU-cab}
\end{figure}
\begin{center}
{\small Set static fraction to 0.5: In \textit{half}, threads do the
  first half of their iterations statically. See rightmost bar. }
\end{center}
\begin{itemize}
\tiny \item {\tiny We expect \textit{half} to cut the dynamic overhead in half.}
\item {\tiny The \textit{half} scheduling provides 10.1\% gain over
  OpenMP guided for the nbody code, due to preservation of locality
  across invocations of threaded computation regions and thus less
  data movement across invocations of the threaded computation
  region.}
\item {\tiny For the NASLU code, the \textit{half} scheduling still has high overhead.}
\end{itemize}
\end{frame}

\begin{frame}[label=dmTimes]
\frametitle{Data Movement with Naive Mixed Static/Dynamic Scheduling}
\visible<1->{
\begin{figure}[t]
\label{fig:cacheMisses-cab-withsd}
\begin{center}
\includegraphics[scale=0.30]{../plots/L2-cacheMisses-nbody-cab-withsd}
\includegraphics[scale=0.30]{../plots/L2-cacheMisses-NASLU-cab-withsd}\\
\includegraphics[scale=0.30]{../plots/L3-cacheMisses-nbody-cab-withsd}
\includegraphics[scale=0.30]{../plots/L3-cacheMisses-NASLU-cab-withsd}
\end{center}
%\caption{\label{fig:cacheMisses-cab-withsd} L2 cache misses (top) and L3 cache misses (bottom).}
\end{figure}
\begin{center}
{\small L2 cache misses shown in the top graphs; L3 in the
bottom. \comments{The \textit{half} is added on the rightmost bar.}
% The data shown is based on 100 trials run on a single node of the
% respective machines. Data movement time is obtained by subtracting
% the sum of idle time, dequeue time and computation time, from total
% execution time.
}
\end{center}
}
\visible<1->{\begin{itemize} 
{\tiny \item \tiny With \textit{half}, cache misses decrease with respect to OpenMP guided, for both nbody and NASLU codes.}
\end{itemize}}
\end{frame}

\begin{frame}[label=noPreset]
\frametitle{Problem: A Preset Static Fraction Doesn't Work Effectively for all Application-architecture Pairs}
\begin{itemize}
\item The above shows that the basic static/dynamic scheduling 
improves performance only for nbody on cab. So, we make the strategy more flexible. 
\end{itemize}
\begin{itemize}
\item We expose the static fraction to the application programmer.  
\item What is right blend of static and dynamic scheduling to use? 
\item The application programmer tunes the blend of static and dynamic scheduling.
\item We refer to this as \textit{Hybrid Static/Dynamic Scheduling}. 
\end{itemize}
\end{frame}

\begin{frame}[label=hybridstatdyn]
\frametitle{Hybrid Static/Dynamic Scheduling}
\begin{columns}
  \column{0.5\columnwidth}
  \lstinputlisting{../listings/threadedCompRegion-static.c}
  \column{0.5\columnwidth}
  \includegraphics[scale=0.15]{../images/legend-dynamic}\\
  \begin{center}
    \includegraphics[scale=0.35]{../images/threadedCompRegion-static} \\
  \end{center} 
\end{columns}
%\end{block} 
%\begin{block}
\begin{columns}
\column{0.5\columnwidth}
\lstinputlisting{../listings/threadedCompRegion-dynamic.c}
\column{0.5\columnwidth}
  \begin{center}
    \includegraphics[scale=0.35]{../images/threadedCompRegion-dynamic}\\
  \end{center}
\end{columns}
%\end{block} 

\begin{columns}
\column{0.5\columnwidth}
%TODO: fix code  to be hybrid sched
\lstinputlisting{../listings/threadedCompRegion-hybrid.c}
\column{0.5\columnwidth}
\begin{center}
\includegraphics[scale=0.35]{../images/threadedCompRegion-hybrid}
\end{center}
\end{columns}
\end{frame}

%TODO: think of a better name for title. 
\begin{frame}[label=expTuningSF]
\frametitle{Experimentally Tuning the Static Fraction}
%TODO: explain why increments of 0.01 
%TODO: explain number of trials run 
{\tiny We vary the static fraction environment variable in increments of 0.01. \comments{(We do 0.01 to get enough data points, but more refinement might give less improvement. )} } 
\begin{figure}[ht]
\begin{center} 
\includegraphics[scale=0.19]{../plots/expTuningSF-nbody-cab-WallTime}
\includegraphics[scale=0.19]{../plots/expTuningSF-NASLU-cab-WallTime}\\
\end{center} 
\begin{center}
{\small Performance for different static fractions when using hybrid static/dynamic scheduling strategy for nbody and NASLU. }
\end{center}
\end{figure}

\begin{itemize}
\tiny \item \tiny For n-body, fully dynamic is better than fully static, but static fraction of 49\% gives even better performance. 
\item \tiny For NASLU, the fully static is much better than fully
  dynamic. However, static fraction of 96\% is the best (although
  compared with the 100\% dynamic, the difference between ``best'' and
  fully static may appear small, it is still a significant
  improvement).  
\item \tiny Note that the best static fraction varies across
  application. 
%Comment: make note about half
\end{itemize}
\end{frame} 

\begin{frame}[label=halftimings]
\frametitle{Hybrid Static/Dynamic Scheduling}
\begin{figure}[t]
\begin{center}
\includegraphics[scale=0.34]{../plots/dmTime-nbody-cab-withsdandbesf}
\includegraphics[scale=0.34]{../plots/dmTime-NASLU-cab-withsdandbesf}\\
\end{center}
\label{fig:dmTimes-NASLU-cab}
\end{figure}
\begin{center}
{\small In \textit{besf}, we use the best static fraction. See rightmost bar.}
\end{center}
%TODO: fix this so it doesn't duplicate

\visible<1->{
\begin{itemize}
\tiny \item \tiny When using \textit{besf} scheduling, cache misses decrease w.r.t. \textit{half}, for both nbody and NASLU codes. This is
because we tune to balance the tradeoff between load balance and locality, to minimize the costs in each of the three challenges, simultaneously.
\item \tiny Paper [statdyn1] studies this idea in more depth.
\end{itemize}}
\end{frame}

\begin{frame}
\frametitle{Data Movement with Hybrid Static/Dynamic Scheduling}
\visible<1->{
\begin{figure}[t]
\label{fig:cacheMisses-cab}
\begin{center}
\includegraphics[scale=0.30]{../plots/L2-cacheMisses-nbody-cab-withbesf}
\includegraphics[scale=0.30]{../plots/L2-cacheMisses-NASLU-cab-withbesf}\\
\includegraphics[scale=0.30]{../plots/L3-cacheMisses-nbody-cab-withbesf}
\includegraphics[scale=0.30]{../plots/L3-cacheMisses-NASLU-cab-withbesf}
\end{center}
%\caption{\label{fig:cacheMisses-nbody-cab} L2 cache misses (top) and L3 cache misses (bottom).}
\caption{\small Cache misses for \textit{besf}, added on the rightmost bar. }
\end{figure}
}
%\begin{center}
%The data shown is based on a 100 trials run on a single node of the respective machines. Data movement time is obtained by subtracting the sum of idle time, dequeue time and computation time, from total execution time.
%} 
%\end{center} 
\visible<1->{
\begin{itemize} 
\tiny \item \tiny When using \textit{besf} scheduling, cache misses 
decrease w.r.t. \textit{half}, for both nbody and NASLU codes. This is
because we tune to balance the tradeoff between load balance and
locality, to minimize the costs in each of the three challenges,
simultaneously. 
\item \tiny Paper [statdyn1] \comments{\footcite{dynwork}} studies
  this idea in more depth. 
\end{itemize}}
\end{frame}

\begin{frame}[label=3Dmesh]
\frametitle{3D Regular Mesh process+thread Domain Decomposition}
\begin{columns}
  \column{0.35\textwidth}
  \begin{center}
    \includegraphics[scale=0.35]{../images/appTimestep-noslack}
 \end{center}
\column{0.65\textwidth}

\begin{figure}[ht]
  \label{fig:3DStencil-domainDecomp-stat}
  \begin{center}
    \includegraphics[scale=0.18]{../images/3DStencil-domainDecomp-stat}
  \end{center}
  \caption{\label{fig:3DStencil-domainDecomp-stat} {\small 3D Regular Mesh domain
  decomposition across processes, along with thread partitioning
  of work within each process.}}
\end{figure}
\end{columns}
\end{frame}


\begin{frame}[label=perfVar]
\frametitle{Performance Variation and Absolute Performance for Static Scheduling}
{\tiny \underline{\textbf{Setup and Methodology:}} Run on IBM Power5+ cluster at UIUC which has 16 cores per node\comments{32MB L1 cache and 128MB L2 cache} and AIX operating system.}
\begin{figure}[h!]
\label{fig:stat-histo}
\begin{center}
\includegraphics[scale=0.29]{../images/IterTimesHisto-outliers-static}\\
%\includegraphics[scale=0.35]{../images/IterTimingsHisto-dynamic}
\end{center}
\caption{\label{fig:stat-histo} \small Iteration timing distribution across 1000 timesteps of the statically scheduled\comments{\hyperlink{3Dmesh}{\beamerbutton{3D reg. mesh}}} 3D regular mesh computation.}
\end{figure}
\begin{center}
{\small $\rightarrow$ \textcolor{gray}{Bi-modal distribution for static scheduling.} \comments{Uni-modal distribution for dynamic scheduling, but increased spread and higher mean for dynamic scheduling.}}
\end{center}
\end{frame}

\begin{frame}[label=lightweightDynDefs]
\frametitle{Lightweight Dynamic Scheduling}
\textbf{Tasklet}: A tasklet specifies a set of iterations and records the core that it was executed on in the preceding iteration. 
\begin{enumerate}
\item \textbf{Skewed Workload}: The distribution of tasklets in the shared queue, from beginning to end, is 16 tasklets of size 16, 32 tasklet of size 8, and 64 tasklets of size 4. 
\item \textbf{Locality-aware Scheduling}: On each iteration, a thread tries to search for the tasklet that it executed during the preceding iteration. 
%\item \textbf{Work-stealing}: On each iteration, a thread tries to search for the tasklet that it executed during the preceding iteration. 
\end{enumerate}
\end{frame}

\begin{frame}[label=HybStatDynAppliedto3DMesh]
\frametitle{Hybrid Static/Dynamic Scheduling Applied to 3D Regular Mesh}
\begin{columns}
  \column{0.10\textwidth}

  \column{0.90\textwidth}
  \begin{figure}
    \label{fig:3DStencil-domainDecomp-sd}
    \begin{center}
      \includegraphics[scale=0.18]{../images/3DStencil-domainDecomp-sd}
    \end{center}
    \caption{\label{fig:3DStencil-domainDecomp-hybrid} \small 3D Regular Mesh domain decomposition with Hybrid Static/Dynamic Scheduling Strategy Applied.}
  \end{figure}
\end{columns}
\end{frame}

\begin{frame}
\frametitle{Hybrid Static/Dynamic Scheduling with Lightweight Dynamic Scheduling in Dynamic Section}
{\tiny Setup and methodology the same as \hyperlink{perfVar}{\textcolor{blue}{here}}.} 
\begin{figure}[h!]
\label{fig:skewed2}
\begin{center}
\includegraphics[scale=0.28]{../images/IterTimesHisto-skewedWorkload2}
\end{center}
\caption{\label{fig:skewed2} Iteration timing histograms for 100\% dynamic scheduling, 100\% dynamic scheduling with locality, and 50\% dynamic scheduling.}
\end{figure}
\begin{itemize}
\tiny \item \tiny 50\% dynamic provides better absolute performance,
as well as less performance variation than 100\% dynamic scheduling
without locality and 100\% dynamic scheduling with locality. 
% \tiny \item \tiny Each percent dynamic has a minima for the number of
%  tasklets in the queue, and skewed does well for larger percent
%  dynamic. 
%\item \tiny The best performance is with 4 planes per planes per
%tasklet, so we use this for the subsequent experimentation. 
%As we vary the number of tasklets in the queue, ther
\end{itemize}
\end{frame}

\begin{frame}[label=noiseAmpExample]
\frametitle{Amplification Problem in the Context of Noise}
\begin{itemize}
\item \small On one node running 1000 iterations of length 10 ms. 
\begin{itemize}
\tiny \item \tiny Assume noise (PI) is of length 5 ms, and it occurs approximately every second.
\item \tiny Execution time without PI 10 ms * 1000 iterations = 10 seconds. 
\item \tiny With PI: 10 out of 1000 iterations will be affected (that is delayed by 5 ms) .  So, the total delay is 50 ms.
\item \tiny Execution time is 10.05 seconds, i.e., 0.5\% delay. 
\end{itemize}
\item \small On two nodes (weak scaling):
\begin{itemize}
\item \tiny Probability that an iteration is delayed on one of the nodes is 1\%, same as above.
\item \tiny Probability that a given iteration is delayed on any processor is : $1.0 - (1 - 0.01)^2$ = $1.99$\% .
\item \tiny The delay is then $1000 \times 0.0199 \times 5$  = $99.5$ ms for a total execution time of $10.0995$ s. 
\end{itemize}
\item \small On p nodes :
\begin{itemize}
\item \tiny The delay is $1000 \times (1.0 - 0.99^p) \times 5$ ms
\item \tiny For 10 nodes: $1000 \times 0.0956 \times 5$ ms =  478 ms
\item \tiny For 100 nodes: $1000 \times 0.634 \times 5$ ms = 3170 ms
\item \tiny For 1000 nodes: $1000 \times 0.9999 \times 5$ = 4999.5 ms
\end{itemize}
\item \small That's a 50\% performance loss. 
\item \small As $p$ increases, the probability that one node experiences a delay on a given iteration starts approaching 100\%. 
\small \item \small Caveat for loosely synchronous: amplification will come at a larger number of nodes than for bulk-synchronous, though magnitude of performance degradation may be the same as bulk-synchronous code. 
\end{itemize}
\end{frame}

\begin{frame}[label=noiseampformula]
\frametitle{Amplification Problem in Context of Noise} 
\begin{itemize}
\small \item \small Let $T_p$ be iteration time on one node with $p$ cores, with no noise. We assume weak scaling. 
\item \small  Let $\delta$ be the noise duration.
\item \small Let $\phi$ be the expected number of noise events per second, on one node. \item \small We assume global synchronization with zero overhead, e.g., allreduce.
\end{itemize}

\textbf{----------------------------------------------------------}
\begin{itemize}
\small \item \small Probability of noise happening in one iteration on one node = $\phi \times T_p$. 
\item \small Probability that there is no delay on any nodes while running on k nodes is $(1 - \phi \times T_p)^k$. 
\item \small Probability that \textit{some} node experiences a delay in a given iteration is $1 - (1- \phi \times T_p)^k$. 
\item \small So, expected execution time of an iteration on k nodes is: $E[T_k]  >=  T_p + \delta \times  [1 - (1- \phi \times T_p)^k]$. 
\end{itemize}
%( Caveat of slack) 
\end{frame}

\begin{frame}
\frametitle{Improvement of Scalability for 3D Stencil}
\begin{figure}
\includegraphics[scale=0.34]{../images/stencil-scaling}
\caption{Scaling 3D Regular Mesh for different schedulers.}
\begin{itemize}
\tiny {\item \tiny Mixed static/dynamic scheduling results in improved 
scalability over both static scheduling and dynamic scheduling.}
\end{itemize}
\end{figure}
\end{frame}

%TODO: show graph for fully dynamic scheduling.
%\vivek{fix layout}
\begin{frame}
\frametitle{Impact on Scaling}
\begin{figure}[ht!]
\label{fig:scaling-histograms}
\begin{center}
\includegraphics[scale=0.20]{../images/scaling-histograms}
\end{center}
%TODO: word the below better
\caption{\label{fig:scaling-histograms} Iteration timing histograms 
for static scheduling~(left) for mixed static/dynamic scheduling~(right) as we increase the 
number of nodes.}
\end{figure}

\begin{enumerate}
\tiny \item \tiny Distribution of two equal peaks goes to one small
peak and one large peak. 
% Performance variation increases with increasing numbers of
% nodes
\item \tiny The mean of the distribution for mixed 
static/dynamic scheduling stays consistent 
as we increase the number of nodes. 
\item \tiny The mean of the distribution at 64 nodes is better 
  with static/dynamic scheduling than that of static
  scheduling. %TODO: think of a better point here. 
\end{enumerate}
\end{frame}

\begin{frame}
\frametitle{Improvement Over an Already Optimized Dense Matrix Factorization using Hybrid Static/Dynamic Scheduling}
%\frametitle{For Problem 1: Does our scheduling approach perform competitively or better than other industry wide schedulers?}
\begin{columns}
\column{0.5\textwidth}
\begin{figure}
\includegraphics[scale=0.22]{../plots/calu-fastNUMA4-perf}
\caption{\tiny Comparison of hybrid static/dynamic scheduled CALU
  to statically scheduled CALU (first version) as well as 
  other industry standards such as PLASMA and MKL.\comments{showing significant
  performance improvements over the baseline (statically scheduled)
  CALU, particularly with large matrix sizes.}}
\visible<1->{\tiny $\rightarrow$ We get significant
  performance improvements over the baseline (statically scheduled)
  CALU, particularly with large matrix sizes. Our scheduler is 30\% 
  better than PLASMA library's implementation of the same 
  numerical algorithm, and 38\% better than Intel MKL library's 
  impementation of the same numerical algorithm.}
%\end{itemize}
%(Problem: PLASMA has inefficient work-stealing algorithm.) 
\end{figure}
\column{0.5\textwidth}
\begin{figure}
\includegraphics[scale=0.10]{../plots/calu-fastNUMA3-histogram}
\caption{\tiny Performance Variation for static CALU, dynamic CALU, and hybrid static/dynamic CALU.}
\end{figure}
{\tiny $\rightarrow$ Hybrid static/dynamic scheduled CALU reduces
  performance variability while maintaining absolute performance
  better than that of statically scheduled CALU and dynamically
  scheduled CALU, and this is consistent with the performance
  variability results for 3D stencil.} 
\end{columns}
\end{frame}

%TODO: update this
\begin{frame}
\frametitle{Problems with the Basic Lightweight Scheduling Approach}
%\textbf{Two problems with the above basic approach:} 
\begin{enumerate}
\small \item \small Persistent load imbalances not handled. How do we handle both
persistent and transient imbalances together? 
\item \small Experimental tuning: can we reduce the search by using a
model?
\item \small Across-node load imbalance not taken into account in 
scheduling strategy. Can we use runtime-guided adjustments (per-node static
fraction based on slack) to improve performance?
\item \small Our scheduler does not consider spatial locality. How do
  we optimize the mixed static/dynamic scheduling scheme so that we
  improve spatial locality and temporal locality?
\end{enumerate}
\end{frame}

\begin{frame}
\frametitle{Weighted Locality-sensitive Scheduling}
\begin{columns}[T]
% \column{0.50\textwidth}
  \begin{column}{0.5\textwidth}
    {\small \underline{\textbf{Introduction}}} \\
    \begin{itemize}
      \tiny \item \tiny Noise not completely transient; 
      can be a mixture of persistent core noise and transient noise, as observed
      in Jaguar (see next slide). 
    \item \tiny We modify our lightweight scheduling scheme to handle the
      mixture of persistent core noise and transient node-level OS noise. 
    \item \tiny For architectures with 
      the mentioned mixture of noise events, we get noted performance
      gains of 10.4\% over our basic lightweight scheduling scheme. 
    \item \tiny The approach described can be generalized to work for any
      type of mixture of persistent and transient load imbalance, not just
      mixtures of persistent and transient noise. 
    \end{itemize}
  \end{column}
  % \column{0.50\textwidth}
  \begin{column}{0.5\textwidth}
    {\tiny  \underline{\textbf{Scheduling Strategy Optimizations}}}\\
    \begin{itemize}
      \tiny \item \tiny \textcolor{gray}{ mixed static/dynamic (\textit{half}) $\rightarrow$ hybrid static/dynamic \textit{besf} (base)}   
    \item \tiny \textcolor{gray}{ lightweight scheduling (\textit{ligh}) (which includes locality-aware scheduling (\textit{loca}) and skewed workloads \textit{skew})}
    \item \tiny \textcolor{black} {weighted locality-sensitive scheduling (\textit{weig})}
    \item \tiny \textcolor{white} {slack-conscious scheduling: \textit{slac} :  model-guided\textit{Static-Hybrid},   slack-conscious(\textit{callsite\_fd})}
    \item \tiny \textcolor{white} {staggered hybrid static/dynamic scheduling: \textit{ssds}}
    \end{itemize}
        {\tiny \underline{\textbf {Methodology for Tuning Scheduler Parameters}}}
        \begin{itemize}
          \tiny \item \tiny \textcolor{black} {experimental tuning}
        \item \tiny \textcolor{white} {model-guided optimization} %TODO: change to pruning search space 
        \item \tiny \textcolor{black} {runtime adjustment} %TODO: add within-node runtime adjustment and across node adjustment 
        \end{itemize}
  \end{column}
\end{columns}
\end{frame} 

%TODO: figure out whether we need to use coupling. 
%TODO: word the below better 
\begin{frame}
  \frametitle{Persistent Imbalances (Noise-induced) Not Handled} 
  \begin{figure}
    %\includegraphics[scale=0.44]{../images/noise_pes_jaguar}
    \subfloat[\tiny Jaguar quasi-persistence in noise.]{\includegraphics[scale=0.41]{../images/noise_pes_jaguar}}
    \subfloat[\tiny Ranger quasi-persistence in noise.]{ \includegraphics[scale=0.41]{../images/noise_pes_ranger}}
    \caption{\small Time spent by each core in a 1200 core run, for a load balanced code.}
  \end{figure}
  \begin{itemize}
    \tiny \item \tiny The two top bands of indicate slow cores, while
    bottom bands show fast cores (particularly for Jaguar).
  \item \tiny Across all nodes, the subset of cores being fast and
    subset slow is almost the same. 
  \item \tiny We refer to the above phenomenon as persistent core noise,
    or  noise-induced persistent imbalances. 
  %(and note that this could be generalized as a fine-grained application-induced persistent load imbalance across cores).%TODO: explain more on why  
  \end{itemize}
\end{frame}

%TODO: check the question presented in the title of the slide.
\begin{frame}
\frametitle{Blending Weighted Factoring and Mixed Static/Dynamic Scheduling to Handle Persistent Core Noise}
%TODO: add regular diagram for weighted decomp. 
{\small Weighted Factoring: assign different number of iterations to different cores based on history.}
{\small We combine weighted with hybrid static/dynamic scheduling, and
  then tune the static fraction.}\\
{\small Earlier, the number of iterations allocated to thread i was
 $f_s \cdot \frac{N}{p}$}
{\small With weighted, it is: $w_i \cdot \frac{f_s \cdot N}{p}$,
where $w_i$ is calculated by measuring the time spent in the static
portion by different threads, $s_i$, as: $w_i = \frac{avg\{s_i\}}{s_i}$}.
% begining iteration for thread i, b_i = b_{i - 1} +
% w_i*\frac{f_s*N}{p} , with b_0 = 0 
\begin{columns}
  \column{0.5\textwidth}
         {\small Weighted Locality-sensitive Scheduling Strategy Applied to 3D Reg. Mesh.\\}
         \column{0.5\textwidth}
         \begin{center}
           \fbox{\includegraphics[scale=0.2]{../images/weighted_decomp}}\\
         \end{center}
\end{columns}
\end{frame}

\begin{frame}
\frametitle{Weighted Locality-Sensitive Scheduling Results}
\begin{itemize}
\tiny \item \tiny Ranger and Jaguar both experience noise; Ranger is
somewhat noisier.
\item \tiny Only Jaguar has core differentation; some cores experience 
more high-frequency noise than the others.
%\item \tiny 
\end{itemize}

\begin{figure}
\label{fig:uWSched-stencil}
\begin{center}
\includegraphics[scale=0.39]{../plots/timings_jaguar}
\includegraphics[scale=0.39]{../plots/timings_ranger}\\
\end{center}
\caption{\label{fig:uWSched-stencil} 3D Stencil scaling with weighted
  locality-sensitive scheduling for Ranger~(left) and Jaguar~(right).}
\end{figure}

\begin{enumerate}
\tiny \item \tiny Weighted micro-scheduling is the best on Ranger,
although micro-scheduling is almost as good (dynamic load balancing
most important). 
\item \tiny Jaguar: Weighted is better on fewer than 64 nodes and
  micro scheduling is better beyond that. 
\item \tiny Jaguar: Weighted micro-scheduling is better than both of these. 
\end{enumerate}
{\tiny $\rightarrow$ Combining schedulers has no disadvantage.}
%TODO: figure out the avg/max for load imbalance to calculate static fraction. 
%TODO: max - avg . avg / max ; 
\end{frame}

\begin{frame}
\frametitle{Slack-conscious Hybrid Static/Dynamic Scheduling}
\begin{columns}[T]
  \column{0.5\textwidth}
         {\small  \underline{\textbf{Introduction:}}} \\
         \begin{itemize}
           \tiny \item \tiny We first discuss limitations of
           experimentally tuning the lightweight scheduling scheme,
           including the problem of slack. 
         \item \tiny With this, we explain the two techniques of this
           chapter: model-guided lightweight scheduling and
           slack-conscious scheduling technique. 
         \item \tiny We evaluate our slack prediction mechanisms,
           showing scaling results for our model-guided lightweight
           scheduling. 
         \item \tiny We show why our slack-conscious runtime
           adjustment is essential and beneficial at large-scale, by
           showing speedup of slack-conscious scheduling over best
           static fraction (\textit{besf}) scheduling strategy. 
         \end{itemize}
         %\vrule[0.5em]{1pt}{3.5in]
\vrule{} 
         \column{0.5\textwidth}
                {\tiny  \underline{\textbf{Scheduling Strategy Optimizations}}} \\
                \begin{itemize}
                  \tiny \item \tiny \textcolor{gray} {mixed static/dynamic $\rightarrow$ hybrid static/dynamic  (base)}
                \item \tiny  \textcolor{gray} {lightweight scheduling (which includes locality-aware scheduling and skewed workloads)}
                \item \tiny \textcolor{gray} {weighted locality-sensitive scheduling}
                \item \tiny \textcolor{black} {slack-conscious scheduling}
                \item \tiny \textcolor{white} {staggered hybrid static/dynamic scheduling}
                \end{itemize}
                    {\tiny \underline{\textbf { Methodology for Tuning Scheduler Parameters}}}
                    \begin{itemize}
                      \tiny \item \tiny \textcolor{white} {experimental tuning}
                    \item \tiny \textcolor{black} {model-guided optimization} %TODO: change to pruning search space 
                    \item \tiny \textcolor{black} {runtime adjustment}  %TODO: add within-node runtime adjustment and across node adjustment 
                \end{itemize}
\end{columns}
\end{frame}

%TODO: get another graph - this may not show the results as well. 
%TODO: show a caption

\begin{frame}
\frametitle{Large and Complex Search Space}
%TODO: get new picture 
%Also, Could maybe show results for enumeration of search space, and that it's large here. 
%TODO: fix the below 
%Put in modeled graph with two endpoints, showing non-continuous search space. 
\begin{enumerate} 
\item Difficult to find exact static fraction for single node. 
\item Best static fraction changes per node because MPI collective operations involve ``slack''(explained more in depth on next slide).
\end{enumerate}
%\begin{figure}
%\includegraphics[scale=0.20]{../plots/vary-sf-for-diff-threads}
%\caption{\small Graph showing the performance of different static fractions
%  used. Here, we have tried different static fractions, 
%  in increments of 0.1. The minima here is difficult to attain.}
%\includegraphics[scale=0.28]{../plots/best-fs-vs-nodes}
%\end{figure}
\end{frame}

%TODO: add in other data sets.  
%TODO: add in max/avg. 

\comments{
\begin{frame}
\frametitle{Can We Use Theoretical Analysis to Prune the Large Search Space?}
% Model-guided optimization, but with experimental tuning after
% scheduler parameter from model is determined. 
\begin{center}
\includegraphics[scale=0.26]{../plots/app-scaling-strat-SNAP-fastNUMA2}
\includegraphics[scale=0.26]{../plots/app-scaling-strat-nbody-fastNUMA2}\\
\end{center}
\begin{center}
{\small Using a theoretical analysis to acheive the minima helps to provide 13-16\% performance gains for a particle simulation (nbody) and SNAP (reg. mesh), each tested on a 16-core non-NUMA architecture and 16-core NUMA architecture.}
\end{center}
\end{frame}
}

\begin{frame}
\frametitle{Different MPI processes Have Different Cost to Application Critical Path}
{\tiny \underline{\textbf{MPI Slack}} is the time by which a process
  can delay arrival at a synchronizing point without impacting application execution time.}
%
\begin{table}[tr]
\label{fig:collectiveSlack-useq}
  \begin{center}
    \begin{tabular}{ | c || c | c || c |}
      \hline
      \textbf{Collective} &      \textbf{max} & \textbf{avg ($\sigma$)} & \textbf{$\sigma_i$} \\ 
\hline
      \texttt{Allreduce} &             307    &    199 (.792) &            .102 \\ 
\hline
      \texttt{Alltoall}  &             280    &    149 (v.821) &            .176 \\ 
\hline
      \texttt{Barrier}  &              250    &    139 (.178)  &           .061 \\ 
\hline
      \texttt{Allgather} &             268    &    189 (.527) &            .115 \\
\hline
      \texttt{Reduce\_scatter} &       459    &    296  (.649)  &          .129 \\ 
\hline
    \end{tabular}
  \end{center}
  \caption{\label{fig:collectiveSlack-useq} {\small Different slack statistics (in $\mu$s) shown for experiments with different collectives at the end of an iteration. Additional evidence and impact of slack esp. at large scale is discussed in [slackPerf] and [adagio].}}
\end{table}

\begin{itemize}
\tiny \item \tiny Slack distribution is different for different collectives.
\item \tiny Slack varies across processes.
\item \tiny The distribution of slack on different processes evolves slowly for different applications. 
\end{itemize}
\end{frame}

\begin{frame}
\frametitle{Terminology and Performance Model}
\begin{table}[tr]\centering
\begin{tabular} {|c|p{.75\columnwidth}|} \hline
\multicolumn{2}{|c|}{\textit{Model outputs}} \\\hline
$f_d$       & {\tiny Fraction of work scheduled dynamically.} \\\hline
$f_d'$      & {\tiny Fraction of work scheduled dynamically,\newline
                      reduced for slack.} \\\hline
%
\hline
\multicolumn{2}{|c|}{\textit{Model inputs}} \\\hline
$q$                     & {\tiny Overhead of dequeueing single iteration.} \\\hline
$S_i$                   & {\tiny Slack duration on process $i$.} \\\hline
$\delta$        & {\tiny Expected noise duration.} \\\hline
$T_p$       & {\tiny Execution time on $p$ cores with fully static scheduling and no noise.} \\\hline
$N$         & {\tiny Total number of iterations across all threads.}\\\hline
% 
$t_1$       & {\tiny Duration of single iteration on a single thread, on one core.} \\\hline
$d$                     & {\tiny Duration of single dynamic iteration on a single thread.} \\\hline
%$\eta$          & Compute delay due to noise and queue overhead \\\hline
%$\alpha_i$      &  Delay added to the critical path by node $i$\\\hline
\end{tabular}
\caption{\small Overview of all model parameters. Note that we use
  the dynamic fraction $f_d$ instead of static fraction $f_s$ to simplify our
  theoretical analysis.\label{tab:parameters}}
\end{table}
\end{frame}

\begin{frame}[label=mghybridstatdyn]
\frametitle{Model-guided Hybrid Static/Dynamic Scheduling}
{\small Instead of tuning the dynamic fraction, we establish a model to
determine it based on noise duration $\delta$, and other model parameters. \\}
\begin{center}
  \fbox{\includegraphics[scale=0.46]{../images/hybSched-nonoisenoslack}} \\
\end{center}

{\small Considering the above model, along with our goal of avoiding load balancing until it is absolutely needed, 
the dynamic fraction applied to each MPI process is: \\}
\begin{center}
  \fbox{{$f_d = \frac{p\delta}{N\left(t_1 + q\right)}$}}\\
\end{center}
    {\small We refer to the strategy described above as Model-guided Hybrid
      Static/Dynamic Scheduling. This strategy is referred to as 
      \textit{Static-Hybrid} in the graphs.} 
\end{frame}

\begin{frame}
\frametitle{Considering Slack Factor for Hybrid Scheduling}
{\small With this, let us now consider the slack factor for each MPI process.} 

\begin{figure}
  \begin{center}
    \includegraphics[scale=0.46]{../images/slackConsciousSched} \\
  \end{center}
\end{figure}

\begin{itemize}
\tiny \item \tiny Slack allows us to reduce the dynamic fraction
further, because a small amount of delay due to noise will lead to no
delay in the completion of an MPI operation. For example, if you are
in a Reduction and waiting for data from your spanning tree
children. 
\item \tiny Note that the slack is different on different nodes,
  depending on the implementation of the collective operation. 
\item \tiny Therefore, the dynamic fraction will be different on
  different processors. Nodes with large slack can get away with 
small dynamic fraction without impacting application performance.
\end{itemize}
\end{frame}

\begin{frame}[label=slackfs]
\frametitle{Slack-conscious Hybrid Static/Dynamic Scheduling
  a.k.a. Adaptive Scheduling} 
\begin{figure}
  \begin{center}
    \includegraphics[scale=0.41]{../images/slackConsciousSched} \\
  \end{center}
\end{figure}
\begin{center}
\fbox{{$f_d = \frac{p\delta}{N\left(t_1 + q\right)}$}}
\fbox{ {$f_d' = f_d - \frac{S}{(p-1)\frac{Nt_1}{p} - Nq}$}}
\end{center}
\begin{center}
{\tiny Note that $f_d'$ and $S$ are local variables within each MPI process. }
\end{center}

%TODO: 
{\small The dynamic fraction $f_d$ of a particular node
 is reduced by a value proportionate to the slack on that node, 
to get the new $f_d'$.} \\

{\small To estimate the slack on each processor, we use ideas from
 Adagio (see next slide for impl). \\ We compared multiple methods,
 and chose the \textit{callsite\_fd} method.} \\

{\small Reducing dynamic fraction reduces overall overhead.}

\end{frame}

\comments{
\begin{frame} 
\frametitle{Slack Prediction Mechanism and Usage}  
\begin{enumerate}
\tiny \item \tiny On each process, retrieve that process's invocation of the
  last MPI collective, where the invocation of the last MPI collective is
retrieved through the callsite slack-prediction method (see paper for
details on implementation details of slack prediction).
 \item \tiny Given the identifier of the last the MPI collective call
invoked, estimate that collective call's slack value from the
history of slack values stored by the slack-conscious runtime.
The slack estimate is based on the slack value recorded
in the previous MPI collective invocation, as is done in the
cited work of Adagio.
 \item \tiny On each process, adjust its dynamic fraction based on the
   slack value. This adjustment is done using the formula 5 in section
   3 of Chapter 4  and implementation of section 4 of
   Chapter 4. The static fraction used in the
   loop bound is 1- $f_d$.
\end{enumerate}
\end{frame} 
}

\begin{frame}
\frametitle{Software Architecture}
\begin{figure}
\begin{center}
\label{code:architecture}
\includegraphics[scale=0.56]{../images/architecture}
\end{center}
\end{figure} 

\begin{itemize}
\tiny \item \tiny The white rectangles taken by themselves show the
interaction between the MPI+OpenMP application code and the
MPI \& OpenMP runtimes.  
%how a basic MPI+OpenMP
%application is run.  
\item \tiny Our contributions in this work are shown in gray
  rectangles. %Involves source-to-source transformation and adaptiv
\item \tiny Predict static fraction call (arrow going from compilation
  to Adaptive runtime library) is how MPI runtime
  communicates with OpenMP runtime to tune OpenMP scheduler parameters
  (and other parameters). 
\end{itemize}
\end{frame}

\begin{frame}
\frametitle{Code Transformation}
\vspace*{-0.4in}
\begin{columns}
\column{0.5\textwidth}
\begin{figure}
\label{code:origOMPcode}
%\subfloat[\label{code:origOMPcode}{\tiny Loop transformed for slack-conscious scheduling.}]
\begin{center} 
\lstinputlisting{../listings/omp-static.c}
\end{center} 
\end{figure}
\column{0.5\textwidth}
\begin{figure}
\label{code:refactoredOMPcode}
\begin{center} 
\lstinputlisting{../listings/omp-hybrid.c}
\end{center} 
%\caption{\label{code:refactoredOMPcode}\tiny Loop transformed for
%  slack-conscious scheduling.}
\end{figure}
\end{columns}
\begin{enumerate}
\tiny \item \tiny The \texttt{decide\_static\_fraction()} and 
\texttt{endLoop()} functions are calls to the slack-conscious runtime. 
\item \tiny The runtime maintains a \texttt{LoopTimeRecord} struct
  \texttt{lr} across invocations for a threaded computation region. 
  The \texttt{LoopTimeRecord} struct includes storage for
  parameters of loop time, slack, est. sched overhead time, and number of
  iterations, along with various timers obtained at different points during
  the execution of the OpenMP threaded computation region. 
\item \tiny Using wrappers around the MPI collective, the 
  slack-conscious runtime predicts slack, and calculates optimal
  static fraction using the model. 
\item \tiny For the first invocation of the threaded computation
  region, \texttt{lr} is empty, so the runtime system uses static
  fraction of 100\%. 
  % the simpler model without slack to determine the
  % static fraction. 
  % calls around the MPI comm. 
\end{enumerate}
\end{frame}

\begin{frame}[label=runtimeparameterestimation]
\frametitle{Scheduler Parameter Estimation}
%{\small To calculate the value of $\delta$, $\frac{N \cdot t_1}{p}$ 
%  and $q$ for computing the slack-conscious static fraction
%  based on \hyperlink{slackfs}{\beamerbutton{Slide 46}}}:}
\begin{itemize}
\small \item \small The value of the noise parameter $\delta$ is 
estimated in the \texttt{MPI\_Init()} function: we subtract the
maximum and minimum times for the iterations of a basic OpenMP loop
performing a square root operation. 
\comments{and then do an
MPI\_Allreduce() with the max operator to collect the maximum across
all iteration times.}
\item \small We estimate the scheduler overhead $q$ by subtracting 
performance of a statically scheduled OpenMP loop doing a dot product
from a dynamically scheduled OpenMP loop doing a dot product.  
\item \small The time $\frac{N \cdot t_1}{p}$ is estimated by taking 
the time taken for statically scheduled section. 
 \comments{of the last 5 invocations of the threaded computation 
region and taking the minimum execution time of these times.} We 
obtain $t_1$ by dividing this time by the number of iterations
executed statically, and use it to calculate $\frac{N\cdot t_1}{p}$.  
% during the execution of the threaded computation
% region, 

%Note that this gives an estimate of 
%scheduler overhead, in order to give a tighter bound on the static 
%fraction. 
\end{itemize}
%\comments{ 
\begin{enumerate}
\tiny \item \tiny We perform our estimates within the MPI\_Init() 
function (intercepted by PMPI), using a wrapper script. Our calculations needed are done
after the real MPI\_Init() has finished. 
%\tiny \item \tiny The value for noise estimated is 5.4\% greater than that of netgauge
%noise measurements.(possible future direction for netgauge is to make
%it output noise event length delta, rather than percent noise). 
%\tiny \item \tiny Note that values noise and scheduler overhead are estimates, rather
%than the exact values; as we will see in the results section, 
%these estimates are adequate to give us significant performance
%gains. 
\end{enumerate}
%}
\end{frame}

\begin{frame}
\frametitle{Process-local Slack Measurement}
%{\small To calculate the slack S, for computing the slack-conscious static fraction based on 
%\hyperlink{slackfs}{\beamerbutton{Slide 46}}:} 
\begin{itemize} 
\small \item \small \textbf{Callpath.} Associate slack trace values
with only the type of collective operation that had the slack. 
\item \small \textbf{Collective.} Storing the full callpath requires
  overhead of unwinding the stack. Associate slack trace values with
  only the type of the collective operation that had the slack.
\item \small \textbf{Naive.} Forget slack tracing. Predict
  slack based on a precomputed experiment. Specifically, 
  run 1,000 invocations of a simple MPI\_Allreduce() with a
  message size of 4 bytes on all MPI processes before the
  application begins, i.e., right after MPI\_Init(). 
\end{itemize}

\begin{enumerate}
\tiny \item \tiny Tradeoff between runtime overhead for the slack measurement, and the 
advantage due to Adaptive Hybrid Static/Dynamic Scheduling.
\item \tiny Note that this implementation is based on the Adagio
runtime, and that we did not use the Adagio runtime directly. 
\end{enumerate}
\end{frame}

\begin{frame}
\frametitle{Slack Prediction Mechanism Overheads and Errors}
\begin{itemize}
\tiny \item \tiny \textit{Slack prediction error} for a particular MPI
region is the percentage difference 
between the predicted slack before the MPI region begins 
and the actual slack recorded at the end of the MPI region.
\item \tiny \textit{Slack prediction overhead} for a particular MPI
  region is the time spent in our slack prediction runtime.
\end{itemize}
\begin{figure}
\includegraphics[scale=0.30]{../plots/slackErr-fastNUMA2}
\includegraphics[scale=0.30]{../plots/slackOvhd-fastNUMA2} \\
\includegraphics[scale=0.30]{../plots/slackErr-rzuseq}
\includegraphics[scale=0.30]{../plots/slackOvhd-rzuseq}
%TODO: see if schemes or runtimes is better
\caption{\tiny Overheads and errors of different available slack
  prediction schemes in our runtime system. Each process sums the
  total error and overhead across all MPI regions, and we report below
  average overhead and error across MPI processes.}
\end{figure}
\end{frame}

\begin{frame}
\frametitle{Improving Performance through Model-guided Optimization And Slack-Conscious Scheduling}
%\frametitle{Can We Use This Slack Information to Adjust the Scheduler Parameters So As to Reduce this Cost?}
\begin{figure}
\includegraphics[scale=0.34]{../plots/app-scaling-fastNUMA2}
\caption{\small Speedup of slack-conscious hybrid static/dynamic scheduling over OpenMP static scheduling for cab with varying number of nodes.}
\end{figure}
\begin{itemize}
\tiny \item \tiny Using a theoretical analysis to acheive the minima
with Static-Hybrid helps to provide 13-16\% performance gains for a
particle simulation (nbody) and SNAP (reg. mesh), each tested on a
16-core non-NUMA architecture and 16-core NUMA architecture.
\item \tiny Slack-conscious scheduling  provides additional gains
  of 23.5\% for NASLU-MZ and 25.5\% for SNAP.
\end{itemize}
\end{frame}

\begin{frame}
\frametitle{Improving Performance through Model-guided Optimization And Slack-Conscious Scheduling}
%\frametitle{Can We Use This Slack Information to Adjust the Scheduler Parameters So As to Reduce this Cost?}
\begin{figure}
\includegraphics[scale=0.34]{../plots/app-scaling-rzuseq}
\caption{\small Speedup of slack-conscious static fraction over OpenMP
  static for rzuseqwith varying number of nodes.}
\end{figure}
\begin{itemize}
\tiny \item \tiny Using our scheduling strategy does not incur
significant performance degradation (less than 1\% on all node counts
tested) for a non-noisy machine. 
\item \tiny We are able obtain some performance gain (5.5\% at 1024 
  nodes) for our amg experiment which has a small amount of application load imbalance in the code.
\end{itemize}
\end{frame}

\begin{frame}[label=bestfsvsadpt]
%\frametitle{Why Should We Care About Model-guided and Runtime-guided Optimizations to Our Lightweight Scheduling Strategies?}
\frametitle{Benefits of Using Runtime Adjustment over Experimental Tuning}
\begin{figure}
\includegraphics[scale=0.50]{../plots/best-fs-vs-adpt}
\caption{\small Shows how runtime-adjusted static fraction for each application run on cab
works better than the tuning approach (which is best static fraction
(\textit{besf}) strategy). Speedup of slack-conscious static fraction
over best static fraction is shown.}
%This shows the importance of model-guided optimization and
%runtime-based adjustment (esp. with increasing nodes). }
\end{figure}
\end{frame}

\begin{frame}[label=locOptHybrid]
\frametitle{Locality-Optimized Mixed Static/Dynamic Scheduling}
\begin{columns}[T]
\column{0.5\textwidth}
{\small \underline{\textbf{Introduction:}}} \\
\begin{itemize}
  \comments{\tiny \item \tiny We need to avoid disturbing spatial locality while also making our scheduler as lightweight as possible.}
  \tiny \item \tiny We first describe our locality-optimizations,
  explaining why the original disturbs spatial locality and how we can
  improve it while maintaining our original lightweight scheduling
  scheme's benefits. 
\item \tiny We describe our implementation of a library that is
  invoked within an OpenMP threaded computation to schedule
  iterations, explaining that it is low overhead compared to OpenMP
  static.  
%\item \tiny We show results for a Barnes-hut pthread code and regular mesh OpenMP code using our scheduling library. 
\item \tiny We show results for a Barnes-hut and regular mesh code 
  using our scheduling library.
\item \tiny We show results that indicate that our locality-optimized 
strategy controls the performance variation (discussed
\hyperlink{perfVar}{\textcolor{blue}{here}}) just as well or better than
does the basic hybrid static/dynamic scheduling. 
%\comments{(which is beneficial for scalability)}. 
\end{itemize}
\vrule{}
\column{0.5\textwidth}
       {\tiny  \underline{\textbf{Scheduling Strategy Optimizations}}}\\
       \begin{itemize}
         \tiny \item \tiny \textcolor{gray}{ mixed static/dynamic $\rightarrow$ hybrid static/dynamic  (base)}
       \item \tiny  \textcolor{gray}{lightweight scheduling (which includes locality-aware scheduling and skewed workloads)}
       \item \tiny \textcolor{gray} {weighted locality-sensitive scheduling}
       \item \tiny \textcolor{gray} {slack-conscious scheduling}
       \item \tiny \textcolor{black} {staggered hybrid static/dynamic scheduling}
       \end{itemize}
    {\tiny \underline{\textbf { Methodology for Tuning Scheduler Parameters}}}
    \begin{itemize}
      \tiny \item \tiny \textcolor{black} {experimental tuning}
    \item \tiny \textcolor{white} {model-guided optimization} %TODO: change to pruning search space 
    \item \tiny \textcolor{white} {runtime adjustment} %TODO: add within-node runtime adjustment and across node adjustment
    \end{itemize}
\end{columns}
\end{frame}

\comments{
low-probability term  for terms  -     precise  approximation .    

-    correctness of the equality . 
- intuition 

-  storyboard   

simple techniques can make your code faster.   
}

\begin{frame}
\frametitle{Improving Spatial Locality in Dynamically Scheduled Iterations}
{\tiny Ignoring slack for the moment, consider the diagram shown in slide 45
  showing hybrid static/dynamic scheduling. \\}
{\tiny  From this diagram, consider the sequence of loop iterations that
  all 4 threads work on. (Note that the x-axis below is increasing
  iteration numbers, and not time ). \\} 
\begin{center}
\includegraphics[scale=0.35]{../images/sdSched} \\
\end{center}
\begin{itemize}
\small \item \small The consecutive iterations typically access consecutive memory locations. 
\item \small Spatial locality benefits static iterations, but is lost in dynamic iterations. 
\item \small Given this, we could get additional performance benefit if we bias distribution of dynamic iterations, so that each thread is likely to execute dynamic iterations 
that are contiguous with its static iterations.  
\end{itemize}
\end{frame}

\begin{frame}[label=sdsSched]
%\frametitle{Question: Can we Re-distribute the Iterations So that Spatial Locality Isn't Lost?}
\frametitle{Spatial Locality in the Hybrid Static/dynamic Scheduling Approach}
\begin{center}
\begin{figure}
\includegraphics[scale=0.35]{../images/sdsSched}
%TODO: update the below, to word better for intent. 
\caption{\small We stagger the dynamic iterations in order to improve spatial locality of the basic static/dynamic scheduling.}
\end{figure}
\end{center}

\begin{itemize}
\small \item \small Strategy: Each thread finishes its static iterations, then does dynamic iterations marked for it if available, and only then looks for other dynamic work. 
\item \small Spatial locality will not be lost in common cases: 
\begin{enumerate}
\small \item \small When there is no imbalance. 
\item \small Even when there is imbalance espeicially for noise, most threads are able to complete their own dynamic iterations.
\end{enumerate}
\end{itemize}
\end{frame}

\begin{frame}[label=efficientImpl]
%\frametitle{Question: Can we develop an efficient scheduler library that handles different tradeoffs between load balance and locality?}
\frametitle{Note on Efficient Implementation of Scheduler Library}
\begin{center}
\input{../plots/runtime-costs}
\end{center}
\begin{itemize}
\small \item \small Overhead of our library is relatively small. 
\end{itemize}
\end{frame}

\begin{frame}[label=claresults]
%TODO: add in varying number of cores to show improvements. 
\frametitle{Performance Improvements with Varying Problem Sizes}
\begin{figure}
\includegraphics[scale=0.40]{../plots/bh-16cores}
\includegraphics[scale=0.35]{../plots/st-16cores}
\caption{\small Shows performance improvements when using our vSched
  library (having staggered static/dynamic scheduling strategy), for a
  Barnes-Hut pthread particle simulation code (right) from the
  Lonestar Benchmark Suite and the SNAP 2D regular mesh code (right).}
\end{figure}
\begin{itemize}
\tiny \item \tiny Using our staggered hybrid static/dynamic
scheduling (\textit{stag}) for the 2D regular mesh code
with 65536 mesh points, we get reasonable performance gain of 6.45\% 
over OpenMP static scheduling. 
\item \tiny Using our staggered static/dynamic scheduling approach for
 the Barnes-Hut code at 100,000 particles, we get significant gains of
 18.53\%.
\item \tiny Our strategy provides increased performance gains as we
  increase the problem size. 
\end{itemize}
\end{frame}

\begin{frame}[label=controlPerfVar]
\frametitle{Controlling Performance Variability}
%\frametitle{Do the optimizations maintain performance consistency characteristics of static/dynamic scheduli}
\begin{center}
\input{../plots/runtime-stddev}
\end{center}
\begin{itemize}
\tiny \item \tiny With dynamic scheduling, performance can vary 
from run to run (even for static/dynamic scheduling).
\item \tiny Runtime standard deviations are controlled with our
  scheduler optimizations done in vSched, probably because of more
  persistent and predictable behavior(because each timestep each
  thread is likely executing the same dynamic iterations). 
\end{itemize}
\end{frame}

\begin{frame}[label=combinedintro]
\frametitle{Scheduler Composition}
\begin{columns}
\column{0.5\textwidth}
{\small \textbf{\underline{Introduction}:}}\\
\begin{itemize}
\tiny \item \tiny Some schedulers may benefit in some cases, others elsewhere.
\item \tiny Problem: Machine and application has all above mentioned factors involved. 
\item \tiny We show how to combine all schedulers. We also apply it to 3 different application codes. 
\item \tiny We discuss application programmer usability. 
\item \tiny Finally, we re-visit the amplification problem, and discuss how our schedulers work at large-scale.
\end{itemize}

\vrule{}
 
\column{0.5\textwidth}
{\tiny  \underline{\textbf{Scheduling Strategy Optimizations:}}}\\
\begin{itemize}
  \tiny \item \tiny mixed static/dynamic (base)
\item \tiny lightweight scheduling 
\item \tiny weighted locality-sensitive
\item \tiny slack-conscious scheduling
\item \tiny constrained staggering
\item \tiny $\rightarrow$ Each opens up new opportunities to balance the tradeoff between load balance and locality. % The main \point is that there are many dimensions of scheduler 
\end{itemize}

{\tiny \underline{\textbf { Methodology for Tuning Scheduler Parameters}}}\\
\begin{itemize}
  \tiny \item \tiny experimental auto-tuning
\item \tiny model-guided optimization
\item \tiny runtime adjustment
\item \tiny $\rightarrow$ Each are important mechanisms to find the best tradeoff between load balance and locality.
\end{itemize}
\end{columns}
\end{frame}

\begin{frame}[label=combinedSched]
\frametitle{Components of Example Composed Scheduler}
\begin{itemize}
\small \item \small \textbf{uSched}: Use model-guided optimization
  to determine the static fraction (we have separate cases for load
  imbalanced due to noise and load imbalance due to the application), 
  and then experimentally tune the static fraction $0.05$ below static
  fraction and $0.05$ above static fraction ($0.05$ is a configurable
  parameter). 
\item \small \textbf{vSched}: Almost the same as the \textit{ssds}
  scheduling strategy. The difference here is that we use the 
  static fraction obtained from \textbf{uSch} above
  rather than experimentally tuning the static fraction. 
\item \small \textbf{callsite\_fd}: Almost the same as the Slack-conscious Scheduling strategy. 
  The difference here is that initial static fraction to use is obtained from \textbf{uSched} above. 
\item \small \textbf{comboSched}: We start with the staggered static/dynamic
  scheduling scheme defined in \textbf{vSched} above, and then do the 
  slack-conscious adjustment described in slack-conscious scheduling section.
\end{itemize}
\end{frame} 

\comments{
\begin{frame}[label=combined] 
\frametitle{Scheduler Composition of Example Composed Scheduler} 
\begin{figure}[ht]
\label{fig:sched-composition}
\begin{center}
\includegraphics[scale=0.35]{../pictures/sched-composition-diagram}
\end{center}
\caption{\label{fig:sched-composition} Composition of 3 schedulers
which make ComboSched.}
\end{figure} 
\end{frame} 
}

\begin{frame}[label=combinednbody]
\frametitle{Combined Scheduling Results for Rebound}
{\tiny Rebound(*) is an MPI+OpenMP particle simulation code involving large load
imbalances as well as irregular data access patterns.}
\begin{figure}
\includegraphics[scale=0.43]{../plots/app-scaling-strat-nbody-fastNUMA2}
\caption{\tiny Percent speedup over OpenMP static scheduling for each
  scheduling strategy for the particle simulation code. Run on
  \texttt{cab} using 16 threads per node (and 1 MPI process per
  node).}

\end{figure}
\begin{itemize}
\tiny \item \tiny Combining different techniques seems to add on
benefits, i.e., they don't cancel benefit out. 
\item \tiny Because they are based on complementary factors. Spatial
  locality in the static/dynamic scheduling vs. reducing dynamic
  scheduling. 
%\item \tiny better combination could lead to even more improvement 
\end{itemize}

{ \tiny *Rebound is actually the name a suite of particle simulation application
  codes, and we considered the shearing sheet simulation. Here, we use the  
  name Rebound to identify the shearing sheet application run in our experiments.  }
\end{frame}

\begin{frame}[label=combinedfe]
\frametitle{Combined Scheduling Results for MiniFE}
{\tiny  CORAL miniFE MPI+OpenMP code runs a finite element method. 
  This involves irregular computation, and relatively low load imbalance.} 
\begin{figure}
%\includegraphics[scale=0.38]{../plots/app-scaling-strat-SNAP-craybw}\\
\includegraphics[scale=0.43]{../plots/app-scaling-strat-SNAP-fastNUMA2}
\caption{\tiny Shows how the application of full scheduling to miniFE 
works better than the application of other approaches, by showing speedup over OpenMP
static scheduling for each scheduling strategy for different numbers
of nodes. Run on \texttt{cab} using 16 threads per node
(and 1 MPI process per node).}
\end{figure}

\begin{itemize}
\tiny \item \tiny Dynamic scheduling causes significant degradation 
even though there is imbalance. 
\item \tiny Our scheduling strategies provides notable performance 
improvement over static scheduling, though gains less than performance gains for n-body.
\item \tiny One reason is that code\comments{'s computation} has 
relatively significant amount of across-node load imblance relative
to within-node load imbalance, while n-body has little to no load
imbalance. 
\end{itemize}

\end{frame}

\begin{frame}[label=combinedregmesh]
\frametitle{Combined Scheduling Results for Regular Mesh}
{\tiny CORAL SNAP is an MPI+OpenMP code that does a regular mesh
  computation used in heat diffusion. This computation is similar to
  that in chapter 2.}
\begin{figure}
\includegraphics[scale=0.43]{../plots/app-scaling-strat-fe-fastNUMA2}\\
\caption{\tiny Shows how the full scheduling for SNAP 
works better than other approaches, by showing speedup over OpenMP
static scheduling for each scheduling strategy for different numbers
of nodes. The above is run on \texttt{cab} using 16 threads per node, 
and 1 MPI process per node.}
\end{figure}

\begin{itemize}
\tiny \item \tiny Even when dynamic strategies worsen performance, the
combined strategies show a small benefit. 
%\item \tiny uSched doesn't do as well 
\end{itemize}
\end{frame}

\comments{
\begin{frame} 
\frametitle{Code with Scheduling Composition} 
\begin{figure}[h!t]
              \label{code:dvSchedMacro}
              \begin{center}
                \lstinputlisting{../listings/dvSchedMacro-CG.c}
             \end{center}
              \caption{\label{code:dvSchedMacro}{\small NAS CG code
                  modification using our scheduling lib\
rary.}}
            \end{figure}
\end{frame} 
}

\begin{frame}[label=programmerEffort]
\frametitle{Is Our Method Easy to Use for the Application Programmer?}
{\tiny Rebound: (-add total loc-) SNAP: (-add total loc-)  miniFE: (-add total loc-) \\}
\input{../plots/linesOfCodeChanged}
%Experiment: Show lines of code modified, as shown in fourth quadrant
%of SC poster. 

\begin{itemize}
\tiny \item \tiny Lines of code changed is minimal for all three codes. 
\item \tiny Code transformation via ROSE \comments{(and changes to
  GOMP)} can reduce or eliminate need for code change.
\end{itemize}
%TODO: explain
\end{frame}

\comments{
\begin{frame} 
\frametitle{Load Imbalance Metrics With Increasing Numbers of Nodes}
\begin{figure}
\label{fig:bh-withinAcrossImb-cab}
\begin{center}
  \subfloat[Load Imbalance Across Nodes] {\includegraphics[width=0.5\textwidth] {../plots/imb-maxavg-nbody-cab-inodes}}
  \subfloat[Load Imbalance Across Cores] {\includegraphics[width=0.5\textwidth] {../plots/imb-maxavg-nbody-cab-icores}}
\end{center}
\caption{\label{fig:bh-withinAcrossImb-cab}{Standard load imbalance metrics taken for a galaxy 
    simulation running on 1024 nodes of \textit{cab}, which is an Intel Westmere
    machine with 16 cores per node. The code is run with one MPI process per core.}}
% Overheads of our scheduling runtime.\comments{shown through comparison between our static scheduling implementation and OpenMP static scheduling.}
\end{figure} 
\end{frame}
}
\comments{
\begin{frame}[label=ampProbRevisit]
\frametitle{Amplification Problem Revisited}
{\tiny (Ignore for now.)} 
\begin{figure}
\label{fig:modeledScurve}
\begin{center}
\includegraphics[scale=0.38]{../plots/amplification-model-Scurve1}
\end{center}
\caption{\label{fig:modeledScurve}\small Modeled scaling curve based
  on formula for expected length of the $k^{th}$ iteration,
  described~\hyperlink{noiseampformula}{\beamerbutton{here}}. In the formula, we use I=1000, $\delta$ = 5 ms, $\phi$ = 1.}
\end{figure}
\end{frame}
}

\begin{frame}[label=ampProbRevisit]
\frametitle{Impact to Extreme-Scale}
\begin{itemize} 
 \small \item \small As we go to future machines including exascale
\begin{enumerate} 
 \tiny  \item \tiny  number of nodes increases 
\item \tiny cores per node increases 
\end{enumerate} 
\item most importantly, static and dynamic variabilty increases. 
\begin{enumerate}
\tiny \item \tiny Operating voltage might decrease, which increases
  variability and decreases reliability. 
\item \tiny  Fault rates increase; these will be contained
through multiple mechanisms locally, but this will lead to transient
load imbalances. 
 \end{enumerate} 
\item \small Applications will need strong scaling, as exascale reports say,
  because memory per core will be lower, and scientists would want to
  solve the same problem faster rather than a larger problem at the
  same speed. 
\item \small All of the above reasons will increase the impact of
  amplification. 
\item \small Thereby increasing the need for the kind of schedulers we have developed in
  this dissertation 
\begin{itemize} 
 \item \small Number of cores per node will be substantially higher, which our
techniques can leverage.
\end{itemize} 

\item {\small the multiple techniques that we developed are in
  extensible software library that can be extended for exascale
  machines with newer features and future sophisticated applications. }
% can be
% selectively combined in different 
\end{itemize} 

\comments{
adaptive componets of hardware and software, low power  node 

The number of errors will increase, but it is expected that errors
will be contained localily 

impact of transient events is larger impact of applications .. 
This will lead to   
}
\end{frame} 

\begin{frame}[label=relatedWork]
\frametitle{Related Work}
\begin{itemize}
\item \small DPLASMA: dynamic scheduling for specific Numerical Linear
  Algebra application, not tunable static fraction. 
\item \small Work-stealing: Ignores locality, and can't tune the
  dynamic scheduler. 
\item \small OpenMP loop scheduling: Sarkar, Chapman, deSupinski.
\item \small TBB, Habanero.
%Some work by Sriram et al tries to fix outer
%iteration locality 
\item \small Co-scheduling: Schedule all events together to make only 
  some timesteps be impacted by noise, but can't co-schedule all
  events. 
\item \small Hyperthreading with SMT at LLNL: shows benefits with SMT 
  and can be lightweight, but is a hardware solution (not tunable to
  avoid overheads). 
\item \small Runtime for Irregular applications (Galois): Transient
  load imbalances, but dealing with more frequent irregularity on each
  timestep, and load imbalances are coarse-grained. 
\item \small UPC task scheduler: Can move work across threads with
  some affinity, but this can't be tuned (all is done at runtime).
\end{itemize}
%*Some work/discussions have been done to integrate our scheduler into this runtime
\end{frame}

\begin{frame}[label=conclusions]
  \frametitle{Conclusions}
  \begin{itemize}
  \item \small Performance irregularities create scaling challenges
    and are difficult to handle through conventional techniques. 
  \item \small Locality-aware hybrid static/dynamic scheduling
    provides a way to cost efficiently handle within-node 
    load imbalance induced by performance irregularity. 
  \item \small Formalization and Numerical Linear Algebra Application
    Case Studies. 
  \item \small Additional scheduling techniques and optimizations:
    model-guided determination of $f_d$, adaptive per-thread static fraction, slack-conscious static
    fraction (staggered scheduling for spatial locality in dynamic
    section). 
  \item \small Our runtime does not introduce significant overheads. 
\begin{itemize} 
  \item \small So, even if runtime slowdowns due to performance
    irregularity are small, it can be used.
\end{itemize}  
  \item \small Feasible to integrate into applications through
    compiler transformations and adaptive runtime library. 
  \item \small Demonstrated a novel set of scheduling strategies that
    can be used to overcome challenges posed by dynamic
    irregularities, while minimizing data movement and synchronization
    costs. 
\item \small Directions for future work: (a) can use within MPI-shared memory extensions and
  MPI+X models; and (b) can combine with other across-node load balancers.
%  \item \small Can further refine accuracy of slack prediction through
%   risk model.
%  \item \small Can use mixed non-blocking + blocking collectives,
%  inspired by mixed static/dynamic scheduling.  
    % \item \small Plan to provide more loosely coupled interface between OpenMP and MPI, or between any two programming models.
  \end{itemize}
\end{frame}

%Acknowlegdments of people
%\begin{frame}[Acknowledgements]
\frametitle{Acknowledgements}
%\begin{itemize}
%\item -
%\item -
%\item -
%\end{itemize}
\end{frame}

\begin{frame}[People] 
\frametitle{People} 
\begin{itemize} 
\item William Gropp (UIUC)
\item Torsten Hoefler (UIUC)
\item Bronis de Supinski (LLNL)
\item Todd Gamblin (LLNL)
\item Amanda Peters (collaborators for Human Heartbeat simulation at Harvard University) 
\item Simplice Donfack, Laura Grigori, Ulrike Meier Yang (collaborators for numerical linear algebra at INRIA and LLNL) 
\item Jim Dinan and Darius Buntinas 
and Mike Heroux (collaborators Programming Models) 
\end{itemize}
\end{frame}


%NOTE: this marks the end of the presentation. 
%TODO: figure out if we should add ``questions'' slide.

%\input{thanks}
%People involved in the project 
%\input{people} 
