\documentclass[serif, mathserif, final]{beamer}
\mode<presentation>{\usetheme{Lankton}}
\usepackage[orientation=landscape,size=a0,scale=1.6, debug]{beamerposter}
%\usepackage[orientation=landscape,size=a0,scale=1.4,debug]{beamerposter}   
%\usepackage[absolute,overlay]{textpos} % <-- can  use this to
%position text (solves problem of earlier yesterday with poster placement).  
%\documentclass[serif,mathserif,final]{beamer}                               
%\mode<poster>{\usetheme{RicePoster}}                                      
%\documentclass{beamer}

\usepackage{amsmath,amsfonts,amssymb,pxfonts,xspace}

\usepackage{graphicx, ragged2e}
\usepackage{pgffor}
\usepackage{pgfpages}
\usepackage{pgf}


\pgfpagesdeclarelayout{9 on 1}
{
  \edef\pgfpageoptionheight{\the\paperheight}
  \edef\pgfpageoptionwidth{\the\paperwidth}
  \edef\pgfpageoptionborder{0pt}
}
{
  \pgfpagesphysicalpageoptions
  {
    logical pages=9,
    physical height=\pgfpageoptionheight,
    physical width=\pgfpageoptionwidth
  }
  \def\pgfpgtemp{}
  \foreach \i in {1,...,3} {
    \foreach \j in {1,...,3} {
      \pgfmathtruncatemacro\n{(\j-1)*3 + \i}
      \pgfmathsetmacro\ri{1 - (\i - .5)/3}
      \pgfmathsetmacro\rj{(\j - .5)/3}
      \edef\temp{%
        \noexpand\pgfpageslogicalpageoptions{\n}
      {
        border code=\noexpand\pgfsetlinewidth{2pt}\noexpand\pgfstroke,
        border shrink=\noexpand\pgfpageoptionborder,
        resized width=.33\noexpand\pgfphysicalwidth,
        resized height=.33\noexpand\pgfphysicalheight,
        center=\noexpand\pgfpoint{\rj\noexpand\pgfphysicalwidth}{\ri\noexpand\pgfphysicalheight}
      }
      }
      \expandafter\expandafter\expandafter\gdef\expandafter\expandafter\expandafter\pgfpgtemp\expandafter\expandafter\expandafter{\expandafter\pgfpgtemp\temp}
    }
  }
  \pgfpgtemp
}

\usepackage{ulem}
\usepackage{caption}
\usepackage{courier}

\usepackage{verbatim}

%\usepackage[usenames,dvipsnames]{xcolor}                                    
\usepackage{listings}

\usepackage{float}
\usepackage{subfig}
\setbeamertemplate{caption}[numbered]

\usepackage{calendar}
\usepackage{geometry}

%\graphicspath{{./figures/}}                                                   
%\usepackage{beamerposter}                                         
%\graphicspath{{./figures/}}  

 
\newcommand{\bllt}{\item \small}
\newcommand{\doneTaskNoItem}[1]{\sout{#1}}
\newcommand{\doneTask}[1]{\tiny \item \tiny \sout{#1}}
\newcommand{\doneTaskHyp}[1]{\tiny \item \tiny \textcolor{blue}
  {\sout{#1}}}
\newcommand{\optTask}[1]{\tiny \item \tiny \textcolor{green}{#1}}
\newcommand{\prioTask}[1]{\tiny \item \tiny \textcolor{red}{#1}}
\newcommand{\timeEst}[1]{\textit{Time:} \textit{#1}}
\newcommand{\te}[1]{\textit{TimeEst:} \textit{#1}}
\newcommand{\priority}[1]{\textit{Priority:} \textit{#1}}
\newcommand{\pr}[1]{\textit{Priority:} \textit{#1}}
\newcommand{\prio}[1]{\textit{Priority:} \textit{#1}}
\newcommand{\dueBy}[1]{\textit{Deadline:} \textit{#1}}
\newcommand{\MyName}{Vivek~Kale}
\newcommand{\fixme}[1]{\textcolor{blue}{[FIXME: #1]}}

\newcommand{\revision}[1]{\textcolor{blue}{[FIXME comment : #1]}}
\newcommand{\regItem}[1]{\item \textcolor{cyan}{#1}}
\newcommand{\regRoutineItem}[1]{\item
  \textcolor{green}{\textit{Reg. Routine:} #1}}
\newcommand{\situationItem}[1]{\item
  \textcolor{magenta}{\textit{Situation:} #1\
}}
\newcommand{\deadline}[1]{#1}
\newcommand{\dl}[1]{#1}
\newcommand{\comments}[1]{} 

%\graphicspath{{./figures/}}                                             
%\usepackage[orientation=landscape,size=a0,scale=1.6,debug]{beamerposter}

%-- Header and footer information ----------------------------------          
\newcommand{\footleft}{Mgmt-WorkLife-weekPlan} % figure what this
                                % should be     
\newcommand{\footright}{August 2015}

%------------------------------------------------------------------          
%-- Main Document -------------------------------------------------          
\title{Month-Week-Day Plan}
\author{Vivek Kale$^1$}
\institute{$^1$ University of Illinois at Urbana-Champaign}
\date{\today}

\pgfpagesuselayout{9 on 1}[a0paper, border shrink=5mm,landscape]

%\pgfpagesuselayout{16 on 1}[a0paper, border shrink=5mm,landscape]

\begin{document}

%\foreach \k in {1,...,20} {
%
%\begin{frame}{Frame \k}

%This is frame \k.
%It is exciting.
%\end{frame}
%}
%\end{document}
      %-- Block 1-1                                 
      \begin{frame}
        \frametitle{Implementation slides}
        \begin{enumerate}
          
        \item \tiny 
        \end{enumerate}
      \end{frame}

      \begin{frame}
        \frametitle{Stresses}
        % Recognize emotions / recognize (precisely) what you're
        % worried about 
        % clean out old worries, put them in pastWorries doc.      
        % remember t define precisely                                      
        % order from top-to bottom                                              
        \begin{itemize}
        \item \tiny Work: worrying about interviewing: just
          focus on the work.
        \end{itemize}
      \end{frame}

      \begin{frame}
        \frametitle{wl-cheat-sheet-impl}
        \begin{itemize}
          \tiny \item \tiny If you do something sloppy, don't
          think about it and don't say sorry.
        \item \tiny don't think someone is trying to hurt you if
          they say something.
        \item \tiny know the paths of least stress.
        \item \tiny if mom does something, don't let that bite you.
        \end{itemize}
      \end{frame} 

     %-- Block 1-3                                            
      \begin{frame}
        \frametitle{WeekPlanDoc}
        \begin{itemize}
        \end{itemize}
      \end{frame} 


    \begin{frame}
      \frametitle{Week Plan} 
        \begin{itemize}
          \tiny \item \tiny
        \item \tiny
        \item \tiny
        \item \tiny
        \end{itemize}
      \end{frame}

    \begin{frame}
      \frametitle{Running TODO} 
        \begin{itemize}
          \tiny \item \tiny
        \item \tiny
        \item \tiny
        \item \tiny
        \end{itemize}
      \end{frame}


    \begin{frame}
      \frametitle{Running TODO} 
        \begin{itemize}
          \tiny \item \tiny
        \item \tiny
        \item \tiny
        \item \tiny
        \end{itemize}
      \end{frame}


    \begin{frame}
      \frametitle{News} 
        \begin{itemize}
          \tiny \item \tiny
        \item \tiny
        \item \tiny
        \item \tiny
        \end{itemize}
      \end{frame}

    \begin{frame}
      \frametitle{Weather} 
        \begin{itemize}
          \tiny \item \tiny
        \item \tiny
        \item \tiny
        \item \tiny
        \end{itemize}
      \end{frame}

   \begin{frame}
      \frametitle{Situations} 
        \begin{itemize}
          \tiny \item \tiny
        \item \tiny
        \item \tiny
        \item \tiny
        \end{itemize}
      \end{frame}

\end{document}
  
