\documentclass{beamer}
%\usetheme[
%          titlepagelogo=bold50-illinois,% Logo for the first page
%          pageofpages=of,% String used between the current page and the total page count
%          titleline=true,% Show a line below the frame title
%          ]{TorinoTh}

\usetheme{Luebeck}

\useoutertheme{infolines}

\usepackage{beamerthemesplit}
\usepackage{graphics}
\usepackage{graphicx}
\usepackage{hyperref}
%\usepackage{comment}
\usepackage{ulem}
\usepackage{color}
\usepackage{xspace}

\usepackage{amssymb}
\usepackage{caption}
\usepackage{courier,cite}
%\usepackage[usenames,dvipsnames]{xcolor}
%\usepackage{color,cite}
%\usepackage{scalefnt}
\usepackage{listings}


\newcommand{\comments}[1]{}

\setbeamertemplate{footline}{\insertframenumber/\inserttotalframenumber}

\usepackage{outlines} 
\usepackage{enumitem}
\setenumerate[1]{label=\Roman*.}
\setenumerate[2]{label=\Alph*.}
\setenumerate[3]{label=\roman*.}
\setenumerate[4]{label=\alph*.}

\usepackage{flushend}
\usepackage{float}
\usepackage{subfig}
\usepackage{tikz}

%Get this updated in macports
%\usepackage{biblatex}
%\usepackage{footcite}

\lstset{
    language=C++,
    %basicstyle=\footnotesize\ttfamily, % Standardschrift
    basicstyle=\tiny\ttfamily, % Standardschrift
    numbers=left,               % Ort der Zeilennummern
    numberstyle=\footnotesize\ttfamily,          % Stil der Zeilennummern
    %stepnumber=2,               % Abstand zwischen den Zeilennummern
    numbersep=6pt,              % Abstand der Nummern zum Text
    tabsize=2,                  % Groesse von Tabs
    extendedchars=true,         %
         breaklines=true,            % Zeilen werden Umgebrochen
        frame=single,
         keywordstyle=[1]\textbf,    % Stil der Keywords
 %        keywordstyle=[2]\textbf,    %
 %        keywordstyle=[3]\textbf,    %
 %        keywordstyle=[4]\textbf,   \sqrt{\sqrt{}} %
         stringstyle=\ttfamily,
         showspaces=false,           % Leerzeichen anzeigen ?
         showtabs=false,             % Tabs anzeigen ?
%         xleftmargin=17pt,
%         framexleftmargin=17pt,
%        framexrightmargin=5pt,                                                                                                                                                     %         framexbottommargin=4pt,
         backgroundcolor=\color{white!97!black},
         showstringspaces=false      % Leerzeichen in Strings anzeigen ?
}


\newcommand{\htor}[1]{\textcolor{OliveGreen}{[htor: #1]}}
\newcommand{\tg}[1]{\textcolor{orange}{[tg: #1]}}
\newcommand{\vivek}[1]{\textcolor{cyan}{[vivek: #1]}}

\newcommand{\comments}[1]{}
\newcommand{\bllt}{\item \small}
\newcommand{\MyName}{Vivek~Kale}

\newcommand{\doneTaskNoItem}[1]{\sout{#1}}
\newcommand{\doneTask}[1]{\tiny \item \tiny \sout{#1}}
\newcommand{\doneTaskHyp}[1]{\tiny \item \tiny \textcolor{blue}
  {\sout{#1}}}
\newcommand{\optTask}[1]{\tiny \item \tiny \textcolor{green}{#1}}
\newcommand{\prioTask}[1]{\tiny \item \tiny \textcolor{red}{#1}}

\newcommand{\timeEst}[1]{\textit{TimeEst:}\textit{#1}}
\newcommand{\te}[1]{\textit{TimeEst:}\textit{#1}}
\newcommand{\deadline}[1]{\textit{Deadline:}\textit{#1}}
\newcommand{\dueBy}[1]{\textit{Deadline:}\textit{#1}}
\newcommand{\dl}[1]{\textit{Deadline:}\textit{#1}}
\newcommand{\priority}[1]{\textit{Priority:}\textit{#1}}
\newcommand{\prio}[1]{\textit{Priority:}\textit{#1}}
\newcommand{\pr}[1]{\textit{Priority:}\textit{#1}}

\newcommand{\fixme}[1]{\textcolor{blue}{[FIXME: #1]}}
\newcommand{\todo}[1]{\textcolor{red}{[TODO: #1]}}
\newcommand{\revision}[1]{\textcolor{blue}{[FIXME comment : #1]}}
\newcommand{\regItem}[1]{\item \textcolor{cyan}{#1}}
\newcommand{\regRoutineItem}[1]{\item
  \textcolor{green}{\textit{Reg. Routine:} #1}}
\newcommand{\situationItem}[1]{\item
  \textcolor{magenta}{\textit{Situation:} #1}}

\title{Notes}
\author{\MyName}
\date{\today} 
\title{My Information} 
\author[\today]{
  Vivek~Kale\inst{1}

}

\institute[USC]{
  \inst{1}%
 Information Sciences Institute\\
 University of Southern California
}

\date{\today}
\subject{Computational Sciences}

\begin{document}
\AtBeginSection[]
{
\begin{frame}
\frametitle{Outline}
\tableofcontents[part=1, pausesections]
\end{frame}
}

\titlepage
%\section{Prelim Outline Notes}
%\input{./content_defense}
\begin{frame}{My Information - 1}
\begin{itemize}
  \tiny \item \tiny work ID
  \begin{enumerate}
    \tiny \item \tiny UIUC UIN: 659743643
  \item \tiny LLNL ID: 008983 
  \item \tiny USC ID: 910-1254-783 $\leftarrow$ \textcolor{red}{Memorize} 
  \item \tiny USC ID: 2029269
  \end{enumerate}
\item \tiny Work equipment: 
  \begin{enumerate}
          \tiny \item \tiny Model A1707
          \tiny \item \tiny S/N: C02T30EZGTF1  
          \tiny \item \tiny Computer serial number:
          \tiny \item \tiny private159.east.isi.edu
        \item \tiny IP address:  65.114.168.159
        \item \tiny Whole disk encryption? 
        \end{enumerate}
%      \item \tiny insurance number:
     
        \seti 
        \end{itemize}
        \end{frame}
        \begin{frame}{My Information - 2}
        \begin{itemize}
        \conti 
 \tiny  \item \tiny  User ID
        \begin{enumerate}
       % \tiny \item \tiny bcbs : user: vivekkale pass: qazxsw23 memberID: QME9212849609  Group No.: IG2601. 
        \item \tiny USC:  user:   pass: 
      \item \tiny mycarle: vivekk112 password: carle
      \item \tiny ppn:532296999 exp: 3-16-2025  \textcolor{red}{Memorize} 
      	\item \tiny driver's license:K400-8728-4051 exp:2-20-18
          \tiny \item \tiny bcbs : user: vivekkale pass: qazxsw23 memberID: QME9212849609  Group No.: IG2601.
        \item \tiny USC:  user:   pass: 
        \end{enumerate}
      \item \tiny Money:
        \begin{enumerate}
          \tiny \item \tiny CC1 num: CC2 num: Dbtnum:  \textcolor{red}{Memorize} 
        \item \tiny Bank Account:
          Name: JP Morgan Chase
          bank acct num: Nine Five Eight Six Thirty Sixteen Two
          IBAN/routing num: Three Triple Two Seven One Six Two Seven 
          ABA-Fedwire No.: 021000021
        \item \tiny CCnum: Four One Four Threee
        \item \tiny VIN:  License Plate: IL AU 59436 reg.exp:   Atul's lic: P84 VIN:
        \item \tiny Dad's License Plate: IL A
        \item \tiny Mom's License Plate : IL P84 9023 
        \end{enumerate}
        \seti
        \end{itemize}
        \end{frame}
        \begin{frame}{My Information -3 }
          \begin{itemize}
            \conti  
	  \item Travel: 
            \begin{enumerate} 
%            \item \tiny VIN:  License Plate: IL E61 2444 reg.exp:   Atul's lic: P84 VIN: 
            \item \tiny Parking location codes:   Parking signs: 
            \item \tiny AtulInternet’s FightingIllini  wifi pass: 2173900374 WPA
            \item \tiny xfinity account number: 0149024144. / \url{vivek.lkale@gmail.com}
            \item \tiny ComEd account number: 0149024144.
            \item \tiny Dominion Account Number: 4248285449. 
            \item \tiny Renter's Insurance Arlington, VA account number:
            \item \tiny Hilton HHonors Number: 255506941. $\longleftarrow$ \textcolor{red}{Memorize} 
            \item \tiny Marriott Rewards number: 947965950. $\longleftarrow$ \textcolor{red}{Memorize}
            \item \tiny United MileagePlus number: 697 
            \item \tiny AAdvantage number: 847JJP0. password: 
            \item \tiny Delta number : 
            \item \tiny Southwest: vivek.k password: 
            \item \tiny Dad's AAdvantage number: BRU6630 Password: polarwind.
            \item \tiny Emerald Club number:  140294615   pass:   pass: $\leftarrow$         \textcolor{red}{Memorize.} 
            \item \tiny Avis Wizard number: AV0426. 
            \item \tiny Netflix user:  \url{lvk2408@gmail.com}  Netflix password: chitrapat.
            \item \tiny Matrimony ID: R1438138 - R one FOUR three eight one three eight 
            \item \tiny SH ID:

            \item \tiny Contact: 
\begin{itemize}
             \item \tiny
             \item \tiny My phone number: 217-369-7996
\item \tiny Google Voice: 650-733-9327
\item \tiny BlueJeans: 703-812-3704 
\item \tiny Zoom conference call: Dial: 408-638-0968, when prompted, enter the meeting ID: 939 078 1212
\end{itemize}
            \end{enumerate}
          \end{itemize}
          
        \end{frame}

%\input{acks}
%\input{thanks} 

%\bibliographystyle{abbrv}
%\bibliography{bibliography} 

\end{document}
