\documentclass[9pt]{article}

\newcommand{\comments}[1]{}  

%Summary: this serves as the intermediary between the logical word outline
%and the mindmap. The idea is to give the specific purpusose. The
%mindmap should tie in the general purpose and structure the lower
%level nodes. 

\usepackage{outlines}
\usepackage{enumitem}
\usepackage{color}
\setenumerate[1]{label=\Roman*.}
\setenumerate[2]{label=\Alph*.}
\setenumerate[3]{label=\roman*.}
\setenumerate[4]{label=\alph*.}

\newcommand{\fontSize}{\tiny}
\newcommand{\todo}[1]{\textcolor{blue}{[TODO: #1]}}

\begin{document}

%TODO: add points to application timestep 
%TODO: check transition for GPU 
%TODO: L0: check 'this needed'

%TODO: L4: clean up 

\textbf{Low-Overhead Loop Scheduling Depth Talk Outline}

%General Goal: to tell people what I've done.

%Specific Goal: to tell people what I've done.

\begin{outline}[enumerate]
  \fontSize \1 {\fontSize Opening Matter} 
  \fontSize \2 {\fontSize Title } 
  \fontSize \3 {\fontSize Summary: Hi, my name is Vivek Kale. This
    describes the worklife cheatsheet. At the end of this, I hope you
    see that this is useful and what I'm going  to do improve
    it. \todo{make this clearer}.}
  \fontSize \3 {\fontSize Working with Professor Bill Gropp.} 
\fontSize \1 {\fontSize Motivation:}
\fontSize \2 {\fontSize  Opening / introduction: exp. examples,
  happiness}
\fontSize \3 {\fontSize Exp. Quickly explain model program.} 
\fontSize \3 {\fontSize happiness. }
\fontSize \3 {\fontSize  }
\fontSize \2 {\fontSize Example MPI Program.} 
\fontSize \3 {\fontSize Quick explanation of program.} 
\fontSize \3 {\fontSize Example program should have no performance
  problem to deal with.}  
\fontSize \3 {\fontSize Application Timeline Schematics: app/OS
  imbalance.}  

\fontSize \1 {\fontSize Key1: Importance } 
\fontSize \2 {\fontSize Key1.1: Habits} 
\fontSize \2 {\fontSize Key1.2: Long-term static}
\fontSize \2 {\fontSize Key1.3: M1}
\fontSize \3 {\fontSize Worry elimination}
\fontSize \3 {\fontSize Work}
\fontSize \3 {\fontSize Knowledge} 
\fontSize \3 {\fontSize Managment} 
\fontSize \1 {\fontSize Key2: Comm} 
\fontSize \2 {\fontSize Key2.1: S1} 
\fontSize \3 {\fontSize Comm org}
\fontSize \3 {\fontSize General skills}
\fontSize \3 {\fontSize Specific situations}
\fontSize \2 {\fontSize Key2.2: real-time/dynamic}
\fontSize \3 {\fontSize Dress}
\fontSize \3 {\fontSize Body language} 
\fontSize \3 {\fontSize Vocal variety} 
\fontSize \2 {\fontSize Key2.3: Experience}

%TODO: clean up 
\fontSize \3 {\fontSize Guidelines before any situation} 
\fontSize \3 {\fontSize Work /interview}
\fontSize \3 {\fontSize rel/date}  
\fontSize \3 {\fontSize More general situations} 
%TODO: put real-world experience here.

%TODO:  could consider putting experience/project in conclusion as
%example case study 
%TODO: even better : think about  putting this into  experiences 

\fontSize \1 { \fontSize Current Active project  / } 
\fontSize \1 {\fontSize Conclusions and Closing Matter:}
\fontSize \2 {\fontSize Conclusions}
  %TODO : make point more clear 
  \fontSize \3 {\fontSize Multi-core optimized version reduces thread
    idle time - has some benefit.} 
  \fontSize \3 {\fontSize OpenACC does significantly better than MPI version, as
    well as an OpenMP version.} 
  %TODO:  
  \fontSize \3 {\fontSize Further work will be done to check feasibility on
    future architectures.}
\fontSize \2 {\fontSize References relevant to this work} 
\fontSize \2 {\fontSize Acknowledgements - don't talk about, just
  show. }  

%\fontSize \1 { \fontSize Current work is done for improving GPUs.}
%\fontSize \1 { \fontSize  Staggered Scheduling improves spatial
%  locality by putting dynamic loop iterations close to static ones.} 

%{ \fontAt large scales this could have better performance
% improvement. -across-node: can also handle transient} 

% Nodes have accelerators -  We have just started exploring this for
% the PlasComCM code  - John Larson and Simon. Also, worked with 
% Nodes today tend to have accelerators - what about G - Xeon Phi -
% examples of accelerators current clusters:  lanl mic, llnl gpu  
% For something like Xeon Phi, we expect the same techniques will work
% effectively. But what about GPUs? 

% At least today, there may not be importance of variability for
% GPUs. 
%\fontSize \2 {\fontSize More clusters have GPUs - PlasComCM suitability for GPUs} 
%\fontSize \3 {\fontSize Explain experimental setup, and note that data resident on device.}
%\fontSize \3 {\fontSize Main Result: Using a GPU on aselect kernel for
%  PlasComCM improves performance 3x over optimized multi-core strategy.}
% - John


\end{outline}
\end{document}
