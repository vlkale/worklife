\documentclass[]{beamer}
\usetheme{Luebeck}
\useoutertheme{infolines}
\usepackage{inputenc}

\usepackage{graphicx}
\usepackage[square,numbers]{natbib}
\usepackage[final]{listings}
\usepackage{color}
\usepackage{xcolor}
\usepackage{url}
\usepackage[colorinlistoftodos]{todonotes}

\usepackage{hyperref}

\usepackage{booktabs} % For \toprule, \midrule and \bottomrule
\usepackage{microtype}
\usepackage{csvsimple}
\usepackage{framed} % used for putting boxes around content easily
\usepackage{type1cm}
\usepackage{amsmath,amsfonts,amssymb,xspace}
\usepackage{tikz}
\usepackage{beamerthemesplit} 
\usepackage{graphicx,ragged2e,pgffor}

\usepackage{ulem}
\usepackage{courier}
\usepackage{verbatim,listings,float} %TODO: order these 

%\usepackage{dtklogos}
\usepackage{tikz}
\usetikzlibrary{mindmap,shadows}

\setbeamertemplate{caption}[numbered]
\usepackage{etoolbox}

\usepackage{lipsum}
\usepackage{multicol}

\def\PRES{1}
\def\POSTER{1}

\newcounter{saveenumi}
\newcommand{\seti}{\setcounter{saveenumi}{\value{enumi}}}
\newcommand{\conti}{\setcounter{enumi}{\value{saveenumi}}}
\resetcounteronoverlays{saveenumi}

\usepackage{fancyvrb}

\newenvironment{mdn}%
    {\VerbatimEnvironment\begin{VerbatimOut}{tmp.markdown}}%
    {\end{VerbatimOut}%
        \immediate\write18{pandoc tmp.markdown -t latex -o tmp.tex}%
        \input{tmp.tex}}



\usepackage[fencedCode, hybrid]{markdown}

\markdownSetup{rendererPrototypes={
 link = {\href{#2}{#1}},
 headingThree = {\begin{frame}\frametitle{#1}},
 headingFour = {\begin{block}{#1}},
 horizontalRule = {\end{block}}
}}



\usepackage{catchfilebetweentags}

\makeatletter
\newread\pin@file
\newcounter{pinlineno}
\newcommand\pin@accu{}
\newcommand\pin@ext{pintmp}
% inputs #3, selecting only lines #1 to #2 (inclusive)
\newcommand*\partialinput [3] {%
  \IfFileExists{#3}{%
    \openin\pin@file #3
    % skip lines 1 to #1 (exclusive)
    \setcounter{pinlineno}{1}
    \@whilenum\value{pinlineno}<#1 \do{%
      \read\pin@file to\pin@line
      \stepcounter{pinlineno}%
    }
    % prepare reading lines #1 to #2 inclusive
    \addtocounter{pinlineno}{-1}
    \let\pin@accu\empty
    \begingroup
    \endlinechar\newlinechar
    \@whilenum\value{pinlineno}<#2 \do{%
      % use safe catcodes provided by e-TeX's \readline
      \readline\pin@file to\pin@line
      \edef\pin@accu{\pin@accu\pin@line}%
      \stepcounter{pinlineno}%
    }
    \closein\pin@file
    \expandafter\endgroup
    \scantokens\expandafter{\pin@accu}%
  }{%
    \errmessage{File `#3' doesn't exist!}%
  }%
}
\makeatother

%\url{http://tex.stackexchange.com/q/78041/86}
%\documentclass{beamer}
%\url{http://tex.stackexchange.com/q/78041/86}

\usepackage{pgfpages}
\usepackage{pgf}

\pgfpagesdeclarelayout{9 on 1}
{
  \edef\pgfpageoptionheight{\the\paperheight}
  \edef\pgfpageoptionwidth{\the\paperwidth}
  \edef\pgfpageoptionborder{0pt}
}
{
  \pgfpagesphysicalpageoptions
  {
    logical pages=9,
    physical height=\pgfpageoptionheight,
    physical width=\pgfpageoptionwidth
  }
  \def\pgfpgtemp{}
  \foreach \i in {1,...,3} {
    \foreach \j in {1,...,3} {
      \pgfmathtruncatemacro\n{(\j-1)*3 + \i}
      \pgfmathsetmacro\ri{1- (\i - .5)/3}
      \pgfmathsetmacro\rj{(\j - .5)/3}
      \edef\temp{%
        \noexpand\pgfpageslogicalpageoptions{\n}
      {
        border code=\noexpand\pgfsetlinewidth{1pt}\noexpand\pgfstroke,
        border shrink=\noexpand\pgfpageoptionborder,
        resized width=1.2\noexpand\pgfphysicalwidth,
        resized height=.345\noexpand\pgfphysicalheight,
        center=\noexpand\pgfpoint{\rj\noexpand\pgfphysicalwidth}{\ri\noexpand\pgfphysicalheight}
      }
      }
      \expandafter\expandafter\expandafter\gdef\expandafter\expandafter\expandafter\pgfpgtemp\expandafter\expandafter\expandafter{\expandafter\pgfpgtemp\temp}
    }
  }
  \pgfpgtemp
}

\pgfpagesuselayout{9 on 1}[a0paper, border shrink=0.5mm, landscape]

%\begin{document}

%\foreach \k in {1,...,20} {

%\begin{frame}{Frame \k}

%This is frame \k.
%It is exciting.
%\end{frame}
%}

%\usepackage[hidelinks,pdfencoding=auto]{hyperref}
% Information boxes
\newcommand*{\info}[4][16.3]{%
  \node [ annotation, #3, scale=0.65, text width = #1em,
          inner sep = 2mm ] at (#2) {%
  \list{$\bullet$}{\topsep=0pt\itemsep=0pt\parsep=0pt
    \parskip=0pt\labelwidth=8pt\leftmargin=8pt
    \itemindent=0pt\labelsep=2pt}%
    #4
  \endlist
  };
}


%TODO: remove unneeded items 
\newcommand{\bllt}{\item \small}
\newcommand{\doneTaskNoItem}[1]{\sout{#1}}
%TODO: consider changing the name of this macro to be what it's
%used for below

\newcommand{\projectTask}{\tiny \item \tiny}
\newcommand{\pitem}{\tiny \item \tiny}
\newcommand{\ptask}{\tiny \item \tiny}
\newcommand{\mpitem}{\tiny \item \tiny}
\newcommand{\doneTaskNoItemNewLine}[1]{\sout{#1}}

\newcommand{\doneTask}[1]{\tiny \item \tiny \sout{#1}}
\newcommand{\doneTaskHyp}[1]{\tiny \item \tiny \textcolor{blue} {\sout{#1}}}
\newcommand{\optTask}[1]{\tiny \item \tiny \textcolor{green}{#1}}
\newcommand{\prioTask}[1]{\tiny \item \tiny \textcolor{red}{#1}}
\newcommand{\timeEst}[1]{\textit{TimeEst:}\textit{#1}}
\newcommand{\te}[1]{\textit{TimeEst:}\textit{#1}}

\newcommand{\deadline}[1]{\textit{Deadline:}\textit{#1}}
\newcommand{\dueBy}[1]{\textit{Deadline:}\textit{#1}}
\newcommand{\dl}[1]{\textit{Deadline:}\textit{#1}}
\newcommand{\priority}[1]{\textit{Priority:}\textit{#1}}
\newcommand{\prio}[1]{\textit{Priority:}\textit{#1}}
\newcommand{\pr}[1]{\textit{Priority:}\textit{#1}}

\newcommand{\MyName}{Vivek~Kale}
\newcommand{\fixme}[1]{\textcolor{blue}{[FIXME: #1]}}
\newcommand{\revision}[1]{\textcolor{blue}{[FIXME comment : #1]}}
\newcommand{\regItem}[1]{\item \textcolor{cyan}{#1}}
\newcommand{\regRoutineItem}[1]{\item \textcolor{green}{\textit{Reg. Routine:} #1}}
\newcommand{\situationItem}[1]{\item \textcolor{magenta}{\textit{Situation:} #1}} 
\newcommand{\comments}[1]{}


%TODO: think of document as adding around ...
%TODO: consider changing the below so you only use the text in the
%document 
%TODO: change Week plans 
%TODO: consider putting making each frames have its own document. 
%TODO: decide where social intelligence goes. 
%TODO: decide how to separate work from life in social intelligence
%notes.

\newcommand{\footleft}{Mgmt-WorkLife-monthweekPlan} % figure what this should be 
\newcommand{\footright}{May 2019}

%-- Main Document -------------------------------------------------
%\title{Month-Week-Day Plan}\author{Vivek Kale$^1$}\institute{$^1$ University of Illinois at Urbana-Champaign}\date{\today}

\begin{document}

\section{Summary}
\input{summarySlide.tex}

\section{Job Search}
\begin{frame}{Job Search}


%switch to deal with overleaf inconsistency. 
\IfFileExists{../thePast/situations/jobsList.csv}{../thePast/situations/jobsList.csv}{\tiny\csvautotabular{./thePast/situations/jobsList.csv}}


\end{frame}


\section{Long-term planning}


%\begin{frame}{Year Plan}{2019}
\begin{itemize} 
\item Jan/Feb 2019:
\item March/April 2019: 
\item May/June 2019:
\item July/August 2019: 
\item September/October 2019: 
\item November/December 2019:
\end{itemize}
\end{frame}

\begin{frame}{Six-month Plan}{Jan through June 2019}
\begin{itemize}
\item December: continue work on uds, making notes on implementation. 
\item January: uds implementation 
\item February: uds paper for problem, algorithmic strategies 
\item March: uds paper for research questions and algorithmic strategies 
\item April: uds paper for SC19 with applications such as LuLEsh with RAJA. 
\item May: uds paper for SC19 with applications run on supercomputers. 
\item June: UDS paper for SC19. 
\end{itemize}
\end{frame}


\begin{frame}{Year Plan}{2019}
\ExecuteMetaData[workLifeMgr/content-planDoc.docx]{ypn}
\end{frame}


\begin{frame}{Six-month Plan}{Jan through June 2019}
\ExecuteMetaData[workLifeMgr/content-planDoc.docx]{spn}
\end{frame}


\section{Project plans}
\relax

%\begin{markdown}
%\markdownInput{./workLifeMgr/content_projectslist.md}
%\end{markdown}


\section{Projects}
%TODO: get rid of dependencies of Project Name here. 

\begin{frame}{Project CCI}
\ExecuteMetaData[workLifeMgr/content-planDoc.docx]{cci}
\end{frame}
\begin{markdown}
\begin{frame}{Project:Dmy}
\ExecuteMetaData[workLifeMgr/content-planDoc.docx]{dmy}
\end{frame}
\end{markdown}
\begin{frame}{Project:UDS}
\ExecuteMetaData[workLifeMgr/content-planDoc.docx]{uds}
\end{frame}
\section{Planning}
\begin{frame}{Month Plan}
\begin{multicols}{2}
\underline{\bf Week of January $28^{th}$}
\begin{framed}
\begin{itemize}
\tiny \item \tiny Work: 
\item \tiny Work:rel: follow-up from interview with AMD with Anshu and Chris. 
\item \tiny Work:rel: get badge from NVIDIA. 
\end{itemize}
\end{framed}

\underline{\bf Week of February $3^{rd}$}\\
\begin{framed}
\begin{itemize}
\tiny \item \tiny Work:rel: follow-up from interview with AMD. 

\tiny \item \tiny Work: meeting follow-up from OpenMP F2F. 
\item \tiny Work: work on RAJA. 
\item \tiny Work: coding practice.

\item \tiny Mgmt:Spaces:
\item \tiny Mgmt:Spaces:
\item \tiny Comm: figure out meeting with Aditi. 
\end{itemize}
\end{framed}

\underline{\bf Week of February $10^{th}$}\\
\begin{framed}
\begin{itemize}
\tiny \item \tiny Work: charmworks update. 
\item \tiny Work: work on RAJA. 
\item \tiny Mgmt:Spaces:
\item \tiny Comm: 
\end{itemize}
\end{framed}

\underline{\bf Week of February $17^{th}$}
\begin{framed}
\begin{itemize}
\tiny \item \tiny Work: unify all projects into one. Write project report. 
\item \tiny Work:
\item \tiny Work:
\item \tiny Mgmt:Spaces:
\item \tiny Comm:
\end{itemize}
\end{framed}
\end{multicols}

\end{frame}
\begin{frame}{Regular Routines}
\underline{\bf Morning}
\begin{itemize}
\tiny \item \tiny Brush Teeth \timeEst{4 mins}.
\item \tiny Mouthwash, Clean tongue - it helps with breath\timeEst{1mins}
\item \tiny Eating small breakfast.
\item \tiny Check your schedule for meetings / appointments for that
day: Is anything urgent? \timeEst{5 mins}.
\item \tiny Excercise: 1 hour ( Stretch for 10 mins, Treadmill for 30
mins , Lift weights or Yoga 1 hour, Meditation for 10 mins).
\item \tiny Shave (at least every other day) \timeEst{5 mins}.
\item \tiny Shower : (shampoo+conditioner, acne wash,  soap on
underarms/underbody) \timeEst{10 mins}
\item \tiny dry hair, dry face, arms, legs, back, stomach \timeEst{3
mins}
\item \tiny Clothing: belt, check collar, shirt not half-tucked,
matching socks \timeEst{5 mins}
\item \tiny Comb hair, hair gel optional \timeEst{ 1 min}
\item \tiny Check Keys/Wallet/Cell/Badge (get a bucket) \timeEst{1
min}
\item \tiny Open meds cabinet, get water, take medicines \timeEst{1
min}
\item \tiny Wear lenses and rinse out case \timeEst{ 2 mins}
\item \tiny Deodorant or cologne depending on the day \timeEst{1 min}
\item \tiny Wash your face with a face wash - it helps with
acne \timeEst{3 mins}
\item \tiny Moisturize - apply lotion \timeEst{ 2 min}
\item \tiny Neti pot / clean out sinuses - if you’re feeling
congested \timeEst{5 min}
\item \tiny clean out ears, nose \timeEst{5 mins}
\item \tiny Remember to lock door \timeEst{1 mins}
\end{itemize}

\underline{\bf Evening}
\begin{itemize}
\tiny \item \tiny Call family: 30 mins
\item \tiny meet friends: 20 mins
\item \tiny Cleaning dishes / cleaning house: 10 mins
\item \tiny Lay out clothes for the next day, iron if necessary
\item \tiny start laundry if needed: 10 mins,
\item \tiny Collect anything you’ll need for the next day (e.g. dry
cleaning to drop off): 5 mins,
\item \tiny Check Facebook/linkedin/google+: 10 mins, matrimony stuff:10 mins,
\item \tiny Food for lunch/dinner the next day: 10 mins
\end{itemize}
\underline{\bf Night before bed}
%TODO: why isn't enumerate working???
\begin{itemize}
\tiny \item \tiny Finish computer stuff \timeEst{5 mins}
\item \tiny Get into sleeping clothes \timeEst{1 mins}
\item \tiny Before sleeping: brush teeth, floss \timeEst{2 min} 
\item \tiny Properly wash and put away contacts \timeEst{2 mins}
\item \tiny Review the days good things \timeEst{10 min} 
\item \tiny Light Reading \timeEst{ 10 min}
\item \tiny Prayer \timeEst{10 mins}
\end{itemize}

\end{frame}
\begin{frame}{Weekend Plans}
\ifdefined\POSTER
\begin{block}{Weekend Plans - June $1^{st}$, 2018 to July $15^{th}$, 2018} 
\else
{\underline{\bf Weekend Plans - June $1^{st}$, 2018 to July $15^{th}$, 2018}}\\ 
\fi
\begin{itemize}
\tiny \item \tiny \underline{Weekend of June $1^{st}$:}     {\it Details:}
\item \tiny \underline{Weekend of June  $8^{th}$:} Meet Aditi? {\it Details:}
\item \tiny \underline{Weekend of June $15^{th}$:} Go to SF?  {\it Details:} 
\item \tiny \underline{Weekend of June $22^{nd}$:} Bhaskar's wedding.  {\it Details:} 
\item \tiny \underline{Weekend of June $29^{th}$:} {\it Details:} 
\item \tiny \underline{Weekend of July $8^{th}$:}  Go to
England? \textit{Details:}

\item \tiny \underline{Weekend of July $15^{th}$:}  \textit{Details:}
\end{itemize}

\ifdefined\POSTER
\end{block}
\fi

\ifdefined\POSTER
\begin{block}{Weekend Plans - July $15^{th}$, 2018 to September $1^{st}$, 2018} 
\else
{\underline{\bf Weekend Plans - July $15^{th}$, 2018 to September $1^{st}$, 2018}}\\ 
\fi 

\begin{itemize}
\tiny \item \tiny \underline{Weekend of July $15^{th}$:}     {\it Details:}
\item \tiny \underline{Weekend of July $22^{nd}$:}  Atul comes to Illinois {\it Details:}
\item \tiny \underline{Weekend of August $1^{st}$:} Chicago {\it Details:} 
\item \tiny \underline{Weekend of August $9^{th}$:} Chicago {\it Details:} 
\item \tiny \underline{Weekend of August $16^{th}$:} \textit{Details:}

\item \tiny \underline{Weekend of August $23^{rd}$:}  \textit{Details:}

\end{itemize}

\ifdefined\POSTER
\end{block}
\fi

\end{frame}

\section{Week Plan}

\begin{frame}{Worries List}
%\underline{Deadline: September $31^{st}$, 2017}
\begin{itemize}
\item jobs $\rightarrow$ we'll find something , we'll look for something on east. 
\item RS $\rightarrow$ forget her 
\end{itemize} 

\end{frame}
 
\begin{frame}{Lessons} 
\end{frame}
\section{News}
\begin{frame}{News}
\begin{itemize}
\tiny \item \tiny M:  S: 
\end{itemize} 

\underline{Weather}
\begin{itemize}
\tiny \item \tiny M:  S: 
\end{itemize} 


\end{frame} 

%%\documentclass{article}
\immediate\write18{ansiweather -f 7 -s false -a false -l Champaign  > currWeather.tex}
%\begin{document}
%

%\begin{frame}{Weather}{Week of \today}
%\input{|''ansiweather -l Champaign -F -s false -a false''}
%\end{frame}

\begin{frame}{Weather}{Week of \today}

\input{./currWeather.tex}

%\begin{figure}
%\includegraphics[scale=0.4]{currLocWeather.png}
%\end{figure}

\end{frame}


%\end{document}


\section{Week Plan}
\begin{markdown}
\begin{frame}{Week Plan Summary}
\ExecuteMetaData[workLifeMgr/content-planDoc.docx]{wps}
\end{frame}
\end{markdown}

\begin{markdown}
\begin{frame}[allowframebreaks]{Running To DO}
\ExecuteMetaData[workLifeMgr/content-planDoc.docx]{rtd}
\end{frame}
\end{markdown}

\section{Daily Plans}
\begin{markdown} 
\begin{frame}{Day Plans}
\ExecuteMetaData[workLifeMgr/content-planDoc.docx]{dpn}
\end{frame} 
\end{markdown}

%\section{Social Life}

\section{Situations}

\begin{frame}{Situations}


\begin{itemize}
\item Bhaskar's wedding
\item Meet Aditi 
\end{itemize}

\end{frame}


\section{Friends And Rel}
\begin{frame}{FriendsAndRel}

\end{frame}

\section{Meeting Information}
\begin{frame}[label=mtgPSMimg]
\frametitle{Meeting by Vivek}
\underline{\bf Join on computer}
\begin{itemize}
\small \item \small Dial: 199.48.152.152
\item \small Meeting ID: 5061993
\end{itemize}
\underline{\bf Join by phone}
\begin{itemize}
\small \item \small 408.740.7256 or 888.240.2560
\item \small Meeting ID: 5061993
\item \small Press \#
\end{itemize}
\end{frame}

\begin{frame}[label=omp_lang]
\frametitle{Meeting for OMP-lang}
\underline{\bf Join WebEx meeting}
\begin{itemize}
\small \item \small Meet Tuesday at 10AM CT /11AM ET
\small \item \small \url{https://llnl.webex.com/llnl/j.php?MTID=md575d91e6134ce9b0aaf14bc65c011eb}
\small \item \small \url{http://tinyurl.com/ompltc}
%\item \small Meeting number: 805 639 779
\item \small Meeting number: 808 344 165
\item \small Meeting password: openmp
\end{itemize}
\underline{\bf Join by phone}
\begin{itemize}
\small \item \small +1-415-655-0001 US TOLL
\item \small Access code: 808 344 165
\end{itemize}
\end{frame}

\begin{frame}[label=omp_affinity]
\frametitle{Meeting for omp-affinity}
\underline{\bf Join WebEx meeting}
\begin{itemize}
\small \item \small  \url{https://llnl.webex.com/llnl/j.php?MTID=md575d91e6134ce9b0aaf14bc65c011eb}
\item \small Meeting number: 846 335 992
\item \small Meeting password: places
\end{itemize}
\underline{\bf Join by phone}
\begin{itemize}
\item +1-415-655-0001 US TOLL
\item Access code: 846 335 992
\end{itemize}
\end{frame}

%TODO: add this. 

\begin{frame}[label=omp_task]
\frametitle{Meeting for omp-task}
\underline{\bf Join WebEx meeting}
\begin{itemize}
\small \item \small \url{}   
% \url{https://llnl.webex.com/llnl/j.php?MTID=md575d91e6134ce9b0aaf14bc65c011eb}
\item \small Meeting number: 846 335 992
\item \small Meeting password: places
\end{itemize}
\underline{\bf Join by phone}
\begin{itemize}
\item +1-415-655-0001 US TOLL
\item Access code: 846 335 992
\end{itemize}
\end{frame}


\begin{frame}[label=omp_device]
\frametitle{Meeting for omp-device}
\underline{\bf Join WebEx meeting}
\begin{itemize}
\small \item \small \url{}   
% \url{https://llnl.webex.com/llnl/j.php?MTID=md575d91e6134ce9b0aaf14bc65c011eb}
\item \small Meeting number: 846 335 992
\item \small Meeting password: places
\end{itemize}
\underline{\bf Join by phone}
\begin{itemize}
\item +1-415-655-0001 US TOLL
\item Access code: 846 335 992
\end{itemize}
\end{frame}

\section{Dating and Relationships}
\begin{frame}{Rel Plan}{\today}
\IfFileExists{../thePast/situations/girlsList.csv}{\tiny\csvautotabular{../thePast/situations/girlsList.csv}}{\tiny\csvautotabular{./thePast/situations/girlsList.csv}}
\end{frame}
\section{My Information} 
%\begin{frame}{My Information - 1}
\begin{itemize}
  \tiny \item \tiny work ID
  \begin{enumerate}
    \tiny \item \tiny UIUC UIN: 659743643
  \item \tiny LLNL ID: 008983 
  \item \tiny USC ID: 910-1254-783 $\leftarrow$ \textcolor{red}{Memorize} 
  \item \tiny USC ID: 2029269
  \end{enumerate}
\item \tiny Work equipment: 
  \begin{enumerate}
          \tiny \item \tiny Model A1707
          \tiny \item \tiny S/N: C02T30EZGTF1  
          \tiny \item \tiny Computer serial number:
          \tiny \item \tiny private159.east.isi.edu
        \item \tiny IP address:  65.114.168.159
        \item \tiny Whole disk encryption? 
        \end{enumerate}
%      \item \tiny insurance number:
     
        \seti 
        \end{itemize}
        \end{frame}
        \begin{frame}{My Information - 2}
        \begin{itemize}
        \conti 
 \tiny  \item \tiny  User ID
        \begin{enumerate}
       % \tiny \item \tiny bcbs : user: vivekkale pass: qazxsw23 memberID: QME9212849609  Group No.: IG2601. 
        \item \tiny USC:  user:   pass: 
      \item \tiny mycarle: vivekk112 password: carle
      \item \tiny ppn:532296999 exp: 3-16-2025  \textcolor{red}{Memorize} 
      	\item \tiny driver's license:K400-8728-4051 exp:2-20-18
          \tiny \item \tiny bcbs : user: vivekkale pass: qazxsw23 memberID: QME9212849609  Group No.: IG2601.
        \item \tiny USC:  user:   pass: 
        \end{enumerate}
      \item \tiny Money:
        \begin{enumerate}
          \tiny \item \tiny CC1 num: CC2 num: Dbtnum:  \textcolor{red}{Memorize} 
        \item \tiny Bank Account:
          Name: JP Morgan Chase
          bank acct num: Nine Five Eight Six Thirty Sixteen Two
          IBAN/routing num: Three Triple Two Seven One Six Two Seven 
          ABA-Fedwire No.: 021000021
        \item \tiny CCnum: Four One Four Threee
        \item \tiny VIN:  License Plate: IL AU 59436 reg.exp:   Atul's lic: P84 VIN:
        \item \tiny Dad's License Plate: IL A
        \item \tiny Mom's License Plate : IL P84 9023 
        \end{enumerate}
        \seti
        \end{itemize}
        \end{frame}
        \begin{frame}{My Information -3 }
          \begin{itemize}
            \conti  
	  \item Travel: 
            \begin{enumerate} 
%            \item \tiny VIN:  License Plate: IL E61 2444 reg.exp:   Atul's lic: P84 VIN: 
            \item \tiny Parking location codes:   Parking signs: 
            \item \tiny AtulInternet’s FightingIllini  wifi pass: 2173900374 WPA
            \item \tiny xfinity account number: 0149024144. / \url{vivek.lkale@gmail.com}
            \item \tiny ComEd account number: 0149024144.
            \item \tiny Dominion Account Number: 4248285449. 
            \item \tiny Renter's Insurance Arlington, VA account number:
            \item \tiny Hilton HHonors Number: 255506941. $\longleftarrow$ \textcolor{red}{Memorize} 
            \item \tiny Marriott Rewards number: 947965950. $\longleftarrow$ \textcolor{red}{Memorize}
            \item \tiny United MileagePlus number: 697 
            \item \tiny AAdvantage number: 847JJP0. password: 
            \item \tiny Delta number : 
            \item \tiny Southwest: vivek.k password: 
            \item \tiny Dad's AAdvantage number: BRU6630 Password: polarwind.
            \item \tiny Emerald Club number:  140294615   pass:   pass: $\leftarrow$         \textcolor{red}{Memorize.} 
            \item \tiny Avis Wizard number: AV0426. 
            \item \tiny Netflix user:  \url{lvk2408@gmail.com}  Netflix password: chitrapat.
            \item \tiny Matrimony ID: R1438138 - R one FOUR three eight one three eight 
            \item \tiny SH ID:

            \item \tiny Contact: 
\begin{itemize}
             \item \tiny
             \item \tiny My phone number: 217-369-7996
\item \tiny Google Voice: 650-733-9327
\item \tiny BlueJeans: 703-812-3704 
\item \tiny Zoom conference call: Dial: 408-638-0968, when prompted, enter the meeting ID: 939 078 1212
\end{itemize}
            \end{enumerate}
          \end{itemize}
          
        \end{frame}


\begin{markdown} %TODO: figure out how to distribute information ought to be distributed within these slides
\ExecuteMetaData[workLifeMgr/content-planDoc.docx]{myi}
\end{markdown}

\end{document}
