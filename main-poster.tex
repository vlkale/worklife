\documentclass{beamer}
%% Possible paper sizes: a0, a0b, a1, a2, a3, a4.
%% Possible orientations: portrait, landscape
%% Font sizes can be changed using the scale option.
\usepackage[size=a0,orientation=landscape,scale=1.8]{beamerposter}
\usetheme{LLT-poster}
%\usecolortheme{ComingClean}
\usecolortheme{Entrepreneur}
% \usecolortheme{ConspiciousCreep}  %% VERY garish.

\usepackage[utf8]{inputenc}
\usepackage[T1]{fontenc}
\usepackage{libertine}
\usepackage[scaled=0.92]{inconsolata}
\usepackage[libertine]{newtxmath}
\usepackage[numbers]{natbib}

\usepackage{framed} % used for putting a frame, or box, around text

\usepackage{multicol}
\renewcommand{\bibfont}{\small}

\newcommand{\bllt}{\item \small}
\newcommand{\doneTaskNoItem}[1]{\sout{#1}}
%TODO: consider changing the name of this macro to be what it's                                                                            
%used for below.     
\newcommand{\projectTask}{\tiny \item \tiny}
\newcommand{\pitem}{\tiny \item \tiny}
\newcommand{\ptask}{\tiny \item \tiny}
\newcommand{\mpitem}{\tiny \item \tiny}
\newcommand{\doneTaskNoItemNewLine}[1]{\sout{#1}}

\newcommand{\doneTask}[1]{\tiny \item \tiny \sout{#1}}
\newcommand{\doneTaskHyp}[1]{\tiny \item \tiny \textcolor{blue} {\sout{#1}}}
\newcommand{\optTask}[1]{\tiny \item \tiny \textcolor{green}{#1}}
\newcommand{\prioTask}[1]{\tiny \item \tiny \textcolor{red}{#1}}
\newcommand{\timeEst}[1]{\textit{TimeEst:}\textit{#1}}
\newcommand{\te}[1]{\textit{TimeEst:}\textit{#1}}

\newcommand{\deadline}[1]{\textit{Deadline:}\textit{#1}}
\newcommand{\dueBy}[1]{\textit{Deadline:}\textit{#1}}
\newcommand{\dl}[1]{\textit{Deadline:}\textit{#1}}
\newcommand{\priority}[1]{\textit{Priority:}\textit{#1}}
\newcommand{\prio}[1]{\textit{Priority:}\textit{#1}}
\newcommand{\pr}[1]{\textit{Priority:}\textit{#1}}


\newcommand{\MyName}{Vivek~Kale}
\newcommand{\fixme}[1]{\textcolor{blue}{[FIXME: #1]}}
\newcommand{\revision}[1]{\textcolor{blue}{[FIXME comment : #1]}}
\newcommand{\regItem}[1]{\item \textcolor{cyan}{#1}}
\newcommand{\regRoutineItem}[1]{\item \textcolor{green}{\textit{Reg. Routine:} #1}}
\newcommand{\situationItem}[1]{\item \textcolor{magenta}{\textit{Situation:} #1}}
\newcommand{\comments}[1]{}


\newcommand{\texthash}{\#}

%% Load the markdown package
\usepackage[citations,footnotes,definitionLists,hashEnumerators,smartEllipses,tightLists=false,hybrid]{markdown}
\markdownSetup{rendererPrototypes={
 link = {\href{#2}{#1}},
 headingFour = {\begin{block}{#1}},
 horizontalRule = {\end{block}}
}}

\author[vivek.lkale@gmail.com]{Vivek Kale}
\title{Week Plan Doc}
\institute{Charmworks, Inc.}
% Optional foot image
%\footimage{\includegraphics[width=4cm]{}}

\begin{document}

\begin{frame}[fragile]\centering

\markdownInput{workLifeMgr/content_yearPlan_plan.md}

\markdownInput{workLifeMgr/content_monthPlan_plan.md}

\bigskip
{\usebeamercolor[bg]{headline}\hrulefill}
\bigskip

\end{frame}

\begin{frame}[fragile]\centering

\begin{multicols}{2}
\markdownInput{/Users/vivekkale/Desktop/ViveksLaptop2/life/workLifeMgr/content_projects_scp.md}
\markdownInput{/Users/vivekkale/Desktop/ViveksLaptop2/life/workLifeMgr/content_projects_jbs.md}
\end{multicols}


\bigskip
{\usebeamercolor[bg]{headline}\hrulefill}
\bigskip

\begin{columns}[T]

%%%% First Column
\begin{column}{.22\textwidth}
\markdownInput{workLifeMgr/content_worries_plan.md}

\end{column}

%%%% Second Column

\begin{column}{.56\textwidth}
\markdownInput{workLifeMgr/content_runningToDoPlan.md}
\end{column}

%%%% Third Column
\begin{column}{.22\textwidth}
\markdownInput{workLifeMgr/content_news.md}

%\begin{markdown}

%#### This is a sample
%
%- One, two, pick up my shoe
%- Three, four, shut the door
%- Five, six, pick up sticks
%- Seven, eight, lay them straight
%- Nine, ten, a big fat hen
%- One, two, pick up my shoe
%- Three, four, shut the door
%- Five, six, pick up sticks
%- Seven, eight, lay them straight
%- Nine, ten, a big fat hen

%----

%#### This is another sample

%- Some maths material

%\begin{align}
%A &= U \times S \times V^T\\
%\sigma &= \frac{x\times y}{\sqrt[3]{\alpha + \beta}}
%\end{align}

%----

%\end{markdown}
\end{column}
\end{columns}


\begin{markdown}

#### This is a sample of a wiiiide column

- One, two, pick up my shoe
- Three, four, shut the door
- Five, six, pick up sticks
- Seven, eight, lay them straight
- Nine, ten, a big fat hen

----


#### Bibliography

\bibliographystyle{unsrtnat}
\bibliography{refs}

----

\end{markdown}

\end{frame}


\end{document}
